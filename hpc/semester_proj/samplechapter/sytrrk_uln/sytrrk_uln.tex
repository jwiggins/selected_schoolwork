

% The following commands help in generating the index
\index{index}{triangular symmetric rank-k update}%
\index{op}{triangular symmetric rank-k update!$ C \leftarrow A^T A + \hat{C} $|( }%

% Name of the chapter
\chapter{Triangular Symmetric Rank-k Update \\
$ C \leftarrow A^T A + C $ \\
$ C $ symmetric, stored in upper triangle
}
\label{chapter:sytrrk_uln}

% Authors
\ChapterAuthor{
\index{author}{Bradford, Jason}
Jason Bradford
\\[0.1in]
\index{author}{Wiggins, John}
John Wiggins
}

% Add the names of the authors to the table of contents
\addtocontents{toc}{ {by {\bf Jason Bradford, John Wiggins}}}

% Intro
In this chapter, we discuss the implementation of the triangular symmetric rank-k update
\[
C \leftarrow A^T A + C
\]
where $ C $ is an $ n \times n $ symmetric matrix and $ A $ is an $ n \times n $ lower
triangular matrix. We will overwrite $ C $ with the result without requiring a work array.
We start by deriving a couple of different sequential algorithms. Subsequently, we show how
to code the algorithms using FLAME.

The variables for the triangular symmetric rank-k update can be described by the precondition
\[
% precondition
P_{pre} : C = \hat{C} \wedge \RowDim(C) = \ColDim(C) \wedge \RowDim(A) = \ColDim(A)
\wedge \RowDim(C) = \RowDim(A) \wedge Symm(C) \wedge LowTr(A),
\]
where $ \hat{C} $ indicates the original contents of $ C $. Here the predicate $ Symm(C) $
is true iff $ C $ is symmetric, and the predicate $ LowTr(A) $ is true iff $ A $ is lower
triangular. The functions $ \RowDim(C) $, $ \RowDim(A) $  and $ \ColDim(C) $,
$ \ColDim(A) $ return the row and column dimension of $ C $ and $ A $, respectively. The
operation to be performed, $ C \leftarrow A^T A + C $, translates to the postcondition
$ P_{post} : C \leftarrow A^T A + \hat{C} $.

% first section
\[
\]
Let us start by partitioning matrix C.  Since it is symmetric, and stored in the upper
triangle we partition it like
\[
\FlaTwoByTwo{ C_{TL} }        { C_{TR} }
            { \undetermined } { C_{BR} }
\]
where $ C_{TL} $  and $ C_{BR} $ are both submatrices on the diagonal and are symmetric.
Notice that * indicates that that part of the matrix is not stored.
Substituting the partitioning of C into the postcondition yields
\[
\FlaTwoByTwo{ C_{TL} }        { C_{TR} }
            { \undetermined } { C_{BR} }
=
( \mbox{some partitioning of }{A^T} )
( \mbox{some partitioning of }{A} )
+
\FlaTwoByTwo{ \hat{C}_{TL} }  { \hat{C}_{TR} }
            { \undetermined } { \hat{C}_{BR} }
\]
This suggests that $ A $ should be partitioned
into quadrants.  Let us consider
the case where $ A $ is partitioned as described above.  Then
\[
\FlaTwoByTwo{ C_{TL} }       { C_{TR} }
            { \undetermined} { C_{BR} }
=
\FlaTwoByTwo { A_{TL}^T }      { A_{BL}^T }
             { 0 }      { A_{BR}^T }
\FlaTwoByTwo { A_{TL} }      { 0 }
             { A_{BL} }      { A_{BR} }
+
\FlaTwoByTwo{ \hat{C}_{TL} }  { \hat{C}_{TR} }
            { \undetermined } { \hat{C}_{BR} }
\]
In order to be able to multiply and add the matrices on the right out and to
be able to then set the submatrices on the left equal to the result on
the right we find that the following must hold:
\[
\ColDim( C_{TL} ) = \RowDim( A_{TL} )
\wedge
\RowDim( C_{TL} ) = \RowDim( A_{TL} )
\wedge
\RowDim( C_{TL} ) = \ColDim( C_{TL} )
\]
Substituting the partitioned matrices into the postcondition, we find
\begin{eqnarray*}
 \FlaTwoByTwo{ C_{TL} }       { C_{TR} }
            { \undetermined} { C_{BR} }
& = &
\FlaTwoByTwo { A_{TL}^T }      { A_{BL}^T }
             { 0 }      { A_{BR}^T }
\FlaTwoByTwo { A_{TL} }      { 0 }
             { A_{BL} }      { A_{BR} }
+
\FlaTwoByTwo{ \hat{C}_{TL} }  { \hat{C}_{TR} }
            { \undetermined } { \hat{C}_{BR} }
\\ & = &
\FlaTwoByTwo { A_{TL}^T A_{TL} + A_{BL}^T A_{BL} + \hat{C}_{TL} }   { A_{BL}^T A_{BR} + \hat{C}_{TR}}
             {\undetermined}  {A_{BR}^T A_{BR} + \hat{C}_{BR}}
\end{eqnarray*}
which yields the equalities
\begin{equation}
\label{eqn:sytrrk_uln}
\begin{array}{r c l || r c l}
C_{TL} &=& A_{TL}^T A_{TL} + A_{BL}^T A_{BL} + \hat{C}_{TL} & C_{TR} &=& A_{BL}^T A_{BR} + \hat{C}_{TR} \\ \hline \hline
C_{BL} &=& \undetermined & C_{BR} &=&  A_{BR}^T A_{BR} + \hat{C}_{BR}


\end{array}
\end{equation}


From Eqn.~\ref{eqn:sytrrk_uln} we conclude that the operations to be performed are
$ A_{TL}^T A_{TL} + A_{BL}^T A_{BL} + \hat{C}_{TL} $,  $ A_{BL}^T A_{BR} + \hat{C}_{TR}$,
and $ A_{BR}^T A_{BR} + \hat{C}_{BR} $. At an intermediate stage, each of these either has or has not already
been computed, leading to the $ 2^4 = 16 $ possible loop-invariants tabulated in Fig.~\ref{fig:sytrrk_uln_invariants}

% display the table of preconditions
\begin{figure}
\begin{center}
\footnotesize
{
\setlength{\tabcolsep}{4pt}
\begin{tabular}{| c | p{2.5in} | c | } \hline
$ P_X: $

$ \FlaTwoByTwo{ C_{TL} } { C_{TR} }
              { \undetermined } { C_{BR} } =  $ & Comment &
\footnotesize Feasible? \\ \hline \hline
\footnotesize
$
\FlaTwoByTwo{ \hat{C}_{TL} }  { \hat{C}_{TR} }
            { \undetermined } { \hat{C}_{BR} }
$
&
&
NO
\\ \hline
%
%
\footnotesize
$
\FlaTwoByTwo{ \hat{C}_{TL} } { \hat{C}_{TR} }
            { \undetermined } { A_{BR}^T A_{BR} + \hat{C}_{BR} }
$
&
Loop Invariant 2.
&
YES
\\ \hline
%
%
\footnotesize
$
\FlaTwoByTwo{ \hat{C}_{TL} }  { A_{BL}^T A_{BR} + \hat{C}_{TR} }
            { \undetermined } { \hat{C}_{BR} }
$
&
&
NO
\\ \hline
%
%
\footnotesize
$
\FlaTwoByTwo{ \hat{C}_{TL} }  { A_{BL}^T A_{BR} + \hat{C}_{TR} }
            { \undetermined } { A_{BR}^T A_{BR} + \hat{C}_{BR} }
$
&
&
YES
\\ \hline
%
%
\footnotesize
$
\FlaTwoByTwo{ A_{BL}^T A_{BL} + \hat{C}_{TL} } { \hat{C}_{TR} }
            { \undetermined }                  { \hat{C}_{BR} }
$
&
&
NO
\\ \hline
%
%
\footnotesize
$
\FlaTwoByTwo{ A_{BL}^T A_{BL} + \hat{C}_{TL} } { \hat{C}_{TR} }
            { \undetermined }                  { A_{BR}^T A_{BR} + \hat{C}_{BR} }
$
&
&
NO
\\ \hline
%
%
\footnotesize
$
\FlaTwoByTwo{ A_{BL}^T A_{BL} + \hat{C}_{TL} } { A_{BL}^T A_{BR} + \hat{C}_{TR} }
            { \undetermined }                  { \hat{C}_{BR} }
$
&
&
NO
\\ \hline
%
%
\footnotesize
$
\FlaTwoByTwo{ A_{BL}^T A_{BL} + \hat{C}_{TL} } { A_{BL}^T A_{BR} + \hat{C}_{TR} }
            { \undetermined }                  { A_{BR}^T A_{BR} + \hat{C}_{BR} }
$
&
&
NO
\\ \hline
\footnotesize
$
\FlaTwoByTwo{ A_{TL}^T A_{TL} + \hat{C}_{TL} }  { \hat{C}_{TR} }
            { \undetermined } { \hat{C}_{BR} }
$
&
&
NO
\\ \hline
%
%
\footnotesize
$
\FlaTwoByTwo{ A_{TL}^T A_{TL} + \hat{C}_{TL} } { \hat{C}_{TR} }
            { \undetermined } { A_{BR}^T A_{BR} + \hat{C}_{BR} }
$
&
&
NO
\\ \hline
%
%
\footnotesize
$
\FlaTwoByTwo{ A_{TL}^T A_{TL} + \hat{C}_{TL} }  { A_{BL}^T A_{BR} + \hat{C}_{TR} }
            { \undetermined } { \hat{C}_{BR} }
$
&
&
NO
\\ \hline
%
%
\footnotesize
$
\FlaTwoByTwo{ A_{TL}^T A_{TL} + \hat{C}_{TL} }  { A_{BL}^T A_{BR} + \hat{C}_{TR} }
            { \undetermined } { A_{BR}^T A_{BR} + \hat{C}_{BR} }
$
&
&
NO
\\ \hline
%
%
\footnotesize
$
\FlaTwoByTwo{ A_{TL}^T A_{TL} + A_{BL}^T A_{BL} + \hat{C}_{TL} } { \hat{C}_{TR} }
            { \undetermined }                  { \hat{C}_{BR} }
$
&
Loop Invariant 1.
&
YES
\\ \hline
%
%
\footnotesize
$
\FlaTwoByTwo{ A_{TL}^T A_{TL} + A_{BL}^T A_{BL} + \hat{C}_{TL} } { \hat{C}_{TR} }
            { \undetermined }                  { A_{BR}^T A_{BR} + \hat{C}_{BR} }
$
&
&
NO
\\ \hline
%
%
\footnotesize
$
\FlaTwoByTwo{ A_{TL}^T A_{TL} + A_{BL}^T A_{BL} + \hat{C}_{TL} } { A_{BL}^T A_{BR} + \hat{C}_{TR} }
            { \undetermined }                  { \hat{C}_{BR} }
$
&
&
YES
\\ \hline
%
%
\footnotesize
$
\FlaTwoByTwo{ A_{TL}^T A_{TL} + A_{BL}^T A_{BL} + \hat{C}_{TL} } { A_{BL}^T A_{BR} + \hat{C}_{TR} }
            { \undetermined }                  { A_{BR}^T A_{BR} + \hat{C}_{BR} }
$
&
&
NO
\\ \hline
\end{tabular}
}
\end{center}
\caption{Possible loop-invariants when partitioning
$ C $ into quadrants.
Here $ P_X $ is the most prominent part of the loop-invariant
$ \PInv $.}
\label{fig:sytrrk_uln_invariants}
\end{figure}



\section{Loop-invariant 1}
We will first examine the third feasible loop-invariant from Fig.~\ref{fig:sytrrk_uln_invariants}.
\begin{equation}
\label{eqn:sytrrk_uln_inv1}
\PInv:
\FlaTwoByTwo{ C_{TL} }       { C_{TR} }
            { \undetermined} { C_{BR} }
=
\FlaTwoByTwo{ A_{TL}^T A_{TL} + A_{BL}^T A_{BL} + \hat{C}_{TL} }  { \hat{C}_{TR} }
            { \undetermined } { \hat{C}_{BR} }
\wedge
\ldots
\end{equation}


Comparing the loop invariant in Eqn.~\ref{eqn:sytrrk_uln_inv1} with the postcondition $ C =  A^T A + \hat{C}$
we see that as we progress through the matrix, we are only doing work within $ C_{TL} $, and after each iteration
we gradually add more contents to $ C_{TL} $ until all of C is resident in $C_{TL} $. Thus, we must
choose a loop-guard G such that its negation $\neg{G} $, implies that the dimensions of these matrices
match appropriately and therefore that $ ( \PInv \wedge \neg{G} ) \implies \PPost $. The loop-guard
$ G = \neg \SameSize(C_{TL}, C)  $ meets this condition.

Loop-guard G indicates that eventually $C_{TL}$ should equal all of C, at which point G becomes false
and the loop is exited. After the initialization, $C_{TL}$ is $ 0 \times 0 $.  Thus, the repartitioning should be
such that as the computation proceeds, rows and columns are subtracted from $C_{BR}$ and added to $C_{TL}$
while updating $C_{TL}$ consistently with this. This is illustrated by the shifting of the double-lines.

\subsection{Blocked algorithm}

The repartitioning of the variables and the loop-invariant $\PInv $ dictates $\QBefore $

%\begin{eqnarray*}
{
\normalsize
$ \\
\FlaTwoByTwo{ C_{TL} } { C_{TR} }
            { * } { C_{BR} }
\rightarrow
\FlaThreeByThreeBR{ C_{00} }{ C_{01} }{ C_{02} }
              { * }{ C_{11} }{ C_{12} }
              { * }{ * }{ C_{22} }
$,
$
\FlaTwoByTwo{ \hat{C}_{TL} } { \hat{C}_{TR} }
            { * } { \hat{C}_{BR} }
\rightarrow
\FlaThreeByThreeBR{ \hat{C}_{00} }{ \hat{C}_{01} }{ \hat{C}_{02} }
              { * }{ \hat{C}_{11} }{ \hat{C}_{12} }
              { * }{ * }{ \hat{C}_{22} }
$, \\ \\ and
$
\FlaTwoByTwo{ A_{TL} } { 0 }
            { A_{BL} } { A_{BR} }
\rightarrow
\FlaThreeByThreeBR{ A_{00} }{ 0 }{ 0 }
              { A_{10} }{ A_{11} }{ 0 }
              { A_{20} }{ A_{21} }{ A_{22} }
$ \\
}
%\end{eqnarray*}

After updating $ C_{TL} $, we would then continue with

{
\normalsize
$ \\
\FlaTwoByTwo{ C_{TL} } { C_{TR} }
            { * } { C_{BR} }
\leftarrow
\FlaThreeByThreeTL{ C_{00} }{ C_{01} }{ C_{02} }
              { * }{ C_{11} }{ C_{12} }
              { * }{ * }{ C_{22} }
$, \\ \\
$
\FlaTwoByTwo{ \hat{C}_{TL} } { \hat{C}_{TR} }
            { * } { \hat{C}_{BR} }
\leftarrow
\FlaThreeByThreeTL{ \hat{C}_{00} }{ \hat{C}_{01} }{ \hat{C}_{02} }
              { * }{ \hat{C}_{11} }{ \hat{C}_{12} }
              { * }{ * }{ \hat{C}_{22} }
$, and \\ \\
$
\FlaTwoByTwo{ A_{TL} } { 0 }
            { A_{BL} } { A_{BR} }
\leftarrow
\FlaThreeByThreeTL{ A_{00} }{ 0 }{ 0 }
              { A_{10} }{ A_{11} }{ 0 }
              { A_{20} }{ A_{21} }{ A_{22} }
$ \\
}

If we combine the initial partitioning with the loop invariant, we get the intermediate state

{
$  \\
\QBefore:
\FlaThreeByThreeBR{ C_{00} }{ C_{01} }{ C_{02} }
              { * }{ C_{11} }{ C_{12} }
              { * }{ * }{ C_{22} }
=
\FlaThreeByThreeBR{ A_{00}^T A_{00} + A_{10}^T A_{10} + A_{20}^T A_{20} + \hat{C}_{00} }{ \hat{C}_{01} }{ \hat{C}_{02} }
              { * }{ \hat{C}_{11} }{ \hat{C}_{12} }
              { * }{ * }{ \hat{C}_{22} }
$
}
\\
\\

If we combine the post-update partitioning with the loop invariant, we get the post-update state

$  \\
\QAfter:
\FlaThreeByThreeTL{ C_{00} }{ C_{01} }{ C_{02} }
              { * }{ C_{11} }{ C_{12} }
              { * }{ * }{ C_{22} }
=
\FlaThreeByThreeTL{ A_{00}^T A_{00} + A_{10}^T A_{10} + A_{20}^T A_{20} + \hat{C}_{00} }
                  { A_{10}^T A_{11} + A_{20}^T A_{21} + \hat{C}_{01} }{ \hat{C}_{02} }
              { * }{ A_{11}^T A_{11} + A_{21}^T A_{21} + \hat{C}_{11} }{ \hat{C}_{12} }
              { * }{ * }{ \hat{C}_{22} }
$

Comparing $\QBefore$ with $\QAfter$ we find that the updates
\\
\\
$\begin{array}{l}

C_{01} \becomes A_{10}^T A_{11} + A_{20}^T A_{21} + \hat{C}_{01} \\
C_{11} \becomes A_{11}^T A_{11} + A_{21}^T A_{21} + \hat{C}_{11}

\end{array} $
\\
\\
are required to change the state from $\QBefore$ to $\QAfter$, which yields the final algorithm.

In order to cast the algorithm to be rich in matrix-matrix multiplications, the repartitioning and
redefinition of the submatrices in Steps 5a and 5b of Fig.~\ref{fig:sytrrk_uln_downright_blk_ann}
expose multiple rows and/or columns at a time. Block size b can be chosen to be any size that
does not exceed the number of rows in $C_{BR}$. $Q_{before}$ is obtained by plugging the repartitioning
in Step 6 of Fig.~\ref{fig:sytrrk_uln_downright_blk_ann} into $P_{inv}$ while $Q_{after}$ is
obtained by plugging the redefinition of the quadrants in Step 7 of Fig.~\ref{fig:sytrrk_uln_downright_blk_ann}
into $P_{inv}$. The update in Step 8 of Fig.~\ref{fig:sytrrk_uln_downright_blk_ann}
is now obtained by comparing the state in Steps 6 and 7 of Fig.~\ref{fig:sytrrk_uln_downright_blk_ann}.
The bulk of the computation is now in the updates given above.

% display the blocked algorithm


%
% The Algorithm
%

% Initial Partitioning
\renewcommand{\partitionings}{
$
C \rightarrow
\FlaTwoByTwo{ C_{TL} } { C_{TR} }
            { * } { C_{BR} }
$,
$
\hat{C} \rightarrow
\FlaTwoByTwo{ \hat{C}_{TL} } { \hat{C}_{TR} }
            { * } { \hat{C}_{BR} }
$ and
$
A \rightarrow
\FlaTwoByTwo{ A_{TL} } { 0 }
            { A_{BL} } { A_{BR} }
$
}


\renewcommand{\partitionsizes}{$ C_{TL} $, $ \hat{C}_{TL} $, and $ A_{TL} $ is $ 0 \times 0 $ }

% loop guard
\renewcommand{\guard}{ \neg \SameSize( C_{TL}, C ) }


% Repartitioning at start of loop
\renewcommand{\repartitionings}{
\normalsize
$
\FlaTwoByTwo{ C_{TL} } { C_{TR} }
            { * } { C_{BR} }
\rightarrow
\FlaThreeByThreeBR{ C_{00} }{ C_{01} }{ C_{02} }
              { * }{ C_{11} }{ C_{12} }
              { * }{ * }{ C_{22} }
$, \\ \\
$
\FlaTwoByTwo{ \hat{C}_{TL} } { \hat{C}_{TR} }
            { * } { \hat{C}_{BR} }
\rightarrow
\FlaThreeByThreeBR{ \hat{C}_{00} }{ \hat{C}_{01} }{ \hat{C}_{02} }
              { * }{ \hat{C}_{11} }{ \hat{C}_{12} }
              { * }{ * }{ \hat{C}_{22} }
$, and \\ \\
$
\FlaTwoByTwo{ A_{TL} } { 0 }
            { A_{BL} } { A_{BR} }
\rightarrow
\FlaThreeByThreeBR{ A_{00} }{ 0 }{ 0 }
              { A_{10} }{ A_{11} }{ 0 }
              { A_{20} }{ A_{21} }{ A_{22} }
$
}

% the update
\renewcommand{\update}{
$
\begin{array}{l}
C_{01} \becomes A_{10}^T A_{11} + A_{20}^T A_{21} + \hat{C}_{01} \\
C_{11} \becomes A_{11}^T A_{11} + A_{21}^T A_{21} + \hat{C}_{11}
\end{array}
$
}

\renewcommand{\repartitionsizes}{
$ C_{11} $ , $ \hat{C}_{11} $, and $ A_{11} $ are $ b \times b $ matrices}

% Moving double lines at end of loop
\renewcommand{\moveboundaries}{
\normalsize
$
\FlaTwoByTwo{ C_{TL} } { C_{TR} }
            { * } { C_{BR} }
\leftarrow
\FlaThreeByThreeTL{ C_{00} }{ C_{01} }{ C_{02} }
              { * }{ C_{11} }{ C_{12} }
              { * }{ * }{ C_{22} }
$, \\ \\
$
\FlaTwoByTwo{ \hat{C}_{TL} } { \hat{C}_{TR} }
            { * } { \hat{C}_{BR} }
\leftarrow
\FlaThreeByThreeTL{ \hat{C}_{00} }{ \hat{C}_{01} }{ \hat{C}_{02} }
              { * }{ \hat{C}_{11} }{ \hat{C}_{12} }
              { * }{ * }{ \hat{C}_{22} }
$, and \\ \\
$
\FlaTwoByTwo{ A_{TL} } { 0 }
            { A_{BL} } { A_{BR} }
\leftarrow
\FlaThreeByThreeTL{ A_{00} }{ 0 }{ 0 }
              { A_{10} }{ A_{11} }{ 0 }
              { A_{20} }{ A_{21} }{ A_{22} }
$
}

% output
\begin{figure}[tbp]
\begin{center}
\FlaAlgorithm
\end{center}
\caption{Blocked algorithm for $ C \becomes A^T A + \hat{C} $.}
\label{fig:sytrrk_uln_downright_blk}
\end{figure}


\subsection{Unblocked algorithm}
The unblocked algorithm for loop invariant 1 is similar to the blocked version, except that the repartitioning and
redefinition of the submatrices in Steps 5a and 5b of Fig.~\ref{fig:sytrrk_uln_downright_unblk_ann}
expose a single row or column at a time. This in turn has an effect on the partitioning and
repartitioning of the matrices, and thus a different update. The update in Step 8 of
Fig.~\ref{fig:sytrrk_uln_downright_unblk_ann} is now obtained by comparing the state in
Steps 6 and 7 of Fig.~\ref{fig:sytrrk_uln_downright_unblk_ann}. The bulk of the computation is now in the update
given below.
\\
{
$  \\
\QBefore:
\FlaThreeByThreeBR{ C_{00} }{ c_{01} }{ C_{02} }
              { * }{ \gamma_{11}}{ C_{12}^T }
              { * }{ * }{ C_{22} }
=
\FlaThreeByThreeBR{ A_{00}^T A_{00} + a_{10} a_{10}^T + A_{20}^T A_{20} + \hat{C}_{00} }{ \hat{c}_{01} }{ \hat{C}_{02} }
              { * }{ \hat{\gamma}_{11} }{ \hat{c}_{12}^T }
              { * }{ * }{ \hat{C}_{22} }
$
}
\\
\\
If we combine the post-update partitioning with the loop invariant, we get the post-update state %$ \Qafter $.
%The redefinition in Step 5b means that the following state must result from the update:


$  \\
\QAfter:
\FlaThreeByThreeTL{ C_{00} }{ c_{01} }{ C_{02} }
              { * }{ \gamma_{11} }{ c_{12}^T }
              { * }{ * }{ C_{22} }
=
\FlaThreeByThreeTL{  A_{00}^T A_{00} + a_{10} a_{10}^T + A_{20}^T A_{20} + \hat{C}_{00} }
                  { A_{20}^T a_{21} + \alpha_{11} a_{10} + \hat{c}_{01} }{ \hat{C}_{02} }
              { * }{ a_{21}^T a_{21} + \alpha_{11}^2 + \hat{\gamma}_{11} }{ \hat{c}_{12}^T }
              { * }{ * }{ \hat{C}_{22} }
$

Comparing $\QBefore$ with $\QAfter$ we find that the updates
\\
\\
$c_{01} \becomes A_{20}^T a_{21} + \alpha_{11} a_{10} + \hat{C}_{01}$ \\
$ \gamma_{11} \becomes a_{21}^T a_{21} + \alpha_{11} \alpha_{11} + \hat{\gamma}_{11} $
\\

In general, an unblocked algorithm can be used, or a blocked algorithm can call an unblocked algorithm as part
of its update.

% display the unblocked algorithm


%
% The Algorithm
%

% Initial Partitioning
\renewcommand{\partitionings}{
$
C \rightarrow
\FlaTwoByTwo{ C_{TL} } { C_{TR} }
            { * } { C_{BR} }
$,
$
\hat{C} \rightarrow
\FlaTwoByTwo{ \hat{C}_{TL} } { \hat{C}_{TR} }
            { * } { \hat{C}_{BR} }
$ and
$
A \rightarrow
\FlaTwoByTwo{ A_{TL} } { 0 }
            { A_{BL} } { A_{BR} }
$
}


\renewcommand{\partitionsizes}{$ C_{TL} $, $ \hat{C}_{TL} $, and $ A_{TL} $ is $ 0 \times 0 $ }

% loop guard
\renewcommand{\guard}{ \neg \SameSize( C_{TL}, C ) }


% Repartitioning at start of loop
\renewcommand{\repartitionings}{
\normalsize
$
\FlaTwoByTwo{ C_{TL} } { C_{TR} }
            { * } { C_{BR} }
\rightarrow
\FlaThreeByThreeBR{ C_{00} }{ c_{01} }{ C_{02} }
              { * }{ \gamma_{11} }{ c_{12}^T }
              { * }{ * }{ C_{22} }
$, \\ \\
$
\FlaTwoByTwo{ \hat{C}_{TL} } { \hat{C}_{TR} }
            { * } { \hat{C}_{BR} }
\rightarrow
\FlaThreeByThreeBR{ \hat{C}_{00} }{ \hat{c}_{01} }{ \hat{C}_{02} }
              { * }{ \hat{\gamma}_{11} }{ \hat{c}_{12}^T }
              { * }{ * }{ \hat{C}_{22} }
$, and \\ \\
$
\FlaTwoByTwo{ A_{TL} } { 0 }
            { A_{BL} } { A_{BR} }
\rightarrow
\FlaThreeByThreeBR{ A_{00} }{ 0 }{ 0 }
              { a_{10}^T }{ \alpha_{11} }{ 0 }
              { A_{20} }{ a_{21} }{ A_{22} }
$
}

% the update
\renewcommand{\update}{
$
\begin{array}{l}
c_{01} \becomes A_{20}^T a_{21} + \alpha_{11} a_{10} + \hat{c}_{01} \\
\gamma_{11} \becomes a_{21}^T a_{21} + \alpha_{11}^2 + \hat{\gamma}_{11}
\end{array}
$
}

\renewcommand{\repartitionsizes}{
$ \gamma_{11} $ , $ \hat{\gamma}_{11} $, and $ \alpha_{11} $ are scalars,
$ c_{01} $ , $ \hat{c}_{01} $, and $ a_{21} $ are column vectors, \\
and $ c_{12}^T $ , $ \hat{c}_{12}^T $, and $ a_{10}^T $ are row vectors. }

% Moving double lines at end of loop
\renewcommand{\moveboundaries}{
\normalsize
$
\FlaTwoByTwo{ C_{TL} } { C_{TR} }
            { * } { C_{BR} }
\leftarrow
\FlaThreeByThreeTL{ C_{00} }{ c_{01} }{ C_{02} }
              { * }{ \gamma_{11} }{ c_{12}^T }
              { * }{ * }{ C_{22} }
$, \\ \\
$
\FlaTwoByTwo{ \hat{C}_{TL} } { \hat{C}_{TR} }
            { * } { \hat{C}_{BR} }
\leftarrow
\FlaThreeByThreeTL{ \hat{C}_{00} }{ \hat{c}_{01} }{ \hat{C}_{02} }
              { * }{ \hat{\gamma}_{11} }{ \hat{c}_{12}^T }
              { * }{ * }{ \hat{C}_{22} }
$, and \\ \\
$
\FlaTwoByTwo{ A_{TL} } { 0 }
            { A_{BL} } { A_{BR} }
\leftarrow
\FlaThreeByThreeTL{ A_{00} }{ 0 }{ 0 }
              { a_{10}^T }{ \alpha_{11} }{ 0 }
              { A_{20} }{ a_{21} }{ A_{22} }
$
}

% output
\begin{figure}[tbp]
\begin{center}
\FlaAlgorithm
\end{center}
\caption{Unblocked algorithm for $ C \becomes A^T A + \hat{C} $.}
\label{fig:sytrrk_uln_downright_unblk}
\end{figure}


\section{Loop Invariant 2}
We now examine the loop-invariant
\begin{equation}
\label{eqn:sytrrk_uln_inv2}
\PInv:
\FlaTwoByTwo{ C_{TL} }        { C_{TR} }
            { \undetermined } { C_{BR} }
=
\FlaTwoByTwo{ {C}_{TL} } { {C}_{TR} }
            { \undetermined }                { A_{BR}^T A_{BR} + \hat{C}_{BR} }
\wedge
\ldots
\end{equation}
Comparing the loop-invariant in Eqn.~\ref{eqn:sytrrk_uln_inv2} with the postcondition $ C = A^T A + \hat{C} $
we see that if $ C = C_{BR}$ , $A = A_{BR}$, and $\hat{C} = \hat{C}_{BR}$ then the loop-invariant implies the
postcondition, i.e., that the desired result has been computed. The fact that for this partitioning of
C, $\hat{C}$, and A,
\\
\[
\FlaTwoByTwo{ C_{TL} }        { C_{TR} }
            { \undetermined } { C_{BR} }
=
\FlaTwoByTwo{ {C}_{TL} } { {C}_{TR} }
            { \undetermined }                { A_{BR}^T A_{BR} + \hat{C}_{BR} }
\]
and the precondition implies the other parts of the loop-invariant, this initialization has the desired
properties.

Loop-guard $G = \neg \SameSize(C_{BR}, C)$ indicates that eventually $C_{BR}$ should equal all of C, at which
point $G$ becomes false and the loop is exited.  After the initialization, $C_{BR}$ is  $0 \times 0 $. Thus,
the repartitioning should be such that as the computation proceeds, rows and columns are subtracted from $C_{TL} $
and added to $C_{BR}$ while updating $C_{BR}$ consistently with this.

\subsection{Blocked algorithm}
If we combine the initial partitioning with the loop invariant, we get the intermediate state
\\
\\
$
\QBefore:
\FlaThreeByThreeTL{ C_{00} }{ C_{01} }{ C_{02} }
              { * }{ C_{11} }{ C_{12} }
              { * }{ * }{ C_{22} }
=
\FlaThreeByThreeTL{ \hat{C}_{00} }{ \hat{C}_{01} }{ \hat{C}_{02} }
              { * }{ \hat{C}_{11} }{ \hat{C}_{12} }
              { * }{ * }{ A_{22}^T A_{22} + \hat{C}_{22} }
$
\\
\\
If we combine the post-update partitioning with the loop invariant, we get the post-update state
\\
$\\
\QAfter:
\FlaThreeByThreeBR{ C_{00} }{ C_{01} }{ C_{02} }
              { * }{ C_{11} }{ C_{12} }
              { * }{ * }{ C_{22} }
=
\FlaThreeByThreeBR{ \hat{C}_{00} } { \hat{C}_{01} }{ \hat{C}_{02} }
              { * }{ A_{11}^T A_{11} + A_{21}^T A_{21} + \hat{C}_{11} }{ A_{21}^T A_{22} + \hat{C}_{12} }
              { * }{ * }{ A_{22}^T A_{22} + \hat{C}_{22} }
$
\\
\\
Comparing $\QBefore$ with $\QAfter$ we obtain the updates
\\
\\
$C_{11} \becomes A_{11}^T A_{11} + A_{21}^T A_{21} + \hat{C}_{11} \\$
$C_{12} \becomes A_{21}^T A_{22} + \hat{C}_{12}$
\\

\subsection{Unblocked algorithm}

As with the algorithm for loop invariant 1, the differences between blocked and unblocked versions are small.
The notable differences are to the states $ \Qbefore $, $ \Qafter $, and the updates given below.

$
\QBefore:
\FlaThreeByThreeTL{ C_{00} }{ c_{01} }{ C_{02} }
              { * }{ \gamma_{11} }{ c_{12}^T }
              { * }{ * }{ C_{22} }
=
\FlaThreeByThreeTL{ \hat{C}_{00} }{ \hat{c}_{01} }{ \hat{C}_{02} }
              { * }{ \hat{\gamma}_{11} }{ \hat{c}_{12}^T }
              { * }{ * }{ A_{22}^T A_{22} + \hat{C}_{22} }
$
\\
\\
If we combine the post-update partitioning with the loop invariant, we get the post-update state %$ \Qafter $.
\\
$\\
\QAfter:
\FlaThreeByThreeBR{ C_{00} }{ c_{01} }{ C_{02} }
              { * }{ \gamma_{11} }{ c_{12}^T }
              { * }{ * }{ C_{22} }
=
\FlaThreeByThreeBR{ \hat{C}_{00} } { \hat{c}_{01} }{ \hat{C}_{02} }
              { * }{ a_{21}^T a_{21} + \alpha_{11}^2 + \hat{\gamma}_{11} }{ A_{22}^T a_{21} + \hat{c}_{12}^T }
              { * }{ * }{ A_{22}^T A_{22} + \hat{C}_{22} }
$
\\
Comparing $\QBefore$ with $\QAfter$ we find that the updates
\\
\\
$\gamma_{11} \becomes a_{21}^T a_{21} + \alpha_{11}^2 + \hat{\gamma}_{11}$ \\
$c_{12} \becomes A_{22}^T a_{21} + \hat{c}_{12}$
\\

\section{Performance}


In this section we will discuss performance and how our blocked, unblocked, and recursive algorithms compare to
the reference FLAME implementation. Performance was measured on a 600 MHz Pentium III Processor with 512k level 2 cache.
The graph we generated shows each algorithms performance in MFLOPS. We see from loop invariant 1 that the unblocked
algorithm seems to perform best when the matrix diminensions are $ 250 \times 250 $ or below.
This is likely due to cache issues and the fact that the unblocked algorithm calls fewer routines for the update
than the blocked algorithm.

Observing the graphs in Fig.~\ref{fig:sytrrk_uln:perfgraph}, we see that the blocked algorithm performs better than
the unblocked algorithm because it uses matrix-matrix operations to perform its updates. Meanwhile, the unblocked
algorithm uses matrix-vector and vector-vector operations to perform its updates. Our recursive algorithm only has
an advantage for very large block sizes, while its performance is comparable to the blocked algorithm for the
blocksize we used.


% display the blocked annotated algorithm
%
% Annotated Algorithm
%


% our operation
\renewcommand{\operation}{C \becomes A^T A + C}


% precondition
\renewcommand{\precondition}{C = \hat{C} \wedge n(C) = m(C) \wedge n(A) = m(A) \wedge n(C) = n(A)
\wedge UpTr(C) \wedge LowTr(A)}

% postcondition
\renewcommand{\postcondition}{{ C \becomes A^T A + \hat{C} }}

% invariant 1
\renewcommand{\invariant}{
\FlaTwoByTwo{ C_{TL} } { C_{TR} }
            { * } { C_{BR} }
=
\FlaTwoByTwo{ A_{TL}^T A_{TL} + A_{BL}^T A_{BL} + \hat{C}_{TL} } { \hat{C}_{TR} }
            { * } { \hat{C}_{BR} }
}

% loop guard
\renewcommand{\guard}{ \neg \SameSize( C_{TL}, C ) }

% initial partitionings
\renewcommand{\partitionings}{
$
C \rightarrow
\FlaTwoByTwo{ C_{TL} } { C_{TR} }
            { * } { C_{BR} }
$,
$
\hat{C} \rightarrow
\FlaTwoByTwo{ \hat{C}_{TL} } { \hat{C}_{TR} }
            { * } { \hat{C}_{BR} }
$ and
$
A \rightarrow
\FlaTwoByTwo{ A_{TL} } { 0 }
            { A_{BL} } { A_{BR} }
$
}


\renewcommand{\partitionsizes}{$ C_{TL} $, $ \hat{C}_{TL} $, and $ A_{TL} $ is $ 0 \times 0 $ }


% Define the blocksize that appears in step 5a
\renewcommand{\blocksize}{ b }
%

% repartitioning at start of loop (step 5a)
\renewcommand{\repartitionings}{
\normalsize
$
\FlaTwoByTwo{ C_{TL} } { C_{TR} }
            { * } { C_{BR} }
\rightarrow
\FlaThreeByThreeBR{ C_{00} }{ C_{01} }{ C_{02} }
              { * }{ C_{11} }{ C_{12} }
              { * }{ * }{ C_{22} }
$, \\ \\
$
\FlaTwoByTwo{ \hat{C}_{TL} } { \hat{C}_{TR} }
            { * } { \hat{C}_{BR} }
\rightarrow
\FlaThreeByThreeBR{ \hat{C}_{00} }{ \hat{C}_{01} }{ \hat{C}_{02} }
              { * }{ \hat{C}_{11} }{ \hat{C}_{12} }
              { * }{ * }{ \hat{C}_{22} }
$, and \\ \\
$
\FlaTwoByTwo{ A_{TL} } { 0 }
            { A_{BL} } { A_{BR} }
\rightarrow
\FlaThreeByThreeBR{ A_{00} }{ 0 }{ 0 }
              { A_{10} }{ A_{11} }{ 0 }
              { A_{20} }{ A_{21} }{ A_{22} }
$ \\
}




\renewcommand{\repartitionsizes}{
$ C_{11} $ , $ \hat{C}_{11} $, and $ A_{11} $ are $ b \times b $ matrices}



% Moving double lines at end of loop (step 5b)
\renewcommand{\moveboundaries}{
\normalsize
$
\FlaTwoByTwo{ C_{TL} } { C_{TR} }
            { * } { C_{BR} }
\leftarrow
\FlaThreeByThreeTL{ C_{00} }{ C_{01} }{ C_{02} }
              { * }{ C_{11} }{ C_{12} }
              { * }{ * }{ C_{22} }
$, \\ \\
$
\FlaTwoByTwo{ \hat{C}_{TL} } { \hat{C}_{TR} }
            { * } { \hat{C}_{BR} }
\leftarrow
\FlaThreeByThreeTL{ \hat{C}_{00} }{ \hat{C}_{01} }{ \hat{C}_{02} }
              { * }{ \hat{C}_{11} }{ \hat{C}_{12} }
              { * }{ * }{ \hat{C}_{22} }
$, and \\ \\
$
\FlaTwoByTwo{ A_{TL} } { 0 }
            { A_{BL} } { A_{BR} }
\leftarrow
\FlaThreeByThreeTL{ A_{00} }{ 0 }{ 0 }
              { A_{10} }{ A_{11} }{ 0 }
              { A_{20} }{ A_{21} }{ A_{22} }
$ \\
}



% state before the update
\renewcommand{\beforeupdate}{
\FlaThreeByThreeBR{ C_{00} }{ C_{01} }{ C_{02} }
              { * }{ C_{11} }{ C_{12} }
              { * }{ * }{ C_{22} }
=
\FlaThreeByThreeBR{ A_{00}^T A_{00} + A_{10}^T A_{10} + A_{20}^T A_{20} + \hat{C}_{00} }{ \hat{C}_{01} }{ \hat{C}_{02} }
              { * }{ \hat{C}_{11} }{ \hat{C}_{12} }
              { * }{ * }{ \hat{C}_{22} }
}



% state after the update
\renewcommand{\afterupdate}{
\FlaThreeByThreeTL{ C_{00} }{ C_{01} }{ C_{02} }
              { * }{ C_{11} }{ C_{12} }
              { * }{ * }{ C_{22} }
=
\FlaThreeByThreeTL{ A_{00}^T A_{00} + A_{10}^T A_{10} + A_{20}^T A_{20} + \hat{C}_{00} }
                  { A_{10}^T A_{11} + A_{20}^T A_{21} + \hat{C}_{01} }{ \hat{C}_{02} }
              { * }{ A_{11}^T A_{11} + A_{21}^T A_{21} + \hat{C}_{11} }{ \hat{C}_{12} }
              { * }{ * }{ \hat{C}_{22} }
}



% the update
\renewcommand{\update}{
$
\begin{array}{l}
C_{01} \becomes A_{10}^T A_{11} + A_{20}^T A_{21} + \hat{C}_{01} \\
C_{11} \becomes A_{11}^T A_{11} + A_{21}^T A_{21} + \hat{C}_{11}
\end{array}
$
}

% output
\begin{figure}[tbp]
\begin{center}
\worksheet
\end{center}
\caption{Worksheet for deriving blocked algorithm for $ C \becomes A^T A + \hat{C} $.}
\label{fig:sytrrk_uln_downright_blk_ann}
\end{figure}


% display the unblocked annotated algorithm
%
% Annotated Algorithm
%


% our operation
\renewcommand{\operation}{C \becomes A^T A + C}


% precondition
\renewcommand{\precondition}{C = \hat{C} \wedge n(C) = m(C) \wedge n(A) = m(A) \wedge n(C) = n(A)
\wedge UpTr(C) \wedge LowTr(A)}

% postcondition
\renewcommand{\postcondition}{{ C \becomes A^T A + \hat{C} }}

% invariant 1
\renewcommand{\invariant}{
\FlaTwoByTwo{ C_{TL} } { C_{TR} }
            { * } { C_{BR} }
=
\FlaTwoByTwo{ A_{TL}^T A_{TL} + A_{BL}^T A_{BL} \hat{C}_{TL} } { \hat{C}_{TR} }
            { * } { \hat{C}_{BR} }
}

% loop guard
\renewcommand{\guard}{ \neg \SameSize( C_{TL}, C ) }

% initial partitionings
\renewcommand{\partitionings}{
$
C \rightarrow
\FlaTwoByTwo{ C_{TL} } { C_{TR} }
            { * } { C_{BR} }
$,
$
\hat{C} \rightarrow
\FlaTwoByTwo{ \hat{C}_{TL} } { \hat{C}_{TR} }
            { * } { \hat{C}_{BR} }
$ and
$
A \rightarrow
\FlaTwoByTwo{ A_{TL} } { 0 }
            { A_{BL} } { A_{BR} }
$
}


\renewcommand{\partitionsizes}{$ C_{TL} $, $ \hat{C}_{TL} $, and $ A_{TL} $ is $ 0 \times 0 $ }



% repartitioning at start of loop (step 5a)
\renewcommand{\repartitionings}{
\normalsize
$
\FlaTwoByTwo{ C_{TL} } { C_{TR} }
            { * } { C_{BR} }
\rightarrow
\FlaThreeByThreeBR{ C_{00} }{ c_{01} }{ C_{02} }
              { * }{ \gamma_{11} }{ c_{12}^T }
              { * }{ * }{ C_{22} }
$, \\ \\
$
\FlaTwoByTwo{ \hat{C}_{TL} } { \hat{C}_{TR} }
            { * } { \hat{C}_{BR} }
\rightarrow
\FlaThreeByThreeBR{ \hat{C}_{00} }{ \hat{c}_{01} }{ \hat{C}_{02} }
              { * }{ \hat{\gamma}_{11} }{ \hat{c}_{12}^T }
              { * }{ * }{ \hat{C}_{22} }
$, and \\ \\
$
\FlaTwoByTwo{ A_{TL} } { 0 }
            { A_{BL} } { A_{BR} }
\rightarrow
\FlaThreeByThreeBR{ A_{00} }{ 0 }{ 0 }
              { a_{10}^T }{ \alpha_{11} }{ 0 }
              { A_{20} }{ a_{21} }{ A_{22} }
$ \\
}

\renewcommand{\repartitionsizes}{
$ \gamma_{11} $ , $ \hat{\gamma}_{11} $, and $ \alpha_{11} $ are scalars,
$ c_{01} $ , $ \hat{c}_{01} $, and $ a_{21} $ are column vectors, \\
and $ c_{12}^T $ , $ \hat{c}_{12}^T $, and $ a_{10}^T $ are row vectors. }



% Moving double lines at end of loop (step 5b)
\renewcommand{\moveboundaries}{
\normalsize
$
\FlaTwoByTwo{ C_{TL} } { C_{TR} }
            { * } { C_{BR} }
\leftarrow
\FlaThreeByThreeTL{ C_{00} }{ c_{01} }{ C_{02} }
              { * }{ \gamma_{11} }{ c_{12}^T }
              { * }{ * }{ C_{22} }
$, \\ \\
$
\FlaTwoByTwo{ \hat{C}_{TL} } { \hat{C}_{TR} }
            { * } { \hat{C}_{BR} }
\leftarrow
\FlaThreeByThreeTL{ \hat{C}_{00} }{ \hat{c}_{01} }{ \hat{C}_{02} }
              { * }{ \hat{\gamma}_{11} }{ \hat{c}_{12}^T }
              { * }{ * }{ \hat{C}_{22} }
$, and \\ \\
$
\FlaTwoByTwo{ A_{TL} } { 0 }
            { A_{BL} } { A_{BR} }
\leftarrow
\FlaThreeByThreeTL{ A_{00} }{ 0 }{ 0 }
              { a_{10}^T }{ \alpha_{11} }{ 0 }
              { A_{20} }{ a_{21} }{ A_{22} }
$ \\
}



% state before the update
\renewcommand{\beforeupdate}{
\FlaThreeByThreeBR{ C_{00} }{ c_{01} }{ C_{02} }
              { * }{ \gamma_{11} }{ c_{12}^T }
              { * }{ * }{ C_{22} }
=
\FlaThreeByThreeBR{ A_{00}^T A_{00} + a_{10} a_{10}^T + A_{20}^T A_{20} + \hat{C}_{00} }{ \hat{c}_{01} }{ \hat{C}_{02} }
              { * }{ \hat{\gamma}_{11} }{ \hat{c}_{12}^T }
              { * }{ * }{ \hat{C}_{22} }
}



% state after the update
\renewcommand{\afterupdate}{
\FlaThreeByThreeTL{ C_{00} }{ c_{01} }{ C_{02} }
              { * }{ \gamma_{11} }{ c_{12}^T }
              { * }{ * }{ C_{22} }
=
\FlaThreeByThreeTL{ A_{00}^T A_{00} + a_{10} a_{10}^T + A_{20}^T A_{20} + \hat{C}_{00} }
                  { A_{20}^T a_{21} + \alpha_{11} a_{10} + \hat{c}_{01} }{ \hat{C}_{02} }
              { * }{ a_{21} a_{21}^T + \alpha_{11}^2 + \hat{\gamma}_{11} }{ \hat{c}_{12}^T }
              { * }{ * }{ \hat{C}_{22} }
}



% the update
\renewcommand{\update}{
$
\begin{array}{l}
c_{01} \becomes A_{20}^T a_{21} + \alpha_{11} a_{10} + \hat{c}_{01} \\
\gamma_{11} \becomes a_{21}^T a_{21} + \alpha_{11}^2 + \hat{\gamma}_{11}
\end{array}
$
}

% output
\begin{figure}[tbp]
\begin{center}
\worksheet
\end{center}
\caption{Worksheet for deriving unblocked algorithm for $ C \becomes A^T A + \hat{C} $.}
\label{fig:sytrrk_uln_downright_unblk_ann}
\end{figure}


% diplay the blocked code for invariant 1
\begin{figure}
\footnotesize
\listinginput{1}{sytrrk_uln/codes/invariant1/MY_Sytrrk_uln_blk.c}
\label{fig:sytrrk_uln_code_inv1_blk}
\caption{Source listing for loop invariant 1, blocked algorithm}
\end{figure}

% diplay the unblocked code for invariant 1
\begin{figure}
\footnotesize
\listinginput{1}{sytrrk_uln/codes/invariant1/MY_Sytrrk_uln_unblk.c}
\label{fig:sytrrk_uln_code_inv1_unblk}
\caption{Source listing for loop invariant 1, unblocked algorithm}
\end{figure}

% display the blocked annotated algorithm
%
% Annotated Algorithm
%


% our operation
\renewcommand{\operation}{C \becomes A^T A + C}


% precondition
\renewcommand{\precondition}{C = \hat{C} \wedge n(C) = m(C) \wedge n(A) = m(A) \wedge n(C) = n(A)
\wedge UpTr(C) \wedge LowTr(A)}

% postcondition
\renewcommand{\postcondition}{{ C \becomes A^T A + \hat{C} }}

% invariant 2
\renewcommand{\invariant}{
\FlaTwoByTwo{ C_{TL} } { C_{TR} }
            { * } { C_{BR} }
=
\FlaTwoByTwo{ \hat{C}_{TL} } { \hat{C}_{TR} }
            { * } { A_{BR}^T + A_{BR} + \hat{C}_{BR} }
}

% loop guard
\renewcommand{\guard}{ \neg \SameSize( C_{BR}, C ) }

% initial partitionings
\renewcommand{\partitionings}{
$
C \rightarrow
\FlaTwoByTwo{ C_{TL} } { C_{TR} }
            { * } { C_{BR} }
$,
$
\hat{C} \rightarrow
\FlaTwoByTwo{ \hat{C}_{TL} } { \hat{C}_{TR} }
            { * } { \hat{C}_{BR} }
$ and
$
A \rightarrow
\FlaTwoByTwo{ A_{TL} } { 0 }
            { A_{BL} } { A_{BR} }
$
}


\renewcommand{\partitionsizes}{$ C_{BR} $, $ \hat{C}_{BR} $, and $ A_{BR} $ are $ 0 \times 0 $ }


% Define the blocksize that appears in step 5a
\renewcommand{\blocksize}{ b }
%

% repartitioning at start of loop (step 5a)
\renewcommand{\repartitionings}{
\normalsize
$
\FlaTwoByTwo{ C_{TL} } { C_{TR} }
            { * } { C_{BR} }
\rightarrow
\FlaThreeByThreeTL{ C_{00} }{ C_{01} }{ C_{02} }
              { * }{ C_{11} }{ C_{12} }
              { * }{ * }{ C_{22} }
$, \\ \\
$
\FlaTwoByTwo{ \hat{C}_{TL} } { \hat{C}_{TR} }
            { * } { \hat{C}_{BR} }
\rightarrow
\FlaThreeByThreeTL{ \hat{C}_{00} }{ \hat{C}_{01} }{ \hat{C}_{02} }
              { * }{ \hat{C}_{11} }{ \hat{C}_{12} }
              { * }{ * }{ \hat{C}_{22} }
$, and \\ \\
$
\FlaTwoByTwo{ A_{TL} } { 0 }
            { A_{BL} } { A_{BR} }
\rightarrow
\FlaThreeByThreeTL{ A_{00} }{ 0 }{ 0 }
              { A_{10} }{ A_{11} }{ 0 }
              { A_{20} }{ A_{21} }{ A_{22} }
$ \\
}




\renewcommand{\repartitionsizes}{
$ C_{11} $ , $ \hat{C}_{11} $, and $ A_{11} $ are $ b \times b $ matrices}



% Moving double lines at end of loop (step 5b)
\renewcommand{\moveboundaries}{
\normalsize
$
\FlaTwoByTwo{ C_{TL} } { C_{TR} }
            { * } { C_{BR} }
\leftarrow
\FlaThreeByThreeBR{ C_{00} }{ C_{01} }{ C_{02} }
              { * }{ C_{11} }{ C_{12} }
              { * }{ * }{ C_{22} }
$, \\ \\
$
\FlaTwoByTwo{ \hat{C}_{TL} } { \hat{C}_{TR} }
            { * } { \hat{C}_{BR} }
\leftarrow
\FlaThreeByThreeBR{ \hat{C}_{00} }{ \hat{C}_{01} }{ \hat{C}_{02} }
              { * }{ \hat{C}_{11} }{ \hat{C}_{12} }
              { * }{ * }{ \hat{C}_{22} }
$, and \\ \\
$
\FlaTwoByTwo{ A_{TL} } { 0 }
            { A_{BL} } { A_{BR} }
\leftarrow
\FlaThreeByThreeBR{ A_{00} }{ 0 }{ 0 }
              { A_{10} }{ A_{11} }{ 0 }
              { A_{20} }{ A_{21} }{ A_{22} }
$ \\
}



% state before the update
\renewcommand{\beforeupdate}{
\FlaThreeByThreeTL{ C_{00} }{ C_{01} }{ C_{02} }
              { * }{ C_{11} }{ C_{12} }
              { * }{ * }{ C_{22} }
=
\FlaThreeByThreeTL{ \hat{C}_{00} }{ \hat{C}_{01} }{ \hat{C}_{02} }
              { * }{ \hat{C}_{11} }{ \hat{C}_{12} }
              { * }{ * }{ A_{22}^T A_{22} + \hat{C}_{22} }
}



% state after the update
\renewcommand{\afterupdate}{
\FlaThreeByThreeBR{ C_{00} }{ C_{01} }{ C_{02} }
              { * }{ C_{11} }{ C_{12} }
              { * }{ * }{ C_{22} }
=
\FlaThreeByThreeBR{ \hat{C}_{00} } { \hat{C}_{01} }{ \hat{C}_{02} }
              { * }{ A_{11}^T A_{11} + A_{21}^T A_{21} + \hat{C}_{11} }{ A_{21}^T A_{22} + \hat{C}_{12} }
              { * }{ * }{ A_{22}^T A_{22} + \hat{C}_{22} }
}



% the update
\renewcommand{\update}{
$
\begin{array}{l}
C_{11} \becomes A_{11}^T A_{11} + A_{21}^T A_{21} + \hat{C}_{11} \\
C_{12} \becomes A_{21}^T A_{22} + \hat{C}_{12}
\end{array}
$
}

% output
\begin{figure}[tbp]
\begin{center}
\worksheet
\end{center}
\caption{Worksheet for deriving blocked algorithm for $ C \becomes A^T A + \hat{C} $.}
\label{fig:sytrrk_uln_upleft_blk_ann}
\end{figure}


% display the blocked algorithm


%
% The Algorithm
%

% Initial Partitioning
\renewcommand{\partitionings}{
$
C \rightarrow
\FlaTwoByTwo{ C_{TL} } { C_{TR} }
            { * } { C_{BR} }
$,
$
\hat{C} \rightarrow
\FlaTwoByTwo{ \hat{C}_{TL} } { \hat{C}_{TR} }
            { * } { \hat{C}_{BR} }
$ and
$
A \rightarrow
\FlaTwoByTwo{ A_{TL} } { 0 }
            { A_{BL} } { A_{BR} }
$
}


\renewcommand{\partitionsizes}{$ C_{BR} $, $ \hat{C}_{BR} $, and $ A_{BR} $ are $ 0 \times 0 $ }

% loop guard
\renewcommand{\guard}{ \neg \SameSize( C_{BR}, C ) }


% Repartitioning at start of loop
\renewcommand{\repartitionings}{
\normalsize
$
\FlaTwoByTwo{ C_{TL} } { C_{TR} }
            { * } { C_{BR} }
\rightarrow
\FlaThreeByThreeTL{ C_{00} }{ C_{01} }{ C_{02} }
              { * }{ C_{11} }{ C_{12} }
              { * }{ * }{ C_{22} }
$, \\ \\
$
\FlaTwoByTwo{ \hat{C}_{TL} } { \hat{C}_{TR} }
            { * } { \hat{C}_{BR} }
\rightarrow
\FlaThreeByThreeTL{ \hat{C}_{00} }{ \hat{C}_{01} }{ \hat{C}_{02} }
              { * }{ \hat{C}_{11} }{ \hat{C}_{12} }
              { * }{ * }{ \hat{C}_{22} }
$, and \\ \\
$
\FlaTwoByTwo{ A_{TL} } { 0 }
            { A_{BL} } { A_{BR} }
\rightarrow
\FlaThreeByThreeTL{ A_{00} }{ 0 }{ 0 }
              { A_{10} }{ A_{11} }{ 0 }
              { A_{20} }{ A_{21} }{ A_{22} }
$ \\
}

% the update
\renewcommand{\update}{
$
\begin{array}{l}
C_{11} \becomes A_{11}^T A_{11} + A_{21}^T A_{21} + \hat{C}_{11} \\
C_{12} \becomes A_{21}^T A_{22} + \hat{C}_{12}
\end{array}
$
}

\renewcommand{\repartitionsizes}{
$ C_{11} $ , $ \hat{C}_{11} $, and $ A_{11} $ are $ b \times b $ matrices}

% Moving double lines at end of loop
\renewcommand{\moveboundaries}{
\normalsize
$
\FlaTwoByTwo{ C_{TL} } { C_{TR} }
            { * } { C_{BR} }
\leftarrow
\FlaThreeByThreeBR{ C_{00} }{ C_{01} }{ C_{02} }
              { * }{ C_{11} }{ C_{12} }
              { * }{ * }{ C_{22} }
$, \\ \\
$
\FlaTwoByTwo{ \hat{C}_{TL} } { \hat{C}_{TR} }
            { * } { \hat{C}_{BR} }
\leftarrow
\FlaThreeByThreeBR{ \hat{C}_{00} }{ \hat{C}_{01} }{ \hat{C}_{02} }
              { * }{ \hat{C}_{11} }{ \hat{C}_{12} }
              { * }{ * }{ \hat{C}_{22} }
$, and \\ \\
$
\FlaTwoByTwo{ A_{TL} } { 0 }
            { A_{BL} } { A_{BR} }
\leftarrow
\FlaThreeByThreeBR{ A_{00} }{ 0 }{ 0 }
              { A_{10} }{ A_{11} }{ 0 }
              { A_{20} }{ A_{21} }{ A_{22} }
$ \\
}

% output
\begin{figure}[tbp]
\begin{center}
\FlaAlgorithm
\end{center}
\caption{Blocked algorithm for $ C \becomes A^T A + \hat{C} $.}
\label{fig:sytrrk_uln_upleft_blk}
\end{figure}


% display the unblocked annotated algorithm
\input{sytrrk_uln/unblocked_annotated2}

% display the unblocked algorithm


%
% The Algorithm
%

% Initial Partitioning
\renewcommand{\partitionings}{
$
C \rightarrow
\FlaTwoByTwo{ C_{TL} } { C_{TR} }
            { * } { C_{BR} }
$,
$
\hat{C} \rightarrow
\FlaTwoByTwo{ \hat{C}_{TL} } { \hat{C}_{TR} }
            { * } { \hat{C}_{BR} }
$ and
$
A \rightarrow
\FlaTwoByTwo{ A_{TL} } { 0 }
            { A_{BL} } { A_{BR} }
$
}


\renewcommand{\partitionsizes}{$ C_{BR} $, $ \hat{C}_{BR} $, and $ A_{BR} $ are $ 0 \times 0 $ }

% loop guard
\renewcommand{\guard}{ \neg \SameSize( C_{BR}, C ) }


% Repartitioning at start of loop
\renewcommand{\repartitionings}{
\normalsize
$
\FlaTwoByTwo{ C_{TL} } { C_{TR} }
            { * } { C_{BR} }
\rightarrow
\FlaThreeByThreeTL{ C_{00} }{ c_{01} }{ C_{02} }
              { * }{ \gamma_{11} }{ c_{12}^T }
              { * }{ * }{ C_{22} }
$, \\ \\
$
\FlaTwoByTwo{ \hat{C}_{TL} } { \hat{C}_{TR} }
            { * } { \hat{C}_{BR} }
\rightarrow
\FlaThreeByThreeTL{ \hat{C}_{00} }{ \hat{c}_{01} }{ \hat{C}_{02} }
              { * }{ \hat{\gamma}_{11} }{ \hat{c}_{12}^T }
              { * }{ * }{ \hat{C}_{22} }
$, and \\ \\
$
\FlaTwoByTwo{ A_{TL} } { 0 }
            { A_{BL} } { A_{BR} }
\rightarrow
\FlaThreeByThreeTL{ A_{00} }{ 0 }{ 0 }
              { a_{10}^T }{ \alpha_{11} }{ 0 }
              { A_{20} }{ a_{21} }{ A_{22} }
$ \\
}

% the update
\renewcommand{\update}{
$
\begin{array}{l}
\gamma_{11} \becomes a_{21}^T a_{21} + \alpha_{11}^2 + \hat{\gamma}_{11} \\
c_{12} \becomes A_{22}^T a_{21} + \hat{c}_{12}
\end{array}
$
}

\renewcommand{\repartitionsizes}{
$ \gamma_{11} $ , $ \hat{\gamma}_{11} $, and $ \alpha_{11} $ are scalars,
$ c_{01} $ , $ \hat{c}_{01} $, and $ a_{21} $ are column vectors, \\
and $ c_{12}^T $ , $ \hat{c}_{12}^T $, and $ a_{10}^T $ are row vectors. }

% Moving double lines at end of loop
\renewcommand{\moveboundaries}{
\normalsize
$
\FlaTwoByTwo{ C_{TL} } { C_{TR} }
            { * } { C_{BR} }
\leftarrow
\FlaThreeByThreeBR{ C_{00} }{ c_{01} }{ C_{02} }
              { * }{ \gamma_{11} }{ c_{12}^T }
              { * }{ * }{ C_{22} }
$, \\ \\
$
\FlaTwoByTwo{ \hat{C}_{TL} } { \hat{C}_{TR} }
            { * } { \hat{C}_{BR} }
\leftarrow
\FlaThreeByThreeBR{ \hat{C}_{00} }{ \hat{c}_{01} }{ \hat{C}_{02} }
              { * }{ \hat{\gamma}_{11} }{ \hat{c}_{12}^T }
              { * }{ * }{ \hat{C}_{22} }
$, and \\ \\
$
\FlaTwoByTwo{ A_{TL} } { 0 }
            { A_{BL} } { A_{BR} }
\leftarrow
\FlaThreeByThreeBR{ A_{00} }{ 0 }{ 0 }
              { a_{10}^T }{ \alpha_{11} }{ 0 }
              { A_{20} }{ a_{21} }{ A_{22} }
$ \\
}

% output
\begin{figure}[tbp]
\begin{center}
\FlaAlgorithm
\end{center}
\caption{Unblocked algorithm for $ C \becomes A^T A + \hat{C} $.}
\label{fig:sytrrk_uln_upleft_unblk}
\end{figure}


% diplay the blocked code for invariant 2
\begin{figure}
\footnotesize
\listinginput{1}{sytrrk_uln/codes/invariant2/MY_Sytrrk_uln_blk.c}
\label{fig:sytrrk_uln_code_inv2_blk}
\caption{Source listing for loop invariant 2, blocked algorithm}
\end{figure}

% diplay the unblocked code for invariant 2
\begin{figure}
\footnotesize
\listinginput{1}{sytrrk_uln/codes/invariant2/MY_Sytrrk_uln_unblk.c}
\label{fig:sytrrk_uln_code_inv2_unblk}
\caption{Source listing for loop invariant 2, unblocked algorithm}
\end{figure}

% display the performance graphs
\begin{figure}[htbp]
\begin{center}
\begin{tabular}{c}
Variant 1 \\
\psfig{figure=sytrrk_uln/codes/invariant1/sytrrk_uln_inv1.eps,width=4.5in,height=3.25in} \\ \\
Variant 2 \\
\psfig{figure=sytrrk_uln/codes/invariant2/sytrrk_uln_inv2.eps,width=4.5in,height=3.25in}
\\
\end{tabular}
\end{center}
\caption{Performance of the variants  of symmetric triangular rank-k update
algorithms for a block size of $ 128 $.
For these experiments, the FLAME matrix-matrix multiplication
kernel was used.}
\label{fig:sytrrk_uln:perfgraph}
\end{figure}