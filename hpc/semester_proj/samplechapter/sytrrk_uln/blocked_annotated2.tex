%
% Annotated Algorithm
%


% our operation
\renewcommand{\operation}{C \becomes A^T A + C}


% precondition
\renewcommand{\precondition}{C = \hat{C} \wedge n(C) = m(C) \wedge n(A) = m(A) \wedge n(C) = n(A)
\wedge UpTr(C) \wedge LowTr(A)}

% postcondition
\renewcommand{\postcondition}{{ C \becomes A^T A + \hat{C} }}

% invariant 2
\renewcommand{\invariant}{
\FlaTwoByTwo{ C_{TL} } { C_{TR} }
            { * } { C_{BR} }
=
\FlaTwoByTwo{ \hat{C}_{TL} } { \hat{C}_{TR} }
            { * } { A_{BR}^T + A_{BR} + \hat{C}_{BR} }
}

% loop guard
\renewcommand{\guard}{ \neg \SameSize( C_{BR}, C ) }

% initial partitionings
\renewcommand{\partitionings}{
$
C \rightarrow
\FlaTwoByTwo{ C_{TL} } { C_{TR} }
            { * } { C_{BR} }
$,
$
\hat{C} \rightarrow
\FlaTwoByTwo{ \hat{C}_{TL} } { \hat{C}_{TR} }
            { * } { \hat{C}_{BR} }
$ and
$
A \rightarrow
\FlaTwoByTwo{ A_{TL} } { 0 }
            { A_{BL} } { A_{BR} }
$
}


\renewcommand{\partitionsizes}{$ C_{BR} $, $ \hat{C}_{BR} $, and $ A_{BR} $ are $ 0 \times 0 $ }


% Define the blocksize that appears in step 5a
\renewcommand{\blocksize}{ b }
%

% repartitioning at start of loop (step 5a)
\renewcommand{\repartitionings}{
\normalsize
$
\FlaTwoByTwo{ C_{TL} } { C_{TR} }
            { * } { C_{BR} }
\rightarrow
\FlaThreeByThreeTL{ C_{00} }{ C_{01} }{ C_{02} }
              { * }{ C_{11} }{ C_{12} }
              { * }{ * }{ C_{22} }
$, \\ \\
$
\FlaTwoByTwo{ \hat{C}_{TL} } { \hat{C}_{TR} }
            { * } { \hat{C}_{BR} }
\rightarrow
\FlaThreeByThreeTL{ \hat{C}_{00} }{ \hat{C}_{01} }{ \hat{C}_{02} }
              { * }{ \hat{C}_{11} }{ \hat{C}_{12} }
              { * }{ * }{ \hat{C}_{22} }
$, and \\ \\
$
\FlaTwoByTwo{ A_{TL} } { 0 }
            { A_{BL} } { A_{BR} }
\rightarrow
\FlaThreeByThreeTL{ A_{00} }{ 0 }{ 0 }
              { A_{10} }{ A_{11} }{ 0 }
              { A_{20} }{ A_{21} }{ A_{22} }
$ \\
}




\renewcommand{\repartitionsizes}{
$ C_{11} $ , $ \hat{C}_{11} $, and $ A_{11} $ are $ b \times b $ matrices}



% Moving double lines at end of loop (step 5b)
\renewcommand{\moveboundaries}{
\normalsize
$
\FlaTwoByTwo{ C_{TL} } { C_{TR} }
            { * } { C_{BR} }
\leftarrow
\FlaThreeByThreeBR{ C_{00} }{ C_{01} }{ C_{02} }
              { * }{ C_{11} }{ C_{12} }
              { * }{ * }{ C_{22} }
$, \\ \\
$
\FlaTwoByTwo{ \hat{C}_{TL} } { \hat{C}_{TR} }
            { * } { \hat{C}_{BR} }
\leftarrow
\FlaThreeByThreeBR{ \hat{C}_{00} }{ \hat{C}_{01} }{ \hat{C}_{02} }
              { * }{ \hat{C}_{11} }{ \hat{C}_{12} }
              { * }{ * }{ \hat{C}_{22} }
$, and \\ \\
$
\FlaTwoByTwo{ A_{TL} } { 0 }
            { A_{BL} } { A_{BR} }
\leftarrow
\FlaThreeByThreeBR{ A_{00} }{ 0 }{ 0 }
              { A_{10} }{ A_{11} }{ 0 }
              { A_{20} }{ A_{21} }{ A_{22} }
$ \\
}



% state before the update
\renewcommand{\beforeupdate}{
\FlaThreeByThreeTL{ C_{00} }{ C_{01} }{ C_{02} }
              { * }{ C_{11} }{ C_{12} }
              { * }{ * }{ C_{22} }
=
\FlaThreeByThreeTL{ \hat{C}_{00} }{ \hat{C}_{01} }{ \hat{C}_{02} }
              { * }{ \hat{C}_{11} }{ \hat{C}_{12} }
              { * }{ * }{ A_{22}^T A_{22} + \hat{C}_{22} }
}



% state after the update
\renewcommand{\afterupdate}{
\FlaThreeByThreeBR{ C_{00} }{ C_{01} }{ C_{02} }
              { * }{ C_{11} }{ C_{12} }
              { * }{ * }{ C_{22} }
=
\FlaThreeByThreeBR{ \hat{C}_{00} } { \hat{C}_{01} }{ \hat{C}_{02} }
              { * }{ A_{11}^T A_{11} + A_{21}^T A_{21} + \hat{C}_{11} }{ A_{21}^T A_{22} + \hat{C}_{12} }
              { * }{ * }{ A_{22}^T A_{22} + \hat{C}_{22} }
}



% the update
\renewcommand{\update}{
$
\begin{array}{l}
C_{11} \becomes A_{11}^T A_{11} + A_{21}^T A_{21} + \hat{C}_{11} \\
C_{12} \becomes A_{21}^T A_{22} + \hat{C}_{12}
\end{array}
$
}

% output
\begin{figure}[tbp]
\begin{center}
\worksheet
\end{center}
\caption{Worksheet for deriving blocked algorithm for $ C \becomes A^T A + \hat{C} $.}
\label{fig:sytrrk_uln_upleft_blk_ann}
\end{figure}
