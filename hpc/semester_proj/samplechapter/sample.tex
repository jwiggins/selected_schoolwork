\documentclass[leqno,10pt]{report}   %You must set up

\usepackage{ifthen}

\usepackage{psfig}
\usepackage{color}
\usepackage{moreverb}
\usepackage{latexsym}
\usepackage{rotating}
\usepackage{makeidx}  % allows for indexgeneration
% \usepackage{draftcopy}
\usepackage{multind}
\usepackage{multicol}

\makeindex{index}
\makeindex{rout}
\makeindex{const}
\makeindex{op}
\makeindex{author}

\newtheorem{note}{Note}
\newtheorem{algorithm}{Algorithm}
\newtheorem{theorem}{Theorem}
\newtheorem{lemma}{Lemma}
\newtheorem{corollary}{Corollary}
\newtheorem{definition}{Definition}
\newtheorem{exercise}[theorem]{Exercise}
\newtheorem{example}[theorem]{Example}

\newcommand{\PATH}{/u/rvdg/}
\newenvironment{proof}{
\noindent
{\bf Proof:}
}
{
\hfill
$ \Box $
}

% for SIAM, comment out the following lines
\setlength{\textheight}{9.0in}
\setlength{\textwidth}{6.5in}
\setlength{\oddsidemargin}{0in}
\setlength{\evensidemargin}{0in}
\setlength{\topmargin}{-.5in}

\newcommand\R{\hbox{\rm I\kern-2pt R}}
\newcommand\C{{\rm\ \raise0.5pt
        \hbox{\vrule height6.00pt width0.6pt depth0pt}\kern-3.20pt C}}
\newcommand{\Rnn}{\R^{n\times n}}                                               
\newcommand{\Rmn}{\R^{m\times n}}                                               
\newenvironment{indent0}%
  {\begin{list}{\ }{}}%
  {\end{list}}

\newenvironment{indent1}%
  {\begin{list}{\  \hspace{.05in} }{}}%
  {\end{list}}

\newenvironment{indent2}%
  {\begin{list}{\  \hspace{.1in} }{}}%
  {\end{list}}

\newenvironment{indent3}%
  {\begin{list}{\  \hspace{.15in} }{}}%
  {\end{list}}

\input{flatex}
%
% ExampleBox draws a box around an example in the FLAME
% document
%

\newcommand{\ExampleBox}[2]{
\begin{flushright}
\fbox{
\begin{minipage}{5.5in}
{\bf Example #1}
#2
\end{minipage}
}
\end{flushright}
}

\newcommand{\NoteBox}[1]{
\begin{flushright}
\fbox{
\begin{minipage}{6.0in}
{\bf Note:}
#1
\end{minipage}
}
\end{flushright}
}

\newcommand{\ChapterAuthor}[1]{
{
\vspace{-0.2in}
\large 
by \\[0.1in]
\bf
#1
}
\vspace{0.5in}
}

\newcommand{\diagram}[1]{
}

\newcommand{\SummaryBox}[3]{
\index{rout}{{#2}}
\begin{tabular}{| p{2.0in} @{} | @{\hspace{1pt}} p{4.3in} | } \hline
\begin{minipage}{1.8in}
#1 
\end{minipage}
& 
\begin{minipage}{4.3in}
\begin{description}
\item
{\tt #2 #3}
\end{description}
\end{minipage}
\\ \hline
\end{tabular}
}

\newcommand{\un}{\_}%\hspace{-5pt}}

\newcommand{\parameter}[2]{
\begin{minipage}[t]{1.5in}
#1
\end{minipage}
\begin{minipage}[t]{4.25in}
#2
\end{minipage}
}

\newcommand{\purpose}[1]{
\vspace{-0.1in}
{\bf Purpose:} #1
\vspace{0.05in}
}

\newenvironment{FlaSpec}{
\begin{center}
\begin{tabular}{| p{6.25in} |} \hline
\vspace{-0.2in}
}
{
\\ \hline
\end{tabular}
\end{center}
}

\newcommand{\conj}[1]{{\tt conj}({#1})}

% \newcommand{\status}[1]{\addtocontents{toc}{{\\ \bf Status: #1}}}

\newcommand{\status}[1]{}

\newcounter{stepnumber}
% \addtocounter{stepnumber}{1}%

\newcommand{\step}{
\refstepcounter{stepnumber}%
\noindent
{\bf Step \thestepnumber:}
}

\newenvironment{codeexample}
{
\begin{quote}
}
{
\end{quote}
}

\newcommand{\becomes}{:=}

\newcommand{\equals}{=}

\newcommand{\implies}{\Rightarrow}

\newcommand{\longggarrow}{
\setlength{\unitlength}{0.5in}
\begin{picture}(5,1)
\put(5,0){\vector(-1,0){5}}
\end{picture}
}

\newcommand{\PPre}{P_{\rm pre}}
\newcommand{\PPost}{P_{\rm post}}
\newcommand{\PInv}{P_{\rm inv}}

\newcommand{\QBefore}{Q_{\rm before}}
\newcommand{\QAfter}{Q_{\rm after}}

\newcommand{\Qbefore}{Q_{\rm before}}
\newcommand{\Qafter}{Q_{\rm after}}


\newboolean{FORTRAN}
% \setboolean{FORTRAN}{true}
\setboolean{FORTRAN}{false}

\newboolean{CLANGUAGE}
\setboolean{CLANGUAGE}{true}
% \setboolean{CLANGUAGE}{false}

\begin{document}

% \frontmatter

\pagestyle{plain}  % switches on printing of running heads

\title{
Developing 
Linear Algebra Algorithms: \\
Class Projects for Spring 2002\\[0.15in]
{\bf \large STATUS AT END OF SEMESTER} \\[0.15in]
\large FLAME Working Note \#??
}

\author{
Vinod Valsalam \\[0.05in]
Robert A. van de Geijn 
\\[0.25in]
Department of Computer Sciences\\
The University of Texas\\
Austin, TX 78712\\
{ \tt \{vkv,rvdg\}@cs.utexas.edu}
}

\date{
Manuscript in Preparation \\
\today
}

% \date{
% May 31, 2001
% }


\maketitle

\

\newpage

\begin{abstract}
In this document we present a set of class projects by students in
an upper-division undergraduate course titled ``High-Performance
Parallel Computing'' taught by us in the Fall of 2002 at The
University of Texas at Austin.  A similar document was created in
Spring 2001 and Spring 2002 for the same course.  
The idea behind the projects is to
illustrate how formal derivation techniques can be used to derive
families of algorithms for given linear algebra operations.

The philosophy behind the approach is that one should start by
systematically deriving the algorithms.  The methodology used
this time is a refinement of the ``recipe'' used in Spring 2001.
By providing a set of \LaTeX\nocite{LATEX}
macros for the students to use, it becomes convenient and
beneficial to first generate a careful documentation of the
derivation.  Once one or more algorithms have been so
developed, they can be translated into implementations 
using an Application Programming Interface (API),
the Formal Linear Algebra Methods Environment
(FLAME).  This library allows the code to look much like the
algorithms, which greatly reduces the opportunity for
the introduction of logical and typographical errors.  
For all examples in the report we
demonstrate that high performance can be attained on an Intel Pentium
(R) III processor.

As part of the class, the students were organized into
small teams and a different linear algebra operation
was assigned to each team.
The names of the members of the
teams are given as the authors of the chapter on the operation.
Thus we attempt to show that the formal derivation approach makes the
development and implementation of high-performance
algorithms for dense linear algebra operations accessible to
novices.

As was the case for the document generated from the 
class projects for the other times this course
was offered,
this document is meant to capture progress achieved
during a single semester.  Thus, it should
be viewed as a work-in-progress rather than a polished
product.

\end{abstract}

\

\newpage

\tableofcontents
%
% \mainmatter              % start of the contributions


\pagestyle{plain}



%\input{trmm_lln/trmm_lln} 
%% The following commands help in generating the index
\index{index}{triangular solve!multiple RHSs}%
\index{op}{triangular solve!multiple RHSs!$ B \leftarrow B U^{-T} $|(}%

% Name of the chapter

\chapter{Triangular Solve (with Multiple RHSs)\\
$ B \leftarrow B U^{-T} $ }
\label{chapter:trsm_rut}

% Authors

\ChapterAuthor{
\index{author}{Van Zee, Field G.}
Field G. Van Zee
\\[0.1in]
\index{author}{Walkup, Patrick J.}
Patrick J. Walkup
}

% Add the names of the authors to the table of contents

\addtocontents{toc}{ {by {\bf Field G. Van Zee and Patrick J. Walkup}} 
13 May 2002: DONE.}


In this chapter, we discuss the implementation of the triangular matrix
solve with multiple right-hand-sides
\[
B \leftarrow B U^{-T}
\]
where $ U $ is an $ m \times m $ upper triangular matrix and $ B $ is
$ n \times m $.  We will overwrite $ B $ with the result without
requiring a workarray.  We start by deriving a number of different
sequential algorithms.  Subsequently, we show how to code the
algorithms using FLAME.  Later in the chapter the parallel
implementation of the operation is discussed.

The variables for the triangular matrix solve can be
described by the precondition
\[
\PPre:
B = \hat{B} \wedge \SameSize( B, \hat{B} ) \wedge \UppTr( U ) \wedge
\ColDim( B )=\RowDim( U ),
\]
where $ \hat{B} $ indicates the original contents of $ B $.  Here the
predicate $ \UppTr( A ) $ is true iff $ A $ is an upper triangular
matrix.  The functions $ \RowDim( A ) $ and $ \ColDim( A ) $ return
the row and column dimension of $ A $, respectively.  The operation to
be performed $ B \becomes B U^{-T} $ translates to the postcondition $
\PPost: B = \hat{B} U^{-T}$.

\section{Algorithms That Start By Partitioning $ U $}
\label{sec:trsm_rut:U}

Let us start by partitioning matrix $ U $.  Since it is triangular, we
partition it like
\[
U \rightarrow \FlaTwoByTwo{ U_{TL} }{ U_{TR} }
                          {   0    }{ U_{BR} },
\]
where $ U_{TR} $ is square so that both submatrices on the diagonal
are themselves upper triangular.

Substituting the partitioning of $ U $ into the postcondition yields
\[
( \mbox{some partitioning of }{B} )
= 
( \mbox{some partitioning of }\hat{B} )
{ \FlaTwoByTwo{ U_{TL} }{ U_{TR} }
            {   0    }{ U_{BR} } }^{-T}
\]
This suggests that $ B $ and $ \hat{B} $ should be partitioned
vertically into two submatrices, or into quadrants.  Let us consider
the case where $ B $ and $ \hat{B} $ are partitioned vertically into
two submatrices.  Then
\[
\FlaOneByTwo{ B_L }
            { B_R }
= 
\FlaOneByTwo{ \hat{B}_L }
            { \hat{B}_R }
{ \FlaTwoByTwo{ U_{TL} }{ U_{TR} }
              {   0    }{ U_{BR} } }^{-T}     
\]
According to the rules of matrix algebra, the former equation can be expressed as
\[
\FlaOneByTwo{ B_L }
            { B_R }
{ \FlaTwoByTwo{ U_{TL} }{ U_{TR} }
              {   0    }{ U_{BR} } }^{T}    
= 
\FlaOneByTwo{ \hat{B}_L }
            { \hat{B}_R }      
\]
and after applying the transpose operation on $ U $, we have
\[
\FlaOneByTwo{ B_L }
            { B_R }
\FlaTwoByTwo{ U_{TL}^{T} }{   0    }
            { U_{TR}^{T} }{ U_{BR}^{T} }   
= 
\FlaOneByTwo{ \hat{B}_L }
            { \hat{B}_R }      
\]
In order to be able to multiply out the matrices on the left and to
be able to then set the submatrices on the right equal to the result on
the left we find that the following must hold:
\[
\RowDim( U_{TL}^T ) = \ColDim( {B}_L ) 
\wedge
\ColDim( U_{TL}^T ) = \ColDim( \hat{B}_L ) 
\]
which in turn implies that $ \ColDim( B_L )= \ColDim( \hat{B}_L ) $
since $ U_{TL}^T $ is a square matrix.  This is good news since this
means $ B $ and $ \hat{B} $ are partitioned the same way, which is
important since $ B $ and $ \hat{B}$ will reference the same matrix ($
\hat{B} $ is being overwritten by $ B $).  Multiplying out the 
partitioned matrices gives us
\[
\FlaOneByTwo{ B_{L} U_{TL}^{T} + B_{R} U_{TR}^{T} }
            { B_{R} U_{BR}^{T}  }
=
\FlaOneByTwo{ \hat{B}_{L} }
            { \hat{B}_{R} }
\]
which yields the equalities
\begin{equation}
\label{eqn:utrsm_rut}
\begin{array}{r c l}
B_{L} U_{TL}^{T} + B_{R} U_{TR}^{T} &=& \hat{B}_{L} \\
B_{R} U_{BR}^{T}                    &=& \hat{B}_{R}
\end{array}
\end{equation}

% Insert the table of possible loop-invariants from
% file trsm_rut/table.new.tex

\begin{figure}
\begin{center}
\footnotesize
% \begin{sideways}
{
\setlength{\tabcolsep}{4pt}
\begin{tabular}{| c | p{2.5in} | c | } \hline
$ P_X: $

$ \FlaTwoByTwo{ C_{TL} } { C_{TR} }
              { C_{BL} } { C_{BR} } =  $ & Comment &
\footnotesize Feasible? \\ \hline \hline
\footnotesize
$
\FlaTwoByTwo{ \hat{C}_{TL} }  { \hat{C}_{TR} }
            { \undetermined } { \hat{C}_{BR} }
$ 
&
&
NO
\\ \hline
%
%
\footnotesize
$
\FlaTwoByTwo{ A_{T} A_{T}^T + \hat{C}_{TL} }  { A_{T} A_{B}^T + \hat{C}_{TR} }
            { \undetermined }                 { A_{B} A_{B}^T + \hat{C}_{BR} }
$ 
& 
NO
\\ \hline
%
%
\footnotesize
$
\FlaTwoByTwo{ A_{T} A_{T}^T + \hat{C}_{TL} }  { \hat{C}_{TR} }
            { \undetermined }                 { \hat{C}_{BR} }
$ 
& 
YES
\\ \hline
%
%
\footnotesize
$
\FlaTwoByTwo{ \hat{C}_{TL} }  { A_{T} A_{B}^T + \hat{C}_{TR} } 
            { \undetermined } { \hat{C}_{BR} }
$ 
&
NO
\\ \hline
%
%
\footnotesize
$
\FlaTwoByTwo{ \hat{C}_{TL} }   { \hat{C}_{TR} }
            { \undetermined }  { A_{B} A_{B}^T + \hat{C}_{BR} }
$ 
&
Loop-invariant 1
&
YES
\\ \hline
%
%
\footnotesize
$
\FlaTwoByTwo{ \hat{C}_{TL} } { A_{T} A_{B}^T + \hat{C}_{TR} }
            { \undetermined} { A_{B} A_{B}^T + \hat{C}_{BR} }
$ 
&
YES
\\ \hline
%
%
\footnotesize
$
\FlaTwoByTwo{ A_{T} A_{T}^T + \hat{C}_{TL} } { \hat{C}_{TR} }
            { \undetermined }                { A_{B} A_{B}^T + \hat{C}_{BR} }
$ 
&
YES
\\ \hline
%
%
\footnotesize
$
\FlaTwoByTwo{ A_{T} A_{T}^T + \hat{C}_{TL} } { A_{T} A_{B}^T + \hat{C}_{TR} }
            { \undetermined }                { \hat{C}_{BR} }
$ 
&
Loop-invariant 2.
&
YES
\\ \hline
%\\ \hline
%
%
\end{tabular}
}
% \end{sideways}
\end{center}
\caption{Possible loop-invariants when partitioning
$ C $ into quadrants.
Here $ P_X $ is the most prominent part of the loop-invariant
$ \PInv $.}
\label{fig:USYRK_UNN_example}
\end{figure}

From (\ref{eqn:utrsm_rut}) we conclude that the operations to be
performed are $ B_{L} U_{TL}^{T} $, $ B_{R} U_{TR}^{T} $, and $ B_{R} 
U_{BR}^{T} $.  At an intermediate stage, each of these either has or
has not already been computed, leading to $ 2^3 = 8 $ possible
loop-invariants. However, only the six invariants which make sense 
computationally are tabulated in Fig.~\ref{fig:TRSM_RUT_variants}.

\subsection{Loop-invariant 1}

We will first examine the (feasible) loop-invariant
\begin{eqnarray}
\label{eqn:utrsm_rut:p1}
\PInv: 
\lefteqn{
\FlaOneByTwo{ B_L }{ B_R } =
\FlaOneByTwo{ \hat{B}_{L} }
            { \hat{B}_R U_{BR}^{-T} }
\wedge
\UppTr( U_{BR} ) } \\ %\wedge \\
\nonumber
& & 
\RowDim( U_{TL}^T ) = \ColDim( {B}_L ) 
\wedge
\ColDim( U_{TL}^T ) = \ColDim( \hat{B}_L ) 
\end{eqnarray}
For short we will write
\begin{equation}
\label{eqn:utrsm_rut:p2}
\PInv: \FlaOneByTwo{ B_L}{B_R} =
\FlaOneByTwo{ \hat{B}_{L} }
            { \hat{B}_R U_{BR}^{-T} }
\wedge
\ldots
\end{equation}
for (\ref{eqn:utrsm_rut:p1}).

Comparing the loop-invariant in (\ref{eqn:utrsm_rut:p2}) with the
postcondition $ B = \hat{B} U^{-T} $ we see that {\em if} $ U = U_{BR} $, 
$ B = B_R $, and $ \hat{B} = \hat{B}_R $ then the loop-invariant implies 
the postcondition, i.e., that the desired result has been computed.
Notice that $ \SameSize( U, U_{BR} ) \wedge \PInv $ implies $ U =
U_{BR} $, $ B = B_R $, and $ \hat{B} = \hat{B}_R $ since the partitioned
matrices are references into the original matrices $ U $, $ B $, and $ \hat{B} $.  
Thus, the loop-guard $ G: \neg \SameSize( U, U_{BR} ) $ has the 
desired property.

Consider the initialization in Step 4 in Fig.~\ref{fig:ws:utrsm_rut}.
The fact that for this partitioning of $ B $, $ U $, and $ \hat{B} $,
$
\FlaOneByTwo{ B_L}{B_R} =
\FlaOneByTwo{ B_{L} }
            { B_R U_{BR}^{-T} }
$
and the precondition implies the
other parts of the loop-invariant, this initialization has the desired
properties.

Loop-guard $ G $ indicates that eventually $ U_{BR} $ should equal all
of $ U $, at which point $ G $ becomes {\em false} and the loop is
exited.  After the initialization, $ U_{BR} $ is $ 0 \times 0 $.
Thus, the repartitioning should be such that as the computation
proceeds rows and columns are subtracted from $ U_{TL} $ and added to
$ U_{BR} $ while updating $ U_{TR} $ consistently with this.

\subsubsection{Unblocked algorithm}

If we move the partitionings by individual rows and columns, we obtain
the repartitioning and redefinition of the partitioning in Steps 5a
and 5b in Fig.~\ref{fig:ws:utrsm_rut}.  The repartitionings in Step 5a
in Fig.~\ref{fig:ws:utrsm_rut} result in the state
\begin{equation}
\label{eqn:utrsm_rut:bu}
\QBefore: 
\FlaOneByTwo{ \FlaOneByTwoSingleLine{ B_{0} }
                                    { b_{1} } }
	                            { B_{2} } 
=
\FlaOneByTwo{ \FlaOneByTwoSingleLine{ \hat{B}_{0} } { \hat{b}_{1} } } 
            { \hat{B}_{2} U_{22}^{-T}  }
\wedge \ldots
\end{equation}
before the update.  The redefinition in Step 5b in
Fig.~\ref{fig:ws:utrsm_rut} means that the following state must result
from the update:
\[
\QAfter: 
\FlaOneByTwo{ B_{0} }{ \FlaOneByTwoSingleLine{ b_{1} } { B_{2} }  }
=
\FlaOneByTwo{ \hat{B}_{0} }
            {  \FlaOneByTwoSingleLine { \hat{b}_{1} } { \hat{B}_{2} }
	       \FlaTwoByTwoSingleLine { \upsilon_{11} } { u_{12}^{T} }
                                      {    0          } { U_{22}     }^{-T}
            }
\wedge \ldots
\]
Applying matrix transposition and inversion to $ U $ leaves us
\[
\QAfter: 
\FlaOneByTwo{ B_{0} }{ \FlaOneByTwoSingleLine{ b_{1} } { B_{2} }  }
=
\FlaOneByTwo{ \hat{B}_{0} }
            {  \FlaOneByTwoSingleLine { \hat{b}_{1} } { \hat{B}_{2} }
	       \FlaTwoByTwoSingleLine { \upsilon_{11}^{-1}                     } {     0      }
                                      { -U_{22}^{-T} u_{12} \upsilon_{11}^{-1} } { U_{22}^{-T}}
            }
\wedge \ldots
\]
Multiplying out the resulting submatrices yields
\begin{equation}
\label{eqn:utrsm_rut:au}
\QAfter: 
\FlaOneByTwo{ B_{0} }{ \FlaOneByTwoSingleLine{ b_{1} } { B_{2} }  }
=
\FlaOneByTwo{ \hat{B}_{0} }
            { \FlaOneByTwoSingleLine { ( \hat{b}_{1} - B_{2} u_{12} ) \upsilon_{11}^{-1} } 
                                     { \hat{B}_{2} U_{22}^{-T} }
            }
\wedge \ldots
\end{equation}



Comparing Eqns.~\ref{eqn:utrsm_lln:bu} and~\ref{eqn:utrsm_rut:au} we
find that the updates
\begin{eqnarray*}
& b_1 \becomes( \hat{b}_{1} - B_{2} u_{12} ) \upsilon_{11}^{-1}
\end{eqnarray*}
are required to change the state from $ \QBefore $ to $ \QAfter $.

% The following commands will in the ``worksheet''
% given in Fig. 4.2

% Step 0: Operation
\renewcommand{\operation}{B \becomes B U^{-T}}

% Step 1a: Precondition
\renewcommand{\precondition}{B = \hat{B} \wedge \UppTr( U ) \wedge \ColDim( B )=\RowDim( U )}

% Step 1b: Postcondition
\renewcommand{\postcondition}{B = \hat{B} U^{-T}}

% Step 2: Loop-invariant
\renewcommand{\invariant}{
\FlaOneByTwo{ B_L }{ B_R } =
\FlaOneByTwo{ \hat{B}_L }
            { \hat{B}_R U_{BR}^{-T} }
\wedge
\ldots
}

% Step 3: Loop-guard
\renewcommand{\guard}{ \neg \SameSize( U, U_{BR} ) }

% Step 4: Initialization
\renewcommand{\partitionings}{
$ 
B \rightarrow \FlaOneByTwo{ B_L }
                          { B_R }
$,
$ 
\hat{B} \rightarrow \FlaOneByTwo{ \hat{B}_L }
                                { \hat{B}_R }
$, and
$ 
U \rightarrow \FlaOneByTwo{ U_{TL} }{ U_{TR} }
                          {   0    }{ U_{BR} }
$
}
\renewcommand{\partitionsizes}{
$ B_R $ and $ \hat{B}_R $ have $ 0 $ rows
and $ U_{BR} $ is $ 0 \times 0 $
}

% Step 5a: repartitioning
\renewcommand{\repartitionings}{
$ 
\FlaOneByTwo{ B_L }
            { B_R } 
\rightarrow
\FlaOneByThreeL{ B_0 }
               { b_1 }
               { B_2 },
\FlaOneByTwo{ \hat{B}_L }
            { \hat{B}_R } 
\rightarrow
\FlaOneByThreeL{ \hat{B}_0 }
               { \hat{b}_1 }
               { \hat{B}_2 }
$ \\
and
$ 
\FlaTwoByTwo{ U_{TL} }{ U_{TR} }
            {   0    }{ U_{BR} } 
\rightarrow
\FlaThreeByThreeTL{ U_{00} } { u_{01}        } { U_{02}     }
                  {   0    } { \upsilon_{11} } { u_{12}^{T} }
                  {   0    } {    0          } { U_{22}     }
$
}
\renewcommand{\repartitionsizes}{
$ b_1 $ and $ \hat{b}_1 $ are columns 
and $ \upsilon_{11} $ is a scalar
 }

% Step 5b: moving the boundaries
\renewcommand{\moveboundaries}{
$
\FlaOneByTwo{ B_L }
            { B_R } 
\leftarrow
\FlaOneByThreeR{ B_0 } { b_1 } { B_2 },
\FlaOneByTwo{ \hat{B}_L }
            { \hat{B}_R } 
\leftarrow
\FlaOneByThreeR{ \hat{B}_0 } { \hat{b}_1 } { \hat{B}_2 }
$ \\
and
$ 
\FlaTwoByTwo{ U_{TL} }{ U_{TR} }
            {    0   }{ U_{BR} } \leftarrow
\FlaThreeByThreeBR{ U_{00} } { u_{01}        } { U_{02}     }
                  {   0    } { \upsilon_{11} } { u_{12}^{T} }
                  {   0    } {      0        } { U_{22}     }  
$
}

% Step 6: state before update
\renewcommand{\beforeupdate}{
\FlaOneByTwo{ \FlaOneByTwoSingleLine{ B_{0} }
                                    { b_{1} } }
	                            { B_{2} } 
=
\FlaOneByTwo{ \FlaOneByTwoSingleLine{ \hat{B}_{0} } { \hat{b}_{1} } } 
            { \hat{B}_{2} U_{22}^{-T}  }
\wedge \ldots
}

% Step 7: state after update
\renewcommand{\afterupdate}{
\FlaOneByTwo{ B_{0} }{\FlaOneByTwoSingleLine{ b_{1} } { B_{2} }  }
=
\FlaOneByTwo { \hat{B}_{0} }
             { \FlaOneByTwoSingleLine { ( \hat{b}_{1} - B_{2} u_{12} ) \upsilon_{11}^{-1} } 
                                      { \hat{B}_{2} U_{22}^{-T} } }
\wedge \ldots
}

% Step 8: update
\renewcommand{\update}{
\begin{minipage}[t]{4in}
\noindent
% \FlaStartCompute \\
$ b_1 \becomes ( \hat{b}_{1} - B_{2} u_{12} ) \upsilon_{11}^{-1} $ \\
% \FlaEndCompute \\
\end{minipage}
}

% Given the commands defined above, the
% command \worksheet generates the annotated
% algorithm

\begin{figure}[htbp]
\worksheet
\caption{Annotated unblocked algorithm for loop-invariant 1.}
\label{fig:ws:utrsm_rut}
\end{figure}

%By recognizing that $ \hat{B} $ is never referenced we can eliminate
%all parts of the algorithm that refer to this matrix, yielding the
%final algorithm given in Fig.~\ref{fig:alg:utrsm_rut}.

% We now redefine some of the commands
% used to generate Fig. 4.2, taking out all references
% to \hat{B} to come up with the algorithm in Fig. 4.3.

% Step 4
%\renewcommand{\partitionings}{
%$ 
%B \rightarrow \FlaTwoByOne{ B_{T} }
%                          { B_{B} }
%$
%and
%$ 
%L \rightarrow \FlaTwoByTwo{ L_{TL} }{ 0 }
%                          { L_{BL} }{ L_{BR} }
%$
%}
%\renewcommand{\partitionsizes}{
%$ B_{B} $ has $ 0 $ rows
%and $ L_{BR} $ is $ 0 \times 0 $
%}
%
%% Step 5a
%\renewcommand{\repartitionings}{
%$ 
%\FlaTwoByOne{ B_T }{ B_B } 
%\rightarrow
%\FlaThreeByOneT{ B_0 }{ b_1^T }{ B_2 }
%$
%and
%$ \FlaTwoByTwo{ L_{TL} }{ 0 }
%            { L_{BL} }{ L_{BR} }
%\rightarrow
%\FlaThreeByThreeTL{ L_{00} }{ 0 }{ 0 }
%                { l_{10}^T }{ \lambda_{11} }{ 0 }
%                { L_{20} }{ l_{21} }{ L_{22} }
%$
%}
%\renewcommand{\repartitionsizes}{
%$ b_1^T $ is a row 
%and $ \lambda_{11} $ is a scalar
%}
%
%% Step 5b
%\renewcommand{\moveboundaries}{%
%$ 
%\FlaTwoByOne{ B_T }{ B_B } 
%\leftarrow
%\FlaThreeByOneB{ B_0 }{ b_1^T }{ B_2 }
%$
%and
%$ \FlaTwoByTwo{ L_{TL} }{ 0 }
%            { L_{BL} }{ L_{BR} }
%\leftarrow
%\FlaThreeByThreeBR{ L_{00} }{ 0 }{ 0 }
%                { l_{10}^T }{ \lambda_{11} }{ 0 }
%                { L_{20} }{ l_{21} }{ L_{22} }
%$
%}
%
%% The command \FlaAlgorithm now generates the 
%% algorithm without annotations
%
%\begin{figure}[htbp]
%\FlaAlgorithm
%\caption{Unblocked algorithm for loop-invariant 1.}
%\label{fig:alg:utrsm_rut}
%\end{figure}

\subsubsection{Blocked Algorithms}

In order to cast the algorithm to be rich in matrix-matrix
multiplications, the repartitioning and redefinition of the
submatrices in steps 5a and 5b in Fig.~\ref{fig:ws:utrsm_rut_blk}
expose multiple rows and/or columns at a time.  Block size $ b $ can
be chosen to be any size that does not exceed the number of rows in $
B_L $.  Again $ \QBefore $ is obtained by plugging the repartitioning
in Step 6 into $ \PInv $ while $ \QAfter $ is obtained by plugging the
redefinition of the quadrants in Step 7 into $ \PInv $.  The update in
Step 8 is now obtained by comparing the state in Steps 6 and 7.  The
bulk of the computation is now in the update 
$ B_1 \becomes( \hat{B}_{1} - B_{2} U_{12}^T ) U_{11}^{-1} $ .

% Redefine the commands that generate the annotated 
% algorithm for the blocked algorithm

% Define the blocksize that appears in step 5a
\renewcommand{\blocksize}{ b }
%

% Step 5a
\renewcommand{\repartitionings}{
$ 
\FlaOneByTwo{ B_L }
            { B_R } 
\rightarrow
\FlaOneByThreeL{ B_0 }
               { B_1 }
               { B_2 },
\FlaOneByTwo{ \hat{B}_L }
            { \hat{B}_R } 
\rightarrow
\FlaOneByThreeL{ \hat{B}_0 }
               { \hat{B}_1 }
               { \hat{B}_2 }
$ \\
and
$ 
\FlaTwoByTwo{ U_{TL} }{ U_{TR} }
            {   0    }{ U_{BR} } 
\rightarrow
\FlaThreeByThreeTL{ U_{00} } { U_{01} } { U_{02} }
                  {   0    } { U_{11} } { U_{12} }
                  {   0    } {   0    } { U_{22} }
$
}
%
\renewcommand{\repartitionsizes}{
$ B_1 $ and $ \hat{B}_1 $ have columns 
and $ U_{11} $ is $ b \times b $
}

% Step 5b
\renewcommand{\moveboundaries}{%
$
\FlaOneByTwo{ B_L }
            { B_R } 
\leftarrow
\FlaOneByThreeR{ B_0 } { B_1 } { B_2 },
\FlaOneByTwo{ \hat{B}_L }
            { \hat{B}_R } 
\leftarrow
\FlaOneByThreeR{ \hat{B}_0 } { \hat{B}_1 } { \hat{B}_2 }
$ \\
and
$ 
\FlaTwoByTwo{ U_{TL} }{ U_{TR} }
            {    0   }{ U_{BR} } \leftarrow
\FlaThreeByThreeBR{ U_{00} } { U_{01} } { U_{02} }
                  {   0    } { U_{11} } { U_{12} }
                  {   0    } {   0    } { U_{22} }
$
}

% Step 6
\renewcommand{\beforeupdate}{
\FlaOneByTwo{ \FlaOneByTwoSingleLine{ B_{0} }
                                    { B_{1} } }
	                            { B_{2} } 
=
\FlaOneByTwo{ \FlaOneByTwoSingleLine{ \hat{B}_{0} } { \hat{B}_{1} } } 
            { \hat{B}_{2} U_{22}^{-T}  }
\wedge \ldots
}

% Step 7
\renewcommand{\afterupdate}{
\FlaOneByTwo{ B_{0} }{\FlaOneByTwoSingleLine{ B_{1} } { B_{2} }  }
=
\FlaOneByTwo { \hat{B}_{0} }
             { \FlaOneByTwoSingleLine { ( \hat{B}_{1} - B_{2} U_{12} ) U_{11}^{-1} } 
                                      { \hat{B}_{2} U_{22}^{-T} } }
\wedge \ldots
}

% Step 8
\renewcommand{\update}{
\begin{minipage}[t]{4in}
\noindent
% \FlaStartCompute \\
$ B_1 \becomes ( \hat{B}_{1} - B_{2} U_{12} ) U_{11}^{-1} $\\
% \FlaEndCompute \\
\end{minipage}
}

% Generate the annotated algorithm in Fig. 4.4
\begin{figure}[htbp]
\worksheet
\caption{Annotated blocked algorithm for loop-invariant 1.}
\label{fig:ws:utrsm_rut_blk}
\end{figure}
%

\subsubsection{Implementation}

Sequential implementations for the unblocked and blocked algorithms
for this loop-invariant using FLAME are given in
Figs.~\ref{fig:trsm_rut_lazy_unb}--\ref{fig:trsm_rut_lazy_blk}.

\begin{figure}[htbp]
\footnotesize
\begin{quote}
\listinginput{1}{trsm_rut/sequential/Trsm_right_upper_trans_unb.c}
\end{quote}
\caption{Unblocked algorithm for loop-invariants 1 and 2 using FLAME.}
\label{fig:trsm_rut_lazy_unb}
\end{figure}

\begin{figure}[htbp]
\footnotesize
\index{const}{\tt FLA\un RECURSIVE}%
\begin{quote}
\listinginput{1}{trsm_rut/sequential/Trsm_right_upper_trans_blk.c}
\end{quote}
\caption{Blocked algorithm for loop-invariants 1 and 2 using FLAME.
Recursion is optionally supported.}
\label{fig:trsm_rut_lazy_blk}
\end{figure}

\subsection{Loop-invariant 2}

We now examine the loop-invariant
\begin{equation}
\label{eqn:utrsm_rut:p3}
\PInv: 
\FlaOneByTwo{ B_L}{B_R} =
\FlaOneByTwo{ \hat{B}_{L} - B_{R} U_{TR}^{T} }
            { \hat{B}_{R} U_{BR}^{-T}        }
\wedge
\ldots
\end{equation}
Comparing the loop-invariant in (\ref{eqn:utrsm_rut:p3}) with the
postcondition $ B = \hat{B} U^{-T} $ we see again that {\em if} $U=U_{BR}$,
$ B = B_R $, and $ \hat{B} = \hat{B}_R $ then the loop-invariant
implies the postcondition, i.e., that the desired result has been
computed.
%
The initialization in Step 4 in Fig.~\ref{fig:ws:utrsm_rut:var2}
is the same as that in Step 4 in Fig.~\ref{fig:ws:utrsm_rut}.
The fact that for this partitioning of $ B $, $ \hat{B} $, and $ L $,
$
\FlaOneByTwo{ B_L}{B_R} =
\FlaOneByTwo{ \hat{B}_{L} - B_{R} U_{TR}^{T} }
            { \hat{B}_{R} U_{BR}^{-T}        }
$
and the precondition implies the other parts of
the loop-invariant, this initialization has
the desired properties.
%
Similarly, the loop-guard is identical to
the one for loop-invariant 1.

\subsubsection{Unblocked algorithm}

The repartitionings in Step 5a in Fig.~\ref{fig:ws:utrsm_rut:var2}
result in the state:
\begin{equation}
\QBefore: 
\FlaOneByTwo{ \FlaOneByTwoSingleLine{ B_{0} }
                                    { b_{1} } }
	                            { B_{2} } 
=
\FlaOneByTwo{ \FlaOneByTwoSingleLine{ \hat{B}_0 }
                                    { \hat{b}_1 } -
	      B_2
              \FlaTwoByOneSingleLine{ U_{02}     } 
                                    { u_{12}^{T} }^{T} }
             { \hat{B}_{2}U_{22}^{-T} }
\wedge \ldots
\end{equation}
Applying the matrix transpose results in
\[
\QBefore: 
\FlaOneByTwo{ \FlaOneByTwoSingleLine{ B_{0} }
                                    { b_{1} } }
	                            { B_{2} } 
=
\FlaOneByTwo{ \FlaOneByTwoSingleLine{ \hat{B}_0 }
                                    { \hat{b}_1 } -
	      B_2
              \FlaOneByTwoSingleLine{ U_{02}^{T} } 
                                    { u_{12}     } }
             { \hat{B}_{2}U_{22}^{-T} } \\
\wedge \ldots
\]
or
\begin{equation}
\label{eqn:utrsm_rut:var2:before}
\QBefore: 
\FlaOneByTwo{ \FlaOneByTwoSingleLine{ B_{0} }
                                    { b_{1} } }
	                            { B_{2} } 
=
\FlaOneByTwo{ \FlaOneByTwoSingleLine{ \hat{B}_0 - B_{2}U_{02}^T }
                                    { \hat{b}_1 -  B_{2}u_{12}   } }
	    { \hat{B}_{2}U_{22}^{-T} } \\
\wedge \ldots
\end{equation}
before the update.
The redefinition in Step 5b in Fig.~\ref{fig:ws:utrsm_rut:var2}
means that the following state
must result from the update:
\[
\QAfter: 
\FlaOneByTwo{ B_{0} }{ \FlaOneByTwoSingleLine{ b_{1} } { B_{2} }  }
=
\FlaOneByTwo { \hat{B}_0 -
               \FlaOneByTwoSingleLine{ b_1 } { B_2 }
               \FlaTwoByOneSingleLine{ u_{01}^T  } 
                                     { U_{02}^T  } }
             { \FlaOneByTwoSingleLine{ \hat{b}_{1} } { \hat{B}_{2} }
	     { \FlaTwoByTwoSingleLine { \upsilon_{11} } { u_{12}^{T} }
	                              {      0        } { U_{22}     } }^{-T} } 
\wedge \ldots
\]
Applying matrix transposition and inversion to $ U $ leaves us
\[
\QAfter: 
\FlaOneByTwo{ B_{0} }{ \FlaOneByTwoSingleLine{ b_{1} } { B_{2} }  }
=
\FlaOneByTwo { \hat{B}_0 -
               \FlaOneByTwoSingleLine{ b_1 } { B_2 }
               \FlaTwoByOneSingleLine{ u_{01}^T  } 
                                     { U_{02}^T  } }
             { \FlaOneByTwoSingleLine{ \hat{b}_{1} } { \hat{B}_{2} }
	       \FlaTwoByTwoSingleLine { \upsilon_{11}^{-1}                 } {       0     }
                                      { -U_{22}^{-T} u_{12} \upsilon_{11}^{-1} } { U_{22}^{-T} } }
\wedge \ldots
\]
and after various matrix multiplications and simplifications
\begin{equation}
\label{eqn:utrsm_rut:var2:after}
\QAfter: 
\FlaOneByTwo{ B_{0} }{ \FlaOneByTwoSingleLine{ b_{1} } { B_{2} }  }
=
\FlaOneByTwo { ( \hat{B}_0 - B_{2} U_{02}^{T} ) - b_{1} u_{01}^{T} }
             {
               \FlaOneByTwoSingleLine { ( \hat{b}_1 - B_2 u_{12} ) \upsilon_{11}^{-1} }
                                      { \hat{B}_{2} U_{22}^{-T} } } 
\wedge \ldots
\end{equation}


Comparing 
(\ref{eqn:utrsm_rut:var2:before}) and
(\ref{eqn:utrsm_rut:var2:after})
we find that the update
\begin{eqnarray*}
& B_0   & \becomes B_0 -  b_{1} u_{01}^T  \\
& b_{1} & \becomes \hat{b}_{1} \upsilon_{11}^{-1}
\end{eqnarray*}
is required to change the state from $ \QBefore $
to $ \QAfter $.


%Again one recognizes that $ \hat{B} $ is never
%referenced and can be eliminated from the algorithm.

% Define the commands to generate the annotated
% algorithm in Fig. 4.7.

% Step 2
\renewcommand{\invariant}{
\FlaOneByTwo{ B_L}{B_R} =
\FlaOneByTwo{ \hat{B}_{L} - B_{R} U_{TR}^{T} }
            { \hat{B}_{R} U_{BR}^{-T}        }
\wedge
\ldots
}

% Step 3
\renewcommand{\guard}{ \neg \SameSize( B, B_{BR} ) }

% Step 4
\renewcommand{\partitionings}{
$ 
B \rightarrow \FlaOneByTwo{ B_L }
                          { B_R }
$,
$ 
\hat{B} \rightarrow \FlaOneByTwo{ \hat{B}_L }
                                { \hat{B}_R }
$, and
$ 
U \rightarrow \FlaOneByTwo{ U_{TL} }{ U_{TR} }
                          {   0    }{ U_{BR} }
$
}
\renewcommand{\partitionsizes}{
$ B_R $ and $ \hat{B}_R $ have $ 0 $ rows
and $ U_{BR} $ is $ 0 \times 0 $
}

% Step 5a
\renewcommand{\repartitionings}{
$ 
\FlaOneByTwo{ B_L }
            { B_R } 
\rightarrow
\FlaOneByThreeL{ B_0 }
               { b_1 }
               { B_2 },
\FlaOneByTwo{ \hat{B}_L }
            { \hat{B}_R } 
\rightarrow
\FlaOneByThreeL{ \hat{B}_0 }
               { \hat{b}_1 }
               { \hat{B}_2 }
$ \\
and
$ 
\FlaTwoByTwo{ U_{TL} }{ U_{TR} }
            {   0    }{ U_{BR} } 
\rightarrow
\FlaThreeByThreeTL{ U_{00} } { u_{01}        } { U_{02}     }
                  {   0    } { \upsilon_{11} } { u_{12}^{T} }
                  {   0    } {    0          } { U_{22}     }
$
}
\renewcommand{\repartitionsizes}{
$ b_1 $ and $ \hat{b}_1 $ are columns 
and $ \upsilon_{11} $ is a scalar
 }

% Step 5b
\renewcommand{\moveboundaries}{
$
\FlaOneByTwo{ B_L }
            { B_R } 
\leftarrow
\FlaOneByThreeR{ B_0 } { b_1 } { B_2 },
\FlaOneByTwo{ \hat{B}_L }
            { \hat{B}_R } 
\leftarrow
\FlaOneByThreeR{ \hat{B}_0 } { \hat{b}_1 } { \hat{B}_2 }
$ \\
and
$ 
\FlaTwoByTwo{ U_{TL} }{ U_{TR} }
            {    0   }{ U_{BR} } \leftarrow
\FlaThreeByThreeBR{ U_{00} } { u_{01}        } { U_{02}     }
                  {   0    } { \upsilon_{11} } { u_{12}^{T} }
                  {   0    } {      0        } { U_{22}     }  
$
}

% Step 6
\renewcommand{\beforeupdate}{
\FlaOneByTwo{ \FlaOneByTwoSingleLine{ B_{0} }
                                    { b_{1} } }
	                            { B_{2} } 
=
\FlaOneByTwo{ \FlaOneByTwoSingleLine{ \hat{B}_0 - B_{2}U_{02}^T }
                                    { \hat{b}_1 - B_2 u_{12}   } }
	    { \hat{B}_{2}U_{22}^{-T} }
}

% Step 7
\renewcommand{\afterupdate}{
\FlaOneByTwo{ B_{0} }{\FlaOneByTwoSingleLine{ b_{1} } { B_{2} }  }
=
\FlaOneByTwo { ( \hat{B}_0 - B_{2} U_{02}^{T} ) - b_{1} u_{01}^{T} }
             {
               \FlaOneByTwoSingleLine { ( \hat{b}_1 - B_2 u_{12} ) \upsilon_{11}^{-1} }
                                      { \hat{B}_{2} U_{22}^{-T} } } 
\wedge \ldots
}

% Step 8
\renewcommand{\update}{
\begin{minipage}[t]{4in}
\noindent
% \FlaStartCompute \\
$ B_0   \becomes B_0 -  b_{1} u_{01}^T $ \\
$ b_{1} \becomes \hat{b}_{1} \upsilon_{11}^{-1} $ \\
% \FlaEndCompute \\
\end{minipage}
}

% Generate figure 4.7
\begin{figure}[htbp]
\worksheet
\caption{Annotated unblocked algorithm for loop-invariant 2.}
\label{fig:ws:utrsm_rut:var2}
\end{figure}

\subsubsection{Blocked Algorithms}

Again, the algorithm can be cast to be rich in
matrix-matrix multiplications by marching through
the matrices multiple rows and/or columns at a time.
The resulting algorithm is given in 
Fig.~\ref{fig:ws:utrsm_rut:var2:blk}.

% Redefine commands to generate Fig. 4.8

% Step 5a
\renewcommand{\repartitionings}{
$ 
\FlaOneByTwo{ B_L }
            { B_R } 
\rightarrow
\FlaOneByThreeL{ B_0 }
               { B_1 }
               { B_2 },
\FlaOneByTwo{ \hat{B}_L }
            { \hat{B}_R } 
\rightarrow
\FlaOneByThreeL{ \hat{B}_0 }
               { \hat{B}_1 }
               { \hat{B}_2 }
$ \\
and
$ 
\FlaTwoByTwo{ U_{TL} }{ U_{TR} }
            {   0    }{ U_{BR} } 
\rightarrow
\FlaThreeByThreeTL{ U_{00} } { U_{01} } { U_{02} }
                  {   0    } { U_{11} } { U_{12} }
                  {   0    } {    0   } { U_{22} }
$
}
%
\renewcommand{\repartitionsizes}{
$ B_1 $ and $ \hat{B}_1 $ have columns 
and $ U_{11} $ is $ b \times b $
}

% Step 5b
\renewcommand{\moveboundaries}{%
$ 
\FlaOneByTwo{ B_L }
            { B_R } 
\leftarrow
\FlaOneByThreeR{ B_0 } { B_1 } { B_2 },
\FlaOneByTwo{ \hat{B}_L }
            { \hat{B}_R } 
\leftarrow
\FlaOneByThreeR{ \hat{B}_0 } { \hat{B}_1 } { \hat{B}_2 }
$ \\
and
$ 
\FlaTwoByTwo{ U_{TL} }{ U_{TR} }
            {    0   }{ U_{BR} } \leftarrow
\FlaThreeByThreeBR{ U_{00} } { U_{01} } { U_{02} }
                  {   0    } { U_{11} } { U_{12} }
                  {   0    } {    0   } { U_{22} }  
$
}

% Step 6
\renewcommand{\beforeupdate}{
\FlaOneByTwo{ \FlaOneByTwoSingleLine{ B_{0} }
                                    { B_{1} } }
	                            { B_{2} } 
=
\FlaOneByTwo{ \FlaOneByTwoSingleLine{ \hat{B}_0 - B_{2}U_{02}^T }
                                    { \hat{B}_1 - B_2 U_{12}   } }
	    { \hat{B}_{2}U_{22}^{-T} }
\wedge \ldots
}

% Step 7
\renewcommand{\afterupdate}{
\FlaOneByTwo{ B_{0} }{\FlaOneByTwoSingleLine{ B_{1} } { B_{2} }  }
=
\FlaOneByTwo { ( \hat{B}_0 - B_{2} U_{02}^{T} ) - B_{1} U_{01}^{T} }
             {
               \FlaOneByTwoSingleLine { ( \hat{B}_1 - B_2 U_{12} ) U_{11}^{-1} }
                                      { \hat{B}_{2} U_{22}^{-T} } } 
\wedge \ldots
}

% Step 8
\renewcommand{\update}{
\begin{minipage}[t]{4in}
\noindent
% \FlaStartCompute \\
$ B_0 \becomes B_0 -  B_{1} U_{01}^T $ \\
$ B_1 \becomes \hat{B}_{1} U_{11}^{-1} $ \\
% \FlaEndCompute \\
\end{minipage}
}

% Generate Fig. 4.8
\begin{figure}[htbp]
\worksheet
\caption{Annotated blocked algorithm for loop-invariant 2.}
\label{fig:ws:utrsm_rut:var2:blk}
\end{figure}
%

\subsubsection{Implementation}

Sequential implementations for the unblocked and blocked algorithms
for this loop-invariant using FLAME are given in
Figs.~\ref{fig:trsm_rut_lazy_unb}--\ref{fig:trsm_rut_lazy_blk}.

\section{Performance}

In this section, we discuss the performance attained by the different
variants for computing $ B \leftarrow B U^{-T} $.  In each of the Figs.
~\ref{fig:trsm_rut:lazy-row-lazy:ATLAS} and
~\ref{fig:trsm_rut:lazy-row-lazy:ITXGEMM}, we compare the performance
attained by five different implementations:
\begin{itemize}
\item{Reference}
{\tt DTRSM} as implemented as part of ATLAS,
\item{FLAME}
{\tt FLA\_Trsm} implemented as part of FLAME
as of this writing,
\item{Unblocked}
the unblocked implementation in Fig.~\ref{fig:trsm_rut_lazy_unb},
\item{Blocked}
the blocked implementation in Fig.~\ref{fig:trsm_rut_lazy_blk}
called with 
{\tt rec == 
\index{const}{\tt FLA\un NON\un RECURSIVE}%
FLA\_NON\_RECURSIVE}, and
\item{Recursive}
the blocked implementation in Fig.~\ref{fig:trsm_rut_lazy_blk}
called with 
{\tt rec == 
\index{const}{\tt FLA\un RECURSIVE}%
FLA\_RECURSIVE}.
\end{itemize}
In Fig.~\ref{fig:trsm_rut:lazy-row-lazy:ATLAS}, all FLAME
implementations are based on the matrix-matrix multiplication ({\tt
DGEMM}) provided by ATLAS.  Notice that for the platform on which we
performed the experiments, multiples of 40 are good block sizes when
using the ATLAS matrix-matrix multiplication kernel.  We note that the
unblocked algorithms perform badly, since they are rich in
matrix-vector operations that do not benefit much from cache memories.
The blocked algorithms performed better for relatively small block
sizes ({\tt nb\_alg=40}) than for larger block sizes ({\tt
nb\_alg=80}).  The reason for this is that too much of the
computation is performed in the subprogram for which the unblocked
algorithm is used.  The recursive implementations benefit from larger
block sizes, since they overcome this problem that plagues the blocked
algorithm.  In addition they benefit from the fact that the {\tt
DGEMM} kernel performs better for larger blocks.

In Fig.~\ref{fig:trsm_rut:lazy-row-lazy:ITXGEMM}, all FLAME
implementations are based on the matrix-matrix multiplication ({\tt
DGEMM}) provided by ITXGEMM.  This time, as discussed in
Section~\ref{sec:mmmult:impact}, 128 is a magic block size.  Again, as
expected, the unblocked implementations perform badly.  Again, the
blocked algorithm benefits from smaller block sizes and the recursive
algorithm performs best.  

%It is interesting to note that the variant 1
%algorithm performs much better for small problem sizes when the
%outer-most block size is chosen to be 128.  This is probably mostly
%due to the fact that the unblocked lazy algorithm performs better.

It is interesting to note that asymptotically, the recursive implementation
for variant 1 performs somewhat worse than the recursive implementation for
variant 2, when ATLAS is used for the matrix-matrix multiply
(Fig.~\ref{fig:trsm_rut:lazy-row-lazy:ATLAS}).  We attribute this to
the fact that when $ m $ is relatively small and $ n $ and $ k $ are
large (e.g., in a panel-matrix multiply), the ATLAS matrix-matrix
multiplication does not perform as well as when $ k $ is relatively
small and $ m $ and $ n $ are large (e.g., in a panel-panel multiply).
Since the variant 1 and variant 2 algorithms are rich in panel-matrix and
panel-panel multiplies, respectively, the variant 1 algorithm attains
better performance.  This is not observed as dramatically in in the
experiments where ITXGEMM was used for matrix-matrix multiplication
(Fig.~\ref{fig:trsm_rut:lazy-row-lazy:ITXGEMM}).  This is due to the
fact that ITXGEMM attains performance that is similar regardless of
the shape of the matrix.

%Notice that the right-moving algorithms perform miserably
%(Fig.~\ref{fig:trsm_rut:right-moving}).  We point out that the blocked
%and recursive algorithms attain essentially the same performance as
%the unblocked algorithm.  This is not surprising since all computation
%is in a triangular matrix-vector multiply, which does not perform very
%well, regardless of how the blocking proceeds.

Since in the recursive algorithms at each level a different variant
could be called to solve the subproblem, it may be possible to improve
performance further by combining different variants.  We have not yet
studied this.

\begin{figure}[htbp]
\begin{center}
\begin{tabular}{c | c}
Variant 1 & Variant 2 \\ \hline
& \\
\psfig{figure=trsm_rut/graphs/FLA_trsm_rut_variant1_40.eps,width=3.0in,height=3.0in} &
\psfig{figure=trsm_rut/graphs/FLA_trsm_rut_variant2_40.eps,width=3.0in,height=3.0in}
\\ \hline
& \\
\psfig{figure=trsm_rut/graphs/FLA_trsm_rut_variant1_80.eps,width=3.0in,height=3.0in} &
\psfig{figure=trsm_rut/graphs/FLA_trsm_rut_variant2_80.eps,width=3.0in,height=3.0in}
\end{tabular}
\end{center}
\caption{Performance of the variant 1 (left) and variant 2 (right) 
triangular matrix solve (with multiple RHS) algorithms for a block 
size of $ 40 $ (top) and
$ 80 $ (bottom).
For these experiments, the ATLAS matrix-matrix multiplication
kernel was used.}
\label{fig:trsm_rut:lazy-row-lazy:ATLAS}
\end{figure}

%\begin{figure}[htbp]
%\begin{center}
%\begin{tabular}{c | c}
%Row-Lazy & Lazy \\ \hline
%& \\
%\psfig{figure=trmm_lln/graphs/trmm_lln_rowlazy_wrt_L_32.eps,width=3.0in,height=3.0in} &
%\psfig{figure=trmm_lln/graphs/trmm_lln_lazy_wrt_L_32.eps,width=3.0in,height=3.0in}
%\\ \hline
%& \\
%\psfig{figure=trmm_lln/graphs/trmm_lln_rowlazy_wrt_L_128.eps,width=3.0in,height=3.0in} &
%\psfig{figure=trmm_lln/graphs/trmm_lln_lazy_wrt_L_128.eps,width=3.0in,height=3.0in}
%\end{tabular}
%\end{center}
%\caption{Performance of the row lazy (left) and lazy (right) (w.r.t. $ L $) 
%lower triangular matrix-matrix multiplication 
%algorithms for a block size of $ 32 $ (top) and
%$ 128 $ (bottom).
%For these experiments, the ITXGEMM matrix-matrix multiplication
%kernel was used.}
%\label{fig:trmm_lln:lazy-row-lazy:ITXGEMM}
%\end{figure}

 
%% The following commands help in generating the index
\index{index}{symmetric rank-k update}%
\index{op}{symmetric rank-k update!$ C \leftarrow A A^T + \hat{C} $|( }%

% Name of the chapter

\chapter{Symmetric Rank-k Update \\
$ C \leftarrow A A^T + \hat{C} $ \\
$ C $ symmetric, stored in upper triangle
}
\label{chapter:usyrk_unn}

% Authors

\ChapterAuthor{
\index{author}{Chen, Katherine}
Katherine Chen
\\[0.1in]
\index{author}{Ho, Mandy}
Mandy Ho
\\[0.1in]
\index{author}{Lai, Ingrid}
Ingrid Lai
}

% Add the names of the authors to the table of contents

\addtocontents{toc}{ {by {\bf Katherine Chen, Mandy Ho, and Ingrid Lai}}}

In this chapter, we discuss the implementation of the symmetric rank-k update
\[
C \leftarrow A A^T + \hat{C}
\]
where $ C $ is an $ n \times n $ symmetric matrix and $ A $ is
$ n \times k $.  We will overwrite $ C $ with the result without
requiring a workarray.  We start by deriving a number of different
sequential algorithms.  Subsequently, we show how to code the
algorithms using FLAME.  

The variables for the symmetric rank-k update can be
described by the precondition
\[
\PPre:
C = \hat{C} \wedge \SameSize( C, \hat{C} ) \wedge \Symm( C ) \wedge
\RowDim( A )=\RowDim(\hat{C}),
\]
where $ \hat{C} $ indicates the original contents of $ C $.  Here the
predicate $ \Symm( C ) $ is true iff $ C $ is a symmetric
matrix.  The functions $ \RowDim( C ) $, $ \RowDim( A ) $  and $ \ColDim( C ) $,
$ \ColDim( A ) $ return
the row and column dimension of $ C $ and $ A $, respectively.  The operation to
be performed, $ C \becomes A A^T + C $, translates to the postcondition $
\PPost: C = A A^T + \hat{C} $.

\section{Algorithms That Start By Partitioning $ C $}
\label{sec:trmm_lln:L}

Let us start by partitioning matrix $ C $.  Since it is symmetric, we
partition it like
\[
C \rightarrow \FlaTwoByTwo{ C_{TL} }        { C_{TR} }
                          { \undetermined } { C_{BR} },
\]
where $ C_{TL} $ is square so that both submatrices on the diagonal
are themselves symmetric.  Notice that $ \undetermined $ indicates
that that part of the matrix is not stored.

Substituting the partitioning of $ C $ into the postcondition yields
\[
\FlaTwoByTwo{ C_{TL} }        { C_{TR} }
            { \undetermined } { C_{BR} }
= 
( \mbox{some partitioning of }{A} )
( \mbox{some partitioning of }{A^T} )
+
\FlaTwoByTwo{ \hat{C}_{TL} }  { \hat{C}_{TR} }
            { \undetermined } { \hat{C}_{BR} }
\]
This suggests that $ A $ should be partitioned
horizontally into two submatrices, or into quadrants.  Let us consider
the case where $ A $ is partitioned horizontally.  Then
\[
\FlaTwoByTwo{ C_{TL} }       { C_{TR} }
            { \undetermined} { C_{BR} }
= 
\FlaTwoByOne{ A_T }
            { A_B }
\FlaOneByTwo{ {A_T}^T }
            { {A_B}^T }
+
\FlaTwoByTwo{ \hat{C}_{TL} }  { \hat{C}_{TR} }
            { \undetermined } { \hat{C}_{BR} }
\]
In order to be able to multiply the matrices on the right out and to
be able to then set the submatrices on the left equal to the result on
the right we find that the following must hold:
\[
\ColDim( C_{TL} ) = \RowDim( A_T ) 
\]
Substituting the partitioned matrices into the postcondition, we find
\[
\FlaTwoByTwo{ C_{TL} }       { C_{TR} }
            { \undetermined} { C_{BR} }
=
\FlaTwoByOne{ A_T }
            { A_B }
\FlaOneByTwo{ {A_T}^T }
            { {A_B}^T }
+
\FlaTwoByTwo{ \hat{C}_{TL} }  { \hat{C}_{TR} }
            { \undetermined } { \hat{C}_{BR} }
=
\FlaTwoByTwo{ A_T {A_T}^T + \hat{C}_{TL} } { A_T {A_B}^T + \hat{C}_{TR} }
            { \undetermined }              { A_B {A_B}^T + \hat{C}_{BR} }
\]
which yields the equalities
\begin{equation}
\label{eqn:usyrk_unn}
\begin{array}{r c l || r c l}
C_{TL} &=& A_T {A_T}^T + \hat{C}_{TL} & C_{TR} &=& A_T {A_B}^T + \hat{C}_{TR} \\ \hline \hline
C_{BL} &=& \undetermined              & C_{BR} &=& A_B {A_B}^T + \hat{C}_{BR}
\end{array}
\end{equation}

% Insert the table of possible loop-invariants from
% file syrk_unn/table.new.tex

\begin{figure}
\begin{center}
\footnotesize
% \begin{sideways}
{
\setlength{\tabcolsep}{4pt}
\begin{tabular}{| c | p{2.5in} | c | } \hline
$ P_X: $

$ \FlaTwoByTwo{ C_{TL} } { C_{TR} }
              { C_{BL} } { C_{BR} } =  $ & Comment &
\footnotesize Feasible? \\ \hline \hline
\footnotesize
$
\FlaTwoByTwo{ \hat{C}_{TL} }  { \hat{C}_{TR} }
            { \undetermined } { \hat{C}_{BR} }
$ 
&
&
NO
\\ \hline
%
%
\footnotesize
$
\FlaTwoByTwo{ A_{T} A_{T}^T + \hat{C}_{TL} }  { A_{T} A_{B}^T + \hat{C}_{TR} }
            { \undetermined }                 { A_{B} A_{B}^T + \hat{C}_{BR} }
$ 
& 
NO
\\ \hline
%
%
\footnotesize
$
\FlaTwoByTwo{ A_{T} A_{T}^T + \hat{C}_{TL} }  { \hat{C}_{TR} }
            { \undetermined }                 { \hat{C}_{BR} }
$ 
& 
YES
\\ \hline
%
%
\footnotesize
$
\FlaTwoByTwo{ \hat{C}_{TL} }  { A_{T} A_{B}^T + \hat{C}_{TR} } 
            { \undetermined } { \hat{C}_{BR} }
$ 
&
NO
\\ \hline
%
%
\footnotesize
$
\FlaTwoByTwo{ \hat{C}_{TL} }   { \hat{C}_{TR} }
            { \undetermined }  { A_{B} A_{B}^T + \hat{C}_{BR} }
$ 
&
Loop-invariant 1
&
YES
\\ \hline
%
%
\footnotesize
$
\FlaTwoByTwo{ \hat{C}_{TL} } { A_{T} A_{B}^T + \hat{C}_{TR} }
            { \undetermined} { A_{B} A_{B}^T + \hat{C}_{BR} }
$ 
&
YES
\\ \hline
%
%
\footnotesize
$
\FlaTwoByTwo{ A_{T} A_{T}^T + \hat{C}_{TL} } { \hat{C}_{TR} }
            { \undetermined }                { A_{B} A_{B}^T + \hat{C}_{BR} }
$ 
&
YES
\\ \hline
%
%
\footnotesize
$
\FlaTwoByTwo{ A_{T} A_{T}^T + \hat{C}_{TL} } { A_{T} A_{B}^T + \hat{C}_{TR} }
            { \undetermined }                { \hat{C}_{BR} }
$ 
&
Loop-invariant 2.
&
YES
\\ \hline
%\\ \hline
%
%
\end{tabular}
}
% \end{sideways}
\end{center}
\caption{Possible loop-invariants when partitioning
$ C $ into quadrants.
Here $ P_X $ is the most prominent part of the loop-invariant
$ \PInv $.}
\label{fig:USYRK_UNN_example}
\end{figure}


From (\ref{eqn:usyrk_unn}) we conclude that the operations to be
performed are $ A_T {A_T}^T + \hat{C}_{TL} $, $ A_T {A_B}^T + \hat{C}_{TR} $, and $ A_B {A_B}^T + 
\hat{C}_{BR} $.  At an intermediate stage, each of these either has or
has not already been computed, leading to the $ 2^3 = 8 $ possible
loop-invariants tabulated in Fig.~\ref{fig:USYRK_UNN_example}.

\subsection{Loop-invariant 1}

We will first examine the (feasible) loop-invariant
\begin{equation}
\label{eqn:usyrk_unn:p2}
\PInv: \FlaTwoByTwo{ C_{TL} } { C_{TR} }
            { \undetermined}  { C_{BR} }
 =
\FlaTwoByTwo{ \hat{C}_{TL} }  { \hat{C}_{TR} }
            { \undetermined } { A_B {A_B}^T + \hat{C}_{BR} }
\wedge
\ldots
\end{equation}
Comparing the loop-invariant in (\ref{eqn:usyrk_unn:p2}) with the
postcondition $ C = A A^T +  \hat{C} $ we see that {\em if} $C=C_{BR}$, $ A =
A_B $, and $ \hat{C} = \hat{C}_{BR} $ then the loop-invariant implies the
postcondition, i.e., that the desired result has been computed.
Notice that $ \SameSize( C, C_{BR} ) \wedge \PInv $ implies $ C =
C_{BR} $, $ A = A_B $ and $ \hat{C}=\hat{C}_{BR} $ since the partitioned
matrices are references into the original matrices $ C $, $ A $, and $
\hat{C} $.  Thus, the loop-guard $ G: \neg \SameSize( C, C_{BR} ) $
has the desired property.

Consider the initialization in Step 4 in Fig.~\ref{fig:ws:usyrk_unn}.
The fact that for this partitioning of $ C $, $ \hat{C} $, and $ A $,
$
\FlaTwoByTwo{ C_{TL} }       { C_{TR} }
            { \undetermined} { C_{BR} }
=
\FlaTwoByTwo{ \hat{C}_{TL} }  { \hat{C}_{TR} }
            { \undetermined } { A_B {A_B}^T + \hat{C}_{BR} }
$ and the precondition implies the
other parts of the loop-invariant, this initialization has the desired
properties.

Loop-guard $ G $ indicates that eventually $ C_{BR} $ should equal all
of $ C $, at which point $ G $ becomes {\em false} and the loop is
exited.  After the initialization, $ C_{BR} $ is $ 0 \times 0 $.
Thus, the repartitioning should be such that as the computation
proceeds, rows and columns are subtracted from $ C_{TL} $ and added to
$ C_{BR} $ while updating $ C_{BR} $ consistently with this.

\subsubsection{Unblocked algorithm}

If we move the partitionings by individual rows and columns, we obtain
the repartitioning and redefinition of the partitioning in Steps 5a
and 5b in Fig.~\ref{fig:ws:usyrk_unn}.  The repartitionings in Step 5a
in Fig.~\ref{fig:ws:usyrk_unn} result in the state
\begin{equation}
\label{eqn:usyrk_unn:bu}
\QBefore: 
\FlaThreeByThreeTL{ C_{00} }  { c_{01} }  { C_{02} }
                  {\undetermined}{ \gamma_{11}}     { c_{12}^{T} }
                  { \undetermined }  { \undetermined }  { C_{22} }
=
\FlaTwoByTwo{ \FlaTwoByTwoSingleLine{ \hat{C}_{00} }   { \hat{c}_{01} }
                           { \undetermined} { \hat{\gamma}_{11}   } }   
            { \FlaTwoByOneSingleLine{ \hat{C}_{02} }{ \hat{c}_{12}^{T} } }
            { \undetermined }  
            { A_2 A_2^{T} + \hat{C}_{22}}
\wedge \ldots
\end{equation}
before the update.  The redefinition in Step 5b in
Fig.~\ref{fig:ws:usyrk_unn} means that the following state must result
from the update:
\[
\QAfter: 
\FlaThreeByThreeBR{ C_{00} }  { c_{01} }  { C_{02} }
                  { \undetermined }{ \gamma_{11}}     { c_{12}^{T} }
                  { \undetermined }  { \undetermined }         { C_{22}}
=
\FlaTwoByTwo{ \hat{C}_{00} }
            { \FlaOneByTwoSingleLine{\hat{c}_{01}}
                                    {\hat{C}_{02}}
            }
            {\undetermined  }
{
 \FlaTwoByOneSingleLine{a_{1}^T}
                       {A_{2}} 
 \FlaTwoByOneSingleLine{a_{1}^T}
                       {A_{2}}^T +
\FlaTwoByTwoSingleLine{ \hat{\gamma}_{11} }{ \hat{c}_{12}^{T} }
            { \undetermined } { \hat{C}_{22} }
}
\wedge \ldots
\]
Multiplying out the symmetric rank-1 update yields
\begin{equation}
\label{eqn:usyrk_unn:au}
\QAfter: 
\FlaThreeByThreeBR{ C_{00} }  { c_{01} }  { C_{02} }
                  { \undetermined }{ \gamma_{11}}     { c_{12}^{T} }
                  { \undetermined }  { \undetermined }         { C_{22}}
=
\FlaTwoByTwo { \hat{C}_{00} }
             { \FlaOneByTwoSingleLine{ \hat{c}_{01} }{ \hat{C}_{02} } }
             { \undetermined }
             { \FlaTwoByTwoSingleLine{ \hat{\gamma}_{11} + a_1^T a_1 }{ a_1^T A_2^T + \hat{c}_{12}^{T} }
                                     { \undetermined }{ A_2 A_2^{T} + \hat{C}_{22}}}
\wedge \ldots
\end{equation}
Comparing Eqns.~\ref{eqn:usyrk_unn:bu} and~\ref{eqn:usyrk_unn:au} we
find that the updates
\begin{eqnarray*}
& \gamma_{11} & \becomes \gamma_{11} + a_1^T a_1 \\
& c_{12} & \becomes c_{12} +  A_2 a_1
\end{eqnarray*}
are required to change the state from $ \QBefore $ to $ \QAfter $.

% The following commands will in the ``worksheet''
% given in Fig. 4.2

% Step 0: Operation
\renewcommand{\operation}{C \becomes A A^T + C}

% Step 1a: Precondition
\renewcommand{\precondition}{C = \hat{C} \wedge \ldots }

% Step 1b: Postcondition
\renewcommand{\postcondition}{C = A A^T + \hat{C}}

% Step 2: Loop-invariant
\renewcommand{\invariant}{
\FlaTwoByTwo{ C_{TL} }       { C_{TR} }
            { \undetermined} { C_{BR} }
=
\FlaTwoByTwo{ \hat{C}_{TL} }  { \hat{C}_{TR} }
            { \undetermined } { A_B {A_B}^T + \hat{C}_{BR} }
\wedge
\ldots
}

% Step 3: Loop-guard
\renewcommand{\guard}{ \neg \SameSize( C, C_{BR} ) }

% Step 4: Initialization
\renewcommand{\partitionings}{
$
C \rightarrow \FlaTwoByTwo{ C_{TL} }{ C_{TR} }
                          { C_{TR}^{T} }{ C_{BR} }
$,
$ 
\hat{C} \rightarrow \FlaTwoByTwo{ \hat{C}_{TL} }{ \hat{C}_{TR} }
                            { \hat{C}_{TR}^{T} }{ \hat{C}_{BR} }
$, and
$ 
A \rightarrow \FlaTwoByOne{ A_{T} }
                          { A_{B} }
$
}
\renewcommand{\partitionsizes}{
$ C_{BR} $ and $ \hat{ C }_{BR} $ are  $ 0 \times 0 $ 
and $ A_{B} $ has $ 0 $ rows.
}

% Step 5a: repartitioning
\renewcommand{\repartitionings}{
$ 
\FlaTwoByTwo{ C_{TL} }{ C_{TR}}
            { \undetermined }{ C_{BR} } 
\rightarrow 
\FlaThreeByThreeTL{ C_{00} }  { c_{01} }  { C_{02} }
                  {\undetermined}{ \gamma_{11}}     { c_{12}^{T} }
                  { \undetermined }  { \undetermined }  { C_{22} }
,
\FlaTwoByTwo{ \hat{C}_{TL} }{ \hat{C}_{TR}}
            { \undetermined }{ \hat{C}_{BR} } 
\rightarrow 
\FlaThreeByThreeTL{ \hat{C}_{00} }  { \hat{c}_{01} }  { \hat{C}_{02} }
                  {\undetermined}{ \hat{\gamma}_{11}}     { \hat{c}_{12}^{T} }
                  { \undetermined }  { \undetermined }  { \hat{C}_{22} }
,
$ \\
and
$
\FlaTwoByOne{ A_{T} }
            { A_{B} } 
\rightarrow
\FlaThreeByOneT{ A_{0} }
               { a_{1}^T }
               { A_{2} }
$
}
\renewcommand{\repartitionsizes}{
$ \gamma_{11} $ and $ \hat{\gamma}_{11} $ are scalars 
and $ a_{1}^T $ is a row.
}

% Step 5b: moving the boundaries
\renewcommand{\moveboundaries}{%
$ 
\FlaTwoByTwo{ C_{TL} }{ C_{TR}}
            { \undetermined }{ C_{BR} } 
\leftarrow
\FlaThreeByThreeBR{ C_{00} }  { c_{01} }  { C_{02} }
                  { \undetermined }{ \gamma_{11}}     { c_{12}^{T} }
                  { \undetermined }  { \undetermined }         { C_{22}}
,
\FlaTwoByTwo{ \hat{C}_{TL} }{ \hat{C}_{TR}}
            { \undetermined }{ \hat{C}_{BR} } 
\leftarrow
\FlaThreeByThreeBR{ \hat{C}_{00} }  { \hat{c}_{01} }  { \hat{C}_{02} }
                  { \undetermined }{ \hat{\gamma}_{11}}     { \hat{c}_{12}^{T}}
                  { \undetermined }  { \undetermined }      { \hat{C}_{22}},
$ \\
and
$ 
\FlaTwoByOne{ A_{T} }
            { A_{B} } 
\leftarrow
\FlaThreeByOneB{ A_{0} }
               { a_{1}^T }
               { A_{2} }
$
}

% Step 6: state before update
\renewcommand{\beforeupdate}{
\FlaThreeByThreeTL{ C_{00} }  { c_{01} }  { C_{02} }
                  {\undetermined}{ \gamma_{11}}     { c_{12}^{T} }
                  { \undetermined }  { \undetermined }  { C_{22} }
=
\FlaTwoByTwo{ \FlaTwoByTwoSingleLine{ \hat{C}_{00} }   { \hat{c}_{01} }
                           { \undetermined} { \hat{\gamma}_{11}   } }   
            { \FlaTwoByOneSingleLine{ \hat{C}_{02} }{ \hat{c}_{12}^{T} } }
            { \undetermined }  
            { A_2 A_2^{T} + \hat{C}_{22}}
\wedge \ldots
}

% Step 7: state after update
\renewcommand{\afterupdate}{
\FlaThreeByThreeBR{ C_{00} }  { c_{01} }  { C_{02} }
                  { \undetermined }{ \gamma_{11}}     { c_{12}^{T} }
                  { \undetermined }  { \undetermined }         { C_{22}}
=
\FlaTwoByTwo { \hat{C}_{00} }
             { \FlaOneByTwoSingleLine{ \hat{c}_{01} }{ \hat{C}_{02} } }
             { \undetermined }
             { \FlaTwoByTwoSingleLine{ \hat{\gamma}_{11} + a_1^T a_1 }{ a_1^T A_2^T + \hat{c}_{12}^{T} }
                                     { \undetermined }{ A_2 A_2^{T} + \hat{C}_{22}}}
\wedge \ldots
}

% Step 8: update
\renewcommand{\update}{
\begin{minipage}[t]{4in}
\noindent
% \FlaStartCompute \\
$ \gamma_{11} \becomes \gamma_{11} + a_1^T a_1 $\\
$ c_{12} \becomes c_{12} +  A_2 a_1 $\\
% \FlaEndCompute \\
\end{minipage}
}

% Given the commands defined above, the
% command \worksheet generates the annotated
% algorithm

\begin{figure}[htbp]
\worksheet
\caption{Annotated algorithm for the symmetric rank-k update example.}
\label{fig:ws:usyrk_unn}
\end{figure}

By recognizing that $ \hat{C} $ is never referenced we can eliminate
all parts of the algorithm that refer to this matrix, yielding the
final algorithm given in Fig.~\ref{fig:alg:usyrk_unn}.

% We now redefine some of the commands
% used to generate Fig. 4.2, taking out all references
% to \hat{B} to come up with the algorithm in Fig. 4.3.

% Step 4
\renewcommand{\partitionings}{
$ 
C \rightarrow \FlaTwoByTwo{ C_{TL} }{ C_{TR} }
                          { C_{TR}^{T} }{ C_{BR} }
$
and
$ 
A \rightarrow \FlaTwoByOne{ A_{T} }
                          { A_{B} }
$
}
\renewcommand{\partitionsizes}{
$ A_{B} $ has $ 0 $ rows
and $ C_{BR} $ is $ 0 \times 0 $
}

% Step 5a
\renewcommand{\repartitionings}{
$ 
\FlaTwoByTwo{ C_{TL} }{ C_{TR}}
            { \undetermined }{ C_{BR} } 
\rightarrow 
\FlaThreeByThreeTL{ C_{00} }  { c_{01} }  { C_{02} }
                  {\undetermined}{ \gamma_{11}}     { c_{12}^{T} }
                  { \undetermined }  { \undetermined }  { C_{22} }
$
and
$
\FlaTwoByOne{ A_{T} }
            { A_{B} } 
\rightarrow
\FlaThreeByOneT{ A_{0} }
               { a_{1}^T }
               { A_{2} }
$
}
\renewcommand{\repartitionsizes}{
$ a_1^T $ is a row 
and $ \gamma_{11} $ is a scalar
}

% Step 5b
\renewcommand{\moveboundaries}{%
$ 
\FlaTwoByTwo{ C_{TL} }{ C_{TR}}
            { \undetermined }{ C_{BR} } 
\leftarrow
\FlaThreeByThreeBR{ C_{00} }  { c_{01} }  { C_{02} }
                  { \undetermined }{ \gamma_{11}}     { c_{12}^{T} }
                  { \undetermined }  { \undetermined }         { C_{22}}
$
and
$
\FlaTwoByOne{ A_{T} }
            { A_{B} } 
\leftarrow
\FlaThreeByOneB{ A_{0} }
               { a_{1}^T }
               { A_{2} }
$
}

% The command \FlaAlgorithm now generates the 
% algorithm without annotations

\begin{figure}[htbp]
\FlaAlgorithm
\caption{Unblocked algorithm for loop-invariant 1.}
\label{fig:alg:usyrk_unn}
\end{figure}

\subsubsection{Blocked Algorithms}

In order to cast the algorithm to be rich in matrix-matrix
multiplications, the repartitioning and redefinition of the
submatrices in Steps 5a and 5b in Fig.~\ref{fig:ws:usyrk_unn_blk}
expose multiple rows and/or columns at a time.  Block size $ b $ can
be chosen to be any size that does not exceed the number of rows in $
A_T $.  Again $ \QBefore $ is obtained by plugging the repartitioning
in Step 6 into $ \PInv $ while $ \QAfter $ is obtained by plugging the
redefinition of the quadrants in Step 7 into $ \PInv $.  The update in
Step 8 is now obtained by comparing the state in Steps 6 and 7.  The
bulk of the computation is now in the update $ C_{12} \becomes C_{12} +  A_1
A_2^T $ which, provided $ b > 1 $ involves a matrix-matrix
multiplication.  The update $ C_{11} \becomes C_{11} +  A_1 A_1^T $ is itself a
symmetric rank-k update and can be achieved by any
correct algorithm for implementing this operation.  In particular, an
unblocked algorithm can be used, or a blocked algorithm can be called
recursively.

% Redefine the commands that generate the annotated 
% algorithm for the blocked algorithm

% Define the blocksize that appears in step 5a
\renewcommand{\blocksize}{ b }
%

% Step 5a
\renewcommand{\repartitionings}{
$ 
\FlaTwoByTwo{ C_{TL} }{ C_{TR}}
            { \undetermined }{ C_{BR} }
\rightarrow
\FlaThreeByThreeTL{ C_{00} }  { C_{01} }  { C_{02} }
                  {\undetermined}{ C_{11}}     { C_{12} }
                  { \undetermined }  { \undetermined }  { C_{22} }
,
\FlaTwoByTwo{ \hat{C}_{TL} }{ \hat{C}_{TR}}
            { \undetermined }{ \hat{C}_{BR} }
\rightarrow
\FlaThreeByThreeTL{ \hat{C}_{00} }  { \hat{C}_{01} }  { \hat{C}_{02} }
                  {\undetermined}   { \hat{C}_{11} }  { \hat{C}_{12} }
                  { \undetermined } { \undetermined } { \hat{C}_{22} }
,
$ \\
and
$
\FlaTwoByOne{ A_{T} }
            { A_{B} }
\rightarrow
\FlaThreeByOneT{ A_{0} }
               { A_{1} }
               { A_{2} }  
$
}
%
\renewcommand{\repartitionsizes}{
$ C_{11} $ and $ \hat{C}_{11} $ are $ b \times b $
and $ A_{1} $ has $ b $ rows.
}

% Step 5b
\renewcommand{\moveboundaries}{%
$ 
\FlaTwoByTwo{ C_{TL} }{ C_{TR}}
            { \undetermined }{ C_{BR} }
\leftarrow
\FlaThreeByThreeBR{ C_{00} }        { C_{01} }        { C_{02} }
                  { \undetermined } { C_{11}}         { C_{12} }
                  { \undetermined } { \undetermined } { C_{22} }
,
\FlaTwoByTwo{ \hat{C}_{TL} }{ \hat{C}_{TR}}
            { \undetermined }{ \hat{C}_{BR} }
\leftarrow
\FlaThreeByThreeBR{ \hat{C}_{00} }  { \hat{C}_{01} }  { \hat{C}_{02} }
                  { \undetermined } { \hat{C}_{11} }  { \hat{C}_{12} }
                  { \undetermined } { \undetermined } { \hat{C}_{22} },
$ \\
and
$
\FlaTwoByOne{ A_{T} }
            { A_{B} }
\leftarrow
\FlaThreeByOneB{ A_{0} }
               { A_{1} }
               { A_{2} }
$
}

% Step 6
\renewcommand{\beforeupdate}{
\FlaThreeByThreeTL{ C_{00} }        { C_{01} }        { C_{02} }
                  {\undetermined}   { C_{11}}         { C_{12} }
                  { \undetermined } { \undetermined } { C_{22} }
=
\FlaTwoByTwo{ \FlaTwoByTwoSingleLine{ \hat{C}_{00} }   { \hat{C}_{01} }
                           { \undetermined} { \hat{C}_{11}   } }
            { \FlaTwoByOneSingleLine{ \hat{C}_{02} }{ \hat{C}_{12} } }
            { \undetermined }
            { A_2 A_2^{T} + \hat{C}_{22}}
\wedge \ldots
}

% Step 7
\renewcommand{\afterupdate}{
\FlaThreeByThreeBR{ C_{00} }        { C_{01} }        { C_{02} }
                  { \undetermined } { C_{11}}         { C_{12} }
                  { \undetermined } { \undetermined } { C_{22} }
=
\FlaTwoByTwo { \hat{C}_{00} }
             { \FlaOneByTwoSingleLine{ \hat{C}_{01} }{ \hat{C}_{02} } }
             { \undetermined }
             { \FlaTwoByTwoSingleLine{ \hat{C}_{11} + A_1 A_1^T }{ \hat{C}_{12} + A_1 A_2^T }
                                     { \undetermined }{ \hat{C}_{22} + A_2 A_2^T} }   
\wedge \ldots
}

% Step 8
\renewcommand{\update}{
\begin{minipage}[t]{4in}
\noindent
% \FlaStartCompute \\
$ C_{11} \becomes C_{11} + A_1 A_1^T $\\
$ C_{12} \becomes C_{12} + A_1 A_2^T $\\
% \FlaEndCompute \\
\end{minipage}
}

% Generate the annotated algorithm in Fig. 4.4
\begin{figure}[htbp]
\worksheet
\caption{Annotated blocked algorithm for loop-invariant 1.}
\label{fig:ws:usyrk_unn_blk}
\end{figure}
%

\subsubsection{Implementation}

Sequential implementations for the unblocked and blocked algorithms
for this loop-invariant using FLAME are given in
Figs.~\ref{fig:syrk_unn_upleft_unb}--\ref{fig:syrk_unn_upleft_blk}.

\begin{figure}[htbp]
\footnotesize
\begin{quote}
\listinginput{1}{syrk_unn/sequential/up-left/Syrk_unn_upleft_wrt_C_unb.c}
\end{quote}
\caption{Unblocked algorithm for loop-invariant 1 using FLAME.}
\label{fig:syrk_unn_upleft_unb}
\end{figure}

\begin{figure}[htbp]
\footnotesize
\index{const}{\tt FLA\un RECURSIVE}%
\begin{quote}
\listinginput{1}{syrk_unn/sequential/up-left/Syrk_unn_upleft_wrt_C_blk.c}
\end{quote}
\caption{Blocked algorithm for loop-invariant 1 using FLAME.
Recursion is optionally supported.}
\label{fig:syrk_unn_upleft_blk}
\end{figure}

\subsection{Loop-invariant 2}

We now examine the loop-invariant
\begin{equation}
\label{eqn:usyrk_unn:p3}
\PInv: 
\FlaTwoByTwo{ C_{TL} }        { C_{TR} } 
            { \undetermined } { C_{BR} }
=
\FlaTwoByTwo{ A_{T} A_{T}^T + \hat{C}_{TL} } { A_{T} A_{B}^T + \hat{C}_{TR} }
            { \undetermined }                { \hat{C}_{BR} }
\wedge
\ldots
\end{equation}
Comparing the loop-invariant in (\ref{eqn:usyrk_unn:p3}) with the
postcondition $ C = A A^T + \hat{C} $ we see again that {\em if} $C = C_{TL}$,
$ A = A_T $, and $ \hat{C} = \hat{C}_{TL} $ then the loop-invariant
implies the postcondition, i.e., that the desired result has been
computed.
%
%The initialization in Step 4 in Fig.~\ref{fig:ws:usyrk_unn:var2}
%is the same as that in Step 4 in Fig.~\ref{fig:ws:usyrk_unn}.
The fact that for this partitioning of $ C $, $ \hat{C} $, and $ A $,
\[
\FlaTwoByTwo{ C_{TL} }        { C_{TR} } 
            { \undetermined } { C_{BR} }
=
\FlaTwoByTwo{ A_{T} A_{T}^T + \hat{C}_{TL} } { A_{T} A_{B}^T + \hat{C}_{TR} }
            { \undetermined }                { \hat{C}_{BR} }
\]
and the precondition implies the other parts of
the loop-invariant, this initialization has
the desired properties.

Loop-guard $ G $ indicates that eventually $ C_{TL} $ should equal all
of $ C $, at which point $ G $ becomes {\em false} and the loop is
exited.  After the initialization, $ C_{TL} $ is $ 0 \times 0 $.
Thus, the repartitioning should be such that as the computation
proceeds, rows and columns are subtracted from $ C_{BR} $ and added to
$ C_{TL} $ while updating $ C_{TL} $ consistently with this.

%


\subsubsection{Unblocked algorithm}

The repartitionings in Step 5a in Fig.~\ref{fig:ws:usyrk_unn:var2}
result in the state
\begin{equation}
\QBefore: 
\FlaThreeByThreeBR{ C_{00} }        { c_{01} }         { C_{02} }
                  { \undetermined } { \gamma_{11}}     { c_{12}^T }
                  { \undetermined } { \undetermined }  { C_{22} }
=
\FlaTwoByTwo{ A_0 A_0^T + \hat{C}_{00}}
	    { A_0 
              \FlaTwoByOneSingleLine{ a_1^T }{ A_2 }^T + 
	      \FlaOneByTwoSingleLine{\hat{c}_{01}}
                                    {\hat{C}_{02} }
            }
            { \undetermined }  
	    { \FlaTwoByTwoSingleLine{ \hat{\gamma}_{11} }   { \hat{c}_{12}^{T} }
                                    { \undetermined}        { \hat{C}_{22}}}   
\wedge \ldots
\end{equation}
or 
\begin{equation}
\label{eqn:usyrk_unn:var2:before}
\QBefore: 
\FlaThreeByThreeBR{ C_{00} }        { c_{01} }         { C_{02} }
                  { \undetermined } { \gamma_{11}}     { c_{12}^T }
                  { \undetermined } { \undetermined }  { C_{22} }
=
\FlaTwoByTwo{ A_0 A_0^T + \hat{C}_{00}}
	    {  
	      \FlaOneByTwoSingleLine{ A_0 a_1 + \hat{c}_{01}}
                                    { A_0 A_2^T + \hat{C}_{02} }
            }
            { \undetermined }  
	    { \FlaTwoByTwoSingleLine{ \hat{\gamma}_{11} }   { \hat{c}_{12}^{T} }
                                    { \undetermined}        { \hat{C}_{22}}}   
\wedge \ldots
\end{equation}
before the update.
The redefinition in Step 5b in Fig.~\ref{fig:ws:usyrk_unn:var2}
means that the following state
must result from the update:
\[
\QAfter: 
\FlaThreeByThreeTL{ C_{00} }        { c_{01} }         { C_{02} }
                  { \undetermined } { \gamma_{11}}     { c_{12}^T }
                  { \undetermined } { \undetermined }  { C_{22} }
=
\FlaTwoByTwo{
              \FlaTwoByOneSingleLine {A_0} {a_1^T}
	      \FlaTwoByOneSingleLine {A_0} {a_1^T} ^T +
              \FlaTwoByTwoSingleLine {\hat{C}_{00}}        {\hat{c}_{01}}
 				     {\undetermined} {\hat{\gamma}_{11}}
            }
	    { 
              \FlaTwoByOneSingleLine{ A_0 }{ a_1^T } 
              A_2^T +
              \FlaTwoByOneSingleLine{\hat{C}_{02} }{ \hat{c}_{12}^T }  
            }
            { \undetermined }  
	    { \hat{C}_{22} }   
\wedge \ldots
\]
or
\begin{equation}
\label{eqn:usyrk_unn:var2:after}
\QAfter: 
\FlaThreeByThreeTL{ C_{00} }        { c_{01} }         { C_{02} }
                  { \undetermined } { \gamma_{11}}     { c_{12}^T }
                  { \undetermined } { \undetermined }  { C_{22} }
=
\FlaTwoByTwo{
              \FlaTwoByTwoSingleLine { A_0 A_0^T + \hat{C}_{00}} { A_0 a_1 + \hat{c}_{01}}
 				     {\undetermined} { a_1^T a_1 + \hat{\gamma}_{11}}
            }
	    { 
              \FlaTwoByOneSingleLine{ A_0 A_2^T + \hat{C}_{02} }{ a_1^T A_2^T + \hat{c}_{12}^T }  
            }
            { \undetermined }  
	    { \hat{C}_{22} }   
\wedge \ldots
\end{equation}


Comparing 
(\ref{eqn:usyrk_unn:var2:before}) and
(\ref{eqn:usyrk_unn:var2:after})
we find that the update
\begin{eqnarray*}
& \gamma_{11} & \becomes a_1^T a_1 + \gamma_{11}  \\
& c_{12} & \becomes A_2 a_1 + c_{12}
\end{eqnarray*}
is required to change the state from $ \QBefore $
to $ \QAfter $.


Again one recognizes that $ \hat{C} $ is never
referenced and can be eliminated from the algorithm.

% Define the commands to generate the annotated
% algorithm in Fig. 4.7.

% Step 2
\renewcommand{\invariant}{
\FlaTwoByTwo{ C_{TL} }       { C_{TR} }
            { \undetermined} { C_{BR} }
=
\FlaTwoByTwo{ A_{T} A_{T}^T + \hat{C}_{TL} } { A_{T} A_{B}^T + \hat{C}_{TR} }
            { \undetermined }                { \hat{C}_{BR} }
\wedge
\ldots
}

% Step 3
\renewcommand{\guard}{ \neg \SameSize( C, C_{TL} ) }

% Step 4
\renewcommand{\partitionings}{
$ 
C \rightarrow \FlaTwoByTwo{ C_{TL} }{ C_{TR} }
                          { C_{TR}^{T} }{ C_{BR} }
$
and
$
A \rightarrow \FlaTwoByOne{ A_{T} }
                          { A_{B} }
$
}
\renewcommand{\partitionsizes}{
$ C_{TL} $ and $ \hat{ C }_{TL} $ are $ 0 \times 0  $ 
and $ A_T $ has $ 0 $ rows.
}

% Step 5a
\renewcommand{\repartitionings}{
$
\FlaTwoByTwo{ C_{TL} }{ C_{TR}}
            { \undetermined }{ C_{BR} }
\rightarrow
\FlaThreeByThreeBR{ C_{00} }  { c_{01} }  { C_{02} }
                  {\undetermined}{ \gamma_{11}}     { c_{12}^{T} }
                  { \undetermined }  { \undetermined }  { C_{22} }
,
\FlaTwoByTwo{ \hat{C}_{TL} }{ \hat{C}_{TR}}
            { \undetermined }{ \hat{C}_{BR} }
\rightarrow
\FlaThreeByThreeBR{ \hat{C}_{00} }  { \hat{c}_{01} }  { \hat{C}_{02} }
                  {\undetermined}{ \hat{\gamma}_{11}}     { \hat{c}_{12}^{T} }
                  { \undetermined }  { \undetermined }  { \hat{C}_{22} }
,
$ \\
and
$
\FlaTwoByOne{ A_{T} }
            { A_{B} }
\rightarrow
\FlaThreeByOneB{ A_{0} }
               { a_{1}^T }
               { A_{2} }
$
}
\renewcommand{\repartitionsizes}{
$ \gamma_{11} $ and $ \hat{\gamma}_{11} $ are scalars 
and $ a_1^T $ is a row.
}

% Step 5b
\renewcommand{\moveboundaries}{%
$ 
\FlaTwoByTwo{ C_{TL} }{ C_{TR}}
            { \undetermined }{ C_{BR} }
\leftarrow
\FlaThreeByThreeTL{ C_{00} }  { c_{01} }  { C_{02} }
                  { \undetermined }{ \gamma_{11}}     { c_{12}^{T} }
                  { \undetermined }  { \undetermined }         { C_{22}}
,   
\FlaTwoByTwo{ \hat{C}_{TL} }{ \hat{C}_{TR}}
            { \undetermined }{ \hat{C}_{BR} }
\leftarrow
\FlaThreeByThreeTL{ \hat{C}_{00} }  { \hat{c}_{01} }  { \hat{C}_{02} }
                  { \undetermined }{ \hat{\gamma}_{11}}     { \hat{c}_{12}^{T}}
                  { \undetermined }  { \undetermined }      { \hat{C}_{22}},
$ \\
and
$
\FlaTwoByOne{ A_{T} }
            { A_{B} }
\leftarrow
\FlaThreeByOneT{ A_{0} }  
               { a_{1}^T }
               { A_{2} }
$
}

% Step 6
\renewcommand{\beforeupdate}{
\FlaThreeByThreeBR{ C_{00} }        { c_{01} }         { C_{02} }
                  { \undetermined } { \gamma_{11}}     { c_{12}^T }
                  { \undetermined } { \undetermined }  { C_{22} }
=
\FlaTwoByTwo{ A_0 A_0^T + \hat{C}_{00}}
            { A_0
              \FlaTwoByOneSingleLine{ a_1^T }{ A_2 }^T +
              \FlaOneByTwoSingleLine{\hat{c}_{01}}
                                    {\hat{C}_{02} }
            }
            { \undetermined }
            { \FlaTwoByTwoSingleLine{ \hat{\gamma}_{11} }   { \hat{c}_{12}^{T} }
                                    { \undetermined}        { \hat{C}_{22}}}
\wedge \ldots
}

% Step 7
\renewcommand{\afterupdate}{
\FlaThreeByThreeTL{ C_{00} }        { c_{01} }         { C_{02} }
                  { \undetermined } { \gamma_{11}}     { c_{12}^T }
                  { \undetermined } { \undetermined }  { C_{22} }
=
\FlaTwoByTwo{
              \FlaTwoByTwoSingleLine { A_0 A_0^T + \hat{C}_{00}} { A_0 a_1 + \hat{c}_{01}}
                                     {\undetermined} { a_1^T a_1 + \hat{\gamma}_{11}}
            }
            {
              \FlaTwoByOneSingleLine{ A_0 A_2^T + \hat{C}_{02} }{ a_1^T A_2^T + \hat{c}_{12}^T }
            }
            { \undetermined }
            { \hat{C}_{22} }
\wedge \ldots
}

% Step 8
\renewcommand{\update}{
\begin{minipage}[t]{4in}
\noindent
% \FlaStartCompute \\
$ \gamma_{11} \becomes a_1^T a_1 + \gamma_{11} $\\
$ c_{12} \becomes A_2 a_1 + c_{12} $\\
% \FlaEndCompute \\
\end{minipage}
}

% Generate figure 4.7
\begin{figure}[htbp]
\worksheet
\caption{Annotated unblocked algorithm for loop-invariant 2.}
\label{fig:ws:usyrk_unn:var2}
\end{figure}

\subsubsection{Blocked Algorithms}

Again, the algorithm can be cast to be rich in
matrix-matrix multiplications by marching through
the matrices multiple rows and/or columns at a time.
The resulting algorithm is given in 
Fig.~\ref{fig:ws:usyrk_unn:var2:blk}.

% Redefine commands to generate Fig. 4.8

% Step 5a
\renewcommand{\repartitionings}{
$ 
\FlaTwoByTwo{ C_{TL} }{ C_{TR}}
            { \undetermined }{ C_{BR} }
\rightarrow
\FlaThreeByThreeBR{ C_{00} }  { C_{01} }  { C_{02} }
                  {\undetermined}{ C_{11}}     { C_{12} }
                  { \undetermined }  { \undetermined }  { C_{22} }
,
\FlaTwoByTwo{ \hat{C}_{TL} }{ \hat{C}_{TR}}
            { \undetermined }{ \hat{C}_{BR} }
\rightarrow
\FlaThreeByThreeBR{ \hat{C}_{00} }  { \hat{C}_{01} }  { \hat{C}_{02} }
                  {\undetermined}{ \hat{C}_{11}}     { \hat{C}_{12} }
                  { \undetermined }  { \undetermined }  { \hat{C}_{22} }
,
$ \\
and
$
\FlaTwoByOne{ A_{T} }
            { A_{B} }
\rightarrow
\FlaThreeByOneB{ A_{0} }
               { A_{1} }
               { A_{2} }
$
}
%
\renewcommand{\repartitionsizes}{
$ C_{11} $ and $ \hat{C}_{11} $ are $ b \times b $ 
and $ A_1 $ has $ b $ rows.
}

% Step 5b
\renewcommand{\moveboundaries}{%
$ 
\FlaTwoByTwo{ C_{TL} }{ C_{TR}}
            { \undetermined }{ C_{BR} }
\leftarrow
\FlaThreeByThreeTL{ C_{00} }        { C_{01} }        { C_{02} }
                  { \undetermined } { C_{11}}         { C_{12} }
                  { \undetermined } { \undetermined } { C_{22}}
,   
\FlaTwoByTwo{ \hat{C}_{TL} }{ \hat{C}_{TR}}
            { \undetermined }{ \hat{C}_{BR} }
\leftarrow
\FlaThreeByThreeTL{ \hat{C}_{00} }  { \hat{C}_{01} }  { \hat{C}_{02} }
                  { \undetermined } { \hat{C}_{11}}   { \hat{C}_{12}}
                  { \undetermined } { \undetermined } { \hat{C}_{22}},
$ \\
and
$
\FlaTwoByOne{ A_{T} }
            { A_{B} }
\leftarrow
\FlaThreeByOneT{ A_{0} }  
               { A_{1} }
               { A_{2} }
$
}

% Step 6
\renewcommand{\beforeupdate}{
\FlaThreeByThreeBR{ C_{00} }        { C_{01} }         { C_{02} }
                  { \undetermined } { C_{11}}          { C_{12} }
                  { \undetermined } { \undetermined }  { C_{22} }   
=
\FlaTwoByTwo{ A_0 A_0^T + \hat{C}_{00}}
            { A_0
              \FlaTwoByOneSingleLine{ A_1 }{ A_2 }^T +
              \FlaOneByTwoSingleLine{\hat{C}_{01}}
                                    {\hat{C}_{02} }
            }
            { \undetermined }
            { \FlaTwoByTwoSingleLine{ \hat{C}_{11} }   { \hat{C}_{12} }
                                    { \undetermined}   { \hat{C}_{22}}}
\wedge \ldots
}

% Step 7
\renewcommand{\afterupdate}{
\FlaThreeByThreeTL{ C_{00} }        { C_{01} }         { C_{02} }
                  { \undetermined } { C_{11}}          { C_{12} }
                  { \undetermined } { \undetermined }  { C_{22} }
=
\FlaTwoByTwo{
              \FlaTwoByTwoSingleLine { A_0 A_0^T + \hat{C}_{00}} { A_0 A_1^T + \hat{C}_{01}}
                                     {\undetermined} { A_1 A_1^T + \hat{C}_{11}}
            }
            {
              \FlaTwoByOneSingleLine{ A_0 A_2^T + \hat{C}_{02} }{ A_1 A_2 + \hat{C}_{12} }
            }
            { \undetermined }
            { \hat{C}_{22} }
\wedge \ldots
}

% Step 8
\renewcommand{\update}{
\begin{minipage}[t]{4in}
\noindent
% \FlaStartCompute \\
$ C_{11} \becomes A_1 A_1^T + C_{11} $\\
$ C_{12} \becomes A_1 A_2^T + C_{12} $\\
% \FlaEndCompute \\
\end{minipage}
}

% Generate Fig. 4.8
\begin{figure}[htbp]
\worksheet
\caption{Annotated blocked algorithm for loop-invariant 2.}
\label{fig:ws:usyrk_unn:var2:blk}
\end{figure}
%

\subsubsection{Implementation}
 
Sequential implementations for the unblocked and blocked algorithms
for this loop-invariant using FLAME are similar to the implementations of loop-invariant 1.

\section{Performance}

\begin{quote}
{\bf Note:  The experiments reported in the various chapters
were not always performed on exactly the platform described in 
Section~\ref{sec:intro:performance}.  Thus, it is the
relative performance on the different algorithms that
is of significance, rather than the absolute performance.
}
\end{quote}

In this section, we discuss the performance attained by the different
variants for computing $ C \leftarrow A A^T +  \hat{C} $.  In Fig.
~\ref{fig:usyrk_unn:upleft-downright:ATLAS}, we compare the performance
attained by five different implementations:
\begin{itemize}
\item{\sc Reference}
{\tt DTRMM} as implemented as part of ATLAS,
\item{\sc FLAME}
{\tt FLA\_Syrk} implemented as part of FLAME
as of this writing,
\item{\sc Unblocked}
the unblocked implementation in Fig.~\ref{fig:syrk_unn_upleft_unb},
\item{\sc Blocked}
the blocked implementation in Fig.~\ref{fig:syrk_unn_upleft_blk}
called with 
{\tt rec == 
\index{const}{\tt FLA\un NON\un RECURSIVE}%
FLA\_NON\_RECURSIVE}, and
\item{\sc Recursive}
the blocked implementation in Fig.~\ref{fig:syrk_unn_upleft_blk}
called with 
{\tt rec == 
\index{const}{\tt FLA\un RECURSIVE}%
FLA\_RECURSIVE}.
\end{itemize}
In Fig.~\ref{fig:usyrk_unn:upleft-downright:ATLAS}, all FLAME
implementations are based on the matrix-matrix multiplication ({\tt
DGEMM}) provided by ATLAS.  Notice that for the platform on which we
performed the experiments, multiples of 40 are good block sizes when
using the ATLAS matrix-matrix multiplication kernel.  We note that the
unblocked algorithms perform badly, since they are rich in
matrix-vector operations that do not benefit much from cache memories.
The blocked algorithms performed better for relatively small block
sizes ({\tt nb\_alg=40}) than for larger block sizes ({\tt
nb\_alg=80}).  The reason for this is that too much of the
computation is performed in the subprogram for which the unblocked
algorithm is used.  The recursive implementations benefit from larger
block sizes, since they overcome this problem that plagues the blocked
algorithm.  In addition they benefit from the fact that the {\tt
DGEMM} kernel performs better for larger blocks.

Since in the recursive algorithms at each level a different variant
could be called to solve the subproblem, it may be possible to improve
performance further by combining different variants.  We have not yet
studied this.

\begin{figure}[htbp]
\begin{center}
\begin{tabular}{c | c}
Variant 1 & Variant 2 \\ \hline
& \\
\psfig{figure=syrk_unn/graphs/syrk_unn_upleft_wrt_C_40.eps,width=3.0in,height=3.0in} &
\psfig{figure=syrk_unn/graphs/syrk_unn_downright_wrt_C_40.eps,width=3.0in,height=3.0in}
\\ \hline
& \\
\psfig{figure=syrk_unn/graphs/syrk_unn_upleft_wrt_C_80.eps,width=3.0in,height=3.0in} &
\psfig{figure=syrk_unn/graphs/syrk_unn_downright_wrt_C_80.eps,width=3.0in,height=3.0in}
\end{tabular}
\end{center}
\caption{Performance of the variants  of
symmetric rank-k update
algorithms for a block size of $ 40 $ (top) and
$ 80 $ (bottom).
For these experiments, the ATLAS matrix-matrix multiplication
kernel was used.}
\label{fig:usyrk_unn:upleft-downright:ATLAS}
\end{figure}
\index{op}{symmetric rank-k update!$ C \leftarrow A A^T + \hat{C} $|( }%



% The following commands help in generating the index
\index{index}{triangular symmetric rank-k update}%
\index{op}{triangular symmetric rank-k update!$ C \leftarrow A^T A + \hat{C} $|( }%

% Name of the chapter
\chapter{Triangular Symmetric Rank-k Update \\
$ C \leftarrow A^T A + C $ \\
$ C $ symmetric, stored in upper triangle
}
\label{chapter:sytrrk_uln}

% Authors
\ChapterAuthor{
\index{author}{Bradford, Jason}
Jason Bradford
\\[0.1in]
\index{author}{Wiggins, John}
John Wiggins
}

% Add the names of the authors to the table of contents
\addtocontents{toc}{ {by {\bf Jason Bradford, John Wiggins}}}

% Intro
In this chapter, we discuss the implementation of the triangular symmetric rank-k update
\[
C \leftarrow A^T A + C
\]
where $ C $ is an $ n \times n $ symmetric matrix and $ A $ is an $ n \times n $ lower
triangular matrix. We will overwrite $ C $ with the result without requiring a work array.
We start by deriving a couple of different sequential algorithms. Subsequently, we show how
to code the algorithms using FLAME.

The variables for the triangular symmetric rank-k update can be described by the precondition
\[
% precondition
P_{pre} : C = \hat{C} \wedge \RowDim(C) = \ColDim(C) \wedge \RowDim(A) = \ColDim(A)
\wedge \RowDim(C) = \RowDim(A) \wedge Symm(C) \wedge LowTr(A),
\]
where $ \hat{C} $ indicates the original contents of $ C $. Here the predicate $ Symm(C) $
is true iff $ C $ is symmetric, and the predicate $ LowTr(A) $ is true iff $ A $ is lower
triangular. The functions $ \RowDim(C) $, $ \RowDim(A) $  and $ \ColDim(C) $,
$ \ColDim(A) $ return the row and column dimension of $ C $ and $ A $, respectively. The
operation to be performed, $ C \leftarrow A^T A + C $, translates to the postcondition
$ P_{post} : C \leftarrow A^T A + \hat{C} $.

% first section
\[
\]
Let us start by partitioning matrix C.  Since it is symmetric, and stored in the upper
triangle we partition it like
\[
\FlaTwoByTwo{ C_{TL} }        { C_{TR} }
            { \undetermined } { C_{BR} }
\]
where $ C_{TL} $  and $ C_{BR} $ are both submatrices on the diagonal and are symmetric.
Notice that * indicates that that part of the matrix is not stored.
Substituting the partitioning of C into the postcondition yields
\[
\FlaTwoByTwo{ C_{TL} }        { C_{TR} }
            { \undetermined } { C_{BR} }
=
( \mbox{some partitioning of }{A^T} )
( \mbox{some partitioning of }{A} )
+
\FlaTwoByTwo{ \hat{C}_{TL} }  { \hat{C}_{TR} }
            { \undetermined } { \hat{C}_{BR} }
\]
This suggests that $ A $ should be partitioned
into quadrants.  Let us consider
the case where $ A $ is partitioned as described above.  Then
\[
\FlaTwoByTwo{ C_{TL} }       { C_{TR} }
            { \undetermined} { C_{BR} }
=
\FlaTwoByTwo { A_{TL}^T }      { A_{BL}^T }
             { 0 }      { A_{BR}^T }
\FlaTwoByTwo { A_{TL} }      { 0 }
             { A_{BL} }      { A_{BR} }
+
\FlaTwoByTwo{ \hat{C}_{TL} }  { \hat{C}_{TR} }
            { \undetermined } { \hat{C}_{BR} }
\]
In order to be able to multiply and add the matrices on the right out and to
be able to then set the submatrices on the left equal to the result on
the right we find that the following must hold:
\[
\ColDim( C_{TL} ) = \RowDim( A_{TL} )
\wedge
\RowDim( C_{TL} ) = \RowDim( A_{TL} )
\wedge
\RowDim( C_{TL} ) = \ColDim( C_{TL} )
\]
Substituting the partitioned matrices into the postcondition, we find
\begin{eqnarray*}
 \FlaTwoByTwo{ C_{TL} }       { C_{TR} }
            { \undetermined} { C_{BR} }
& = &
\FlaTwoByTwo { A_{TL}^T }      { A_{BL}^T }
             { 0 }      { A_{BR}^T }
\FlaTwoByTwo { A_{TL} }      { 0 }
             { A_{BL} }      { A_{BR} }
+
\FlaTwoByTwo{ \hat{C}_{TL} }  { \hat{C}_{TR} }
            { \undetermined } { \hat{C}_{BR} }
\\ & = &
\FlaTwoByTwo { A_{TL}^T A_{TL} + A_{BL}^T A_{BL} + \hat{C}_{TL} }   { A_{BL}^T A_{BR} + \hat{C}_{TR}}
             {\undetermined}  {A_{BR}^T A_{BR} + \hat{C}_{BR}}
\end{eqnarray*}
which yields the equalities
\begin{equation}
\label{eqn:sytrrk_uln}
\begin{array}{r c l || r c l}
C_{TL} &=& A_{TL}^T A_{TL} + A_{BL}^T A_{BL} + \hat{C}_{TL} & C_{TR} &=& A_{BL}^T A_{BR} + \hat{C}_{TR} \\ \hline \hline
C_{BL} &=& \undetermined & C_{BR} &=&  A_{BR}^T A_{BR} + \hat{C}_{BR}


\end{array}
\end{equation}


From Eqn.~\ref{eqn:sytrrk_uln} we conclude that the operations to be performed are
$ A_{TL}^T A_{TL} + A_{BL}^T A_{BL} + \hat{C}_{TL} $,  $ A_{BL}^T A_{BR} + \hat{C}_{TR}$,
and $ A_{BR}^T A_{BR} + \hat{C}_{BR} $. At an intermediate stage, each of these either has or has not already
been computed, leading to the $ 2^4 = 16 $ possible loop-invariants tabulated in Fig.~\ref{fig:sytrrk_uln_invariants}

% display the table of preconditions
\begin{figure}
\begin{center}
\footnotesize
{
\setlength{\tabcolsep}{4pt}
\begin{tabular}{| c | p{2.5in} | c | } \hline
$ P_X: $

$ \FlaTwoByTwo{ C_{TL} } { C_{TR} }
              { \undetermined } { C_{BR} } =  $ & Comment &
\footnotesize Feasible? \\ \hline \hline
\footnotesize
$
\FlaTwoByTwo{ \hat{C}_{TL} }  { \hat{C}_{TR} }
            { \undetermined } { \hat{C}_{BR} }
$
&
&
NO
\\ \hline
%
%
\footnotesize
$
\FlaTwoByTwo{ \hat{C}_{TL} } { \hat{C}_{TR} }
            { \undetermined } { A_{BR}^T A_{BR} + \hat{C}_{BR} }
$
&
Loop Invariant 2.
&
YES
\\ \hline
%
%
\footnotesize
$
\FlaTwoByTwo{ \hat{C}_{TL} }  { A_{BL}^T A_{BR} + \hat{C}_{TR} }
            { \undetermined } { \hat{C}_{BR} }
$
&
&
NO
\\ \hline
%
%
\footnotesize
$
\FlaTwoByTwo{ \hat{C}_{TL} }  { A_{BL}^T A_{BR} + \hat{C}_{TR} }
            { \undetermined } { A_{BR}^T A_{BR} + \hat{C}_{BR} }
$
&
&
YES
\\ \hline
%
%
\footnotesize
$
\FlaTwoByTwo{ A_{BL}^T A_{BL} + \hat{C}_{TL} } { \hat{C}_{TR} }
            { \undetermined }                  { \hat{C}_{BR} }
$
&
&
NO
\\ \hline
%
%
\footnotesize
$
\FlaTwoByTwo{ A_{BL}^T A_{BL} + \hat{C}_{TL} } { \hat{C}_{TR} }
            { \undetermined }                  { A_{BR}^T A_{BR} + \hat{C}_{BR} }
$
&
&
NO
\\ \hline
%
%
\footnotesize
$
\FlaTwoByTwo{ A_{BL}^T A_{BL} + \hat{C}_{TL} } { A_{BL}^T A_{BR} + \hat{C}_{TR} }
            { \undetermined }                  { \hat{C}_{BR} }
$
&
&
NO
\\ \hline
%
%
\footnotesize
$
\FlaTwoByTwo{ A_{BL}^T A_{BL} + \hat{C}_{TL} } { A_{BL}^T A_{BR} + \hat{C}_{TR} }
            { \undetermined }                  { A_{BR}^T A_{BR} + \hat{C}_{BR} }
$
&
&
NO
\\ \hline
\footnotesize
$
\FlaTwoByTwo{ A_{TL}^T A_{TL} + \hat{C}_{TL} }  { \hat{C}_{TR} }
            { \undetermined } { \hat{C}_{BR} }
$
&
&
NO
\\ \hline
%
%
\footnotesize
$
\FlaTwoByTwo{ A_{TL}^T A_{TL} + \hat{C}_{TL} } { \hat{C}_{TR} }
            { \undetermined } { A_{BR}^T A_{BR} + \hat{C}_{BR} }
$
&
&
NO
\\ \hline
%
%
\footnotesize
$
\FlaTwoByTwo{ A_{TL}^T A_{TL} + \hat{C}_{TL} }  { A_{BL}^T A_{BR} + \hat{C}_{TR} }
            { \undetermined } { \hat{C}_{BR} }
$
&
&
NO
\\ \hline
%
%
\footnotesize
$
\FlaTwoByTwo{ A_{TL}^T A_{TL} + \hat{C}_{TL} }  { A_{BL}^T A_{BR} + \hat{C}_{TR} }
            { \undetermined } { A_{BR}^T A_{BR} + \hat{C}_{BR} }
$
&
&
NO
\\ \hline
%
%
\footnotesize
$
\FlaTwoByTwo{ A_{TL}^T A_{TL} + A_{BL}^T A_{BL} + \hat{C}_{TL} } { \hat{C}_{TR} }
            { \undetermined }                  { \hat{C}_{BR} }
$
&
Loop Invariant 1.
&
YES
\\ \hline
%
%
\footnotesize
$
\FlaTwoByTwo{ A_{TL}^T A_{TL} + A_{BL}^T A_{BL} + \hat{C}_{TL} } { \hat{C}_{TR} }
            { \undetermined }                  { A_{BR}^T A_{BR} + \hat{C}_{BR} }
$
&
&
NO
\\ \hline
%
%
\footnotesize
$
\FlaTwoByTwo{ A_{TL}^T A_{TL} + A_{BL}^T A_{BL} + \hat{C}_{TL} } { A_{BL}^T A_{BR} + \hat{C}_{TR} }
            { \undetermined }                  { \hat{C}_{BR} }
$
&
&
YES
\\ \hline
%
%
\footnotesize
$
\FlaTwoByTwo{ A_{TL}^T A_{TL} + A_{BL}^T A_{BL} + \hat{C}_{TL} } { A_{BL}^T A_{BR} + \hat{C}_{TR} }
            { \undetermined }                  { A_{BR}^T A_{BR} + \hat{C}_{BR} }
$
&
&
NO
\\ \hline
\end{tabular}
}
\end{center}
\caption{Possible loop-invariants when partitioning
$ C $ into quadrants.
Here $ P_X $ is the most prominent part of the loop-invariant
$ \PInv $.}
\label{fig:sytrrk_uln_invariants}
\end{figure}



\section{Loop-invariant 1}
We will first examine the third feasible loop-invariant from Fig.~\ref{fig:sytrrk_uln_invariants}.
\begin{equation}
\label{eqn:sytrrk_uln_inv1}
\PInv:
\FlaTwoByTwo{ C_{TL} }       { C_{TR} }
            { \undetermined} { C_{BR} }
=
\FlaTwoByTwo{ A_{TL}^T A_{TL} + A_{BL}^T A_{BL} + \hat{C}_{TL} }  { \hat{C}_{TR} }
            { \undetermined } { \hat{C}_{BR} }
\wedge
\ldots
\end{equation}


Comparing the loop invariant in Eqn.~\ref{eqn:sytrrk_uln_inv1} with the postcondition $ C =  A^T A + \hat{C}$
we see that as we progress through the matrix, we are only doing work within $ C_{TL} $, and after each iteration
we gradually add more contents to $ C_{TL} $ until all of C is resident in $C_{TL} $. Thus, we must
choose a loop-guard G such that its negation $\neg{G} $, implies that the dimensions of these matrices
match appropriately and therefore that $ ( \PInv \wedge \neg{G} ) \implies \PPost $. The loop-guard
$ G = \neg \SameSize(C_{TL}, C)  $ meets this condition.

Loop-guard G indicates that eventually $C_{TL}$ should equal all of C, at which point G becomes false
and the loop is exited. After the initialization, $C_{TL}$ is $ 0 \times 0 $.  Thus, the repartitioning should be
such that as the computation proceeds, rows and columns are subtracted from $C_{BR}$ and added to $C_{TL}$
while updating $C_{TL}$ consistently with this. This is illustrated by the shifting of the double-lines.

\subsection{Blocked algorithm}

The repartitioning of the variables and the loop-invariant $\PInv $ dictates $\QBefore $

%\begin{eqnarray*}
{
\normalsize
$ \\
\FlaTwoByTwo{ C_{TL} } { C_{TR} }
            { * } { C_{BR} }
\rightarrow
\FlaThreeByThreeBR{ C_{00} }{ C_{01} }{ C_{02} }
              { * }{ C_{11} }{ C_{12} }
              { * }{ * }{ C_{22} }
$,
$
\FlaTwoByTwo{ \hat{C}_{TL} } { \hat{C}_{TR} }
            { * } { \hat{C}_{BR} }
\rightarrow
\FlaThreeByThreeBR{ \hat{C}_{00} }{ \hat{C}_{01} }{ \hat{C}_{02} }
              { * }{ \hat{C}_{11} }{ \hat{C}_{12} }
              { * }{ * }{ \hat{C}_{22} }
$, \\ \\ and
$
\FlaTwoByTwo{ A_{TL} } { 0 }
            { A_{BL} } { A_{BR} }
\rightarrow
\FlaThreeByThreeBR{ A_{00} }{ 0 }{ 0 }
              { A_{10} }{ A_{11} }{ 0 }
              { A_{20} }{ A_{21} }{ A_{22} }
$ \\
}
%\end{eqnarray*}

After updating $ C_{TL} $, we would then continue with

{
\normalsize
$ \\
\FlaTwoByTwo{ C_{TL} } { C_{TR} }
            { * } { C_{BR} }
\leftarrow
\FlaThreeByThreeTL{ C_{00} }{ C_{01} }{ C_{02} }
              { * }{ C_{11} }{ C_{12} }
              { * }{ * }{ C_{22} }
$, \\ \\
$
\FlaTwoByTwo{ \hat{C}_{TL} } { \hat{C}_{TR} }
            { * } { \hat{C}_{BR} }
\leftarrow
\FlaThreeByThreeTL{ \hat{C}_{00} }{ \hat{C}_{01} }{ \hat{C}_{02} }
              { * }{ \hat{C}_{11} }{ \hat{C}_{12} }
              { * }{ * }{ \hat{C}_{22} }
$, and \\ \\
$
\FlaTwoByTwo{ A_{TL} } { 0 }
            { A_{BL} } { A_{BR} }
\leftarrow
\FlaThreeByThreeTL{ A_{00} }{ 0 }{ 0 }
              { A_{10} }{ A_{11} }{ 0 }
              { A_{20} }{ A_{21} }{ A_{22} }
$ \\
}

If we combine the initial partitioning with the loop invariant, we get the intermediate state

{
$  \\
\QBefore:
\FlaThreeByThreeBR{ C_{00} }{ C_{01} }{ C_{02} }
              { * }{ C_{11} }{ C_{12} }
              { * }{ * }{ C_{22} }
=
\FlaThreeByThreeBR{ A_{00}^T A_{00} + A_{10}^T A_{10} + A_{20}^T A_{20} + \hat{C}_{00} }{ \hat{C}_{01} }{ \hat{C}_{02} }
              { * }{ \hat{C}_{11} }{ \hat{C}_{12} }
              { * }{ * }{ \hat{C}_{22} }
$
}
\\
\\

If we combine the post-update partitioning with the loop invariant, we get the post-update state

$  \\
\QAfter:
\FlaThreeByThreeTL{ C_{00} }{ C_{01} }{ C_{02} }
              { * }{ C_{11} }{ C_{12} }
              { * }{ * }{ C_{22} }
=
\FlaThreeByThreeTL{ A_{00}^T A_{00} + A_{10}^T A_{10} + A_{20}^T A_{20} + \hat{C}_{00} }
                  { A_{10}^T A_{11} + A_{20}^T A_{21} + \hat{C}_{01} }{ \hat{C}_{02} }
              { * }{ A_{11}^T A_{11} + A_{21}^T A_{21} + \hat{C}_{11} }{ \hat{C}_{12} }
              { * }{ * }{ \hat{C}_{22} }
$

Comparing $\QBefore$ with $\QAfter$ we find that the updates
\\
\\
$\begin{array}{l}

C_{01} \becomes A_{10}^T A_{11} + A_{20}^T A_{21} + \hat{C}_{01} \\
C_{11} \becomes A_{11}^T A_{11} + A_{21}^T A_{21} + \hat{C}_{11}

\end{array} $
\\
\\
are required to change the state from $\QBefore$ to $\QAfter$, which yields the final algorithm.

In order to cast the algorithm to be rich in matrix-matrix multiplications, the repartitioning and
redefinition of the submatrices in Steps 5a and 5b of Fig.~\ref{fig:sytrrk_uln_downright_blk_ann}
expose multiple rows and/or columns at a time. Block size b can be chosen to be any size that
does not exceed the number of rows in $C_{BR}$. $Q_{before}$ is obtained by plugging the repartitioning
in Step 6 of Fig.~\ref{fig:sytrrk_uln_downright_blk_ann} into $P_{inv}$ while $Q_{after}$ is
obtained by plugging the redefinition of the quadrants in Step 7 of Fig.~\ref{fig:sytrrk_uln_downright_blk_ann}
into $P_{inv}$. The update in Step 8 of Fig.~\ref{fig:sytrrk_uln_downright_blk_ann}
is now obtained by comparing the state in Steps 6 and 7 of Fig.~\ref{fig:sytrrk_uln_downright_blk_ann}.
The bulk of the computation is now in the updates given above.

% display the blocked algorithm


%
% The Algorithm
%

% Initial Partitioning
\renewcommand{\partitionings}{
$
C \rightarrow
\FlaTwoByTwo{ C_{TL} } { C_{TR} }
            { * } { C_{BR} }
$,
$
\hat{C} \rightarrow
\FlaTwoByTwo{ \hat{C}_{TL} } { \hat{C}_{TR} }
            { * } { \hat{C}_{BR} }
$ and
$
A \rightarrow
\FlaTwoByTwo{ A_{TL} } { 0 }
            { A_{BL} } { A_{BR} }
$
}


\renewcommand{\partitionsizes}{$ C_{TL} $, $ \hat{C}_{TL} $, and $ A_{TL} $ is $ 0 \times 0 $ }

% loop guard
\renewcommand{\guard}{ \neg \SameSize( C_{TL}, C ) }


% Repartitioning at start of loop
\renewcommand{\repartitionings}{
\normalsize
$
\FlaTwoByTwo{ C_{TL} } { C_{TR} }
            { * } { C_{BR} }
\rightarrow
\FlaThreeByThreeBR{ C_{00} }{ C_{01} }{ C_{02} }
              { * }{ C_{11} }{ C_{12} }
              { * }{ * }{ C_{22} }
$, \\ \\
$
\FlaTwoByTwo{ \hat{C}_{TL} } { \hat{C}_{TR} }
            { * } { \hat{C}_{BR} }
\rightarrow
\FlaThreeByThreeBR{ \hat{C}_{00} }{ \hat{C}_{01} }{ \hat{C}_{02} }
              { * }{ \hat{C}_{11} }{ \hat{C}_{12} }
              { * }{ * }{ \hat{C}_{22} }
$, and \\ \\
$
\FlaTwoByTwo{ A_{TL} } { 0 }
            { A_{BL} } { A_{BR} }
\rightarrow
\FlaThreeByThreeBR{ A_{00} }{ 0 }{ 0 }
              { A_{10} }{ A_{11} }{ 0 }
              { A_{20} }{ A_{21} }{ A_{22} }
$
}

% the update
\renewcommand{\update}{
$
\begin{array}{l}
C_{01} \becomes A_{10}^T A_{11} + A_{20}^T A_{21} + \hat{C}_{01} \\
C_{11} \becomes A_{11}^T A_{11} + A_{21}^T A_{21} + \hat{C}_{11}
\end{array}
$
}

\renewcommand{\repartitionsizes}{
$ C_{11} $ , $ \hat{C}_{11} $, and $ A_{11} $ are $ b \times b $ matrices}

% Moving double lines at end of loop
\renewcommand{\moveboundaries}{
\normalsize
$
\FlaTwoByTwo{ C_{TL} } { C_{TR} }
            { * } { C_{BR} }
\leftarrow
\FlaThreeByThreeTL{ C_{00} }{ C_{01} }{ C_{02} }
              { * }{ C_{11} }{ C_{12} }
              { * }{ * }{ C_{22} }
$, \\ \\
$
\FlaTwoByTwo{ \hat{C}_{TL} } { \hat{C}_{TR} }
            { * } { \hat{C}_{BR} }
\leftarrow
\FlaThreeByThreeTL{ \hat{C}_{00} }{ \hat{C}_{01} }{ \hat{C}_{02} }
              { * }{ \hat{C}_{11} }{ \hat{C}_{12} }
              { * }{ * }{ \hat{C}_{22} }
$, and \\ \\
$
\FlaTwoByTwo{ A_{TL} } { 0 }
            { A_{BL} } { A_{BR} }
\leftarrow
\FlaThreeByThreeTL{ A_{00} }{ 0 }{ 0 }
              { A_{10} }{ A_{11} }{ 0 }
              { A_{20} }{ A_{21} }{ A_{22} }
$
}

% output
\begin{figure}[tbp]
\begin{center}
\FlaAlgorithm
\end{center}
\caption{Blocked algorithm for $ C \becomes A^T A + \hat{C} $.}
\label{fig:sytrrk_uln_downright_blk}
\end{figure}


\subsection{Unblocked algorithm}
The unblocked algorithm for loop invariant 1 is similar to the blocked version, except that the repartitioning and
redefinition of the submatrices in Steps 5a and 5b of Fig.~\ref{fig:sytrrk_uln_downright_unblk_ann}
expose a single row or column at a time. This in turn has an effect on the partitioning and
repartitioning of the matrices, and thus a different update. The update in Step 8 of
Fig.~\ref{fig:sytrrk_uln_downright_unblk_ann} is now obtained by comparing the state in
Steps 6 and 7 of Fig.~\ref{fig:sytrrk_uln_downright_unblk_ann}. The bulk of the computation is now in the update
given below.
\\
{
$  \\
\QBefore:
\FlaThreeByThreeBR{ C_{00} }{ c_{01} }{ C_{02} }
              { * }{ \gamma_{11}}{ C_{12}^T }
              { * }{ * }{ C_{22} }
=
\FlaThreeByThreeBR{ A_{00}^T A_{00} + a_{10} a_{10}^T + A_{20}^T A_{20} + \hat{C}_{00} }{ \hat{c}_{01} }{ \hat{C}_{02} }
              { * }{ \hat{\gamma}_{11} }{ \hat{c}_{12}^T }
              { * }{ * }{ \hat{C}_{22} }
$
}
\\
\\
If we combine the post-update partitioning with the loop invariant, we get the post-update state %$ \Qafter $.
%The redefinition in Step 5b means that the following state must result from the update:


$  \\
\QAfter:
\FlaThreeByThreeTL{ C_{00} }{ c_{01} }{ C_{02} }
              { * }{ \gamma_{11} }{ c_{12}^T }
              { * }{ * }{ C_{22} }
=
\FlaThreeByThreeTL{  A_{00}^T A_{00} + a_{10} a_{10}^T + A_{20}^T A_{20} + \hat{C}_{00} }
                  { A_{20}^T a_{21} + \alpha_{11} a_{10} + \hat{c}_{01} }{ \hat{C}_{02} }
              { * }{ a_{21}^T a_{21} + \alpha_{11}^2 + \hat{\gamma}_{11} }{ \hat{c}_{12}^T }
              { * }{ * }{ \hat{C}_{22} }
$

Comparing $\QBefore$ with $\QAfter$ we find that the updates
\\
\\
$c_{01} \becomes A_{20}^T a_{21} + \alpha_{11} a_{10} + \hat{C}_{01}$ \\
$ \gamma_{11} \becomes a_{21}^T a_{21} + \alpha_{11} \alpha_{11} + \hat{\gamma}_{11} $
\\

In general, an unblocked algorithm can be used, or a blocked algorithm can call an unblocked algorithm as part
of its update.

% display the unblocked algorithm


%
% The Algorithm
%

% Initial Partitioning
\renewcommand{\partitionings}{
$
C \rightarrow
\FlaTwoByTwo{ C_{TL} } { C_{TR} }
            { * } { C_{BR} }
$,
$
\hat{C} \rightarrow
\FlaTwoByTwo{ \hat{C}_{TL} } { \hat{C}_{TR} }
            { * } { \hat{C}_{BR} }
$ and
$
A \rightarrow
\FlaTwoByTwo{ A_{TL} } { 0 }
            { A_{BL} } { A_{BR} }
$
}


\renewcommand{\partitionsizes}{$ C_{TL} $, $ \hat{C}_{TL} $, and $ A_{TL} $ is $ 0 \times 0 $ }

% loop guard
\renewcommand{\guard}{ \neg \SameSize( C_{TL}, C ) }


% Repartitioning at start of loop
\renewcommand{\repartitionings}{
\normalsize
$
\FlaTwoByTwo{ C_{TL} } { C_{TR} }
            { * } { C_{BR} }
\rightarrow
\FlaThreeByThreeBR{ C_{00} }{ c_{01} }{ C_{02} }
              { * }{ \gamma_{11} }{ c_{12}^T }
              { * }{ * }{ C_{22} }
$, \\ \\
$
\FlaTwoByTwo{ \hat{C}_{TL} } { \hat{C}_{TR} }
            { * } { \hat{C}_{BR} }
\rightarrow
\FlaThreeByThreeBR{ \hat{C}_{00} }{ \hat{c}_{01} }{ \hat{C}_{02} }
              { * }{ \hat{\gamma}_{11} }{ \hat{c}_{12}^T }
              { * }{ * }{ \hat{C}_{22} }
$, and \\ \\
$
\FlaTwoByTwo{ A_{TL} } { 0 }
            { A_{BL} } { A_{BR} }
\rightarrow
\FlaThreeByThreeBR{ A_{00} }{ 0 }{ 0 }
              { a_{10}^T }{ \alpha_{11} }{ 0 }
              { A_{20} }{ a_{21} }{ A_{22} }
$
}

% the update
\renewcommand{\update}{
$
\begin{array}{l}
c_{01} \becomes A_{20}^T a_{21} + \alpha_{11} a_{10} + \hat{c}_{01} \\
\gamma_{11} \becomes a_{21}^T a_{21} + \alpha_{11}^2 + \hat{\gamma}_{11}
\end{array}
$
}

\renewcommand{\repartitionsizes}{
$ \gamma_{11} $ , $ \hat{\gamma}_{11} $, and $ \alpha_{11} $ are scalars,
$ c_{01} $ , $ \hat{c}_{01} $, and $ a_{21} $ are column vectors, \\
and $ c_{12}^T $ , $ \hat{c}_{12}^T $, and $ a_{10}^T $ are row vectors. }

% Moving double lines at end of loop
\renewcommand{\moveboundaries}{
\normalsize
$
\FlaTwoByTwo{ C_{TL} } { C_{TR} }
            { * } { C_{BR} }
\leftarrow
\FlaThreeByThreeTL{ C_{00} }{ c_{01} }{ C_{02} }
              { * }{ \gamma_{11} }{ c_{12}^T }
              { * }{ * }{ C_{22} }
$, \\ \\
$
\FlaTwoByTwo{ \hat{C}_{TL} } { \hat{C}_{TR} }
            { * } { \hat{C}_{BR} }
\leftarrow
\FlaThreeByThreeTL{ \hat{C}_{00} }{ \hat{c}_{01} }{ \hat{C}_{02} }
              { * }{ \hat{\gamma}_{11} }{ \hat{c}_{12}^T }
              { * }{ * }{ \hat{C}_{22} }
$, and \\ \\
$
\FlaTwoByTwo{ A_{TL} } { 0 }
            { A_{BL} } { A_{BR} }
\leftarrow
\FlaThreeByThreeTL{ A_{00} }{ 0 }{ 0 }
              { a_{10}^T }{ \alpha_{11} }{ 0 }
              { A_{20} }{ a_{21} }{ A_{22} }
$
}

% output
\begin{figure}[tbp]
\begin{center}
\FlaAlgorithm
\end{center}
\caption{Unblocked algorithm for $ C \becomes A^T A + \hat{C} $.}
\label{fig:sytrrk_uln_downright_unblk}
\end{figure}


\section{Loop Invariant 2}
We now examine the loop-invariant
\begin{equation}
\label{eqn:sytrrk_uln_inv2}
\PInv:
\FlaTwoByTwo{ C_{TL} }        { C_{TR} }
            { \undetermined } { C_{BR} }
=
\FlaTwoByTwo{ {C}_{TL} } { {C}_{TR} }
            { \undetermined }                { A_{BR}^T A_{BR} + \hat{C}_{BR} }
\wedge
\ldots
\end{equation}
Comparing the loop-invariant in Eqn.~\ref{eqn:sytrrk_uln_inv2} with the postcondition $ C = A^T A + \hat{C} $
we see that if $ C = C_{BR}$ , $A = A_{BR}$, and $\hat{C} = \hat{C}_{BR}$ then the loop-invariant implies the
postcondition, i.e., that the desired result has been computed. The fact that for this partitioning of
C, $\hat{C}$, and A,
\\
\[
\FlaTwoByTwo{ C_{TL} }        { C_{TR} }
            { \undetermined } { C_{BR} }
=
\FlaTwoByTwo{ {C}_{TL} } { {C}_{TR} }
            { \undetermined }                { A_{BR}^T A_{BR} + \hat{C}_{BR} }
\]
and the precondition implies the other parts of the loop-invariant, this initialization has the desired
properties.

Loop-guard $G = \neg \SameSize(C_{BR}, C)$ indicates that eventually $C_{BR}$ should equal all of C, at which
point $G$ becomes false and the loop is exited.  After the initialization, $C_{BR}$ is  $0 \times 0 $. Thus,
the repartitioning should be such that as the computation proceeds, rows and columns are subtracted from $C_{TL} $
and added to $C_{BR}$ while updating $C_{BR}$ consistently with this.

\subsection{Blocked algorithm}
If we combine the initial partitioning with the loop invariant, we get the intermediate state
\\
\\
$
\QBefore:
\FlaThreeByThreeTL{ C_{00} }{ C_{01} }{ C_{02} }
              { * }{ C_{11} }{ C_{12} }
              { * }{ * }{ C_{22} }
=
\FlaThreeByThreeTL{ \hat{C}_{00} }{ \hat{C}_{01} }{ \hat{C}_{02} }
              { * }{ \hat{C}_{11} }{ \hat{C}_{12} }
              { * }{ * }{ A_{22}^T A_{22} + \hat{C}_{22} }
$
\\
\\
If we combine the post-update partitioning with the loop invariant, we get the post-update state
\\
$\\
\QAfter:
\FlaThreeByThreeBR{ C_{00} }{ C_{01} }{ C_{02} }
              { * }{ C_{11} }{ C_{12} }
              { * }{ * }{ C_{22} }
=
\FlaThreeByThreeBR{ \hat{C}_{00} } { \hat{C}_{01} }{ \hat{C}_{02} }
              { * }{ A_{11}^T A_{11} + A_{21}^T A_{21} + \hat{C}_{11} }{ A_{21}^T A_{22} + \hat{C}_{12} }
              { * }{ * }{ A_{22}^T A_{22} + \hat{C}_{22} }
$
\\
\\
Comparing $\QBefore$ with $\QAfter$ we obtain the updates
\\
\\
$C_{11} \becomes A_{11}^T A_{11} + A_{21}^T A_{21} + \hat{C}_{11} \\$
$C_{12} \becomes A_{21}^T A_{22} + \hat{C}_{12}$
\\

\subsection{Unblocked algorithm}

As with the algorithm for loop invariant 1, the differences between blocked and unblocked versions are small.
The notable differences are to the states $ \Qbefore $, $ \Qafter $, and the updates given below.

$
\QBefore:
\FlaThreeByThreeTL{ C_{00} }{ c_{01} }{ C_{02} }
              { * }{ \gamma_{11} }{ c_{12}^T }
              { * }{ * }{ C_{22} }
=
\FlaThreeByThreeTL{ \hat{C}_{00} }{ \hat{c}_{01} }{ \hat{C}_{02} }
              { * }{ \hat{\gamma}_{11} }{ \hat{c}_{12}^T }
              { * }{ * }{ A_{22}^T A_{22} + \hat{C}_{22} }
$
\\
\\
If we combine the post-update partitioning with the loop invariant, we get the post-update state %$ \Qafter $.
\\
$\\
\QAfter:
\FlaThreeByThreeBR{ C_{00} }{ c_{01} }{ C_{02} }
              { * }{ \gamma_{11} }{ c_{12}^T }
              { * }{ * }{ C_{22} }
=
\FlaThreeByThreeBR{ \hat{C}_{00} } { \hat{c}_{01} }{ \hat{C}_{02} }
              { * }{ a_{21}^T a_{21} + \alpha_{11}^2 + \hat{\gamma}_{11} }{ A_{22}^T a_{21} + \hat{c}_{12}^T }
              { * }{ * }{ A_{22}^T A_{22} + \hat{C}_{22} }
$
\\
Comparing $\QBefore$ with $\QAfter$ we find that the updates
\\
\\
$\gamma_{11} \becomes a_{21}^T a_{21} + \alpha_{11}^2 + \hat{\gamma}_{11}$ \\
$c_{12} \becomes A_{22}^T a_{21} + \hat{c}_{12}$
\\

\section{Performance}


In this section we will discuss performance and how our blocked, unblocked, and recursive algorithms compare to
the reference FLAME implementation. Performance was measured on a 600 MHz Pentium III Processor with 512k level 2 cache.
The graph we generated shows each algorithms performance in MFLOPS. We see from loop invariant 1 that the unblocked
algorithm seems to perform best when the matrix diminensions are $ 250 \times 250 $ or below.
This is likely due to cache issues and the fact that the unblocked algorithm calls fewer routines for the update
than the blocked algorithm.

Observing the graphs in Fig.~\ref{fig:sytrrk_uln:perfgraph}, we see that the blocked algorithm performs better than
the unblocked algorithm because it uses matrix-matrix operations to perform its updates. Meanwhile, the unblocked
algorithm uses matrix-vector and vector-vector operations to perform its updates. Our recursive algorithm only has
an advantage for very large block sizes, while its performance is comparable to the blocked algorithm for the
blocksize we used.


% display the blocked annotated algorithm
%
% Annotated Algorithm
%


% our operation
\renewcommand{\operation}{C \becomes A^T A + C}


% precondition
\renewcommand{\precondition}{C = \hat{C} \wedge n(C) = m(C) \wedge n(A) = m(A) \wedge n(C) = n(A)
\wedge UpTr(C) \wedge LowTr(A)}

% postcondition
\renewcommand{\postcondition}{{ C \becomes A^T A + \hat{C} }}

% invariant 1
\renewcommand{\invariant}{
\FlaTwoByTwo{ C_{TL} } { C_{TR} }
            { * } { C_{BR} }
=
\FlaTwoByTwo{ A_{TL}^T A_{TL} + A_{BL}^T A_{BL} + \hat{C}_{TL} } { \hat{C}_{TR} }
            { * } { \hat{C}_{BR} }
}

% loop guard
\renewcommand{\guard}{ \neg \SameSize( C_{TL}, C ) }

% initial partitionings
\renewcommand{\partitionings}{
$
C \rightarrow
\FlaTwoByTwo{ C_{TL} } { C_{TR} }
            { * } { C_{BR} }
$,
$
\hat{C} \rightarrow
\FlaTwoByTwo{ \hat{C}_{TL} } { \hat{C}_{TR} }
            { * } { \hat{C}_{BR} }
$ and
$
A \rightarrow
\FlaTwoByTwo{ A_{TL} } { 0 }
            { A_{BL} } { A_{BR} }
$
}


\renewcommand{\partitionsizes}{$ C_{TL} $, $ \hat{C}_{TL} $, and $ A_{TL} $ is $ 0 \times 0 $ }


% Define the blocksize that appears in step 5a
\renewcommand{\blocksize}{ b }
%

% repartitioning at start of loop (step 5a)
\renewcommand{\repartitionings}{
\normalsize
$
\FlaTwoByTwo{ C_{TL} } { C_{TR} }
            { * } { C_{BR} }
\rightarrow
\FlaThreeByThreeBR{ C_{00} }{ C_{01} }{ C_{02} }
              { * }{ C_{11} }{ C_{12} }
              { * }{ * }{ C_{22} }
$, \\ \\
$
\FlaTwoByTwo{ \hat{C}_{TL} } { \hat{C}_{TR} }
            { * } { \hat{C}_{BR} }
\rightarrow
\FlaThreeByThreeBR{ \hat{C}_{00} }{ \hat{C}_{01} }{ \hat{C}_{02} }
              { * }{ \hat{C}_{11} }{ \hat{C}_{12} }
              { * }{ * }{ \hat{C}_{22} }
$, and \\ \\
$
\FlaTwoByTwo{ A_{TL} } { 0 }
            { A_{BL} } { A_{BR} }
\rightarrow
\FlaThreeByThreeBR{ A_{00} }{ 0 }{ 0 }
              { A_{10} }{ A_{11} }{ 0 }
              { A_{20} }{ A_{21} }{ A_{22} }
$ \\
}




\renewcommand{\repartitionsizes}{
$ C_{11} $ , $ \hat{C}_{11} $, and $ A_{11} $ are $ b \times b $ matrices}



% Moving double lines at end of loop (step 5b)
\renewcommand{\moveboundaries}{
\normalsize
$
\FlaTwoByTwo{ C_{TL} } { C_{TR} }
            { * } { C_{BR} }
\leftarrow
\FlaThreeByThreeTL{ C_{00} }{ C_{01} }{ C_{02} }
              { * }{ C_{11} }{ C_{12} }
              { * }{ * }{ C_{22} }
$, \\ \\
$
\FlaTwoByTwo{ \hat{C}_{TL} } { \hat{C}_{TR} }
            { * } { \hat{C}_{BR} }
\leftarrow
\FlaThreeByThreeTL{ \hat{C}_{00} }{ \hat{C}_{01} }{ \hat{C}_{02} }
              { * }{ \hat{C}_{11} }{ \hat{C}_{12} }
              { * }{ * }{ \hat{C}_{22} }
$, and \\ \\
$
\FlaTwoByTwo{ A_{TL} } { 0 }
            { A_{BL} } { A_{BR} }
\leftarrow
\FlaThreeByThreeTL{ A_{00} }{ 0 }{ 0 }
              { A_{10} }{ A_{11} }{ 0 }
              { A_{20} }{ A_{21} }{ A_{22} }
$ \\
}



% state before the update
\renewcommand{\beforeupdate}{
\FlaThreeByThreeBR{ C_{00} }{ C_{01} }{ C_{02} }
              { * }{ C_{11} }{ C_{12} }
              { * }{ * }{ C_{22} }
=
\FlaThreeByThreeBR{ A_{00}^T A_{00} + A_{10}^T A_{10} + A_{20}^T A_{20} + \hat{C}_{00} }{ \hat{C}_{01} }{ \hat{C}_{02} }
              { * }{ \hat{C}_{11} }{ \hat{C}_{12} }
              { * }{ * }{ \hat{C}_{22} }
}



% state after the update
\renewcommand{\afterupdate}{
\FlaThreeByThreeTL{ C_{00} }{ C_{01} }{ C_{02} }
              { * }{ C_{11} }{ C_{12} }
              { * }{ * }{ C_{22} }
=
\FlaThreeByThreeTL{ A_{00}^T A_{00} + A_{10}^T A_{10} + A_{20}^T A_{20} + \hat{C}_{00} }
                  { A_{10}^T A_{11} + A_{20}^T A_{21} + \hat{C}_{01} }{ \hat{C}_{02} }
              { * }{ A_{11}^T A_{11} + A_{21}^T A_{21} + \hat{C}_{11} }{ \hat{C}_{12} }
              { * }{ * }{ \hat{C}_{22} }
}



% the update
\renewcommand{\update}{
$
\begin{array}{l}
C_{01} \becomes A_{10}^T A_{11} + A_{20}^T A_{21} + \hat{C}_{01} \\
C_{11} \becomes A_{11}^T A_{11} + A_{21}^T A_{21} + \hat{C}_{11}
\end{array}
$
}

% output
\begin{figure}[tbp]
\begin{center}
\worksheet
\end{center}
\caption{Worksheet for deriving blocked algorithm for $ C \becomes A^T A + \hat{C} $.}
\label{fig:sytrrk_uln_downright_blk_ann}
\end{figure}


% display the unblocked annotated algorithm
%
% Annotated Algorithm
%


% our operation
\renewcommand{\operation}{C \becomes A^T A + C}


% precondition
\renewcommand{\precondition}{C = \hat{C} \wedge n(C) = m(C) \wedge n(A) = m(A) \wedge n(C) = n(A)
\wedge UpTr(C) \wedge LowTr(A)}

% postcondition
\renewcommand{\postcondition}{{ C \becomes A^T A + \hat{C} }}

% invariant 1
\renewcommand{\invariant}{
\FlaTwoByTwo{ C_{TL} } { C_{TR} }
            { * } { C_{BR} }
=
\FlaTwoByTwo{ A_{TL}^T A_{TL} + A_{BL}^T A_{BL} \hat{C}_{TL} } { \hat{C}_{TR} }
            { * } { \hat{C}_{BR} }
}

% loop guard
\renewcommand{\guard}{ \neg \SameSize( C_{TL}, C ) }

% initial partitionings
\renewcommand{\partitionings}{
$
C \rightarrow
\FlaTwoByTwo{ C_{TL} } { C_{TR} }
            { * } { C_{BR} }
$,
$
\hat{C} \rightarrow
\FlaTwoByTwo{ \hat{C}_{TL} } { \hat{C}_{TR} }
            { * } { \hat{C}_{BR} }
$ and
$
A \rightarrow
\FlaTwoByTwo{ A_{TL} } { 0 }
            { A_{BL} } { A_{BR} }
$
}


\renewcommand{\partitionsizes}{$ C_{TL} $, $ \hat{C}_{TL} $, and $ A_{TL} $ is $ 0 \times 0 $ }



% repartitioning at start of loop (step 5a)
\renewcommand{\repartitionings}{
\normalsize
$
\FlaTwoByTwo{ C_{TL} } { C_{TR} }
            { * } { C_{BR} }
\rightarrow
\FlaThreeByThreeBR{ C_{00} }{ c_{01} }{ C_{02} }
              { * }{ \gamma_{11} }{ c_{12}^T }
              { * }{ * }{ C_{22} }
$, \\ \\
$
\FlaTwoByTwo{ \hat{C}_{TL} } { \hat{C}_{TR} }
            { * } { \hat{C}_{BR} }
\rightarrow
\FlaThreeByThreeBR{ \hat{C}_{00} }{ \hat{c}_{01} }{ \hat{C}_{02} }
              { * }{ \hat{\gamma}_{11} }{ \hat{c}_{12}^T }
              { * }{ * }{ \hat{C}_{22} }
$, and \\ \\
$
\FlaTwoByTwo{ A_{TL} } { 0 }
            { A_{BL} } { A_{BR} }
\rightarrow
\FlaThreeByThreeBR{ A_{00} }{ 0 }{ 0 }
              { a_{10}^T }{ \alpha_{11} }{ 0 }
              { A_{20} }{ a_{21} }{ A_{22} }
$ \\
}

\renewcommand{\repartitionsizes}{
$ \gamma_{11} $ , $ \hat{\gamma}_{11} $, and $ \alpha_{11} $ are scalars,
$ c_{01} $ , $ \hat{c}_{01} $, and $ a_{21} $ are column vectors, \\
and $ c_{12}^T $ , $ \hat{c}_{12}^T $, and $ a_{10}^T $ are row vectors. }



% Moving double lines at end of loop (step 5b)
\renewcommand{\moveboundaries}{
\normalsize
$
\FlaTwoByTwo{ C_{TL} } { C_{TR} }
            { * } { C_{BR} }
\leftarrow
\FlaThreeByThreeTL{ C_{00} }{ c_{01} }{ C_{02} }
              { * }{ \gamma_{11} }{ c_{12}^T }
              { * }{ * }{ C_{22} }
$, \\ \\
$
\FlaTwoByTwo{ \hat{C}_{TL} } { \hat{C}_{TR} }
            { * } { \hat{C}_{BR} }
\leftarrow
\FlaThreeByThreeTL{ \hat{C}_{00} }{ \hat{c}_{01} }{ \hat{C}_{02} }
              { * }{ \hat{\gamma}_{11} }{ \hat{c}_{12}^T }
              { * }{ * }{ \hat{C}_{22} }
$, and \\ \\
$
\FlaTwoByTwo{ A_{TL} } { 0 }
            { A_{BL} } { A_{BR} }
\leftarrow
\FlaThreeByThreeTL{ A_{00} }{ 0 }{ 0 }
              { a_{10}^T }{ \alpha_{11} }{ 0 }
              { A_{20} }{ a_{21} }{ A_{22} }
$ \\
}



% state before the update
\renewcommand{\beforeupdate}{
\FlaThreeByThreeBR{ C_{00} }{ c_{01} }{ C_{02} }
              { * }{ \gamma_{11} }{ c_{12}^T }
              { * }{ * }{ C_{22} }
=
\FlaThreeByThreeBR{ A_{00}^T A_{00} + a_{10} a_{10}^T + A_{20}^T A_{20} + \hat{C}_{00} }{ \hat{c}_{01} }{ \hat{C}_{02} }
              { * }{ \hat{\gamma}_{11} }{ \hat{c}_{12}^T }
              { * }{ * }{ \hat{C}_{22} }
}



% state after the update
\renewcommand{\afterupdate}{
\FlaThreeByThreeTL{ C_{00} }{ c_{01} }{ C_{02} }
              { * }{ \gamma_{11} }{ c_{12}^T }
              { * }{ * }{ C_{22} }
=
\FlaThreeByThreeTL{ A_{00}^T A_{00} + a_{10} a_{10}^T + A_{20}^T A_{20} + \hat{C}_{00} }
                  { A_{20}^T a_{21} + \alpha_{11} a_{10} + \hat{c}_{01} }{ \hat{C}_{02} }
              { * }{ a_{21} a_{21}^T + \alpha_{11}^2 + \hat{\gamma}_{11} }{ \hat{c}_{12}^T }
              { * }{ * }{ \hat{C}_{22} }
}



% the update
\renewcommand{\update}{
$
\begin{array}{l}
c_{01} \becomes A_{20}^T a_{21} + \alpha_{11} a_{10} + \hat{c}_{01} \\
\gamma_{11} \becomes a_{21}^T a_{21} + \alpha_{11}^2 + \hat{\gamma}_{11}
\end{array}
$
}

% output
\begin{figure}[tbp]
\begin{center}
\worksheet
\end{center}
\caption{Worksheet for deriving unblocked algorithm for $ C \becomes A^T A + \hat{C} $.}
\label{fig:sytrrk_uln_downright_unblk_ann}
\end{figure}


% diplay the blocked code for invariant 1
\begin{figure}
\footnotesize
\listinginput{1}{sytrrk_uln/codes/invariant1/MY_Sytrrk_uln_blk.c}
\label{fig:sytrrk_uln_code_inv1_blk}
\caption{Source listing for loop invariant 1, blocked algorithm}
\end{figure}

% diplay the unblocked code for invariant 1
\begin{figure}
\footnotesize
\listinginput{1}{sytrrk_uln/codes/invariant1/MY_Sytrrk_uln_unblk.c}
\label{fig:sytrrk_uln_code_inv1_unblk}
\caption{Source listing for loop invariant 1, unblocked algorithm}
\end{figure}

% display the blocked annotated algorithm
%
% Annotated Algorithm
%


% our operation
\renewcommand{\operation}{C \becomes A^T A + C}


% precondition
\renewcommand{\precondition}{C = \hat{C} \wedge n(C) = m(C) \wedge n(A) = m(A) \wedge n(C) = n(A)
\wedge UpTr(C) \wedge LowTr(A)}

% postcondition
\renewcommand{\postcondition}{{ C \becomes A^T A + \hat{C} }}

% invariant 2
\renewcommand{\invariant}{
\FlaTwoByTwo{ C_{TL} } { C_{TR} }
            { * } { C_{BR} }
=
\FlaTwoByTwo{ \hat{C}_{TL} } { \hat{C}_{TR} }
            { * } { A_{BR}^T + A_{BR} + \hat{C}_{BR} }
}

% loop guard
\renewcommand{\guard}{ \neg \SameSize( C_{BR}, C ) }

% initial partitionings
\renewcommand{\partitionings}{
$
C \rightarrow
\FlaTwoByTwo{ C_{TL} } { C_{TR} }
            { * } { C_{BR} }
$,
$
\hat{C} \rightarrow
\FlaTwoByTwo{ \hat{C}_{TL} } { \hat{C}_{TR} }
            { * } { \hat{C}_{BR} }
$ and
$
A \rightarrow
\FlaTwoByTwo{ A_{TL} } { 0 }
            { A_{BL} } { A_{BR} }
$
}


\renewcommand{\partitionsizes}{$ C_{BR} $, $ \hat{C}_{BR} $, and $ A_{BR} $ are $ 0 \times 0 $ }


% Define the blocksize that appears in step 5a
\renewcommand{\blocksize}{ b }
%

% repartitioning at start of loop (step 5a)
\renewcommand{\repartitionings}{
\normalsize
$
\FlaTwoByTwo{ C_{TL} } { C_{TR} }
            { * } { C_{BR} }
\rightarrow
\FlaThreeByThreeTL{ C_{00} }{ C_{01} }{ C_{02} }
              { * }{ C_{11} }{ C_{12} }
              { * }{ * }{ C_{22} }
$, \\ \\
$
\FlaTwoByTwo{ \hat{C}_{TL} } { \hat{C}_{TR} }
            { * } { \hat{C}_{BR} }
\rightarrow
\FlaThreeByThreeTL{ \hat{C}_{00} }{ \hat{C}_{01} }{ \hat{C}_{02} }
              { * }{ \hat{C}_{11} }{ \hat{C}_{12} }
              { * }{ * }{ \hat{C}_{22} }
$, and \\ \\
$
\FlaTwoByTwo{ A_{TL} } { 0 }
            { A_{BL} } { A_{BR} }
\rightarrow
\FlaThreeByThreeTL{ A_{00} }{ 0 }{ 0 }
              { A_{10} }{ A_{11} }{ 0 }
              { A_{20} }{ A_{21} }{ A_{22} }
$ \\
}




\renewcommand{\repartitionsizes}{
$ C_{11} $ , $ \hat{C}_{11} $, and $ A_{11} $ are $ b \times b $ matrices}



% Moving double lines at end of loop (step 5b)
\renewcommand{\moveboundaries}{
\normalsize
$
\FlaTwoByTwo{ C_{TL} } { C_{TR} }
            { * } { C_{BR} }
\leftarrow
\FlaThreeByThreeBR{ C_{00} }{ C_{01} }{ C_{02} }
              { * }{ C_{11} }{ C_{12} }
              { * }{ * }{ C_{22} }
$, \\ \\
$
\FlaTwoByTwo{ \hat{C}_{TL} } { \hat{C}_{TR} }
            { * } { \hat{C}_{BR} }
\leftarrow
\FlaThreeByThreeBR{ \hat{C}_{00} }{ \hat{C}_{01} }{ \hat{C}_{02} }
              { * }{ \hat{C}_{11} }{ \hat{C}_{12} }
              { * }{ * }{ \hat{C}_{22} }
$, and \\ \\
$
\FlaTwoByTwo{ A_{TL} } { 0 }
            { A_{BL} } { A_{BR} }
\leftarrow
\FlaThreeByThreeBR{ A_{00} }{ 0 }{ 0 }
              { A_{10} }{ A_{11} }{ 0 }
              { A_{20} }{ A_{21} }{ A_{22} }
$ \\
}



% state before the update
\renewcommand{\beforeupdate}{
\FlaThreeByThreeTL{ C_{00} }{ C_{01} }{ C_{02} }
              { * }{ C_{11} }{ C_{12} }
              { * }{ * }{ C_{22} }
=
\FlaThreeByThreeTL{ \hat{C}_{00} }{ \hat{C}_{01} }{ \hat{C}_{02} }
              { * }{ \hat{C}_{11} }{ \hat{C}_{12} }
              { * }{ * }{ A_{22}^T A_{22} + \hat{C}_{22} }
}



% state after the update
\renewcommand{\afterupdate}{
\FlaThreeByThreeBR{ C_{00} }{ C_{01} }{ C_{02} }
              { * }{ C_{11} }{ C_{12} }
              { * }{ * }{ C_{22} }
=
\FlaThreeByThreeBR{ \hat{C}_{00} } { \hat{C}_{01} }{ \hat{C}_{02} }
              { * }{ A_{11}^T A_{11} + A_{21}^T A_{21} + \hat{C}_{11} }{ A_{21}^T A_{22} + \hat{C}_{12} }
              { * }{ * }{ A_{22}^T A_{22} + \hat{C}_{22} }
}



% the update
\renewcommand{\update}{
$
\begin{array}{l}
C_{11} \becomes A_{11}^T A_{11} + A_{21}^T A_{21} + \hat{C}_{11} \\
C_{12} \becomes A_{21}^T A_{22} + \hat{C}_{12}
\end{array}
$
}

% output
\begin{figure}[tbp]
\begin{center}
\worksheet
\end{center}
\caption{Worksheet for deriving blocked algorithm for $ C \becomes A^T A + \hat{C} $.}
\label{fig:sytrrk_uln_upleft_blk_ann}
\end{figure}


% display the blocked algorithm


%
% The Algorithm
%

% Initial Partitioning
\renewcommand{\partitionings}{
$
C \rightarrow
\FlaTwoByTwo{ C_{TL} } { C_{TR} }
            { * } { C_{BR} }
$,
$
\hat{C} \rightarrow
\FlaTwoByTwo{ \hat{C}_{TL} } { \hat{C}_{TR} }
            { * } { \hat{C}_{BR} }
$ and
$
A \rightarrow
\FlaTwoByTwo{ A_{TL} } { 0 }
            { A_{BL} } { A_{BR} }
$
}


\renewcommand{\partitionsizes}{$ C_{BR} $, $ \hat{C}_{BR} $, and $ A_{BR} $ are $ 0 \times 0 $ }

% loop guard
\renewcommand{\guard}{ \neg \SameSize( C_{BR}, C ) }


% Repartitioning at start of loop
\renewcommand{\repartitionings}{
\normalsize
$
\FlaTwoByTwo{ C_{TL} } { C_{TR} }
            { * } { C_{BR} }
\rightarrow
\FlaThreeByThreeTL{ C_{00} }{ C_{01} }{ C_{02} }
              { * }{ C_{11} }{ C_{12} }
              { * }{ * }{ C_{22} }
$, \\ \\
$
\FlaTwoByTwo{ \hat{C}_{TL} } { \hat{C}_{TR} }
            { * } { \hat{C}_{BR} }
\rightarrow
\FlaThreeByThreeTL{ \hat{C}_{00} }{ \hat{C}_{01} }{ \hat{C}_{02} }
              { * }{ \hat{C}_{11} }{ \hat{C}_{12} }
              { * }{ * }{ \hat{C}_{22} }
$, and \\ \\
$
\FlaTwoByTwo{ A_{TL} } { 0 }
            { A_{BL} } { A_{BR} }
\rightarrow
\FlaThreeByThreeTL{ A_{00} }{ 0 }{ 0 }
              { A_{10} }{ A_{11} }{ 0 }
              { A_{20} }{ A_{21} }{ A_{22} }
$ \\
}

% the update
\renewcommand{\update}{
$
\begin{array}{l}
C_{11} \becomes A_{11}^T A_{11} + A_{21}^T A_{21} + \hat{C}_{11} \\
C_{12} \becomes A_{21}^T A_{22} + \hat{C}_{12}
\end{array}
$
}

\renewcommand{\repartitionsizes}{
$ C_{11} $ , $ \hat{C}_{11} $, and $ A_{11} $ are $ b \times b $ matrices}

% Moving double lines at end of loop
\renewcommand{\moveboundaries}{
\normalsize
$
\FlaTwoByTwo{ C_{TL} } { C_{TR} }
            { * } { C_{BR} }
\leftarrow
\FlaThreeByThreeBR{ C_{00} }{ C_{01} }{ C_{02} }
              { * }{ C_{11} }{ C_{12} }
              { * }{ * }{ C_{22} }
$, \\ \\
$
\FlaTwoByTwo{ \hat{C}_{TL} } { \hat{C}_{TR} }
            { * } { \hat{C}_{BR} }
\leftarrow
\FlaThreeByThreeBR{ \hat{C}_{00} }{ \hat{C}_{01} }{ \hat{C}_{02} }
              { * }{ \hat{C}_{11} }{ \hat{C}_{12} }
              { * }{ * }{ \hat{C}_{22} }
$, and \\ \\
$
\FlaTwoByTwo{ A_{TL} } { 0 }
            { A_{BL} } { A_{BR} }
\leftarrow
\FlaThreeByThreeBR{ A_{00} }{ 0 }{ 0 }
              { A_{10} }{ A_{11} }{ 0 }
              { A_{20} }{ A_{21} }{ A_{22} }
$ \\
}

% output
\begin{figure}[tbp]
\begin{center}
\FlaAlgorithm
\end{center}
\caption{Blocked algorithm for $ C \becomes A^T A + \hat{C} $.}
\label{fig:sytrrk_uln_upleft_blk}
\end{figure}


% display the unblocked annotated algorithm
\input{sytrrk_uln/unblocked_annotated2}

% display the unblocked algorithm


%
% The Algorithm
%

% Initial Partitioning
\renewcommand{\partitionings}{
$
C \rightarrow
\FlaTwoByTwo{ C_{TL} } { C_{TR} }
            { * } { C_{BR} }
$,
$
\hat{C} \rightarrow
\FlaTwoByTwo{ \hat{C}_{TL} } { \hat{C}_{TR} }
            { * } { \hat{C}_{BR} }
$ and
$
A \rightarrow
\FlaTwoByTwo{ A_{TL} } { 0 }
            { A_{BL} } { A_{BR} }
$
}


\renewcommand{\partitionsizes}{$ C_{BR} $, $ \hat{C}_{BR} $, and $ A_{BR} $ are $ 0 \times 0 $ }

% loop guard
\renewcommand{\guard}{ \neg \SameSize( C_{BR}, C ) }


% Repartitioning at start of loop
\renewcommand{\repartitionings}{
\normalsize
$
\FlaTwoByTwo{ C_{TL} } { C_{TR} }
            { * } { C_{BR} }
\rightarrow
\FlaThreeByThreeTL{ C_{00} }{ c_{01} }{ C_{02} }
              { * }{ \gamma_{11} }{ c_{12}^T }
              { * }{ * }{ C_{22} }
$, \\ \\
$
\FlaTwoByTwo{ \hat{C}_{TL} } { \hat{C}_{TR} }
            { * } { \hat{C}_{BR} }
\rightarrow
\FlaThreeByThreeTL{ \hat{C}_{00} }{ \hat{c}_{01} }{ \hat{C}_{02} }
              { * }{ \hat{\gamma}_{11} }{ \hat{c}_{12}^T }
              { * }{ * }{ \hat{C}_{22} }
$, and \\ \\
$
\FlaTwoByTwo{ A_{TL} } { 0 }
            { A_{BL} } { A_{BR} }
\rightarrow
\FlaThreeByThreeTL{ A_{00} }{ 0 }{ 0 }
              { a_{10}^T }{ \alpha_{11} }{ 0 }
              { A_{20} }{ a_{21} }{ A_{22} }
$ \\
}

% the update
\renewcommand{\update}{
$
\begin{array}{l}
\gamma_{11} \becomes a_{21}^T a_{21} + \alpha_{11}^2 + \hat{\gamma}_{11} \\
c_{12} \becomes A_{22}^T a_{21} + \hat{c}_{12}
\end{array}
$
}

\renewcommand{\repartitionsizes}{
$ \gamma_{11} $ , $ \hat{\gamma}_{11} $, and $ \alpha_{11} $ are scalars,
$ c_{01} $ , $ \hat{c}_{01} $, and $ a_{21} $ are column vectors, \\
and $ c_{12}^T $ , $ \hat{c}_{12}^T $, and $ a_{10}^T $ are row vectors. }

% Moving double lines at end of loop
\renewcommand{\moveboundaries}{
\normalsize
$
\FlaTwoByTwo{ C_{TL} } { C_{TR} }
            { * } { C_{BR} }
\leftarrow
\FlaThreeByThreeBR{ C_{00} }{ c_{01} }{ C_{02} }
              { * }{ \gamma_{11} }{ c_{12}^T }
              { * }{ * }{ C_{22} }
$, \\ \\
$
\FlaTwoByTwo{ \hat{C}_{TL} } { \hat{C}_{TR} }
            { * } { \hat{C}_{BR} }
\leftarrow
\FlaThreeByThreeBR{ \hat{C}_{00} }{ \hat{c}_{01} }{ \hat{C}_{02} }
              { * }{ \hat{\gamma}_{11} }{ \hat{c}_{12}^T }
              { * }{ * }{ \hat{C}_{22} }
$, and \\ \\
$
\FlaTwoByTwo{ A_{TL} } { 0 }
            { A_{BL} } { A_{BR} }
\leftarrow
\FlaThreeByThreeBR{ A_{00} }{ 0 }{ 0 }
              { a_{10}^T }{ \alpha_{11} }{ 0 }
              { A_{20} }{ a_{21} }{ A_{22} }
$ \\
}

% output
\begin{figure}[tbp]
\begin{center}
\FlaAlgorithm
\end{center}
\caption{Unblocked algorithm for $ C \becomes A^T A + \hat{C} $.}
\label{fig:sytrrk_uln_upleft_unblk}
\end{figure}


% diplay the blocked code for invariant 2
\begin{figure}
\footnotesize
\listinginput{1}{sytrrk_uln/codes/invariant2/MY_Sytrrk_uln_blk.c}
\label{fig:sytrrk_uln_code_inv2_blk}
\caption{Source listing for loop invariant 2, blocked algorithm}
\end{figure}

% diplay the unblocked code for invariant 2
\begin{figure}
\footnotesize
\listinginput{1}{sytrrk_uln/codes/invariant2/MY_Sytrrk_uln_unblk.c}
\label{fig:sytrrk_uln_code_inv2_unblk}
\caption{Source listing for loop invariant 2, unblocked algorithm}
\end{figure}

% display the performance graphs
\begin{figure}[htbp]
\begin{center}
\begin{tabular}{c}
Variant 1 \\
\psfig{figure=sytrrk_uln/codes/invariant1/sytrrk_uln_inv1.eps,width=4.5in,height=3.25in} \\ \\
Variant 2 \\
\psfig{figure=sytrrk_uln/codes/invariant2/sytrrk_uln_inv2.eps,width=4.5in,height=3.25in}
\\
\end{tabular}
\end{center}
\caption{Performance of the variants  of symmetric triangular rank-k update
algorithms for a block size of $ 128 $.
For these experiments, the FLAME matrix-matrix multiplication
kernel was used.}
\label{fig:sytrrk_uln:perfgraph}
\end{figure}

% \printindex{index}{Index}
% \printindex{author}{Author Index}
% \printindex{const}{Constant Index}
% \printindex{rout}{Function Index}
% \printindex{op}{Operation Index}

\end{document}



