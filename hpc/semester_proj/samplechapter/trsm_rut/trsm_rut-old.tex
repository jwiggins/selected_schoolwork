\index{op}{triangular solve!multiple RHSs}%
\index{op}{triangular solve!multiple RHSs!$ B \leftarrow B U^{-T} $|(}%
\chapter{Triangular Solve (with Multiple RHSs)\\
$ B \leftarrow B U^{-T} $ }
\label{chapter:trsm_rut}

\ChapterAuthor{
\index{author}{Van Zee, Field G.}
Field G. Van Zee
\\[0.1in]
\index{author}{Walkup, Patrick J.}
Patrick J. Walkup
}

\addtocontents{toc}{ {by {\bf Field G. Van Zee and Patrick J. Walkup}}}

\status{Codes updated, graphs update, naming conventions
checked}

In this chapter, we discuss the computation of the
solution of a triangular system with multiple right-hand-sides:
\[
B \leftarrow B U^{-T}
\]
where $ U $ is a $ m \times m $ upper triangular matrix
and $ B $ is $ n \times m $.

Let us consider
\begin{equation}
D = B U^{-T}
\end{equation} 
keeping in mind that $ D $ will
overwrite $ B $.
Notice that this is equivalent to solving
\begin{equation}
\label{eqn:utrsm1}
D U^{T} = B 
\end{equation} 
for matrix $ D$.  
Thus the name {\em triangular solve with multiple right-hand-sides} (RHSs).

\section{Algorithms that start by splitting $ U $}
\label{sec:teaml_ltrmmU}

Partition
\[
U \rightarrow \FlaTwoByTwo{ U_{TL} }{ U_{TR} }
                          {   0    }{ U_{BR} }
\]
where $ U_{BR} $ is $ k \times k $.
Also, partition
\[
D \rightarrow \FlaOneByTwo{ D_{L} }
                          { D_{R} },
\quad
\mbox{and}
\quad
B \rightarrow \FlaOneByTwo{ B_{L} }
                          { B_{R} }
\]
where $ D_{L} $ and $ B_{L} $ have $ k$ cols.

Now, (\ref{eqn:utrsm1}) becomes
\[
\FlaOneByTwo{ D_{L} }
            { D_{R} }
{\FlaTwoByTwo{ U_{TL} }{ U_{TR} }
             {     0  }{ U_{BR} }}^{T}
=
\FlaOneByTwo{ D_{L} }
            { D_{R} }
\FlaTwoByTwo{ U_{TL}^{T} }{     0      }
            { U_{TR}^{T} }{ U_{BR}^{T} }
=
\FlaOneByTwo{ B_{L} }
            { B_{R} }
\]
Multiplying out the left-hand-side yields
\[
\FlaOneByTwo{ D_{L} U_{TL}^{T} + D_{R} U_{TR}^{T} }
            { D_{R} U_{BR}^{T}  }
=
\FlaOneByTwo{ B_{L} }
            { B_{R} }
\]
This in turn exposes the equalities
\begin{eqnarray*}
D_{L} U_{TL}^{T} + D_{R} U_{TR}^{T} &=& B_{L} \\
D_{R} U_{BR}^{T}                    &=& B_{R}
\end{eqnarray*}
which must hold.
Notice that these equalities are also 
equivalent to
\begin{eqnarray*}
D_{L}  &=& ( B_{L} - D_{R} U_{TR}^{T} ) U_{TL}^{-T} \\
D_{R}  &=&   B_{R} U_{BR}^{-T}
\end{eqnarray*}
Note that in order to test these equalities, their subexpressions must be evaluated in the following order \\
\[
B_{R} U_{BR}^{-T}
\rightarrow 
(B_{L} - D_{R} U_{TR}^{T}) 
\rightarrow 
(B_{L} - D_{R} U_{TR}^{T}) U_{TL}^{-T}
\]
where the terms to the left of an arrow must be evaluated before the terms to the right.\\ \\
Possible contents of $ D $
at this intermediate step are given
in Fig.~\ref{fig:dtrsm:possibilities}.
\begin{figure}[htbp]
\begin{center}
\begin{tabular}{| p{2.5in} | p{3.0in} | c |} \hline
$ D $ contains & Comments & Viable?\\ \hline
\begin{minipage}{2.5in}
$
\FlaOneByTwo{ B_L }
            { B_R }
$ 
\end{minipage}
&
\begin{minipage}[c]{3in}
This condition indicates no
progress has been made.
\end{minipage}
&
NO
\\ \hline
%
%
\begin{minipage}{2.5in}
$
\FlaOneByTwo{ B_L }
            { B_{R} U_{BR}^{-T} }
$ 
\end{minipage}
& 
\begin{minipage}{3in}
\end{minipage}
&
YES
\\ \hline
%
%
\begin{minipage}{2.5in}
$
\FlaOneByTwo{ B_L - D_{R} U_{TR}^{T} }
            { B_R }
$ 
\end{minipage}
& 
\begin{minipage}{3in}
This one is not a viable option since 
it requires $ D_R $ without actually
storing $ D_R $. (See ordered equalities above.)
\end{minipage}
&
NO
\\ \hline
%
%
\begin{minipage}{2.5in}
$
\FlaOneByTwo{ B_L - D_{R} U_{TR}^{T} }
            { B_{R} U_{BR}^{-T} }
$ 
\end{minipage}
& 
\begin{minipage}{3in}
\end{minipage}
&
YES
\\ \hline
%
%
\begin{minipage}{2.5in}
$
\FlaOneByTwo{ ( B_{L} - D_R U_{TR}^{T} ) U_{TL}^{-T} }
            { B_R }
$ 
\end{minipage}
& 
\begin{minipage}{3in}
This one is not a viable option since 
it requires $ D_R $ without actually
storing $ D_R $. (See ordered equalities above.)
\end{minipage}
&
NO
\\ \hline
%
%
\begin{minipage}{2.5in}
$
\FlaOneByTwo{ ( B_L - D_R U_{TR}^{T} ) U_{TL}^{-T} }
            { B_R U_{BR}^{-T} }
$ 
\end{minipage}
& 
\begin{minipage}{3in}
This condition indicates that the computation has completed.
\end{minipage}
&
NO \\
\hline
\end{tabular}
\end{center}
\caption{Possible contents of $ D $ at 
an intermediate step.}
\label{fig:dtrsm:possibilities}
\end{figure}
Considering the comments in that table, only two viable
conditions are left, the ones for which there are no comments.

\subsection{Column-lazy algorithm (relative to $ U $)}

Consider the condition that currently 
\begin{equation}
\label{eqn:utrsm2}
D \quad \mbox{contains}
\quad
\FlaOneByTwo{ B_{L} - D_{R} U_{TR}^{T} }
            { B_{R} U_{BR}^{-T}        }
\end{equation}
Notice that this indicates that 
$ U_{TR} $ and $ U_{BR} $ have been used
to update $ D $, which we will call
a column-lazy algorithm relative to operand
$ U $.

In order to move the boundary that indicates
how far the computation has proceeded, 
that boundary must be moved right.
Thus, this algorithm naturally moves through
matrices $ D $ and $ B $ in the ``left'' direction.

\subsubsection{Unblocked algorithm}

Repartition 
\[
  \FlaOneByTwo{ D_{L} }
              { D_{R} } \rightarrow
  \FlaOneByThreeL{ D_{0} } 
                 { d_{1} }
                 { D_{2} }
\]
where
$ d_{1} $ is a column.
Similarly, we repartition 
\[ 
  \FlaOneByTwo{ B_{L} }
              { B_{R} } \rightarrow
  \FlaOneByThreeL{ B_{0} }
                 { b_{1} }
                 { B_{2} }
\quad \mbox{and} \quad
      \FlaTwoByTwo{ U_{TL} }{ U_{TR} }
                  {   0    }{ U_{BR} } \rightarrow
      \FlaThreeByThreeTL{ U_{00} } { u_{01}       } { U_{02}     }
                        {   0    } { \upsilon_{11}} { u_{12}^{T} }
                        {   0    } {    0         } { U_{22}     }  
\]
where $ b_{1} $ is a column and 
$ \upsilon_{11} $ is a scalar.

Notice that the double lines have meaning:
\begin{equation}
\label{eqn:teaml_ltrmm:meaning}
\FlaOneByTwo  { D_L = \FlaOneByTwoSingleLine{ D_0 }
                                            { d_1 } }
              { D_R = D_2 },
%
\FlaOneByTwo  { B_L = \FlaOneByTwoSingleLine{ B_0 }
                                            { b_1 } }
              { B_R = B_2 }
\end{equation}
\[
\mbox{ and }
\FlaTwoByTwo  
              { U_{TL} = \FlaTwoByTwoSingleLine{ U_{00} } { u_{01}        }
                                               {   0    } { \upsilon_{11} } }
              { U_{TR} = \FlaTwoByOneSingleLine{ U_{02}   } 
                                               { u_{12}^T } }
	      { 0 }
              { U_{BR} = U_{22} } 
\]
Considering (\ref{eqn:utrsm2}) and these repartitionings, 
$ D $ currently contains
\begin{eqnarray*}
\begin{array}{c @{\hspace{1pt}} c @{\hspace{1pt}} c @{\hspace{1pt}} c @{\hspace{1pt}} c} 
\FlaOneByTwo{ B_{L} - D_{R} U_{TR}^{T} }
            { B_{R} U_{BR}^{-T}        }
&=& 
\FlaOneByTwo{ \FlaOneByTwoSingleLine{ B_0 }
                                    { b_1 } -
	      D_2
              \FlaOneByTwoSingleLine{ U_{02}^{T} } 
                                    { u_{12}     } }
             { B_{2}U_{22}^{-T} }
 \\
&=&
\FlaOneByTwo{ \FlaOneByTwoSingleLine{ B_0 - D_{2}U_{02}^T }
                                    { b_1 -  D_{2}u_{12}   } }
	    { B_{2}U_{22}^{-T} }\\
&=&
\FlaOneByTwo{ \FlaOneByTwoSingleLine{ B_0 - D_{2}U_{02}^T }
                                    { b_1 -  B_{2}U_{22}^{-T}u_{12}   } }
	    { B_{2}U_{22}^{-T} }\\    
\end{array}
\end{eqnarray*}
since $ D_{2} = B_{2}U_{22}^{-T} $. \\
\\
After moving the double lines ahead, in preparation
of the next iteration, 
\begin{equation}
\label{eqn:teaml_ltrmm:meaning2}
\FlaOneByTwo  { D_L = D_0 }
              { D_R = \FlaOneByTwoSingleLine{ d_1 }
                                            { D_2 } },
\FlaOneByTwo  { B_L = B_0 }
              { B_R = \FlaOneByTwoSingleLine{ b_1 }
                                            { B_2 } }
\end{equation}
\[
\mbox{ and }
\FlaTwoByTwo  { U_{TL} = U_{00} }
              { U_{TR} = \FlaOneByTwoSingleLine{ u_{01} } 
                                               { U_{02} } }
              { 0 }
              { U_{BR} = \FlaTwoByTwoSingleLine{ \upsilon_{11} } { u_{12}^T }
                                               { 0             } { U_{22}   } }
\]
After the double lines move (in other words, at the top of the loop 
during the next iteration),
the contents of $ D $ must equal
\begin{eqnarray*}
\begin{array}{c @{\hspace{1pt}} c @{\hspace{1pt}} c @{\hspace{1pt}} c @{\hspace{1pt}} c}
\FlaOneByTwo{ B_{L} - D_{R} U_{TR}^{T} }
            { B_{R} U_{BR}^{-T}        }
&=& 
\FlaOneByTwo { B_0 -
               \FlaOneByTwoSingleLine{ d_1 } { D_2 }
               \FlaTwoByOneSingleLine{ u_{01}^T  } 
                                     { U_{02}^T  } }
             { \FlaOneByTwoSingleLine{ b_{1} } { B_{2} }
	       { \FlaTwoByTwoSingleLine { \upsilon_{11} } { u_{12}^{T} }
	                                {      0        } { U_{22}     } }^{-T} } \\
&=& 
\FlaOneByTwo { B_0 -
               \FlaOneByTwoSingleLine{ d_1 } { D_2 }
               \FlaTwoByOneSingleLine{ u_{01}^T  } 
                                     { U_{02}^T  } }
             { \FlaOneByTwoSingleLine{ b_{1} } { B_{2} }
	       { \FlaTwoByTwoSingleLine { \upsilon_{11} } {    0     }
                                        {     u_{12}    } { U_{22}^T } }^{-1} } \\
&=&
\FlaOneByTwo { B_0 -
               \FlaOneByTwoSingleLine{ d_1 } { D_2 }
               \FlaTwoByOneSingleLine{ u_{01}^T  } 
                                     { U_{02}^T  } }
             { \FlaOneByTwoSingleLine{ b_{1} } { B_{2} }
	       \FlaTwoByTwoSingleLine { \upsilon_{11}^{-1}                 } {       0     }
                                      { -U_{22}^{-T} u_{12} \upsilon_{11}^{-1} } { U_{22}^{-T} } } \\
&=&
\FlaOneByTwo { B_0 -
               \FlaOneByTwoSingleLine{ d_1 } { D_2 }
               \FlaTwoByOneSingleLine{ u_{01}^T  } 
                                     { U_{02}^T  } }
             {
               \FlaOneByTwoSingleLine { b_1 \upsilon_{11}^{-1} - B_{2}U_{22}^{-T}u_{12} \upsilon_{11}^{-1} }
                                      { B_{2} U_{22}^{-T} } } \\
&=&
\FlaOneByTwo { B_0 -
               \FlaOneByTwoSingleLine{ d_1 } { D_2 }
               \FlaTwoByOneSingleLine{ u_{01}^T  } 
                                     { U_{02}^T  } }
             {
               \FlaOneByTwoSingleLine { ( b_1 - B_{2}U_{22}^{-T}u_{12} ) \upsilon_{11}^{-1} }
                                      { B_{2} U_{22}^{-T} } } \\
&=&
\FlaOneByTwo { B_0 - ( d_{1} u_{01}^{T} + D_{2} U_{02}^{T} ) }
             {
               \FlaOneByTwoSingleLine { d_1 = ( b_1 - B_{2}U_{22}^{-T}u_{12} ) \upsilon_{11}^{-1} }
                                      { B_{2} U_{22}^{-T} } } \\
&=&
\FlaOneByTwo { ( B_0 - D_{2} U_{02}^{T} ) - d_{1} u_{01}^{T} }
             {
               \FlaOneByTwoSingleLine { d_1 = ( b_1 - B_{2}U_{22}^{-T}u_{12} ) \upsilon_{11}^{-1} }
                                      { B_{2}U_{22}^{-T} } } \\
\end{array}
\end{eqnarray*}
since 
\[
            \FlaTwoByTwoSingleLine{ \upsilon_{11} } {    0     }
                                  {     u_{12}    } { U_{22}^T }^{-1}
=
            \FlaTwoByTwoSingleLine{ \upsilon_{11}^{-1}                     } {       0     }
                                  { -U_{22}^{-T} u_{12} \upsilon_{11}^{-1} } { U_{22}^{-T} }
\]
Thus, the contents of $ D $ must be updated like:
\begin{equation}
\FlaOneByTwo{ \FlaOneByTwoSingleLine{ B_0 - D_{2}U_{02}^T }
                                    { b_1 -  B_{2}U_{22}^{-T}u_{12}   } }
	    { B_{2}U_{22}^{-T} }
\rightarrow
\end{equation}
\[
\FlaOneByTwo { ( B_0 - D_{2} U_{02}^{T} ) - d_{1} u_{01}^{T} }
             {
               \FlaOneByTwoSingleLine { d_1 = ( b_1 - B_{2} U_{22}^{-T} u_{12} ) \upsilon_{11}^{-1} }
                                      { B_{2} U_{22}^{-T} } } \\
\]
Thus we conclude that
an algorithm that maintains the condition in (\ref{eqn:utrsm2}) is
given in Fig.~\ref{fig:utrsm:collazy_alg}~(left).
Notice that the algorithm overwrites matrix $ B $ with the
result $ B U^{-T} $.

\subsubsection{Blocked algorithm}

In order to derive a blocked algorithm, instead repartition like
\[
  \FlaOneByTwo{ D_{L} }
              { D_{R} } \rightarrow
  \FlaOneByThreeL{ D_{0} }
                 { D_{1} }
                 { D_{2} }
\]
where
$ D_{1} $ is a block of $ b $ columns.
Similarly, we repartition 
\[ 
  \FlaOneByTwo{ B_{L} }
              { B_{R} } \rightarrow
  \FlaOneByThreeL{ B_{0} }
                 { B_{1} }
                 { B_{2} }
\quad \mbox{and} \quad
      \FlaTwoByTwo{ U_{TL} }{ U_{TR} }
                  {   0    }{ U_{BR} } \rightarrow
      \FlaThreeByThreeTL{ U_{00} } { U_{01} } { U_{02} }
                        {   0    } { U_{11} } { U_{12} }
                        {   0    } {    0   } { U_{22} }  
\]
where $ B_{1} $ is a block of $ b $ columns and 
$ U_{11} $ is a $ b \times b $.

Again, the double lines have meaning:
\begin{equation}
\label{eqn:teaml_ltrmmB:meaning}
\FlaOneByTwo  { D_L = \FlaOneByTwoSingleLine{ D_0 }
                                            { D_1 } }
              { D_R = D_2 },
%
\FlaOneByTwo  { B_L = \FlaOneByTwoSingleLine{ B_0 }
                                            { D_1 } }
              { B_R = B_2 }
\end{equation}
\[
\mbox{ and }
%
\FlaTwoByTwo  
              { U_{TL} = \FlaTwoByTwoSingleLine{ U_{00} } { U_{01} }
                                               {   0    } { U_{11} } }
              { U_{TR} = \FlaTwoByOneSingleLine{ U_{02} } 
                                               { U_{12} } }
	      { 0 }
              { U_{BR} = U_{22} } 
\]
Considering (\ref{eqn:utrsm2}) and these repartitionings, 
$ D $ currently contains
\begin{eqnarray*}
\begin{array}{c @{\hspace{1pt}} c @{\hspace{1pt}} c @{\hspace{1pt}} c @{\hspace{1pt}} c} 
\FlaOneByTwo{ B_{L} - D_{R} U_{TR}^{T} }
            { B_{R} U_{BR}^{-T}        }
&=& 
\FlaOneByTwo{ \FlaOneByTwoSingleLine{ B_0 }
                                    { B_1 } -
	      D_2
              \FlaOneByTwoSingleLine{ U_{02}^{T} } 
                                    { U_{12}     } }
             { B_{2}U_{22}^{-T} } \\
&=&
\FlaOneByTwo{ \FlaOneByTwoSingleLine{ B_0 - D_{2}U_{02}^{T} }
                                    { B_1 - D_{2}U_{12}     } }
	    { B_{2}U_{22}^{-T} } \\
&=&
\FlaOneByTwo{ \FlaOneByTwoSingleLine{ B_0 - D_{2}U_{02}^{T}         }
                                    { B_1 - B_{2}U_{22}^{-T}U_{12}  } }
	    { B_{2}U_{22}^{-T} } \\    
\end{array}
\end{eqnarray*}
since $ D_{2} = B_{2}U_{22}^{-T} $. \\
\\
After moving the double lines ahead, in preparation
of the next iteration, 
\begin{equation}
\label{eqn:teaml_ltrmmB:meaning2}
\FlaOneByTwo  { D_L = D_0 }
              { D_R = \FlaOneByTwoSingleLine{ D_1 }
                                            { D_2 } },
\FlaOneByTwo  { B_L = B_0 }
              { B_R = \FlaOneByTwoSingleLine{ B_1 }
                                            { B_2 } }
\end{equation}
\[
\mbox{ and }
\FlaTwoByTwo  { U_{TL} = U_{00} }
              { U_{TR} = \FlaOneByTwoSingleLine{ U_{01} } 
                                               { U_{02} } }
              { 0 }
              { U_{BR} = \FlaTwoByTwoSingleLine{ U_{11} } { U_{12} }
                                               {   0    } { U_{22} } }
\]
Thus, the contents of $ D $ must be updated like:
\begin{eqnarray*}
\begin{array}{c @{\hspace{1pt}} c @{\hspace{1pt}} c @{\hspace{1pt}} c @{\hspace{1pt}} c}
\FlaOneByTwo{ B_{L} - D_{R} U_{TR}^{T} }
            { B_{R} U_{BR}^{-T}        }
&=& 
\FlaOneByTwo { B_0 -
               \FlaOneByTwoSingleLine{ D_1 } { D_2 }
               \FlaTwoByOneSingleLine{ U_{01}^T  } 
                                     { U_{02}^T  } }
             { \FlaOneByTwoSingleLine{ B_{1} } { B_{2} }
	       { \FlaTwoByTwoSingleLine { U_{11} } { U_{12} }
	                                {   0    } { U_{22} } }^{-T} } \\
&=& 
\FlaOneByTwo { B_0 -
               \FlaOneByTwoSingleLine{ D_1 } { D_2 }
               \FlaTwoByOneSingleLine{ U_{01}^T  } 
                                     { U_{02}^T  } }
             { \FlaOneByTwoSingleLine{ B_{1} } { B_{2} }
	       { \FlaTwoByTwoSingleLine { U_{11} } {    0     }
                                        { U_{12} } { U_{22}^T } }^{-1} } \\
&=&
\FlaOneByTwo { B_0 -
               \FlaOneByTwoSingleLine{ D_1 } { D_2 }
               \FlaTwoByOneSingleLine{ U_{01}^T  } 
                                     { U_{02}^T  } }
             { \FlaOneByTwoSingleLine{ B_{1} } { B_{2} }
	       \FlaTwoByTwoSingleLine {  U_{11}^{-1}                    } {       0     }
                                      { -U_{22}^{-T} U_{12} U_{11}^{-1} } { U_{22}^{-T} } } \\
&=&
\FlaOneByTwo { B_0 -
               \FlaOneByTwoSingleLine{ D_1 } { D_2 }
               \FlaTwoByOneSingleLine{ U_{01}^T  } 
                                     { U_{02}^T  } }
             { \FlaOneByTwoSingleLine { B_1 U_{11}^{-1} - B_{2}U_{22}^{-T}U_{12} U_{11}^{-1} }
                                      { B_{2} U_{22}^{-T} } } \\
&=&
\FlaOneByTwo { B_0 -
               \FlaOneByTwoSingleLine{ D_1 } { D_2 }
               \FlaTwoByOneSingleLine{ U_{01}^T  } 
                                     { U_{02}^T  } }
             { \FlaOneByTwoSingleLine { ( B_{1} - B_{2}U_{22}^{-T}U_{12} ) U_{11}^{-1} }
                                      { B_{2} U_{22}^{-T} } } \\
&=&
\FlaOneByTwo { B_0 - ( D_{1} U_{01}^{T} + D_{2} U_{02}^{T} ) }
             { \FlaOneByTwoSingleLine { D_{1} = ( B_{1} - B_{2}U_{22}^{-T}U_{12} ) U_{11}^{-1} }
                                      { B_{2} U_{22}^{-T} } } \\
&=&
\FlaOneByTwo { ( B_0 - D_{2} U_{02}^{T} ) - D_{1} U_{01}^{T} }
             { \FlaOneByTwoSingleLine { D_{1} = ( B_{1} - B_{2}U_{22}^{-T}U_{12} ) U_{11}^{-1} }
                                      { B_{2}U_{22}^{-T} } } \\
\end{array}
\end{eqnarray*}
Thus, the contents of $ D $ must be updated like:
\begin{equation}
\FlaOneByTwo{ \FlaOneByTwoSingleLine{ B_0 - D_{2}U_{02}^{T} }
                                    { B_1 - B_{2}U_{22}^{-T}U_{12}  } }
	    { B_{2}U_{22}^{-T} }
\rightarrow
\end{equation}
\[
\FlaOneByTwo { ( B_0 - D_{2}U_{02}^{T} ) - D_{1}U_{01}^{T} }
             {
               \FlaOneByTwoSingleLine { D_1 = ( B_1 - B_{2}U_{22}^{-T}U_{12} ) U_{11}^{-1} }
                                      { B_{2}U_{22}^{-T} } } \\
\]
Thus we conclude that
an algorithm that maintains the condition in (\ref{eqn:utrsm2}) is
given in Fig.~\ref{fig:utrsm:collazy_alg}~(left).
Notice that the algorithm overwrites matrix $ B $ with the
result $ B U^{-T} $.
\begin{figure}[htbp]
\begin{center}
\begin{tabular}{p{3.1in} | p{3.1in}}
\begin{minipage}{3.1in}
\begin{algorithm}{
\label{alg:unblocked_collazy_utrsm}
\begin{minipage}[t]{2in}
$ B \leftarrow B U^{-T} $ \\(unblocked col.-lazy w.r.t. $ U $)
\end{minipage}
}
\\[0.2in]
\begin{minipage}{4in}
\begin{FlameAlgNarrow}
\FlaPartition{ $
B \rightarrow
     \FlaOneByTwo{ B_{L} } { B_{R} } 
$ }
{
  \FlaWhere{$ B_{R} $ has $ 0 $ columns}
} \\
\FlaPartition { $ 
U \rightarrow 
     \FlaTwoByTwo{ U_{TL} }{ U_{TR} }
                 {   0    }{ U_{BR} } 
$ }
{
  \FlaWhere{ $ U_{BR} $ is $ 0 \times 0 $ }
}
\\
\FlaDoUntil{$ U_{TL} $ is $ 0 \times 0 $} \\
  \\
  \FlaRepartition{ $ 
  \FlaOneByTwo{ B_{L} } { B_{R} } \rightarrow
  \FlaOneByThreeL{ B_{0} }
                 { b_{1} }
                 { B_{2} }
  $ }
  {
    \FlaWhere{$ b_{1} $ is a column}
  } \\
  \FlaRepartition{ $ 
      \FlaTwoByTwo{ U_{TL} }{ U_{TR} }
                  {    0   }{ U_{BR} } \rightarrow
      \FlaThreeByThreeTL{ U_{00} } { u_{01}        } { U_{02}     }
                        {   0    } { \upsilon_{11} } { u_{12}^{T} }
                        {   0    } {      0        } { U_{22}     }  
  $ }
  {
    \FlaWhere{$ \upsilon_{11} $ is a scalar}
  } \\
  \FlaStartCompute \\  % ******************

  $ B_0 \leftarrow B_0 -  b_{1} u_{01}^T $ \\
  $ b_{1} \leftarrow b_{1} \upsilon_{11}^{-1} $ \\
  
  \FlaEndCompute\\     % ******************
  \FlaContinue{ $ 
      \FlaTwoByTwo{ U_{TL} }{ U_{TR} }
                  {    0   }{ U_{BR} } \leftarrow
      \FlaThreeByThreeBR{ U_{00} } { u_{01}        } { U_{02}     }
                        {   0    } { \upsilon_{11} } { u_{12}^{T} }
                        {   0    } {      0        } { U_{22}     }  
  $ } \\ 
  \FlaContinue{ $ 
  \FlaOneByTwo{ B_{L} }
              { B_{R} } \leftarrow
  \FlaOneByThreeR{B_{0}} {b_{1}} {B_{2}}
  $ } \\           % Note: it is important to 
\FlaEndDo \\       % have \\ before \FlaEndDo
\end{FlameAlgNarrow}
\end{minipage}
\end{algorithm}
\end{minipage}
&
\begin{minipage}{3.1in}
\begin{algorithm}{
\label{alg:blocked_collazy_utrsm}
\begin{minipage}[t]{2in}
$ B \leftarrow BU^{-T} $ \\(blocked col.-lazy w.r.t. $ U $)
\end{minipage}
}
\\[0.2in]
\begin{minipage}{4in}
\begin{FlameAlgNarrow}
\FlaPartition{ $
B \rightarrow
     \FlaOneByTwo{ B_{L} } { B_{R} } 
$ }
{
  \FlaWhere{$ B_{R} $ has $ 0 $ columns}
} \\
\FlaPartition { $ 
U \rightarrow 
     \FlaTwoByTwo{ U_{TL} }{ U_{TR} }
                 {   0    }{ U_{BR} } 
$ }
{
  \FlaWhere{ $ U_{BR} $ is $ 0 \times 0 $ }
}
\\
\FlaDoUntil{$ U_{TL} $ is $ 0 \times 0 $} \\
   {\bf determine block size} $ b $ \\
  \FlaRepartition{ $ 
  \FlaOneByTwo{ B_{L} } { B_{R} } \rightarrow
  \FlaOneByThreeL{ B_{0} } { B_{1} } { B_{2} }
  $ }
  {
    \FlaWhere{$ B_{1} $ has $ b $ columns}
  } \\
  \FlaRepartition{ $ 
      \FlaTwoByTwo{ U_{TL} }{ U_{TR} }
                  {    0   }{ U_{BR} } \rightarrow
      \FlaThreeByThreeTL{ U_{00} } { U_{01} } { U_{02} }
                        {   0    } { U_{11} } { U_{12} }
                        {   0    } {   0    } { U_{22} }  
  $ }
  {
    \FlaWhere{$ U_{11} $ is $ b \times b $}
  } \\
  \FlaStartCompute \\  % ******************

  $ B_{0} \leftarrow B_{0} - B_{1}U_{01}^{T} $ \\
  $ B_{1} \leftarrow B_{1}U_{11}^{-1} $ \\

  \FlaEndCompute\\     % ******************
  \FlaContinue{ $ 
      \FlaTwoByTwo{ U_{TL} }{ U_{TR} }
                  {    0   }{ U_{BR} } \leftarrow
      \FlaThreeByThreeBR{ U_{00} } { U_{01} } { U_{02} }
                        {   0    } { U_{11} } { U_{12} }
                        {   0    } {   0    } { U_{22} }  
  $ } \\ 
  \FlaContinue{ $ 
  \FlaOneByTwo{ B_{L} }
              { B_{R} } \leftarrow
  \FlaOneByThreeR{ B_{0} } { B_{1} } { B_{2} }
  $ } \\           % Note: it is important to 
\FlaEndDo \\       % have \\ before \FlaEndDo
\end{FlameAlgNarrow}
\end{minipage}
\end{algorithm}
\end{minipage}
\end{tabular}
\end{center}
\caption{Unblocked (left) and blocked (right) column-lazy triangular solution with
multiple right-hand-sides
algorithms.}
\label{fig:utrsm:collazy_alg}
\end{figure}

\subsection{Lazy algorithm (relative to $ U $)}

Next, consider the condition that currently 
\begin{equation}
\label{eqn:teaml_ltrmm4}
D \quad \mbox{contains}
\quad
\FlaOneByTwo{ B_L }
            { B_R U_{BR}^{-T} }
\end{equation}
Notice that this indicates that only 
$ U_{BR}^{-T} $ has been used
to update $ D $, which will call
a lazy algorithm relative to operand
$ U $.

Again, this algorithm naturally moves through matrices
$ D $ and $ B $ in the ``left'' direction.

\subsubsection{Unblocked algorithm}

Repartition
\[
  \FlaOneByTwo{ D_{L} }
              { D_{R} } \rightarrow
  \FlaOneByThreeL{ D_{0} }
                 { d_{1} }
                 { D_{2} }
\]
where
$ d_{1} $ is a column.
Similarly, we repartition 
\[ 
  \FlaOneByTwo{ B_{L} }
              { B_{R} } \rightarrow
  \FlaOneByThreeL{ B_{0} }
                 { b_{1} }
                 { B_{2} }
\quad \mbox{and} \quad
      \FlaTwoByTwo{ U_{TL} }{ U_{TR} }
                  {   0    }{ U_{BR} } \rightarrow
      \FlaThreeByThreeTL{ U_{00} } { u_{01}        } { U_{02}     }
                        {   0    } { \upsilon_{11} } { u_{12}^{T} }
                        {   0    } {    0          } { U_{22}     }  
\]
where $ b_{1} $ is a column and $ \upsilon_{11} $ is a scalar.

Again, the double lines have the same meaning as illustrated
in (\ref{eqn:teaml_ltrmm:meaning}). Considering (\ref{eqn:teaml_ltrmm4}) and these repartitionings,
$ D $ currently contains
\begin{eqnarray*}
\begin{array}{c @{\hspace{1pt}} c @{\hspace{1pt}} c} 
\FlaOneByTwo{ B_{L} }
            { B_R U_{TR}^{-T} }
&=& 
\FlaOneByThreeL{ B_{0} }
               { b_{1} }
	       { B_{2} U_{22}^{-T} } \\      
\end{array}
\end{eqnarray*}
\\
After moving the double lines ahead, in preparation
of the next iteration, the partitioning takes on the same
meaning in (\ref{eqn:teaml_ltrmm4}) and $ D $ must be updated to contain 
\begin{eqnarray*}
\begin{array}{c @{\hspace{1pt}} c @{\hspace{1pt}} c @{\hspace{1pt}} c @{\hspace{1pt}} c}
&=& 
\FlaOneByTwo { 
	       B_{0} }
             { 
               \FlaOneByTwoSingleLine { b_{1} } { B_{2} }
	       \FlaTwoByTwoSingleLine { \upsilon_{11} } { u_{12}^{T} }
                                      {    0          } { U_{22}     }^{-T}
              } \\
&=& 
\FlaOneByTwo { 
	       B_{0} }
             { 
               \FlaOneByTwoSingleLine { b_{1} } { B_{2} }
	       \FlaTwoByTwoSingleLine { \upsilon_{11} } {     0      }
                                      { u_{12}        } { U_{22}^{T} }^{-1}
              } \\
&=& 
\FlaOneByTwo { 
	       B_{0} }
             { 
               \FlaOneByTwoSingleLine { b_{1} } { B_{2} }
	       \FlaTwoByTwoSingleLine { 1/\upsilon_{11}                     } {     0      }
                                      { -U_{22}^{-T} u_{12} / \upsilon_{11} } { U_{22}^{-T}}
              } \\
&=&
\FlaOneByTwo { 
	       B_{0} }
             { 
               \FlaOneByTwoSingleLine { b_{1} / \upsilon_{11} - B_{2} U_{22}^{-T} u_{12} / \upsilon_{11} } { B_{2} U_{22}^{-T} }
              } \\
&=&
\FlaOneByTwo { 
	       B_{0} }
             { 
               \FlaOneByTwoSingleLine { ( b_{1} - B_{2} U_{22}^{-T} u_{12} ) / \upsilon_{11} } { B_{2} U_{22}^{-T} }
              } \\
&=&
\FlaOneByTwo { 
	       B_{0} }
             { 
               \FlaOneByTwoSingleLine { ( b_{1} - D_{2} u_{12} ) / \upsilon_{11} } { B_{2} U_{22}^{-T} }
              } \\
\end{array}
\end{eqnarray*}
Thus, the contents of $ D $ must be updated like:
\[
\FlaOneByTwo{ \FlaOneByTwoSingleLine{ B_{0} }
                                    { b_{1} } }
	    { B_{2} U_{22}^{-T}  } \\ 
\rightarrow
\FlaOneByTwo { 
	       B_{0} }
             { 
               \FlaOneByTwoSingleLine { ( b_{1} - D_{2} u_{12} ) / \upsilon_{11} } { B_{2} U_{22}^{-T} }
              } \\
\]
Thus we conclude that
an algorithm that maintains the condition in (\ref{eqn:teaml_ltrmm4}) is
given in Fig.~\ref{fig:utrsm:lazy_alg}~(left).
Notice that the algorithm overwrites matrix $ B $ with the
result $ B U^{-T} $.

\subsubsection{Blocked algorithm}

Repartition
\[
  \FlaOneByTwo{ D_{L} }
              { D_{R} } \rightarrow
  \FlaOneByThreeL{ D_{0} }
                 { D_{1} }
                 { D_{2} }
\]
where
$ D_{1} $ is a block of $ b $ columns.
Similarly, we repartition 
\[ 
  \FlaOneByTwo{ B_{L} }
              { B_{R} } \rightarrow
  \FlaOneByThreeL{ B_{0} }
                 { B_{1} }
                 { B_{2} }
\quad \mbox{and} \quad
      \FlaTwoByTwo{ U_{TL} }{ U_{TR} }
                  {   0    }{ U_{BR} } \rightarrow
      \FlaThreeByThreeTL{ U_{00} } { U_{01} }  { U_{02} }
                        {   0    } { U_{11} }  { U_{12} }
                        {   0    } {   0    }  { U_{22} }  
\]
where $ B_{1} $ is a block of $ b $ columns and 
$ U_{11} $ is a $ b \times b $.

Again, the double lines have the same meaning as illustrated
in (\ref{eqn:teaml_ltrmm:meaning}). Considering (\ref{eqn:teaml_ltrmm4}) and these repartitionings,
$ D $ currently contains
\begin{eqnarray*}
\begin{array}{c @{\hspace{1pt}} c @{\hspace{1pt}} c} 
\FlaOneByTwo{ B_{L} }
            { B_R U_{TR}^{-T} }
&=& 
\FlaOneByThreeL{ B_{0} }
               { B_{1} }
	       { B_{2} U_{22}^{-T} } \\      
\end{array}
\end{eqnarray*}
\\
After moving the double lines ahead, in preparation
of the next iteration, the partitioning takes on the same
meaning in (\ref{eqn:teaml_ltrmm4}) and $ D $ must be updated to contain 
\begin{eqnarray*}
\begin{array}{c @{\hspace{1pt}} c @{\hspace{1pt}} c @{\hspace{1pt}} c @{\hspace{1pt}} c}
&=& 
\FlaOneByTwo { 
	       B_{0} }
             { 
               \FlaOneByTwoSingleLine { B_{1}  } { B_{2}  }
	       \FlaTwoByTwoSingleLine { U_{11} } { U_{12} }
                                      {    0   } { U_{22} }^{-T}
              } \\
&=& 
\FlaOneByTwo { 
	       B_{0} }
             { 
               \FlaOneByTwoSingleLine { B_{1}  } { B_{2}  }
	       \FlaTwoByTwoSingleLine { U_{11}^{T} } {     0      }
                                      { U_{12}^{T} } { U_{22}^{T} }^{-1}
              } \\
&=& 
\FlaOneByTwo { 
	       B_{0} }
             { 
               \FlaOneByTwoSingleLine { B_{1}       } { B_{2} }
	       \FlaTwoByTwoSingleLine { U_{11}^{-T} } {     0      }
                                      { -U_{22}^{-T} U_{12}^{T} U_{11}^{-T} } { U_{22}^{-T}}
              } \\
&=&
\FlaOneByTwo { 
	       B_{0} }
             { 
               \FlaOneByTwoSingleLine { B_{1} U_{11}^{-T} - B_{2} U_{22}^{-T} U_{12}^{T} U_{11}^{-T} } { B_{2} U_{22}^{-T} }
              } \\
&=&
\FlaOneByTwo { 
	       B_{0} }
             { 
               \FlaOneByTwoSingleLine { ( B_{1} - B_{2} U_{22}^{-T} U_{12}^{T} ) U_{11}^{-T} } { B_{2} U_{22}^{-T} }
              } \\
&=&
\FlaOneByTwo { 
	       B_{0} }
             { 
               \FlaOneByTwoSingleLine { ( B_{1} - D_{2} U_{12}^{T} ) U_{11}^{-T} } { B_{2} U_{22}^{-T} }
              } \\
\end{array}
\end{eqnarray*}
Thus, the contents of $ D $ must be updated like:
\[
\FlaOneByTwo{ \FlaOneByTwoSingleLine{ B_{0} }
                                    { B_{1} } }
	    { B_{2} U_{22}^{-T}  } \\ 
\rightarrow
\FlaOneByTwo { 
	       B_{0} }
             { 
               \FlaOneByTwoSingleLine { ( B_{1} - D_{2} U_{12}^{T} ) U_{11}^{-T} } { B_{2} U_{22}^{-T} }
              } \\
\]
Thus we conclude that
an algorithm that maintains the condition in (\ref{eqn:teaml_ltrmm4}) is
given in Fig.~\ref{fig:utrsm:lazy_alg}~(left).
Notice that the algorithm overwrites matrix $ B $ with the
result $ B U^{-T} $.

\begin{figure}[htbp]
\begin{center}
\begin{tabular}{p{3.1in} | p{3.1in}}
\begin{minipage}{3.1in}
\begin{algorithm}{
\label{alg:unblocked_lazy_utrsm}
\begin{minipage}[t]{2in}
$ B \leftarrow B U^{-T} $ \\(unblocked lazy w.r.t. $ U $)
\end{minipage}
}
\\[0.2in]
\begin{minipage}{4in}
\begin{FlameAlgNarrow}
\FlaPartition{ $
B \rightarrow
     \FlaOneByTwo{ B_{L} } { B_{R} } 
$ }
{
  \FlaWhere{$ B_{R} $ has $ 0 $ columns}
} \\
\FlaPartition { $ 
U \rightarrow 
     \FlaTwoByTwo{ U_{TL} }{ U_{TR} }
                 {   0    }{ U_{BR} } 
$ }
{
  \FlaWhere{ $ U_{BR} $ is $ 0 \times 0 $ }
}
\\
\FlaDoUntil{$ B_{L} $ has $ 0 $ columns} \\
  \\
  \FlaRepartition{ $ 
  \FlaOneByTwo{ B_{L} } { B_{R} } \rightarrow
  \FlaOneByThreeL{ B_{0} }
                 { b_{1} }
                 { B_{2} }
  $ }
  {
    \FlaWhere{$ b_{1} $ is a column}
  } \\
  \FlaRepartition{ $ 
      \FlaTwoByTwo{ U_{TL} }{ U_{TR} }
                  {    0   }{ U_{BR} } \rightarrow
      \FlaThreeByThreeTL{ U_{00} } { u_{01} }       { U_{02} }
                        {   0    } { \upsilon_{11}} { u_{12}^{T} }
                        {   0    } {    0   }       { U_{22} }  
  $ }
  {
    \FlaWhere{$ \upsilon_{11} $ is a scalar}
  } \\
  \FlaStartCompute \\  % ******************

  $ b_{1} \leftarrow  ( b_{1} - B_{2} u_{12} ) \upsilon_{11}^{-1}  $ \\

  \FlaEndCompute\\     % ******************
  \FlaContinue{ $ 
      \FlaTwoByTwo{ U_{TL} }{ U_{TR} }
                  {    0   }{ U_{BR} } \leftarrow
      \FlaThreeByThreeBR{ U_{00} } { u_{01} }       { U_{02} }
                        {   0    } { \upsilon_{11}} { u_{12}^{T} }
                        {   0    } {    0   }       { U_{22} }  
  $ } \\ 
  \FlaContinue{ $ 
  \FlaOneByTwo{ B_{L} }
              { B_{R} } \leftarrow
  \FlaOneByThreeR{B_{0}} {b_{1}} {B_{2}}
  $ } \\           % Note: it is important to 
\FlaEndDo \\       % have \\ before \FlaEndDo
\end{FlameAlgNarrow}
\end{minipage}
\end{algorithm}
\end{minipage}
&
\begin{minipage}{3.1in}
\begin{algorithm}{
\label{alg:blocked_lazy_utrsm}
\begin{minipage}[t]{2in}
$ B \leftarrow BU^{-T} $ \\(blocked lazy w.r.t. $ U $)
\end{minipage}
}
\\[0.2in]
\begin{minipage}{4in}
\begin{FlameAlgNarrow}
\FlaPartition{ $
B \rightarrow
     \FlaOneByTwo{ B_{L} } { B_{R} } 
$ }
{
  \FlaWhere{$ B_{L} $ has $ 0 $ columns}
} \\
\FlaPartition { $ 
U \rightarrow 
     \FlaTwoByTwo{ U_{TL} } { U_{TR} }
                 {   0    } { U_{BR} } 
$ }
{
  \FlaWhere{ $ U_{BR} $ is $ 0 \times 0 $ }
}
\\
\FlaDoUntil{$ U_{BR} $ is $ 0 \times 0 $} \\
   {\bf determine block size} $ b $ \\
  \FlaRepartition{ $ 
  \FlaOneByTwo{ B_{L} } { B_{R} } \rightarrow
  \FlaOneByThreeL{B_{0}} {B_{1}} {B_{2}}
  $ }
  {
    \FlaWhere{$ B_{1} $ has $ b $ columns}
  } \\
  \FlaRepartition{ $ 
      \FlaTwoByTwo{ U_{TL} } { U_{TR} }
                  {    0   } { U_{BR} } \rightarrow
      \FlaThreeByThreeTL{ U_{00} } { U_{01} } { U_{02} }
                        {   0    } { U_{11} } { U_{12} }
                        {   0    } {   0    } { U_{22} }  
  $ }
  {
    \FlaWhere{$ U_{11} $ is $ b \times b $}
  } \\
  \FlaStartCompute \\  % ******************

  $ B_{1} \leftarrow ( B_{1} - B_{2} U_{12}^{T} ) U_{11}^{-T} $ \\

  \FlaEndCompute\\     % ******************
  \FlaContinue{ $ 
      \FlaTwoByTwo{ U_{TL} }{ U_{TR} }
                  {    0   }{ U_{BR} } \leftarrow
      \FlaThreeByThreeBR{ U_{00} } { U_{01} } { U_{02} }
                        {   0    } { U_{11} } { U_{12} }
                        {   0    } {   0    } { U_{22} }  
  $ } \\ 
  \FlaContinue{ $ 
  \FlaOneByTwo{ B_{L} }
              { B_{R} } \leftarrow
  \FlaOneByThreeR{B_{0}} {B_{1}} {B_{2}}
  $ } \\           % Note: it is important to 
\FlaEndDo \\       % have \\ before \FlaEndDo
\end{FlameAlgNarrow}
\end{minipage}
\end{algorithm}
\end{minipage}
\end{tabular}
\end{center}
\caption{Unblocked (left) and blocked (right) lazy triangular solution with
multiple right-hand-sides
algorithms.}
\label{fig:utrsm:lazy_alg}
\end{figure}

\section{Implementation}

These codes, and related test routines, can be found at
\begin{center}
\tt
http://www.cs.utexas.edu/users/flame/materials/trsm\_rut/
\end{center}

\begin{figure}[htbp]
\footnotesize
\begin{quote}
\listinginput{1}{materials/trsm_rut/lazy-collazy/Trsm_right_upper_trans_unb.c}
\end{quote}
\caption{Unblocked lazy and column-lazy (w.r.t. $ U $) triangular 
solve with multiple right-hand-sides using FLAME.}
\label{fig:trsm_rut_unb}
\end{figure}

\begin{figure}[htbp]
\footnotesize
\begin{quote}
\listinginput{1}{materials/trsm_rut/lazy-collazy/Trsm_right_upper_trans_blk.c}
\end{quote}
\caption{Blocked lazy and column-lazy (w.r.t. $ U $) triangular 
solve with multiple right-hand-sides using FLAME.}
\label{fig:trsm_rut_blk}
\end{figure}

\section{Performance}

\begin{figure}[htbp]
\begin{center}
\begin{tabular}{c | c}
Column-lazy & Lazy \\ \hline
& \\
%\psfig{figure=trsm_rut/graphs/trsm_rut_collazy_wrt_U_32.eps,width=3.0in,height=3.0in} &
%\psfig{figure=trsm_rut/graphs/trsm_rut_lazy_wrt_U_32.eps,width=3.0in,height=3.0in}
\psfig{figure=materials/trsm_rut/lazy-collazy/trsm_rut_collazy_wrt_U_32.eps,width=3.0in,height=3.0in} &
\psfig{figure=materials/trsm_rut/lazy-collazy/trsm_rut_lazy_wrt_U_32.eps,width=3.0in,height=3.0in}
\\ \hline
& \\
%\psfig{figure=trsm_rut/graphs/trsm_rut_collazy_wrt_U_128.eps,width=3.0in,height=3.0in} &
%\psfig{figure=trsm_rut/graphs/trsm_rut_lazy_wrt_U_128.eps,width=3.0in,height=3.0in}
\psfig{figure=materials/trsm_rut/lazy-collazy/trsm_rut_collazy_wrt_U_128.eps,width=3.0in,height=3.0in} &
\psfig{figure=materials/trsm_rut/lazy-collazy/trsm_rut_lazy_wrt_U_128.eps,width=3.0in,height=3.0in}
\end{tabular}
\end{center}
\caption{Performance of the column-lazy (left) and lazy (right) (w.r.t. $ U $) 
triangular solve with multiple right-hand-sides
for a block size of $ 32 $ (top) and
$ 128 $ (bottom).
For these experiments, the ITXGEMM matrix-matrix multiplication
kernel was used.}
\label{perf:trsm_rut_ITXGEMM}
\end{figure}
\index{op}{triangular solve!multiple RHSs!$ B \leftarrow B U^{-T} $|)}%
