\begin{figure}
\begin{center}
\footnotesize
% \begin{sideways}
{
\setlength{\tabcolsep}{4pt}
\begin{tabular}{| c | c | c || c | p{1.1in} | c | } \hline
\multicolumn{3}{| c ||}{Computed?} & $ P_X: $
& & \\ \cline{1-3}

\footnotesize
$ L_{TL} B_T $ & 
\footnotesize
$ L_{BL} B_T $ & 
\footnotesize
$ L_{BR} B_B $ &
$ \FlaTwoByOne{ B_T}{B_B} =  $ & Comment &
\footnotesize Feasible? \\ \hline \hline
NO & NO & NO &
\footnotesize
$
\FlaTwoByOne{ \hat{B}_T }
            { \hat{B}_B }
$ 
&
Infeasible (Reason 2).
&
NO
\\ \hline
%
%
YES & NO & NO &
\footnotesize
$
\FlaTwoByOne{ L_{TL} \hat{B}_T }
            { \hat{B}_B }
$ 
&
Infeasible (Reason 1).
& 
NO
\\ \hline
NO & YES & NO &
%
%
\footnotesize
$
\FlaTwoByOne{ \hat{B}_T }
            { L_{BL} \hat{B}_T  }
$ 
&
Infeasible (Reason 1).
& 
NO
\\ \hline
%
%
YES & YES & NO &
\footnotesize
$
\FlaTwoByOne{ L_{TL} \hat{B}_T }
            { L_{BL} \hat{B}_T }
$ 
&
Infeasible (Reason 1).
&
NO
\\ \hline
NO & NO & YES &
%
%
\footnotesize
$
\FlaTwoByOne{ \hat{B}_T }
            { L_{BR} \hat{B}_B }
$ 
&
Loop-invariant 1
&
YES
\\ \hline
YES & NO & YES &
%
%
\footnotesize
$
\FlaTwoByOne{ L_{TL} \hat{B}_T }
            { L_{BR} \hat{B}_B }
$ 
&
Infeasible (Reason 1).
&
NO
\\ \hline
%
%
NO & YES & YES &
\footnotesize
$
\FlaTwoByOne{ \hat{B}_T }
            { L_{BL} \hat{B}_T  + L_{BR} \hat{B}_B }
$ 
&
Loop-invariant 2
&
YES
\\ \hline
YES & YES & YES &
%
%
\footnotesize
$
\FlaTwoByOne{ L_{TL} \hat{B}_T }
            { L_{BL} \hat{B}_T  + L_{BR} \hat{B}_B }
$ 
&
Infeasible (Reason 3).
&
NO
\\ \hline
%\\ \hline
%
%
\end{tabular}
}
% \end{sideways}
\end{center}
\caption{Possible loop-invariants when partitioning
$ L $ into quadrants.
Here $ P_X $ is the most prominent part of the loop-invariant
$ \PInv $.
The reasons for declaring the loop-invariant infeasible
refer to the reasons given on page 11 of the handout.
The reason for rejecting the loop-invariant may not be
the only reason for doing so.}
\label{fig:LTRMM_LLN_example}
\end{figure}
