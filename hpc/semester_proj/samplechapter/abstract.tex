In this document we present a set of class projects by students in
an upper-division undergraduate course titled ``High-Performance
Parallel Computing'' taught by us in the Fall of 2002 at The
University of Texas at Austin.  A similar document was created in
Spring 2001 and Spring 2002 for the same course.  
The idea behind the projects is to
illustrate how formal derivation techniques can be used to derive
families of algorithms for given linear algebra operations.

The philosophy behind the approach is that one should start by
systematically deriving the algorithms.  The methodology used
this time is a refinement of the ``recipe'' used in Spring 2001.
By providing a set of \LaTeX\nocite{LATEX}
macros for the students to use, it becomes convenient and
beneficial to first generate a careful documentation of the
derivation.  Once one or more algorithms have been so
developed, they can be translated into implementations 
using an Application Programming Interface (API),
the Formal Linear Algebra Methods Environment
(FLAME).  This library allows the code to look much like the
algorithms, which greatly reduces the opportunity for
the introduction of logical and typographical errors.  
For all examples in the report we
demonstrate that high performance can be attained on an Intel Pentium
(R) III processor.

As part of the class, the students were organized into
small teams and a different linear algebra operation
was assigned to each team.
The names of the members of the
teams are given as the authors of the chapter on the operation.
Thus we attempt to show that the formal derivation approach makes the
development and implementation of high-performance
algorithms for dense linear algebra operations accessible to
novices.

As was the case for the document generated from the 
class projects for the other times this course
was offered,
this document is meant to capture progress achieved
during a single semester.  Thus, it should
be viewed as a work-in-progress rather than a polished
product.
