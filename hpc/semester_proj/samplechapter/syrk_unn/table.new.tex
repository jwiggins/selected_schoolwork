\begin{figure}
\begin{center}
\footnotesize
% \begin{sideways}
{
\setlength{\tabcolsep}{4pt}
\begin{tabular}{| c | p{2.5in} | c | } \hline
$ P_X: $

$ \FlaTwoByTwo{ C_{TL} } { C_{TR} }
              { C_{BL} } { C_{BR} } =  $ & Comment &
\footnotesize Feasible? \\ \hline \hline
\footnotesize
$
\FlaTwoByTwo{ \hat{C}_{TL} }  { \hat{C}_{TR} }
            { \undetermined } { \hat{C}_{BR} }
$ 
&
&
NO
\\ \hline
%
%
\footnotesize
$
\FlaTwoByTwo{ A_{T} A_{T}^T + \hat{C}_{TL} }  { A_{T} A_{B}^T + \hat{C}_{TR} }
            { \undetermined }                 { A_{B} A_{B}^T + \hat{C}_{BR} }
$ 
& 
NO
\\ \hline
%
%
\footnotesize
$
\FlaTwoByTwo{ A_{T} A_{T}^T + \hat{C}_{TL} }  { \hat{C}_{TR} }
            { \undetermined }                 { \hat{C}_{BR} }
$ 
& 
YES
\\ \hline
%
%
\footnotesize
$
\FlaTwoByTwo{ \hat{C}_{TL} }  { A_{T} A_{B}^T + \hat{C}_{TR} } 
            { \undetermined } { \hat{C}_{BR} }
$ 
&
NO
\\ \hline
%
%
\footnotesize
$
\FlaTwoByTwo{ \hat{C}_{TL} }   { \hat{C}_{TR} }
            { \undetermined }  { A_{B} A_{B}^T + \hat{C}_{BR} }
$ 
&
Loop-invariant 1
&
YES
\\ \hline
%
%
\footnotesize
$
\FlaTwoByTwo{ \hat{C}_{TL} } { A_{T} A_{B}^T + \hat{C}_{TR} }
            { \undetermined} { A_{B} A_{B}^T + \hat{C}_{BR} }
$ 
&
YES
\\ \hline
%
%
\footnotesize
$
\FlaTwoByTwo{ A_{T} A_{T}^T + \hat{C}_{TL} } { \hat{C}_{TR} }
            { \undetermined }                { A_{B} A_{B}^T + \hat{C}_{BR} }
$ 
&
YES
\\ \hline
%
%
\footnotesize
$
\FlaTwoByTwo{ A_{T} A_{T}^T + \hat{C}_{TL} } { A_{T} A_{B}^T + \hat{C}_{TR} }
            { \undetermined }                { \hat{C}_{BR} }
$ 
&
Loop-invariant 2.
&
YES
\\ \hline
%\\ \hline
%
%
\end{tabular}
}
% \end{sideways}
\end{center}
\caption{Possible loop-invariants when partitioning
$ C $ into quadrants.
Here $ P_X $ is the most prominent part of the loop-invariant
$ \PInv $.}
\label{fig:USYRK_UNN_example}
\end{figure}
