% The following commands help in generating the index
\index{index}{symmetric rank-k update}%
\index{op}{symmetric rank-k update!$ C \leftarrow A A^T + \hat{C} $|( }%

% Name of the chapter

\chapter{Symmetric Rank-k Update \\
$ C \leftarrow A A^T + \hat{C} $ \\
$ C $ symmetric, stored in upper triangle
}
\label{chapter:usyrk_unn}

% Authors

\ChapterAuthor{
\index{author}{Chen, Katherine}
Katherine Chen
\\[0.1in]
\index{author}{Ho, Mandy}
Mandy Ho
\\[0.1in]
\index{author}{Lai, Ingrid}
Ingrid Lai
}

% Add the names of the authors to the table of contents

\addtocontents{toc}{ {by {\bf Katherine Chen, Mandy Ho, and Ingrid Lai}}}

In this chapter, we discuss the implementation of the symmetric rank-k update
\[
C \leftarrow A A^T + \hat{C}
\]
where $ C $ is an $ n \times n $ symmetric matrix and $ A $ is
$ n \times k $.  We will overwrite $ C $ with the result without
requiring a workarray.  We start by deriving a number of different
sequential algorithms.  Subsequently, we show how to code the
algorithms using FLAME.  

The variables for the symmetric rank-k update can be
described by the precondition
\[
\PPre:
C = \hat{C} \wedge \SameSize( C, \hat{C} ) \wedge \Symm( C ) \wedge
\RowDim( A )=\RowDim(\hat{C}),
\]
where $ \hat{C} $ indicates the original contents of $ C $.  Here the
predicate $ \Symm( C ) $ is true iff $ C $ is a symmetric
matrix.  The functions $ \RowDim( C ) $, $ \RowDim( A ) $  and $ \ColDim( C ) $,
$ \ColDim( A ) $ return
the row and column dimension of $ C $ and $ A $, respectively.  The operation to
be performed, $ C \becomes A A^T + C $, translates to the postcondition $
\PPost: C = A A^T + \hat{C} $.

\section{Algorithms That Start By Partitioning $ C $}
\label{sec:trmm_lln:L}

Let us start by partitioning matrix $ C $.  Since it is symmetric, we
partition it like
\[
C \rightarrow \FlaTwoByTwo{ C_{TL} }        { C_{TR} }
                          { \undetermined } { C_{BR} },
\]
where $ C_{TL} $ is square so that both submatrices on the diagonal
are themselves symmetric.  Notice that $ \undetermined $ indicates
that that part of the matrix is not stored.

Substituting the partitioning of $ C $ into the postcondition yields
\[
\FlaTwoByTwo{ C_{TL} }        { C_{TR} }
            { \undetermined } { C_{BR} }
= 
( \mbox{some partitioning of }{A} )
( \mbox{some partitioning of }{A^T} )
+
\FlaTwoByTwo{ \hat{C}_{TL} }  { \hat{C}_{TR} }
            { \undetermined } { \hat{C}_{BR} }
\]
This suggests that $ A $ should be partitioned
horizontally into two submatrices, or into quadrants.  Let us consider
the case where $ A $ is partitioned horizontally.  Then
\[
\FlaTwoByTwo{ C_{TL} }       { C_{TR} }
            { \undetermined} { C_{BR} }
= 
\FlaTwoByOne{ A_T }
            { A_B }
\FlaOneByTwo{ {A_T}^T }
            { {A_B}^T }
+
\FlaTwoByTwo{ \hat{C}_{TL} }  { \hat{C}_{TR} }
            { \undetermined } { \hat{C}_{BR} }
\]
In order to be able to multiply the matrices on the right out and to
be able to then set the submatrices on the left equal to the result on
the right we find that the following must hold:
\[
\ColDim( C_{TL} ) = \RowDim( A_T ) 
\]
Substituting the partitioned matrices into the postcondition, we find
\[
\FlaTwoByTwo{ C_{TL} }       { C_{TR} }
            { \undetermined} { C_{BR} }
=
\FlaTwoByOne{ A_T }
            { A_B }
\FlaOneByTwo{ {A_T}^T }
            { {A_B}^T }
+
\FlaTwoByTwo{ \hat{C}_{TL} }  { \hat{C}_{TR} }
            { \undetermined } { \hat{C}_{BR} }
=
\FlaTwoByTwo{ A_T {A_T}^T + \hat{C}_{TL} } { A_T {A_B}^T + \hat{C}_{TR} }
            { \undetermined }              { A_B {A_B}^T + \hat{C}_{BR} }
\]
which yields the equalities
\begin{equation}
\label{eqn:usyrk_unn}
\begin{array}{r c l || r c l}
C_{TL} &=& A_T {A_T}^T + \hat{C}_{TL} & C_{TR} &=& A_T {A_B}^T + \hat{C}_{TR} \\ \hline \hline
C_{BL} &=& \undetermined              & C_{BR} &=& A_B {A_B}^T + \hat{C}_{BR}
\end{array}
\end{equation}

% Insert the table of possible loop-invariants from
% file syrk_unn/table.new.tex

\begin{figure}
\begin{center}
\footnotesize
% \begin{sideways}
{
\setlength{\tabcolsep}{4pt}
\begin{tabular}{| c | p{2.5in} | c | } \hline
$ P_X: $

$ \FlaTwoByTwo{ C_{TL} } { C_{TR} }
              { C_{BL} } { C_{BR} } =  $ & Comment &
\footnotesize Feasible? \\ \hline \hline
\footnotesize
$
\FlaTwoByTwo{ \hat{C}_{TL} }  { \hat{C}_{TR} }
            { \undetermined } { \hat{C}_{BR} }
$ 
&
&
NO
\\ \hline
%
%
\footnotesize
$
\FlaTwoByTwo{ A_{T} A_{T}^T + \hat{C}_{TL} }  { A_{T} A_{B}^T + \hat{C}_{TR} }
            { \undetermined }                 { A_{B} A_{B}^T + \hat{C}_{BR} }
$ 
& 
NO
\\ \hline
%
%
\footnotesize
$
\FlaTwoByTwo{ A_{T} A_{T}^T + \hat{C}_{TL} }  { \hat{C}_{TR} }
            { \undetermined }                 { \hat{C}_{BR} }
$ 
& 
YES
\\ \hline
%
%
\footnotesize
$
\FlaTwoByTwo{ \hat{C}_{TL} }  { A_{T} A_{B}^T + \hat{C}_{TR} } 
            { \undetermined } { \hat{C}_{BR} }
$ 
&
NO
\\ \hline
%
%
\footnotesize
$
\FlaTwoByTwo{ \hat{C}_{TL} }   { \hat{C}_{TR} }
            { \undetermined }  { A_{B} A_{B}^T + \hat{C}_{BR} }
$ 
&
Loop-invariant 1
&
YES
\\ \hline
%
%
\footnotesize
$
\FlaTwoByTwo{ \hat{C}_{TL} } { A_{T} A_{B}^T + \hat{C}_{TR} }
            { \undetermined} { A_{B} A_{B}^T + \hat{C}_{BR} }
$ 
&
YES
\\ \hline
%
%
\footnotesize
$
\FlaTwoByTwo{ A_{T} A_{T}^T + \hat{C}_{TL} } { \hat{C}_{TR} }
            { \undetermined }                { A_{B} A_{B}^T + \hat{C}_{BR} }
$ 
&
YES
\\ \hline
%
%
\footnotesize
$
\FlaTwoByTwo{ A_{T} A_{T}^T + \hat{C}_{TL} } { A_{T} A_{B}^T + \hat{C}_{TR} }
            { \undetermined }                { \hat{C}_{BR} }
$ 
&
Loop-invariant 2.
&
YES
\\ \hline
%\\ \hline
%
%
\end{tabular}
}
% \end{sideways}
\end{center}
\caption{Possible loop-invariants when partitioning
$ C $ into quadrants.
Here $ P_X $ is the most prominent part of the loop-invariant
$ \PInv $.}
\label{fig:USYRK_UNN_example}
\end{figure}


From (\ref{eqn:usyrk_unn}) we conclude that the operations to be
performed are $ A_T {A_T}^T + \hat{C}_{TL} $, $ A_T {A_B}^T + \hat{C}_{TR} $, and $ A_B {A_B}^T + 
\hat{C}_{BR} $.  At an intermediate stage, each of these either has or
has not already been computed, leading to the $ 2^3 = 8 $ possible
loop-invariants tabulated in Fig.~\ref{fig:USYRK_UNN_example}.

\subsection{Loop-invariant 1}

We will first examine the (feasible) loop-invariant
\begin{equation}
\label{eqn:usyrk_unn:p2}
\PInv: \FlaTwoByTwo{ C_{TL} } { C_{TR} }
            { \undetermined}  { C_{BR} }
 =
\FlaTwoByTwo{ \hat{C}_{TL} }  { \hat{C}_{TR} }
            { \undetermined } { A_B {A_B}^T + \hat{C}_{BR} }
\wedge
\ldots
\end{equation}
Comparing the loop-invariant in (\ref{eqn:usyrk_unn:p2}) with the
postcondition $ C = A A^T +  \hat{C} $ we see that {\em if} $C=C_{BR}$, $ A =
A_B $, and $ \hat{C} = \hat{C}_{BR} $ then the loop-invariant implies the
postcondition, i.e., that the desired result has been computed.
Notice that $ \SameSize( C, C_{BR} ) \wedge \PInv $ implies $ C =
C_{BR} $, $ A = A_B $ and $ \hat{C}=\hat{C}_{BR} $ since the partitioned
matrices are references into the original matrices $ C $, $ A $, and $
\hat{C} $.  Thus, the loop-guard $ G: \neg \SameSize( C, C_{BR} ) $
has the desired property.

Consider the initialization in Step 4 in Fig.~\ref{fig:ws:usyrk_unn}.
The fact that for this partitioning of $ C $, $ \hat{C} $, and $ A $,
$
\FlaTwoByTwo{ C_{TL} }       { C_{TR} }
            { \undetermined} { C_{BR} }
=
\FlaTwoByTwo{ \hat{C}_{TL} }  { \hat{C}_{TR} }
            { \undetermined } { A_B {A_B}^T + \hat{C}_{BR} }
$ and the precondition implies the
other parts of the loop-invariant, this initialization has the desired
properties.

Loop-guard $ G $ indicates that eventually $ C_{BR} $ should equal all
of $ C $, at which point $ G $ becomes {\em false} and the loop is
exited.  After the initialization, $ C_{BR} $ is $ 0 \times 0 $.
Thus, the repartitioning should be such that as the computation
proceeds, rows and columns are subtracted from $ C_{TL} $ and added to
$ C_{BR} $ while updating $ C_{BR} $ consistently with this.

\subsubsection{Unblocked algorithm}

If we move the partitionings by individual rows and columns, we obtain
the repartitioning and redefinition of the partitioning in Steps 5a
and 5b in Fig.~\ref{fig:ws:usyrk_unn}.  The repartitionings in Step 5a
in Fig.~\ref{fig:ws:usyrk_unn} result in the state
\begin{equation}
\label{eqn:usyrk_unn:bu}
\QBefore: 
\FlaThreeByThreeTL{ C_{00} }  { c_{01} }  { C_{02} }
                  {\undetermined}{ \gamma_{11}}     { c_{12}^{T} }
                  { \undetermined }  { \undetermined }  { C_{22} }
=
\FlaTwoByTwo{ \FlaTwoByTwoSingleLine{ \hat{C}_{00} }   { \hat{c}_{01} }
                           { \undetermined} { \hat{\gamma}_{11}   } }   
            { \FlaTwoByOneSingleLine{ \hat{C}_{02} }{ \hat{c}_{12}^{T} } }
            { \undetermined }  
            { A_2 A_2^{T} + \hat{C}_{22}}
\wedge \ldots
\end{equation}
before the update.  The redefinition in Step 5b in
Fig.~\ref{fig:ws:usyrk_unn} means that the following state must result
from the update:
\[
\QAfter: 
\FlaThreeByThreeBR{ C_{00} }  { c_{01} }  { C_{02} }
                  { \undetermined }{ \gamma_{11}}     { c_{12}^{T} }
                  { \undetermined }  { \undetermined }         { C_{22}}
=
\FlaTwoByTwo{ \hat{C}_{00} }
            { \FlaOneByTwoSingleLine{\hat{c}_{01}}
                                    {\hat{C}_{02}}
            }
            {\undetermined  }
{
 \FlaTwoByOneSingleLine{a_{1}^T}
                       {A_{2}} 
 \FlaTwoByOneSingleLine{a_{1}^T}
                       {A_{2}}^T +
\FlaTwoByTwoSingleLine{ \hat{\gamma}_{11} }{ \hat{c}_{12}^{T} }
            { \undetermined } { \hat{C}_{22} }
}
\wedge \ldots
\]
Multiplying out the symmetric rank-1 update yields
\begin{equation}
\label{eqn:usyrk_unn:au}
\QAfter: 
\FlaThreeByThreeBR{ C_{00} }  { c_{01} }  { C_{02} }
                  { \undetermined }{ \gamma_{11}}     { c_{12}^{T} }
                  { \undetermined }  { \undetermined }         { C_{22}}
=
\FlaTwoByTwo { \hat{C}_{00} }
             { \FlaOneByTwoSingleLine{ \hat{c}_{01} }{ \hat{C}_{02} } }
             { \undetermined }
             { \FlaTwoByTwoSingleLine{ \hat{\gamma}_{11} + a_1^T a_1 }{ a_1^T A_2^T + \hat{c}_{12}^{T} }
                                     { \undetermined }{ A_2 A_2^{T} + \hat{C}_{22}}}
\wedge \ldots
\end{equation}
Comparing Eqns.~\ref{eqn:usyrk_unn:bu} and~\ref{eqn:usyrk_unn:au} we
find that the updates
\begin{eqnarray*}
& \gamma_{11} & \becomes \gamma_{11} + a_1^T a_1 \\
& c_{12} & \becomes c_{12} +  A_2 a_1
\end{eqnarray*}
are required to change the state from $ \QBefore $ to $ \QAfter $.

% The following commands will in the ``worksheet''
% given in Fig. 4.2

% Step 0: Operation
\renewcommand{\operation}{C \becomes A A^T + C}

% Step 1a: Precondition
\renewcommand{\precondition}{C = \hat{C} \wedge \ldots }

% Step 1b: Postcondition
\renewcommand{\postcondition}{C = A A^T + \hat{C}}

% Step 2: Loop-invariant
\renewcommand{\invariant}{
\FlaTwoByTwo{ C_{TL} }       { C_{TR} }
            { \undetermined} { C_{BR} }
=
\FlaTwoByTwo{ \hat{C}_{TL} }  { \hat{C}_{TR} }
            { \undetermined } { A_B {A_B}^T + \hat{C}_{BR} }
\wedge
\ldots
}

% Step 3: Loop-guard
\renewcommand{\guard}{ \neg \SameSize( C, C_{BR} ) }

% Step 4: Initialization
\renewcommand{\partitionings}{
$
C \rightarrow \FlaTwoByTwo{ C_{TL} }{ C_{TR} }
                          { C_{TR}^{T} }{ C_{BR} }
$,
$ 
\hat{C} \rightarrow \FlaTwoByTwo{ \hat{C}_{TL} }{ \hat{C}_{TR} }
                            { \hat{C}_{TR}^{T} }{ \hat{C}_{BR} }
$, and
$ 
A \rightarrow \FlaTwoByOne{ A_{T} }
                          { A_{B} }
$
}
\renewcommand{\partitionsizes}{
$ C_{BR} $ and $ \hat{ C }_{BR} $ are  $ 0 \times 0 $ 
and $ A_{B} $ has $ 0 $ rows.
}

% Step 5a: repartitioning
\renewcommand{\repartitionings}{
$ 
\FlaTwoByTwo{ C_{TL} }{ C_{TR}}
            { \undetermined }{ C_{BR} } 
\rightarrow 
\FlaThreeByThreeTL{ C_{00} }  { c_{01} }  { C_{02} }
                  {\undetermined}{ \gamma_{11}}     { c_{12}^{T} }
                  { \undetermined }  { \undetermined }  { C_{22} }
,
\FlaTwoByTwo{ \hat{C}_{TL} }{ \hat{C}_{TR}}
            { \undetermined }{ \hat{C}_{BR} } 
\rightarrow 
\FlaThreeByThreeTL{ \hat{C}_{00} }  { \hat{c}_{01} }  { \hat{C}_{02} }
                  {\undetermined}{ \hat{\gamma}_{11}}     { \hat{c}_{12}^{T} }
                  { \undetermined }  { \undetermined }  { \hat{C}_{22} }
,
$ \\
and
$
\FlaTwoByOne{ A_{T} }
            { A_{B} } 
\rightarrow
\FlaThreeByOneT{ A_{0} }
               { a_{1}^T }
               { A_{2} }
$
}
\renewcommand{\repartitionsizes}{
$ \gamma_{11} $ and $ \hat{\gamma}_{11} $ are scalars 
and $ a_{1}^T $ is a row.
}

% Step 5b: moving the boundaries
\renewcommand{\moveboundaries}{%
$ 
\FlaTwoByTwo{ C_{TL} }{ C_{TR}}
            { \undetermined }{ C_{BR} } 
\leftarrow
\FlaThreeByThreeBR{ C_{00} }  { c_{01} }  { C_{02} }
                  { \undetermined }{ \gamma_{11}}     { c_{12}^{T} }
                  { \undetermined }  { \undetermined }         { C_{22}}
,
\FlaTwoByTwo{ \hat{C}_{TL} }{ \hat{C}_{TR}}
            { \undetermined }{ \hat{C}_{BR} } 
\leftarrow
\FlaThreeByThreeBR{ \hat{C}_{00} }  { \hat{c}_{01} }  { \hat{C}_{02} }
                  { \undetermined }{ \hat{\gamma}_{11}}     { \hat{c}_{12}^{T}}
                  { \undetermined }  { \undetermined }      { \hat{C}_{22}},
$ \\
and
$ 
\FlaTwoByOne{ A_{T} }
            { A_{B} } 
\leftarrow
\FlaThreeByOneB{ A_{0} }
               { a_{1}^T }
               { A_{2} }
$
}

% Step 6: state before update
\renewcommand{\beforeupdate}{
\FlaThreeByThreeTL{ C_{00} }  { c_{01} }  { C_{02} }
                  {\undetermined}{ \gamma_{11}}     { c_{12}^{T} }
                  { \undetermined }  { \undetermined }  { C_{22} }
=
\FlaTwoByTwo{ \FlaTwoByTwoSingleLine{ \hat{C}_{00} }   { \hat{c}_{01} }
                           { \undetermined} { \hat{\gamma}_{11}   } }   
            { \FlaTwoByOneSingleLine{ \hat{C}_{02} }{ \hat{c}_{12}^{T} } }
            { \undetermined }  
            { A_2 A_2^{T} + \hat{C}_{22}}
\wedge \ldots
}

% Step 7: state after update
\renewcommand{\afterupdate}{
\FlaThreeByThreeBR{ C_{00} }  { c_{01} }  { C_{02} }
                  { \undetermined }{ \gamma_{11}}     { c_{12}^{T} }
                  { \undetermined }  { \undetermined }         { C_{22}}
=
\FlaTwoByTwo { \hat{C}_{00} }
             { \FlaOneByTwoSingleLine{ \hat{c}_{01} }{ \hat{C}_{02} } }
             { \undetermined }
             { \FlaTwoByTwoSingleLine{ \hat{\gamma}_{11} + a_1^T a_1 }{ a_1^T A_2^T + \hat{c}_{12}^{T} }
                                     { \undetermined }{ A_2 A_2^{T} + \hat{C}_{22}}}
\wedge \ldots
}

% Step 8: update
\renewcommand{\update}{
\begin{minipage}[t]{4in}
\noindent
% \FlaStartCompute \\
$ \gamma_{11} \becomes \gamma_{11} + a_1^T a_1 $\\
$ c_{12} \becomes c_{12} +  A_2 a_1 $\\
% \FlaEndCompute \\
\end{minipage}
}

% Given the commands defined above, the
% command \worksheet generates the annotated
% algorithm

\begin{figure}[htbp]
\worksheet
\caption{Annotated algorithm for the symmetric rank-k update example.}
\label{fig:ws:usyrk_unn}
\end{figure}

By recognizing that $ \hat{C} $ is never referenced we can eliminate
all parts of the algorithm that refer to this matrix, yielding the
final algorithm given in Fig.~\ref{fig:alg:usyrk_unn}.

% We now redefine some of the commands
% used to generate Fig. 4.2, taking out all references
% to \hat{B} to come up with the algorithm in Fig. 4.3.

% Step 4
\renewcommand{\partitionings}{
$ 
C \rightarrow \FlaTwoByTwo{ C_{TL} }{ C_{TR} }
                          { C_{TR}^{T} }{ C_{BR} }
$
and
$ 
A \rightarrow \FlaTwoByOne{ A_{T} }
                          { A_{B} }
$
}
\renewcommand{\partitionsizes}{
$ A_{B} $ has $ 0 $ rows
and $ C_{BR} $ is $ 0 \times 0 $
}

% Step 5a
\renewcommand{\repartitionings}{
$ 
\FlaTwoByTwo{ C_{TL} }{ C_{TR}}
            { \undetermined }{ C_{BR} } 
\rightarrow 
\FlaThreeByThreeTL{ C_{00} }  { c_{01} }  { C_{02} }
                  {\undetermined}{ \gamma_{11}}     { c_{12}^{T} }
                  { \undetermined }  { \undetermined }  { C_{22} }
$
and
$
\FlaTwoByOne{ A_{T} }
            { A_{B} } 
\rightarrow
\FlaThreeByOneT{ A_{0} }
               { a_{1}^T }
               { A_{2} }
$
}
\renewcommand{\repartitionsizes}{
$ a_1^T $ is a row 
and $ \gamma_{11} $ is a scalar
}

% Step 5b
\renewcommand{\moveboundaries}{%
$ 
\FlaTwoByTwo{ C_{TL} }{ C_{TR}}
            { \undetermined }{ C_{BR} } 
\leftarrow
\FlaThreeByThreeBR{ C_{00} }  { c_{01} }  { C_{02} }
                  { \undetermined }{ \gamma_{11}}     { c_{12}^{T} }
                  { \undetermined }  { \undetermined }         { C_{22}}
$
and
$
\FlaTwoByOne{ A_{T} }
            { A_{B} } 
\leftarrow
\FlaThreeByOneB{ A_{0} }
               { a_{1}^T }
               { A_{2} }
$
}

% The command \FlaAlgorithm now generates the 
% algorithm without annotations

\begin{figure}[htbp]
\FlaAlgorithm
\caption{Unblocked algorithm for loop-invariant 1.}
\label{fig:alg:usyrk_unn}
\end{figure}

\subsubsection{Blocked Algorithms}

In order to cast the algorithm to be rich in matrix-matrix
multiplications, the repartitioning and redefinition of the
submatrices in Steps 5a and 5b in Fig.~\ref{fig:ws:usyrk_unn_blk}
expose multiple rows and/or columns at a time.  Block size $ b $ can
be chosen to be any size that does not exceed the number of rows in $
A_T $.  Again $ \QBefore $ is obtained by plugging the repartitioning
in Step 6 into $ \PInv $ while $ \QAfter $ is obtained by plugging the
redefinition of the quadrants in Step 7 into $ \PInv $.  The update in
Step 8 is now obtained by comparing the state in Steps 6 and 7.  The
bulk of the computation is now in the update $ C_{12} \becomes C_{12} +  A_1
A_2^T $ which, provided $ b > 1 $ involves a matrix-matrix
multiplication.  The update $ C_{11} \becomes C_{11} +  A_1 A_1^T $ is itself a
symmetric rank-k update and can be achieved by any
correct algorithm for implementing this operation.  In particular, an
unblocked algorithm can be used, or a blocked algorithm can be called
recursively.

% Redefine the commands that generate the annotated 
% algorithm for the blocked algorithm

% Define the blocksize that appears in step 5a
\renewcommand{\blocksize}{ b }
%

% Step 5a
\renewcommand{\repartitionings}{
$ 
\FlaTwoByTwo{ C_{TL} }{ C_{TR}}
            { \undetermined }{ C_{BR} }
\rightarrow
\FlaThreeByThreeTL{ C_{00} }  { C_{01} }  { C_{02} }
                  {\undetermined}{ C_{11}}     { C_{12} }
                  { \undetermined }  { \undetermined }  { C_{22} }
,
\FlaTwoByTwo{ \hat{C}_{TL} }{ \hat{C}_{TR}}
            { \undetermined }{ \hat{C}_{BR} }
\rightarrow
\FlaThreeByThreeTL{ \hat{C}_{00} }  { \hat{C}_{01} }  { \hat{C}_{02} }
                  {\undetermined}   { \hat{C}_{11} }  { \hat{C}_{12} }
                  { \undetermined } { \undetermined } { \hat{C}_{22} }
,
$ \\
and
$
\FlaTwoByOne{ A_{T} }
            { A_{B} }
\rightarrow
\FlaThreeByOneT{ A_{0} }
               { A_{1} }
               { A_{2} }  
$
}
%
\renewcommand{\repartitionsizes}{
$ C_{11} $ and $ \hat{C}_{11} $ are $ b \times b $
and $ A_{1} $ has $ b $ rows.
}

% Step 5b
\renewcommand{\moveboundaries}{%
$ 
\FlaTwoByTwo{ C_{TL} }{ C_{TR}}
            { \undetermined }{ C_{BR} }
\leftarrow
\FlaThreeByThreeBR{ C_{00} }        { C_{01} }        { C_{02} }
                  { \undetermined } { C_{11}}         { C_{12} }
                  { \undetermined } { \undetermined } { C_{22} }
,
\FlaTwoByTwo{ \hat{C}_{TL} }{ \hat{C}_{TR}}
            { \undetermined }{ \hat{C}_{BR} }
\leftarrow
\FlaThreeByThreeBR{ \hat{C}_{00} }  { \hat{C}_{01} }  { \hat{C}_{02} }
                  { \undetermined } { \hat{C}_{11} }  { \hat{C}_{12} }
                  { \undetermined } { \undetermined } { \hat{C}_{22} },
$ \\
and
$
\FlaTwoByOne{ A_{T} }
            { A_{B} }
\leftarrow
\FlaThreeByOneB{ A_{0} }
               { A_{1} }
               { A_{2} }
$
}

% Step 6
\renewcommand{\beforeupdate}{
\FlaThreeByThreeTL{ C_{00} }        { C_{01} }        { C_{02} }
                  {\undetermined}   { C_{11}}         { C_{12} }
                  { \undetermined } { \undetermined } { C_{22} }
=
\FlaTwoByTwo{ \FlaTwoByTwoSingleLine{ \hat{C}_{00} }   { \hat{C}_{01} }
                           { \undetermined} { \hat{C}_{11}   } }
            { \FlaTwoByOneSingleLine{ \hat{C}_{02} }{ \hat{C}_{12} } }
            { \undetermined }
            { A_2 A_2^{T} + \hat{C}_{22}}
\wedge \ldots
}

% Step 7
\renewcommand{\afterupdate}{
\FlaThreeByThreeBR{ C_{00} }        { C_{01} }        { C_{02} }
                  { \undetermined } { C_{11}}         { C_{12} }
                  { \undetermined } { \undetermined } { C_{22} }
=
\FlaTwoByTwo { \hat{C}_{00} }
             { \FlaOneByTwoSingleLine{ \hat{C}_{01} }{ \hat{C}_{02} } }
             { \undetermined }
             { \FlaTwoByTwoSingleLine{ \hat{C}_{11} + A_1 A_1^T }{ \hat{C}_{12} + A_1 A_2^T }
                                     { \undetermined }{ \hat{C}_{22} + A_2 A_2^T} }   
\wedge \ldots
}

% Step 8
\renewcommand{\update}{
\begin{minipage}[t]{4in}
\noindent
% \FlaStartCompute \\
$ C_{11} \becomes C_{11} + A_1 A_1^T $\\
$ C_{12} \becomes C_{12} + A_1 A_2^T $\\
% \FlaEndCompute \\
\end{minipage}
}

% Generate the annotated algorithm in Fig. 4.4
\begin{figure}[htbp]
\worksheet
\caption{Annotated blocked algorithm for loop-invariant 1.}
\label{fig:ws:usyrk_unn_blk}
\end{figure}
%

\subsubsection{Implementation}

Sequential implementations for the unblocked and blocked algorithms
for this loop-invariant using FLAME are given in
Figs.~\ref{fig:syrk_unn_upleft_unb}--\ref{fig:syrk_unn_upleft_blk}.

\begin{figure}[htbp]
\footnotesize
\begin{quote}
\listinginput{1}{syrk_unn/sequential/up-left/Syrk_unn_upleft_wrt_C_unb.c}
\end{quote}
\caption{Unblocked algorithm for loop-invariant 1 using FLAME.}
\label{fig:syrk_unn_upleft_unb}
\end{figure}

\begin{figure}[htbp]
\footnotesize
\index{const}{\tt FLA\un RECURSIVE}%
\begin{quote}
\listinginput{1}{syrk_unn/sequential/up-left/Syrk_unn_upleft_wrt_C_blk.c}
\end{quote}
\caption{Blocked algorithm for loop-invariant 1 using FLAME.
Recursion is optionally supported.}
\label{fig:syrk_unn_upleft_blk}
\end{figure}

\subsection{Loop-invariant 2}

We now examine the loop-invariant
\begin{equation}
\label{eqn:usyrk_unn:p3}
\PInv: 
\FlaTwoByTwo{ C_{TL} }        { C_{TR} } 
            { \undetermined } { C_{BR} }
=
\FlaTwoByTwo{ A_{T} A_{T}^T + \hat{C}_{TL} } { A_{T} A_{B}^T + \hat{C}_{TR} }
            { \undetermined }                { \hat{C}_{BR} }
\wedge
\ldots
\end{equation}
Comparing the loop-invariant in (\ref{eqn:usyrk_unn:p3}) with the
postcondition $ C = A A^T + \hat{C} $ we see again that {\em if} $C = C_{TL}$,
$ A = A_T $, and $ \hat{C} = \hat{C}_{TL} $ then the loop-invariant
implies the postcondition, i.e., that the desired result has been
computed.
%
%The initialization in Step 4 in Fig.~\ref{fig:ws:usyrk_unn:var2}
%is the same as that in Step 4 in Fig.~\ref{fig:ws:usyrk_unn}.
The fact that for this partitioning of $ C $, $ \hat{C} $, and $ A $,
\[
\FlaTwoByTwo{ C_{TL} }        { C_{TR} } 
            { \undetermined } { C_{BR} }
=
\FlaTwoByTwo{ A_{T} A_{T}^T + \hat{C}_{TL} } { A_{T} A_{B}^T + \hat{C}_{TR} }
            { \undetermined }                { \hat{C}_{BR} }
\]
and the precondition implies the other parts of
the loop-invariant, this initialization has
the desired properties.

Loop-guard $ G $ indicates that eventually $ C_{TL} $ should equal all
of $ C $, at which point $ G $ becomes {\em false} and the loop is
exited.  After the initialization, $ C_{TL} $ is $ 0 \times 0 $.
Thus, the repartitioning should be such that as the computation
proceeds, rows and columns are subtracted from $ C_{BR} $ and added to
$ C_{TL} $ while updating $ C_{TL} $ consistently with this.

%


\subsubsection{Unblocked algorithm}

The repartitionings in Step 5a in Fig.~\ref{fig:ws:usyrk_unn:var2}
result in the state
\begin{equation}
\QBefore: 
\FlaThreeByThreeBR{ C_{00} }        { c_{01} }         { C_{02} }
                  { \undetermined } { \gamma_{11}}     { c_{12}^T }
                  { \undetermined } { \undetermined }  { C_{22} }
=
\FlaTwoByTwo{ A_0 A_0^T + \hat{C}_{00}}
	    { A_0 
              \FlaTwoByOneSingleLine{ a_1^T }{ A_2 }^T + 
	      \FlaOneByTwoSingleLine{\hat{c}_{01}}
                                    {\hat{C}_{02} }
            }
            { \undetermined }  
	    { \FlaTwoByTwoSingleLine{ \hat{\gamma}_{11} }   { \hat{c}_{12}^{T} }
                                    { \undetermined}        { \hat{C}_{22}}}   
\wedge \ldots
\end{equation}
or 
\begin{equation}
\label{eqn:usyrk_unn:var2:before}
\QBefore: 
\FlaThreeByThreeBR{ C_{00} }        { c_{01} }         { C_{02} }
                  { \undetermined } { \gamma_{11}}     { c_{12}^T }
                  { \undetermined } { \undetermined }  { C_{22} }
=
\FlaTwoByTwo{ A_0 A_0^T + \hat{C}_{00}}
	    {  
	      \FlaOneByTwoSingleLine{ A_0 a_1 + \hat{c}_{01}}
                                    { A_0 A_2^T + \hat{C}_{02} }
            }
            { \undetermined }  
	    { \FlaTwoByTwoSingleLine{ \hat{\gamma}_{11} }   { \hat{c}_{12}^{T} }
                                    { \undetermined}        { \hat{C}_{22}}}   
\wedge \ldots
\end{equation}
before the update.
The redefinition in Step 5b in Fig.~\ref{fig:ws:usyrk_unn:var2}
means that the following state
must result from the update:
\[
\QAfter: 
\FlaThreeByThreeTL{ C_{00} }        { c_{01} }         { C_{02} }
                  { \undetermined } { \gamma_{11}}     { c_{12}^T }
                  { \undetermined } { \undetermined }  { C_{22} }
=
\FlaTwoByTwo{
              \FlaTwoByOneSingleLine {A_0} {a_1^T}
	      \FlaTwoByOneSingleLine {A_0} {a_1^T} ^T +
              \FlaTwoByTwoSingleLine {\hat{C}_{00}}        {\hat{c}_{01}}
 				     {\undetermined} {\hat{\gamma}_{11}}
            }
	    { 
              \FlaTwoByOneSingleLine{ A_0 }{ a_1^T } 
              A_2^T +
              \FlaTwoByOneSingleLine{\hat{C}_{02} }{ \hat{c}_{12}^T }  
            }
            { \undetermined }  
	    { \hat{C}_{22} }   
\wedge \ldots
\]
or
\begin{equation}
\label{eqn:usyrk_unn:var2:after}
\QAfter: 
\FlaThreeByThreeTL{ C_{00} }        { c_{01} }         { C_{02} }
                  { \undetermined } { \gamma_{11}}     { c_{12}^T }
                  { \undetermined } { \undetermined }  { C_{22} }
=
\FlaTwoByTwo{
              \FlaTwoByTwoSingleLine { A_0 A_0^T + \hat{C}_{00}} { A_0 a_1 + \hat{c}_{01}}
 				     {\undetermined} { a_1^T a_1 + \hat{\gamma}_{11}}
            }
	    { 
              \FlaTwoByOneSingleLine{ A_0 A_2^T + \hat{C}_{02} }{ a_1^T A_2^T + \hat{c}_{12}^T }  
            }
            { \undetermined }  
	    { \hat{C}_{22} }   
\wedge \ldots
\end{equation}


Comparing 
(\ref{eqn:usyrk_unn:var2:before}) and
(\ref{eqn:usyrk_unn:var2:after})
we find that the update
\begin{eqnarray*}
& \gamma_{11} & \becomes a_1^T a_1 + \gamma_{11}  \\
& c_{12} & \becomes A_2 a_1 + c_{12}
\end{eqnarray*}
is required to change the state from $ \QBefore $
to $ \QAfter $.


Again one recognizes that $ \hat{C} $ is never
referenced and can be eliminated from the algorithm.

% Define the commands to generate the annotated
% algorithm in Fig. 4.7.

% Step 2
\renewcommand{\invariant}{
\FlaTwoByTwo{ C_{TL} }       { C_{TR} }
            { \undetermined} { C_{BR} }
=
\FlaTwoByTwo{ A_{T} A_{T}^T + \hat{C}_{TL} } { A_{T} A_{B}^T + \hat{C}_{TR} }
            { \undetermined }                { \hat{C}_{BR} }
\wedge
\ldots
}

% Step 3
\renewcommand{\guard}{ \neg \SameSize( C, C_{TL} ) }

% Step 4
\renewcommand{\partitionings}{
$ 
C \rightarrow \FlaTwoByTwo{ C_{TL} }{ C_{TR} }
                          { C_{TR}^{T} }{ C_{BR} }
$
and
$
A \rightarrow \FlaTwoByOne{ A_{T} }
                          { A_{B} }
$
}
\renewcommand{\partitionsizes}{
$ C_{TL} $ and $ \hat{ C }_{TL} $ are $ 0 \times 0  $ 
and $ A_T $ has $ 0 $ rows.
}

% Step 5a
\renewcommand{\repartitionings}{
$
\FlaTwoByTwo{ C_{TL} }{ C_{TR}}
            { \undetermined }{ C_{BR} }
\rightarrow
\FlaThreeByThreeBR{ C_{00} }  { c_{01} }  { C_{02} }
                  {\undetermined}{ \gamma_{11}}     { c_{12}^{T} }
                  { \undetermined }  { \undetermined }  { C_{22} }
,
\FlaTwoByTwo{ \hat{C}_{TL} }{ \hat{C}_{TR}}
            { \undetermined }{ \hat{C}_{BR} }
\rightarrow
\FlaThreeByThreeBR{ \hat{C}_{00} }  { \hat{c}_{01} }  { \hat{C}_{02} }
                  {\undetermined}{ \hat{\gamma}_{11}}     { \hat{c}_{12}^{T} }
                  { \undetermined }  { \undetermined }  { \hat{C}_{22} }
,
$ \\
and
$
\FlaTwoByOne{ A_{T} }
            { A_{B} }
\rightarrow
\FlaThreeByOneB{ A_{0} }
               { a_{1}^T }
               { A_{2} }
$
}
\renewcommand{\repartitionsizes}{
$ \gamma_{11} $ and $ \hat{\gamma}_{11} $ are scalars 
and $ a_1^T $ is a row.
}

% Step 5b
\renewcommand{\moveboundaries}{%
$ 
\FlaTwoByTwo{ C_{TL} }{ C_{TR}}
            { \undetermined }{ C_{BR} }
\leftarrow
\FlaThreeByThreeTL{ C_{00} }  { c_{01} }  { C_{02} }
                  { \undetermined }{ \gamma_{11}}     { c_{12}^{T} }
                  { \undetermined }  { \undetermined }         { C_{22}}
,   
\FlaTwoByTwo{ \hat{C}_{TL} }{ \hat{C}_{TR}}
            { \undetermined }{ \hat{C}_{BR} }
\leftarrow
\FlaThreeByThreeTL{ \hat{C}_{00} }  { \hat{c}_{01} }  { \hat{C}_{02} }
                  { \undetermined }{ \hat{\gamma}_{11}}     { \hat{c}_{12}^{T}}
                  { \undetermined }  { \undetermined }      { \hat{C}_{22}},
$ \\
and
$
\FlaTwoByOne{ A_{T} }
            { A_{B} }
\leftarrow
\FlaThreeByOneT{ A_{0} }  
               { a_{1}^T }
               { A_{2} }
$
}

% Step 6
\renewcommand{\beforeupdate}{
\FlaThreeByThreeBR{ C_{00} }        { c_{01} }         { C_{02} }
                  { \undetermined } { \gamma_{11}}     { c_{12}^T }
                  { \undetermined } { \undetermined }  { C_{22} }
=
\FlaTwoByTwo{ A_0 A_0^T + \hat{C}_{00}}
            { A_0
              \FlaTwoByOneSingleLine{ a_1^T }{ A_2 }^T +
              \FlaOneByTwoSingleLine{\hat{c}_{01}}
                                    {\hat{C}_{02} }
            }
            { \undetermined }
            { \FlaTwoByTwoSingleLine{ \hat{\gamma}_{11} }   { \hat{c}_{12}^{T} }
                                    { \undetermined}        { \hat{C}_{22}}}
\wedge \ldots
}

% Step 7
\renewcommand{\afterupdate}{
\FlaThreeByThreeTL{ C_{00} }        { c_{01} }         { C_{02} }
                  { \undetermined } { \gamma_{11}}     { c_{12}^T }
                  { \undetermined } { \undetermined }  { C_{22} }
=
\FlaTwoByTwo{
              \FlaTwoByTwoSingleLine { A_0 A_0^T + \hat{C}_{00}} { A_0 a_1 + \hat{c}_{01}}
                                     {\undetermined} { a_1^T a_1 + \hat{\gamma}_{11}}
            }
            {
              \FlaTwoByOneSingleLine{ A_0 A_2^T + \hat{C}_{02} }{ a_1^T A_2^T + \hat{c}_{12}^T }
            }
            { \undetermined }
            { \hat{C}_{22} }
\wedge \ldots
}

% Step 8
\renewcommand{\update}{
\begin{minipage}[t]{4in}
\noindent
% \FlaStartCompute \\
$ \gamma_{11} \becomes a_1^T a_1 + \gamma_{11} $\\
$ c_{12} \becomes A_2 a_1 + c_{12} $\\
% \FlaEndCompute \\
\end{minipage}
}

% Generate figure 4.7
\begin{figure}[htbp]
\worksheet
\caption{Annotated unblocked algorithm for loop-invariant 2.}
\label{fig:ws:usyrk_unn:var2}
\end{figure}

\subsubsection{Blocked Algorithms}

Again, the algorithm can be cast to be rich in
matrix-matrix multiplications by marching through
the matrices multiple rows and/or columns at a time.
The resulting algorithm is given in 
Fig.~\ref{fig:ws:usyrk_unn:var2:blk}.

% Redefine commands to generate Fig. 4.8

% Step 5a
\renewcommand{\repartitionings}{
$ 
\FlaTwoByTwo{ C_{TL} }{ C_{TR}}
            { \undetermined }{ C_{BR} }
\rightarrow
\FlaThreeByThreeBR{ C_{00} }  { C_{01} }  { C_{02} }
                  {\undetermined}{ C_{11}}     { C_{12} }
                  { \undetermined }  { \undetermined }  { C_{22} }
,
\FlaTwoByTwo{ \hat{C}_{TL} }{ \hat{C}_{TR}}
            { \undetermined }{ \hat{C}_{BR} }
\rightarrow
\FlaThreeByThreeBR{ \hat{C}_{00} }  { \hat{C}_{01} }  { \hat{C}_{02} }
                  {\undetermined}{ \hat{C}_{11}}     { \hat{C}_{12} }
                  { \undetermined }  { \undetermined }  { \hat{C}_{22} }
,
$ \\
and
$
\FlaTwoByOne{ A_{T} }
            { A_{B} }
\rightarrow
\FlaThreeByOneB{ A_{0} }
               { A_{1} }
               { A_{2} }
$
}
%
\renewcommand{\repartitionsizes}{
$ C_{11} $ and $ \hat{C}_{11} $ are $ b \times b $ 
and $ A_1 $ has $ b $ rows.
}

% Step 5b
\renewcommand{\moveboundaries}{%
$ 
\FlaTwoByTwo{ C_{TL} }{ C_{TR}}
            { \undetermined }{ C_{BR} }
\leftarrow
\FlaThreeByThreeTL{ C_{00} }        { C_{01} }        { C_{02} }
                  { \undetermined } { C_{11}}         { C_{12} }
                  { \undetermined } { \undetermined } { C_{22}}
,   
\FlaTwoByTwo{ \hat{C}_{TL} }{ \hat{C}_{TR}}
            { \undetermined }{ \hat{C}_{BR} }
\leftarrow
\FlaThreeByThreeTL{ \hat{C}_{00} }  { \hat{C}_{01} }  { \hat{C}_{02} }
                  { \undetermined } { \hat{C}_{11}}   { \hat{C}_{12}}
                  { \undetermined } { \undetermined } { \hat{C}_{22}},
$ \\
and
$
\FlaTwoByOne{ A_{T} }
            { A_{B} }
\leftarrow
\FlaThreeByOneT{ A_{0} }  
               { A_{1} }
               { A_{2} }
$
}

% Step 6
\renewcommand{\beforeupdate}{
\FlaThreeByThreeBR{ C_{00} }        { C_{01} }         { C_{02} }
                  { \undetermined } { C_{11}}          { C_{12} }
                  { \undetermined } { \undetermined }  { C_{22} }   
=
\FlaTwoByTwo{ A_0 A_0^T + \hat{C}_{00}}
            { A_0
              \FlaTwoByOneSingleLine{ A_1 }{ A_2 }^T +
              \FlaOneByTwoSingleLine{\hat{C}_{01}}
                                    {\hat{C}_{02} }
            }
            { \undetermined }
            { \FlaTwoByTwoSingleLine{ \hat{C}_{11} }   { \hat{C}_{12} }
                                    { \undetermined}   { \hat{C}_{22}}}
\wedge \ldots
}

% Step 7
\renewcommand{\afterupdate}{
\FlaThreeByThreeTL{ C_{00} }        { C_{01} }         { C_{02} }
                  { \undetermined } { C_{11}}          { C_{12} }
                  { \undetermined } { \undetermined }  { C_{22} }
=
\FlaTwoByTwo{
              \FlaTwoByTwoSingleLine { A_0 A_0^T + \hat{C}_{00}} { A_0 A_1^T + \hat{C}_{01}}
                                     {\undetermined} { A_1 A_1^T + \hat{C}_{11}}
            }
            {
              \FlaTwoByOneSingleLine{ A_0 A_2^T + \hat{C}_{02} }{ A_1 A_2 + \hat{C}_{12} }
            }
            { \undetermined }
            { \hat{C}_{22} }
\wedge \ldots
}

% Step 8
\renewcommand{\update}{
\begin{minipage}[t]{4in}
\noindent
% \FlaStartCompute \\
$ C_{11} \becomes A_1 A_1^T + C_{11} $\\
$ C_{12} \becomes A_1 A_2^T + C_{12} $\\
% \FlaEndCompute \\
\end{minipage}
}

% Generate Fig. 4.8
\begin{figure}[htbp]
\worksheet
\caption{Annotated blocked algorithm for loop-invariant 2.}
\label{fig:ws:usyrk_unn:var2:blk}
\end{figure}
%

\subsubsection{Implementation}
 
Sequential implementations for the unblocked and blocked algorithms
for this loop-invariant using FLAME are similar to the implementations of loop-invariant 1.

\section{Performance}

\begin{quote}
{\bf Note:  The experiments reported in the various chapters
were not always performed on exactly the platform described in 
Section~\ref{sec:intro:performance}.  Thus, it is the
relative performance on the different algorithms that
is of significance, rather than the absolute performance.
}
\end{quote}

In this section, we discuss the performance attained by the different
variants for computing $ C \leftarrow A A^T +  \hat{C} $.  In Fig.
~\ref{fig:usyrk_unn:upleft-downright:ATLAS}, we compare the performance
attained by five different implementations:
\begin{itemize}
\item{\sc Reference}
{\tt DTRMM} as implemented as part of ATLAS,
\item{\sc FLAME}
{\tt FLA\_Syrk} implemented as part of FLAME
as of this writing,
\item{\sc Unblocked}
the unblocked implementation in Fig.~\ref{fig:syrk_unn_upleft_unb},
\item{\sc Blocked}
the blocked implementation in Fig.~\ref{fig:syrk_unn_upleft_blk}
called with 
{\tt rec == 
\index{const}{\tt FLA\un NON\un RECURSIVE}%
FLA\_NON\_RECURSIVE}, and
\item{\sc Recursive}
the blocked implementation in Fig.~\ref{fig:syrk_unn_upleft_blk}
called with 
{\tt rec == 
\index{const}{\tt FLA\un RECURSIVE}%
FLA\_RECURSIVE}.
\end{itemize}
In Fig.~\ref{fig:usyrk_unn:upleft-downright:ATLAS}, all FLAME
implementations are based on the matrix-matrix multiplication ({\tt
DGEMM}) provided by ATLAS.  Notice that for the platform on which we
performed the experiments, multiples of 40 are good block sizes when
using the ATLAS matrix-matrix multiplication kernel.  We note that the
unblocked algorithms perform badly, since they are rich in
matrix-vector operations that do not benefit much from cache memories.
The blocked algorithms performed better for relatively small block
sizes ({\tt nb\_alg=40}) than for larger block sizes ({\tt
nb\_alg=80}).  The reason for this is that too much of the
computation is performed in the subprogram for which the unblocked
algorithm is used.  The recursive implementations benefit from larger
block sizes, since they overcome this problem that plagues the blocked
algorithm.  In addition they benefit from the fact that the {\tt
DGEMM} kernel performs better for larger blocks.

Since in the recursive algorithms at each level a different variant
could be called to solve the subproblem, it may be possible to improve
performance further by combining different variants.  We have not yet
studied this.

\begin{figure}[htbp]
\begin{center}
\begin{tabular}{c | c}
Variant 1 & Variant 2 \\ \hline
& \\
\psfig{figure=syrk_unn/graphs/syrk_unn_upleft_wrt_C_40.eps,width=3.0in,height=3.0in} &
\psfig{figure=syrk_unn/graphs/syrk_unn_downright_wrt_C_40.eps,width=3.0in,height=3.0in}
\\ \hline
& \\
\psfig{figure=syrk_unn/graphs/syrk_unn_upleft_wrt_C_80.eps,width=3.0in,height=3.0in} &
\psfig{figure=syrk_unn/graphs/syrk_unn_downright_wrt_C_80.eps,width=3.0in,height=3.0in}
\end{tabular}
\end{center}
\caption{Performance of the variants  of
symmetric rank-k update
algorithms for a block size of $ 40 $ (top) and
$ 80 $ (bottom).
For these experiments, the ATLAS matrix-matrix multiplication
kernel was used.}
\label{fig:usyrk_unn:upleft-downright:ATLAS}
\end{figure}
\index{op}{symmetric rank-k update!$ C \leftarrow A A^T + \hat{C} $|( }%
