% The following commands help in generating the index
\index{index}{symmetric rank-k update}%
\index{op}{symmetric rank-k update!$ C \leftarrow A A^T + \hat{C} $|( }%

% Name of the chapter

\chapter{Symmetric Rank-k Update \\
$ C \leftarrow A A^T + \hat{C} $ \\
$ C $ symmetric, stored in upper triangle
}
\label{chapter:syrk_un}

% Authors

\ChapterAuthor{
\index{author}{Chen, Katherine}
Katherine Chen
\\[0.1in]
\index{author}{Ho, Mandy}
Mandy Ho
\\[0.1in]
\index{author}{Lai, Ingrid}
Ingrid Lai
}

% Add the names of the authors to the table of contents

\addtocontents{toc}{ {by {\bf Katherine Chen, Mandy Ho, and Ingrid Lai} } 4/23/02 - 
finished the first 2 and 1/2 pages, made changes suggested by Paolo}

In this chapter, we discuss the implementation of the symmetric rank-k update
\[
C \leftarrow A A^T + \hat{C}
\]
where $ C $ is an $ n \times n $ upper triangular matrix and $ A $ is
$ n \times k $.  We will overwrite $ C $ with the result without
requiring a workarray.  We start by deriving a number of different
sequential algorithms.  Subsequently, we show how to code the
algorithms using FLAME.  Later in the chapter the parallel
implementation of the operation is discussed.

The variables for the symmetric rank-k update can be
described by the precondition
\[
\PPre:
C = \hat{C} \wedge \SameSize( C, \hat{C} ) \wedge \UpTr( C ) \wedge
\RowDim( A )=\RowDim(\hat{C}) \wedge \ColDim( A )=\ColDim(\hat{C}),
\]
where $ \hat{C} $ indicates the original contents of $ C $.  Here the
predicate $ \UpTr( C ) $ is true iff $ C $ is a upper triangular
matrix.  The functions $ \RowDim( C ) $, $ \RowDim( A ) $  and $ \ColDim( C ) $,
$ \ColDim( A ) $ return
the row and column dimension of $ C $ and $ A $, respectively.  The operation to
be performed, $ C \becomes A A^T + C $, translates to the postcondition $
\PPost: C = A A^T + \hat{C} $.

\section{Algorithms That Start By Partitioning $ C $}
\label{sec:trmm_lln:L}

Let us start by partitioning matrix $ C $.  Since it is triangular, we
partition it like
\[
C \rightarrow \FlaTwoByTwo{ C_{TL} }        { C_{TR} }
                          { \undetermined } { C_{BR} },
\]
where $ C_{TL} $ is square so that both submatrices on the diagonal
are themselves upper triangular.

Substituting the partitioning of $ C $ into the postcondition yields
\[
\FlaTwoByTwo{ C_{TL} }        { C_{TR} }
            { \undetermined } { C_{BR} }
= 
( \mbox{some partitioning of }{A} )
( \mbox{some partitioning of }{A^T} )
+
\FlaTwoByTwo{ \hat{C}_{TL} }  { \hat{C}_{TR} }
            { \undetermined } { \hat{C}_{BR} }
\]
This suggests that $ A $ should be partitioned
vertically into two submatrices, or into quadrants.  Let us consider
the case where $ A $ is partitioned as described above.  Then
\[
\FlaTwoByTwo{ C_{TL} }       { C_{TR} }
            { \undetermined} { C_{BR} }
= 
\FlaTwoByOne{ A_T }
            { A_B }
\FlaOneByTwo{ {A_T}^T }
            { {A_B}^T }
+
\FlaTwoByTwo{ \hat{C}_{TL} }  { \hat{C}_{TR} }
            { \undetermined } { \hat{C}_{BR} }
\]
In order to be able to multiply the matrices on the right out and to
be able to then set the submatrices on the left equal to the result on
the right we find that the following must hold:
\[
\ColDim( C_{TL} ) = \RowDim( A_T ) 
\wedge
\RowDim( C_{TL} ) = \RowDim( A_T ) 
\]
Substituting the partitioned matrices into the postcondition, we find
\[
\FlaTwoByTwo{ C_{TL} }       { C_{TR} }
            { \undetermined} { C_{BR} }
=
\FlaTwoByOne{ A_T }
            { A_B }
\FlaOneByTwo{ {A_T}^T }
            { {A_B}^T }
+
\FlaTwoByTwo{ \hat{C}_{TL} }  { \hat{C}_{TR} }
            { \undetermined } { \hat{C}_{BR} }
=
\FlaTwoByTwo{ A_T {A_T}^T + \hat{C}_{TL} } { A_T {A_B}^T + \hat{C}_{TR} }
            { \undetermined }              { A_B {A_B}^T + \hat{C}_{BR} }
\]
which yields the equalities
\begin{equation}
\label{eqn:ltrmm_lln}
\begin{array}{r c l || r c l}
C_{TL} &=& A_T {A_T}^T + \hat{C}_{TL} & C_{TR} &=& A_T {A_B}^T + \hat{C}_{TR} \\ \hline \hline
C_{BL} &=& \undetermined              & C_{BR} &=& A_B {A_B}^T + \hat{C}_{BR}
\end{array}
\end{equation}

% Insert the table of possible loop-invariants from
% file syrk_un/table.new.tex

\begin{figure}
\begin{center}
\footnotesize
% \begin{sideways}
{
\setlength{\tabcolsep}{4pt}
\begin{tabular}{| c | p{2.5in} | c | } \hline
$ P_X: $

$ \FlaTwoByTwo{ C_{TL} } { C_{TR} }
              { C_{BL} } { C_{BR} } =  $ & Comment &
\footnotesize Feasible? \\ \hline \hline
\footnotesize
$
\FlaTwoByTwo{ \hat{C}_{TL} }  { \hat{C}_{TR} }
            { \undetermined } { \hat{C}_{BR} }
$ 
&
&
NO
\\ \hline
%
%
\footnotesize
$
\FlaTwoByTwo{ A_{T} A_{T}^T + \hat{C}_{TL} }  { A_{T} A_{B}^T + \hat{C}_{TR} }
            { \undetermined }                 { A_{B} A_{B}^T + \hat{C}_{BR} }
$ 
& 
NO
\\ \hline
%
%
\footnotesize
$
\FlaTwoByTwo{ A_{T} A_{T}^T + \hat{C}_{TL} }  { \hat{C}_{TR} }
            { \undetermined }                 { \hat{C}_{BR} }
$ 
& 
YES
\\ \hline
%
%
\footnotesize
$
\FlaTwoByTwo{ \hat{C}_{TL} }  { A_{T} A_{B}^T + \hat{C}_{TR} } 
            { \undetermined } { \hat{C}_{BR} }
$ 
&
NO
\\ \hline
%
%
\footnotesize
$
\FlaTwoByTwo{ \hat{C}_{TL} }   { \hat{C}_{TR} }
            { \undetermined }  { A_{B} A_{B}^T + \hat{C}_{BR} }
$ 
&
Loop-invariant 1
&
YES
\\ \hline
%
%
\footnotesize
$
\FlaTwoByTwo{ \hat{C}_{TL} } { A_{T} A_{B}^T + \hat{C}_{TR} }
            { \undetermined} { A_{B} A_{B}^T + \hat{C}_{BR} }
$ 
&
YES
\\ \hline
%
%
\footnotesize
$
\FlaTwoByTwo{ A_{T} A_{T}^T + \hat{C}_{TL} } { \hat{C}_{TR} }
            { \undetermined }                { A_{B} A_{B}^T + \hat{C}_{BR} }
$ 
&
YES
\\ \hline
%
%
\footnotesize
$
\FlaTwoByTwo{ A_{T} A_{T}^T + \hat{C}_{TL} } { A_{T} A_{B}^T + \hat{C}_{TR} }
            { \undetermined }                { \hat{C}_{BR} }
$ 
&
Loop-invariant 2.
&
YES
\\ \hline
%\\ \hline
%
%
\end{tabular}
}
% \end{sideways}
\end{center}
\caption{Possible loop-invariants when partitioning
$ C $ into quadrants.
Here $ P_X $ is the most prominent part of the loop-invariant
$ \PInv $.}
\label{fig:USYRK_UNN_example}
\end{figure}

From (\ref{eqn:ltrmm_lln}) we conclude that the operations to be
performed are $ A_T {A_T}^T + \hat{C}_{TL} $, $ A_T {A_B}^T + \hat{C}_{TR} $, and $ A_B {A_B}^T + 
\hat{C}_{BR} $.  At an intermediate stage, each of these either has or
has not already been computed, leading to the $ 2^3 = 8 $ possible
loop-invariants tabulated in Fig.~\ref{fig:LTRMM_LLNexample}.

\subsection{Loop-invariant 1}

We will first examine the (feasible) loop-invariant
\begin{eqnarray}
\label{eqn:ltrmm_lln:p1}
\PInv: 
\lefteqn{
\FlaTwoByTwo{ C_{TL} }       { C_{TR} }
            { \undetermined} { C_{BR} }
=
\FlaTwoByTwo{ \hat{C}_{TL} }  { \hat{C}_{TR} }
            { \undetermined } { A_B {A_B}^T + \hat{C}_{BR} }
\wedge
\UpTr( C_{TL} ) }  \\
\nonumber
& & 
\ColDim( C_{TL} ) = \RowDim( A_T ) 
\wedge
\RowDim( C_{TL} ) = \RowDim( A_T )
\end{eqnarray}
For short we will write
\begin{equation}
\label{eqn:ltrmm_lln:p2}
\PInv: \FlaTwoByTwo{ C_{TL} } { C_{TR} }
            { \undetermined}  { C_{BR} }
 =
\FlaTwoByTwo{ \hat{C}_{TL} }  { \hat{C}_{TR} }
            { \undetermined } { A_B {A_B}^T + \hat{C}_{BR} }
\wedge
\ldots
\end{equation}
for (\ref{eqn:ltrmm_lln:p1}).

Comparing the loop-invariant in (\ref{eqn:ltrmm_lln:p2}) with the
postcondition $ C = A A^T +  \hat{C} $ we see that {\em if} $C=C_{BR}$, $ A =
A_B $, and $ \hat{C} = \hat{C}_{BR} $ then the loop-invariant implies the
postcondition, i.e., that the desired result has been computed.
Notice that $ \SameSize( C, C_{BR} ) \wedge \PInv $ implies $ C =
C_{BR} $, $ A = A_B $ and $ \hat{C}=\hat{C}_{BR} $ since the partitioned
matrices are references into the original matrices $ C $, $ A $, and $
\hat{C} $.  Thus, the loop-guard $ G: \neg \SameSize( C, C_{BR} ) $
has the desired property.

Consider the initialization in Step 4 in Fig.~\ref{fig:ws:ltrmm_lln}.
The fact that for this partitioning of $ C $, $ \hat{C} $, and $ A $,
$
\FlaTwoByTwo{ C_{TL} }       { C_{TR} }
            { \undetermined} { C_{BR} }
=
\FlaTwoByTwo{ \hat{C}_{TL} }  { \hat{C}_{TR} }
            { \undetermined } { A_B {A_B}^T + \hat{C}_{BR} }
$ and the precondition implies the
other parts of the loop-invariant, this initialization has the desired
properties.

Loop-guard $ G $ indicates that eventually $ C_{BR} $ should equal all
of $ C $, at which point $ G $ becomes {\em false} and the loop is
exited.  After the initialization, $ C_{BR} $ is $ 0 \times 0 $.
Thus, the repartitioning should be such that as the computation
proceeds, rows and columns are subtracted from $ C_{TL} $ and added to
$ C_{BR} $ while updating $ C_{BR} $ consistently with this.

\subsubsection{Unblocked algorithm}

If we move the partitionings by individual rows and columns, we obtain
the repartitioning and redefinition of the partitioning in Steps 5a
and 5b in Fig.~\ref{fig:ws:ltrmm_lln}.  The repartitionings in Step 5a
in Fig.~\ref{fig:ws:ltrmm_lln} result in the state
\begin{equation}
\label{eqn:ltrmm_lln:bu}
\QBefore: 
\FlaTwoByTwo{ C_{TL} }{ C_{TR}}
            { \undetermined}{ A_{B} A_{B}^{T} + C_{BR}}
=
\FlaTwoByTwo{ \FlaTwoByTwoSingleLine{ \hat{C}_{00} }   { \hat{c}_{01} }
                           { \undetermined} { \hat{\gamma}_{11}   } }   
            { \FlaTwoByOneSingleLine{ \hat{C}_{02} }{ \hat{c}_{12}^{T} } }
            { \undetermined }  
            { A_2 A_2^{T} + \hat{C}_{22}}
\wedge \ldots
\end{equation}
before the update.  The redefinition in Step 5b in
Fig.~\ref{fig:ws:ltrmm_lln} means that the following state must result
from the update:
\[
\QAfter: 
\FlaTwoByTwo{ C_{TL} }{ C_{TR} }
            { \undetermined }{ A_{B} A_{B}^{T} + C_{BR}}
=
\FlaTwoByTwo{ \hat{C}_{00} }
            { \FlaOneByTwoSingleLine{\hat{c}_{01}}
                                    {\hat{C}_{02}}
            }
            {\undetermined  }
{
 \FlaTwoByOneSingleLine{a_{1}^T}
                       {A_{2}} 
 \FlaTwoByOneSingleLine{a_{1}^T}
                       {A_{2}}^T +
\FlaTwoByTwoSingleLine{ \hat{\gamma}_{11} }{ \hat{c}_{12}^{T} }
            { \undetermined } { \hat{C}_{22} }
}
\wedge \ldots
\]
Multiplying out the triangular matrix multiplication yields
\begin{equation}
\label{eqn:ltrmm_lln:au}
\QAfter: 
\FlaTwoByTwo{ C_{TL} }{ C_{TR} }
            { \undetermined }{ A_{B} A_{B}^{T} + C_{BR}}
=
\FlaTwoByTwo { \hat{C}_{00} }
             { \FlaOneByTwoSingleLine{ \hat{c}_{01} }{ \hat{C}_{02} } }
             { \undetermined }
             { \FlaTwoByTwoSingleLine{ \hat{\gamma}_{11} + a_{1}^{T} a_{1} }{ a_{1}^{T} A_{2}^{T} + \hat{c}_{12}^{T} }
                                     { \undetermined }{ A_{2} A_{2}^{T} + \hat{C}_{22}}}
\wedge \ldots
\end{equation}
Comparing Eqns.~\ref{eqn:ltrmm_lln:bu} and~\ref{eqn:ltrmm_lln:au} we
find that the updates
\begin{eqnarray*}
& \gamma_{11} & \becomes \hat{\gamma}_{11} + a_1^T a_1  \\
& c_{12}^{T} & \becomes \hat{c}_{12}^{T} +  a_1^T A_2^T 
\end{eqnarray*}
are required to change the state from $ \QBefore $ to $ \QAfter $.

% The following commands will in the ``worksheet''
% given in Fig. 4.2

% Step 0: Operation
\renewcommand{\operation}{C \becomes A A^T + \hat{C}}

% Step 1a: Precondition
\renewcommand{\precondition}{B = \hat{B} \wedge \LowTr( L ) \wedge \ColDim( L )=\RowDim( B )}

% Step 1b: Postcondition
\renewcommand{\postcondition}{B = L \hat{B}}

% Step 2: Loop-invariant
\renewcommand{\invariant}{\wedge \\
\FlaTwoByOne{ B_T}{B_B} =
\FlaTwoByOne{ \hat{B}_T }
            { L_{BR} \hat{B}_B }
\wedge
\ldots
}

% Step 3: Loop-guard
\renewcommand{\guard}{ \neg \SameSize( L, L_{BR} ) }

% Step 4: Initialization
\renewcommand{\partitionings}{
$ 
B \rightarrow \FlaTwoByOne{ B_{T} }
                          { B_{B} }
$,
$ 
\hat{B} \rightarrow \FlaTwoByOne{ \hat{B}_{T} }
                          { \hat{B}_{B} }
$, and
$ 
L \rightarrow \FlaTwoByTwo{ L_{TL} }{ 0 }
                          { L_{BL} }{ L_{BR} }
$
}
\renewcommand{\partitionsizes}{
$ B_{B} $ and $ \hat{ B }_B $ have $ 0 $ rows
and $ L_{BR} $ is $ 0 \times 0 $
}

% Step 5a: repartitioning
\renewcommand{\repartitionings}{
$ 
\FlaTwoByOne{ B_T }{ B_B } 
\rightarrow
\FlaThreeByOneT{ B_0 }{ b_1^T }{ B_2 },
\FlaTwoByOne{ \hat{B}_T }{ \hat{B}_B } 
\rightarrow
\FlaThreeByOneT{ \hat{B}_0 }{ \hat{b}_1^T }{ \hat{B}_2 },
$ \\
and
$ \FlaTwoByTwo{ L_{TL} }{ 0 }
            { L_{BL} }{ L_{BR} }
\rightarrow
\FlaThreeByThreeTL{ L_{00} }{ 0 }{ 0 }
                { l_{10}^T }{ \lambda_{11} }{ 0 }
                { L_{20} }{ l_{21} }{ L_{22} }
$
}
\renewcommand{\repartitionsizes}{
$ b_1^T $ and $ \hat{b}_1^T $ are rows 
and $ \lambda_{11} $ is a scalar
}

% Step 5b: moving the boundaries
\renewcommand{\moveboundaries}{%
$ 
\FlaTwoByOne{ B_T }{ B_B } 
\leftarrow
\FlaThreeByOneB{ B_0 }{ b_1^T }{ B_2 },
\FlaTwoByOne{ \hat{B}_T }{ \hat{B}_B } 
\leftarrow
\FlaThreeByOneB{ \hat{B}_0 }{ \hat{b}_1^T }{ \hat{B}_2 },
$ \\
and
$ \FlaTwoByTwo{ L_{TL} }{ 0 }
            { L_{BL} }{ L_{BR} }
\leftarrow
\FlaThreeByThreeBR{ L_{00} }{ 0 }{ 0 }
                { l_{10}^T }{ \lambda_{11} }{ 0 }
                { L_{20} }{ l_{21} }{ L_{22} }
$
}

% Step 6: state before update
\renewcommand{\beforeupdate}{
\FlaTwoByOne{ 
\FlaTwoByOneSingleLine{ B_0 }{ b_1^T }
}
{
B_2
} =
\FlaTwoByOne{ 
\FlaTwoByOneSingleLine{ \hat{B}_0 }{ \hat{b}_1^T }
}
{
L_{22} \hat{B}_2
}
\wedge \ldots
}

% Step 7: state after update
\renewcommand{\afterupdate}{
\FlaTwoByOne{ 
B_0
}
{
\FlaTwoByOneSingleLine{ b_1^T }{ B_2 }
} =
\FlaTwoByOne{ 
\hat{B}_0
}
{
\FlaTwoByOneSingleLine{ \lambda_{11} \hat{b}_1^T }{ l_{21} \hat{b}_1^T + L_{22} \hat{B}_2 }
}
\wedge \ldots
}

% Step 8: update
\renewcommand{\update}{
\begin{minipage}[t]{4in}
\noindent
% \FlaStartCompute \\
$ B_2 \becomes l_{21} b_1^T + B_2 $\\
$ b_1^T \becomes \lambda_{11} b_1^T $\\
% \FlaEndCompute \\
\end{minipage}
}

% Given the commands defined above, the
% command \worksheet generates the annotated
% algorithm

\begin{figure}[htbp]
\worksheet
\caption{Annotated algorithm for the lower triangular
matrix multiplication example.}
\label{fig:ws:ltrmm_lln}
\end{figure}

By recognizing that $ \hat{B} $ is never referenced we can eliminate
all parts of the algorithm that refer to this matrix, yielding the
final algorithm given in Fig.~\ref{fig:alg:ltrmm_lln}.

% We now redefine some of the commands
% used to generate Fig. 4.2, taking out all references
% to \hat{B} to come up with the algorithm in Fig. 4.3.

% Step 4
\renewcommand{\partitionings}{
$ 
B \rightarrow \FlaTwoByOne{ B_{T} }
                          { B_{B} }
$
and
$ 
L \rightarrow \FlaTwoByTwo{ L_{TL} }{ 0 }
                          { L_{BL} }{ L_{BR} }
$
}
\renewcommand{\partitionsizes}{
$ B_{B} $ has $ 0 $ rows
and $ L_{BR} $ is $ 0 \times 0 $
}

% Step 5a
\renewcommand{\repartitionings}{
$ 
\FlaTwoByOne{ B_T }{ B_B } 
\rightarrow
\FlaThreeByOneT{ B_0 }{ b_1^T }{ B_2 }
$
and
$ \FlaTwoByTwo{ L_{TL} }{ 0 }
            { L_{BL} }{ L_{BR} }
\rightarrow
\FlaThreeByThreeTL{ L_{00} }{ 0 }{ 0 }
                { l_{10}^T }{ \lambda_{11} }{ 0 }
                { L_{20} }{ l_{21} }{ L_{22} }
$
}
\renewcommand{\repartitionsizes}{
$ b_1^T $ is a row 
and $ \lambda_{11} $ is a scalar
}

% Step 5b
\renewcommand{\moveboundaries}{%
$ 
\FlaTwoByOne{ B_T }{ B_B } 
\leftarrow
\FlaThreeByOneB{ B_0 }{ b_1^T }{ B_2 }
$
and
$ \FlaTwoByTwo{ L_{TL} }{ 0 }
            { L_{BL} }{ L_{BR} }
\leftarrow
\FlaThreeByThreeBR{ L_{00} }{ 0 }{ 0 }
                { l_{10}^T }{ \lambda_{11} }{ 0 }
                { L_{20} }{ l_{21} }{ L_{22} }
$
}

% The command \FlaAlgorithm now generates the 
% algorithm without annotations

\begin{figure}[htbp]
\FlaAlgorithm
\caption{Unblocked algorithm for loop-invariant 1.}
\label{fig:alg:ltrmm_lln}
\end{figure}

\subsubsection{Blocked Algorithms}

In order to cast the algorithm to be rich in matrix-matrix
multiplications, the repartitioning and redefinition of the
submatrices in steps 5a and 5b in Fig.~\ref{fig:ws:ltrmm_lln_blk}
expose multiple rows and/or columns at a time.  Block size $ b $ can
be chosen to be any size that does not exceed the number of rows in $
B_T $.  Again $ \QBefore $ is obtained by plugging the repartitioning
in Step 6 into $ \PInv $ while $ \QAfter $ is obtained by plugging the
redefinition of the quadrants in Step 7 into $ \PInv $.  The update in
Step 8 is now obtained by comparing the state in Steps 6 and 7.  The
bulk of the computation is now in the update $ B_2 \becomes L_{21} B_1
+ B_2 $ which, provided $ b > 1 $ involves a matrix-matrix
multiplication.  The update $ B_1 \becomes L_{11} B_1 $ is itself a
triangular matrix-matrix multiplication and can be achieved by any
correct algorithm for implementing this operation.  In particular, an
unblocked algorithm can be used, or a blocked algorithm can be called
recursively.

% Redefine the commands that generate the annotated 
% algorithm for the blocked algorithm

% Define the blocksize that appears in step 5a
\renewcommand{\blocksize}{ b }
%

% Step 5a
\renewcommand{\repartitionings}{
$ 
\FlaTwoByOne{ B_T }{ B_B } 
\rightarrow
\FlaThreeByOneT{ B_0 }{ B_1 }{ B_2 },
\FlaTwoByOne{ \hat{B}_T }{ \hat{B}_B } 
\rightarrow
\FlaThreeByOneT{ \hat{B}_0 }{ \hat{B}_1 }{ \hat{B}_2 },
$ \\
and
$ \FlaTwoByTwo{ L_{TL} }{ 0 }
            { L_{BL} }{ L_{BR} }
\rightarrow
\FlaThreeByThreeTL{ L_{00} }{ 0 }{ 0 }
                { L_{10} }{ L_{11} }{ 0 }
                { L_{20} }{ L_{21} }{ L_{22} }
$
}
%
\renewcommand{\repartitionsizes}{
$ B_1 $ and $ \hat{B}_1 $ have $ b $ rows
and $ L_{11} $ is $ b \times b $
}

% Step 5b
\renewcommand{\moveboundaries}{%
$ 
\FlaTwoByOne{ B_T }{ B_B } 
\leftarrow
\FlaThreeByOneB{ B_0 }{ B_1 }{ B_2 },
\FlaTwoByOne{ \hat{B}_T }{ \hat{B}_B } 
\leftarrow
\FlaThreeByOneB{ \hat{B}_0 }{ \hat{B}_1 }{ \hat{B}_2 },
$ \\
and
$ \FlaTwoByTwo{ L_{TL} }{ 0 }
            { L_{BL} }{ L_{BR} }
\leftarrow
\FlaThreeByThreeBR{ L_{00} }{ 0 }{ 0 }
                { L_{10} }{ L_{11} }{ 0 }
                { L_{20} }{ L_{21} }{ L_{22} }
$
}

% Step 6
\renewcommand{\beforeupdate}{
\FlaTwoByOne{ 
\FlaTwoByOneSingleLine{ B_0 }{ B_1 }
}
{
B_2
} =
\FlaTwoByOne{ 
\FlaTwoByOneSingleLine{ \hat{B}_0 }{ \hat{B}_1 }
}
{
L_{22} \hat{B}_2
}
\wedge \ldots
}

% Step 7
\renewcommand{\afterupdate}{
\FlaTwoByOne{ 
B_0
}
{
\FlaTwoByOneSingleLine{ B_1 }{ B_2 }
} =
\FlaTwoByOne{ 
\hat{B}_0
}
{
\FlaTwoByOneSingleLine{ L_{11} \hat{B}_1 }{ L_{21} \hat{B}_1 + L_{22} \hat{B}_2 }
}
\wedge \ldots
}

% Step 8
\renewcommand{\update}{
\begin{minipage}[t]{4in}
\noindent
% \FlaStartCompute \\
$ B_2 \becomes L_{21} B_1 + B_2 $\\
$ B_1 \becomes L_{11} B_1 $\\
% \FlaEndCompute \\
\end{minipage}
}

% Generate the annotated algorithm in Fig. 4.4
\begin{figure}[htbp]
\worksheet
\caption{Annotated blocked algorithm for loop-invariant 1.}
\label{fig:ws:ltrmm_lln_blk}
\end{figure}
%

\subsubsection{Implementation}

Sequential implementations for the unblocked and blocked algorithms
for this loop-invariant using FLAME are given in
Figs.~\ref{fig:syrk_un_lazy_unb}--\ref{fig:syrk_un_lazy_blk}.

\begin{figure}[htbp]
\footnotesize
\begin{quote}
\listinginput{1}{materials/syrk_un/lazy-rowlazy/Syrk_left_lower_notrans_unb.c}
\end{quote}
\caption{Unblocked algorithm for loop-invariants 1 and 2 using FLAME.}
\label{fig:syrk_un_lazy_unb}
\end{figure}

\begin{figure}[htbp]
\footnotesize
\index{const}{\tt FLA\un RECURSIVE}%
\begin{quote}
\listinginput{1}{materials/syrk_un/lazy-rowlazy/Syrk_left_lower_notrans_blk.c}
\end{quote}
\caption{Blocked algorithm for loop-invariants 1 and 2 using FLAME.
Recursion is optionally supported.}
\label{fig:syrk_un_lazy_blk}
\end{figure}

\subsection{Loop-invariant 2}

We now examine the loop-invariant
\begin{equation}
\label{eqn:ltrmm_lln:p3}
\PInv: 
\FlaTwoByOne{ B_T}{B_B} =
\FlaTwoByOne{ \hat{B}_T }
            { L_{BL} \hat{B}_T + L_{BR} \hat{B}_B }
\wedge
\ldots
\end{equation}
Comparing the loop-invariant in (\ref{eqn:ltrmm_lln:p3}) with the
postcondition $ B = L \hat{B} $ we see again that {\em if} $L=L_{BR}$,
$ B = B_B $, and $ \hat{B} = \hat{B}_B $ then the loop-invariant
implies the postcondition, i.e., that the desired result has been
computed.
%
The initialization in Step 4 in Fig.~\ref{fig:ws:ltrmm_lln:var2}
is the same as that in Step 4 in Fig.~\ref{fig:ws:ltrmm_lln}.
The fact that for this partitioning of $ B $, $ \hat{B} $, and $ L $,
$
\FlaTwoByOne{ \hat{B}_T}
            { \hat{B}_B} =
\FlaTwoByOne{ \hat{B}_T}
            { L_{BL} \hat{B}_T + L_{BR} \hat{B}_B }
$
and the precondition implies the other parts of
the loop-invariant, this initialization has
the desired properties.
%
Similarly, the loop-guard is identical to
the one for loop-invariant 1.

\subsubsection{Unblocked algorithm}

The repartitionings in Step 5a in Fig.~\ref{fig:ws:ltrmm_lln:var2}
result in the state
\begin{equation}
\QBefore: 
\FlaTwoByOne{ 
\FlaTwoByOneSingleLine{ B_0 }{ b_1^T }
}
{
B_2
} =
\FlaTwoByOne{ 
\FlaTwoByOneSingleLine{ \hat{B}_0 }{ \hat{b}_1^T }
}
{
\FlaOneByTwoSingleLine{ L_{20} }{ l_{21} }
\FlaTwoByOneSingleLine{ \hat{B}_0 }{ \hat{b}_1^T } +
L_{22} \hat{B}_2
}
\wedge \ldots
\end{equation}
or 
\begin{equation}
\label{eqn:ltrmm_lln:var2:before}
\QBefore: 
\FlaTwoByOne{ 
\FlaTwoByOneSingleLine{ B_0 }{ b_1^T }
}
{
B_2
} =
\FlaTwoByOne{ 
\FlaTwoByOneSingleLine{ \hat{B}_0 }{ \hat{b}_1^T }
}
{
L_{20} \hat{B}_0 + l_{21} \hat{b}_1^T +
L_{22} \hat{B}_2
}
\wedge \ldots
\end{equation}
before the update.
The redefinition in Step 5b in Fig.~\ref{fig:ws:ltrmm_lln:var2}
means that the following state
must result from the update:
\[
\QAfter: 
\FlaTwoByOne{ 
B_0
}
{
\FlaTwoByOneSingleLine{ b_1^T }{ B_2 }
} =
\FlaTwoByOne{ 
\hat{B}_0
}
{
\FlaTwoByOneSingleLine{ l_{10}^T }
                      { L_{20} }
\hat{B}_0 + 
\FlaTwoByTwoSingleLine{ \lambda_{11} }{ 0 }
                      { l_{21} }{ L_{22} }
\FlaTwoByOneSingleLine{ \hat{b}_1^T }{ \hat{B}_2 }
}
\wedge \ldots
\]
or
\begin{equation}
\label{eqn:ltrmm_lln:var2:after}
\QAfter: 
\FlaTwoByOne{ 
B_0
}
{
\FlaTwoByOneSingleLine{ b_1^T }{ B_2 }
} =
\FlaTwoByOne{ 
\hat{B}_0
}
{
\FlaTwoByOneSingleLine{ l_{10}^T \hat{B}_0 + \lambda_{11} \hat{b}_1^T }
                      { L_{20} \hat{B}_0 + l_{21} \hat{b}_1^T + L_{22} \hat{B}_2 }
}
\wedge \ldots
\end{equation}


Comparing 
(\ref{eqn:ltrmm_lln:var2:before}) and
(\ref{eqn:ltrmm_lln:var2:after})
we find that the update
\begin{eqnarray*}
& b_1^T & \becomes l_{10}^T B_{0} + \lambda_{11} b_1^T 
\end{eqnarray*}
is required to change the state from $ \QBefore $
to $ \QAfter $.


Again one recognizes that $ \hat{B} $ is never
referenced and can be eliminated from the algorithm.

% Define the commands to generate the annotated
% algorithm in Fig. 4.7.

% Step 2
\renewcommand{\invariant}{
\FlaTwoByOne{ B_T}{B_B} =
\FlaTwoByOne{ \hat{B}_T }
            { L_{BL} \hat{B}_T + L_{BR} \hat{B}_B }
\wedge
\ldots
}

% Step 3
\renewcommand{\guard}{ \neg \SameSize( L, L_{BR} ) }

% Step 4
\renewcommand{\partitionings}{
$ 
B \rightarrow \FlaTwoByOne{ B_{T} }
                          { B_{B} }
$,
$ 
\hat{B} \rightarrow \FlaTwoByOne{ \hat{B}_{T} }
                          { \hat{B}_{B} }
$, and
$ 
L \rightarrow \FlaTwoByTwo{ L_{TL} }{ 0 }
                          { L_{BL} }{ L_{BR} }
$
}
\renewcommand{\partitionsizes}{
$ B_{B} $ and $ \hat{ B }_B $ have $ 0 $ rows
and $ L_{BR} $ is $ 0 \times 0 $
}

% Step 5a
\renewcommand{\repartitionings}{
$ 
\FlaTwoByOne{ B_T }{ B_B } 
\rightarrow
\FlaThreeByOneT{ B_0 }{ b_1^T }{ B_2 },
\FlaTwoByOne{ \hat{B}_T }{ \hat{B}_B } 
\rightarrow
\FlaThreeByOneT{ \hat{B}_0 }{ \hat{b}_1^T }{ \hat{B}_2 },
$ \\
and
$ \FlaTwoByTwo{ L_{TL} }{ 0 }
            { L_{BL} }{ L_{BR} }
\rightarrow
\FlaThreeByThreeTL{ L_{00} }{ 0 }{ 0 }
                { l_{10}^T }{ \lambda_{11} }{ 0 }
                { L_{20} }{ l_{21} }{ L_{22} }
$
}
\renewcommand{\repartitionsizes}{
$ b_1^T $ and $ \hat{b}_1^T $ are rows 
and $ \lambda_{11} $ is a scalar
}

% Step 5b
\renewcommand{\moveboundaries}{%
$ 
\FlaTwoByOne{ B_T }{ B_B } 
\leftarrow
\FlaThreeByOneB{ B_0 }{ b_1^T }{ B_2 },
\FlaTwoByOne{ \hat{B}_T }{ \hat{B}_B } 
\leftarrow
\FlaThreeByOneB{ \hat{B}_0 }{ \hat{b}_1^T }{ \hat{B}_2 },
$ \\
and
$ \FlaTwoByTwo{ L_{TL} }{ 0 }
            { L_{BL} }{ L_{BR} }
\leftarrow
\FlaThreeByThreeBR{ L_{00} }{ 0 }{ 0 }
                { l_{10}^T }{ \lambda_{11} }{ 0 }
                { L_{20} }{ l_{21} }{ L_{22} }
$
}

% Step 6
\renewcommand{\beforeupdate}{
\FlaTwoByOne{ 
\FlaTwoByOneSingleLine{ B_0 }{ b_1^T }
}
{
B_2
} =
\FlaTwoByOne{ 
\FlaTwoByOneSingleLine{ \hat{B}_0 }{ \hat{b}_1^T }
}
{
\FlaOneByTwoSingleLine{ L_{20} }{ l_{21} }
\FlaTwoByOneSingleLine{ \hat{B}_0 }{ \hat{b}_1^T } +
L_{22} \hat{B}_2 
}
\wedge \ldots
}

% Step 7
\renewcommand{\afterupdate}{
\FlaTwoByOne{ 
B_0
}
{
\FlaTwoByOneSingleLine{ b_1^T }{ B_2 }
} =
\FlaTwoByOne{ 
\hat{B}_0
}
{
\FlaTwoByOneSingleLine{ \lambda_{11} \hat{b}_1^T }{ l_{21} \hat{b}_1^T + L_{22} \hat{B}_2 }
}
\wedge \ldots
}

% Step 8
\renewcommand{\update}{
\begin{minipage}[t]{4in}
\noindent
% \FlaStartCompute \\
$b_1^T \becomes l_{10}^T B_{0} + \lambda_{11} b_1^T $ \\
% \FlaEndCompute \\
\end{minipage}
}

% Generate figure 4.7
\begin{figure}[htbp]
\worksheet
\caption{Annotated unblocked algorithm for loop-invariant 2.}
\label{fig:ws:ltrmm_lln:var2}
\end{figure}

\subsubsection{Blocked Algorithms}

Again, the algorithm can be cast to be rich in
matrix-matrix multiplications by marching through
the matrices multiple rows and/or columns at a time.
The resulting algorithm is given in 
Fig.~\ref{fig:ws:ltrmm_lln:var2:blk}.

% Redefine commands to generate Fig. 4.8

% Step 5a
\renewcommand{\repartitionings}{
$ 
\FlaTwoByOne{ B_T }{ B_B } 
\rightarrow
\FlaThreeByOneT{ B_0 }{ B_1 }{ B_2 },
\FlaTwoByOne{ \hat{B}_T }{ \hat{B}_B } 
\rightarrow
\FlaThreeByOneT{ \hat{B}_0 }{ \hat{B}_1 }{ \hat{B}_2 },
$ \\
and
$ \FlaTwoByTwo{ L_{TL} }{ 0 }
            { L_{BL} }{ L_{BR} }
\rightarrow
\FlaThreeByThreeTL{ L_{00} }{ 0 }{ 0 }
                { L_{10} }{ L_{11} }{ 0 }
                { L_{20} }{ L_{21} }{ L_{22} }
$
}
%
\renewcommand{\repartitionsizes}{
$ B_1 $ and $ \hat{B}_1 $ have $ b $ rows
and $ L_{11} $ is $ b \times b $
}

% Step 5b
\renewcommand{\moveboundaries}{%
$ 
\FlaTwoByOne{ B_T }{ B_B } 
\leftarrow
\FlaThreeByOneB{ B_0 }{ B_1 }{ B_2 },
\FlaTwoByOne{ \hat{B}_T }{ \hat{B}_B } 
\leftarrow
\FlaThreeByOneB{ \hat{B}_0 }{ \hat{B}_1 }{ \hat{B}_2 },
$ \\
and
$ \FlaTwoByTwo{ L_{TL} }{ 0 }
            { L_{BL} }{ L_{BR} }
\leftarrow
\FlaThreeByThreeBR{ L_{00} }{ 0 }{ 0 }
                { L_{10} }{ L_{11} }{ 0 }
                { L_{20} }{ L_{21} }{ L_{22} }
$
}

% Step 6
\renewcommand{\beforeupdate}{
\FlaTwoByOne{ 
\FlaTwoByOneSingleLine{ B_0 }{ B_1 }
}
{
B_2
} =
\FlaTwoByOne{ 
\FlaTwoByOneSingleLine{ \hat{B}_0 }{ \hat{B}_1 }
}
{
\FlaOneByTwoSingleLine{L_{20}}{L_{21}}
\FlaTwoByOneSingleLine{\hat{B}_0}{ \hat{B}_1 } +
L_{22} \hat{B}_2
}
\wedge \ldots
}

% Step 7
\renewcommand{\afterupdate}{
\FlaTwoByOne{ 
B_0
}
{
\FlaTwoByOneSingleLine{ B_1 }{ B_2 }
} =
\FlaTwoByOne{ 
\hat{B}_0
}
{
\FlaTwoByOneSingleLine{ L_{10} }{ L_{20} }
\hat{B}_0 +
\FlaTwoByTwoSingleLine{ L_{11} }{ L_{12} }
                      { L_{21} }{ L_{22} }
\FlaTwoByOneSingleLine{ \hat{B}_1 }{ \hat{B}_2 }
}
\wedge \ldots
}

% Step 8
\renewcommand{\update}{
\begin{minipage}[t]{4in}
\noindent
% \FlaStartCompute \\
$ B_1 \becomes L_{10} B_0 + L_{11} B_1 $\\
% \FlaEndCompute \\
\end{minipage}
}

% Generate Fig. 4.8
\begin{figure}[htbp]
\worksheet
\caption{Annotated blocked algorithm for loop-invariant 2.}
\label{fig:ws:ltrmm_lln:var2:blk}
\end{figure}
%

\subsubsection{Implementation}

Sequential implementations for the unblocked and blocked algorithms
for this loop-invariant using FLAME are given in
Figs.~\ref{fig:syrk_un_lazy_unb}--\ref{fig:syrk_un_lazy_blk}.

\section{Algorithms that start by partitioning $ B $}

In Section~\ref{sec:syrk_un:L} we saw that when we start by
partitioning matrix $ L $, two feasible loop-invariants result.
However, the recipe for deriving algorithms indicates that one should
start by picking {\em a} variable without indicating {\em which}
variable.  An alternative set of loop-invariants results when one
starts by partitioning matrix $ B $.  The partitioning of $ B $
introduces an identical partitioning of $ \hat{B} $ since they
reference the same data.

\subsection{Horizontal partitioning of $ B$}

Revisiting Step 2, we first partition
\[
B \rightarrow \FlaTwoByOne{ B_{T}}{ B_{B} }
\quad
\mbox{and}
\quad
\hat{B} \rightarrow \FlaTwoByOne{ \hat{B}_{T}}{ \hat{B}_{B} }.
\]
Plugging these partitionings into the postcondition yields
\[
\FlaTwoByOne{ B_{T}}{ B_{B} }
\quad
=
\quad
% \FlaTwoByTwo{ L_{TL} }{ 0 }
%            { L_{BL} }{ L_{BR} }
( \mbox{ some partitioning of $ L $ } )
\FlaTwoByOne{ \hat{B}_{T}}{ \hat{B}_{B} }.
\]
In order to be able to multiply out the expression on the right we
will have to partition $ L $ into quadrants.  The astute reader will
recognize that now the partitionings of $ B $, $ \hat{B} $, and $ L $
are identical to those introduced in the example in
Section~\ref{sec:trmm_lln:L} and will thus lead to the same
loop-invariants and algorithms.

\subsection{Vertical partitioning of $ B$}

Now, let us partition instead like
\[
B \rightarrow \FlaOneByTwo{ B_{L}}{ B_{R} }
\quad
\mbox{and}
\quad
\hat{B} \rightarrow \FlaOneByTwo{ \hat{B}_{L}}{ \hat{B}_{R} }.
\]
Plugging these partitionings into the postcondition yields
\[
\FlaOneByTwo{ B_{L}}{ B_{R} }
\quad
=
\quad
( \mbox{ some partitioning of $ L $ } )
\FlaOneByTwo{ \hat{B}_{L}}{ \hat{B}_{R} }.
\]
In order to be able to multiply out the expression on the right we
recognize that $ L $ should {\em not} be partitioned.

The condition from which we will determine possible loop-invariants
now becomes
\[
\FlaOneByTwo{ B_{L}}{ B_{R} }
=
L
\FlaOneByTwo{ \hat{B}_{L}}{ \hat{B}_{R} }
=
\FlaOneByTwo{ L \hat{B}_{L}}{ L \hat{B}_{R} }
\]
The two feasible loop-invariants that can be identified from this
equality are
\begin{center}
\begin{tabular}{p{1.5in}  l}
Loop-invariant 3: & 
$
\FlaOneByTwo{ B_{L}}{ B_{R} }
=
\FlaOneByTwo{ L \hat{B}_{L}}{ \hat{B}_{R} }
$ \\ 
Loop-invariant 4: & 
$
\FlaOneByTwo{ B_{L}}{ B_{R} }
=
\FlaOneByTwo{ \hat{B}_{L}}{ L \hat{B}_{R} }.
$
\end{tabular}
\end{center}
An annotated blocked algorithm corresponding to Loop-invariant 3 is
given in Fig.~\ref{fig:ws:ltrmm_lln3_blk}.

\renewcommand{\precondition}{B = \hat{B} \wedge \LowTr( L ) \wedge \ColDim( L )=\RowDim( B )}

\renewcommand{\postcondition}{B = L \hat{B}}

\renewcommand{\invariant}{
\FlaOneByTwo{ B_L}{B_R} =
\FlaOneByTwo{ L \hat{B}_L }{ \hat{B}_R }
\wedge
\ldots
}

\renewcommand{\guard}{ \neg \SameSize( B, B_L ) }

\renewcommand{\partitionings}{
$ 
B \rightarrow \FlaOneByTwo{ B_{L} }{ B_{R} }
$ and
$ 
\hat{B} \rightarrow \FlaOneByTwo{ \hat{B}_{L} }
                          { \hat{B}_{R} }
$
}
\renewcommand{\partitionsizes}{
$ B_{L} $ and $ \hat{ B }_L $ have $ 0 $ columns
}

\renewcommand{\blocksize}{b}

\renewcommand{\repartitionings}{
$ 
\FlaOneByTwo{ B_L }{ B_R } 
\rightarrow
\FlaOneByThreeR{ B_0 }{ B_1 }{ B_2 }
$
and
$ 
\FlaOneByTwo{ \hat{B}_L }{ \hat{B}_R } 
\rightarrow
\FlaOneByThreeR{ \hat{B}_0 }{ \hat{B}_1 }{ \hat{B}_2 }
$
}
\renewcommand{\repartitionsizes}{
$ B_1 $ and $ \hat{B}_1 $ have $ b $ columns
}

\renewcommand{\moveboundaries}{%
$ 
\FlaOneByTwo{ B_L }{ B_R } 
\leftarrow
\FlaOneByThreeL{ B_0 }{ B_1 }{ B_2 }
$
and
$ 
\FlaOneByTwo{ \hat{B}_L }{ \hat{B}_R } 
\leftarrow
\FlaOneByThreeL{ \hat{B}_0 }{ \hat{B}_1 }{ \hat{B}_2 }
$
}

\renewcommand{\beforeupdate}{
\FlaOneByTwo{ B_0 }{ \FlaOneByTwoSingleLine{ B_1 }{ B_2 } }
=
\FlaOneByTwo{ L \hat{B}_0 }{ \FlaOneByTwoSingleLine{ \hat{B}_1 }{ \hat{B}_2 } }
\wedge \ldots
}

\renewcommand{\afterupdate}{
\FlaOneByTwo{ \FlaOneByTwoSingleLine{ B_0 }{ B_1 } }
{ B_2 }
=
\FlaOneByTwo{ L \FlaOneByTwoSingleLine{ \hat{B}_0 }{ \hat{B}_1 } }{ \hat{B}_2 }
\wedge \ldots
}

\renewcommand{\update}{
\begin{minipage}[t]{4in}
\noindent
% \FlaStartCompute \\
$ B_1 \becomes L B_1 $\\
% \FlaEndCompute \\
\end{minipage}
}

\begin{figure}[htbp]
\worksheet
\caption{Annotated blocked algorithm for the lower triangular
matrix multiplication example.}
\label{fig:ws:ltrmm_lln3_blk}
\end{figure}

\subsection{Implementation}

An implementation for the blocked algorithm corresponding to
Loop-invariant 3 is given Figs.~\ref{fig:syrk_un_right}.

\begin{figure}[htbp]
\begin{quote}
\index{const}{\tt FLA\un RECURSIVE}%
\listinginput{1}{materials/syrk_un/rightmoving/Syrk_un_right_wrt_B_blk.c}
\end{quote}
\caption{Blocked algorithm for loop-invariants 3 using FLAME.}
\label{fig:syrk_un_right}
\end{figure}

\begin{quote}
\bf Note to the students: I have eliminated as much
as possible references to ``lazy'' and ``row-lazy'' algorithms.  The
rest will be removed from the remainder of the chapter in the near
future.  You need to number the feasible loop-invariants, and refere
to these numbers instead.
\end{quote}

\section{Performance}

In this section, we discuss the performance attained by the different
variants for computing $ B \leftarrow L B $.  In each of the Figs.
~\ref{fig:syrk_un:lazy-row-lazy:ATLAS} and
~\ref{fig:syrk_un:lazy-row-lazy:ITXGEMM}, we compare the performance
attained by five different implementations:
\begin{itemize}
\item{Reference}
{\tt DTRMM} as implemented as part of ATLAS,
\item{FLAME}
{\tt FLA\_Trmm} implemented as part of FLAME
as of this writing,
\item{Unblocked}
the unblocked implementation in Fig.~\ref{fig:syrk_un_lazy_unb},
\item{Blocked}
the blocked implementation in Fig.~\ref{fig:syrk_un_lazy}
called with 
{\tt rec == 
\index{const}{\tt FLA\un NON\un RECURSIVE}%
FLA\_NON\_RECURSIVE}, and
\item{Recursive}
the blocked implementation in Fig.~\ref{fig:syrk_un_lazy}
called with 
{\tt rec == 
\index{const}{\tt FLA\un RECURSIVE}%
FLA\_RECURSIVE}.
\end{itemize}
In Fig.~\ref{fig:syrk_un:lazy-row-lazy:ATLAS}, all FLAME
implementations are based on the matrix-matrix multiplication ({\tt
DGEMM}) provided by ATLAS.  Notice that for the platform on which we
performed the experiments, multiples of 40 are good block sizes when
using the ATLAS matrix-matrix multiplication kernel.  We note that the
unblocked algorithms perform badly, since they are rich in
matrix-vector operations that do not benefit much from cache memories.
The blocked algorithms performed better for relatively small block
sizes ({\tt nb\_alg=80}) than for larger block sizes ({\tt
nb\_alg=160}).  The reason for this is that too much of the
computation is performed in the subprogram for which the unblocked
algorithm is used.  The recursive implementations benefit from larger
block sizes, since they overcome this problem that plagues the blocked
algorithm.  In addition they benefit from the fact that the {\tt
DGEMM} kernel performs better for larger blocks.

In Fig.~\ref{fig:syrk_un:lazy-row-lazy:ITXGEMM}, all FLAME
implementations are based on the matrix-matrix multiplication ({\tt
DGEMM}) provided by ITXGEMM.  This time, as discussed in
Section~\ref{sec:mmmult:impact}, 128 is a magic block size.  Again, as
expected, the unblocked implementations perform badly.  Again, the
blocked algorithm benefits from smaller block sizes and the recursive
algorithm performs best.  It is interesting to note that the lazy
algorithm performs much better for small problem sizes when the
outer-most block size is chosen to be 128.  This is probably mostly
due to the fact that the unblocked lazy algorithm performs better.

It is interesting to note that asymptotically, the recursive row-lazy
implementation performs somewhat worse than the recursive lazy
implementation, when ATLAS is used for the matrix-matrix multiply
(Fig.~\ref{fig:syrk_un:lazy-row-lazy:ATLAS}).  We attribute this to
the fact that when $ m $ is relatively small and $ n $ and $ k $ are
large (e.g., in a panel-matrix multiply), the ATLAS matrix-matrix
multiplication does not perform as well as when $ k $ is relatively
small and $ m $ and $ n $ are large (e.g., in a panel-panel multiply).
Since the row-lazy and lazy algorithms are rich in panel-matrix and
panel-panel multiplies, respectively, the lazy algorithm attains
better performance.  This is not observed as dramatically in in the
experiments where ITXGEMM was used for matrix-matrix multiplication
(Fig.~\ref{fig:syrk_un:lazy-row-lazy:ITXGEMM}).  This is due to the
fact that ITXGEMM attains performance that is similar regardless of
the shape of the matrix.

Notice that the right-moving algorithms perform miserably
(Fig.~\ref{fig:syrk_un:right-moving}).  We point out that the blocked
and recursive algorithms attain essentially the same performance as
the unblocked algorithm.  This is not surprising since all computation
is in a triangular matrix-vector multiply, which does not perform very
well, regardless of how the blocking proceeds.

Since in the recursive algorithms at each level a different variant
could be called to solve the subproblem, it may be possible to improve
performance further by combining different variants.  We have not yet
studied this.

\begin{figure}[htbp]
\begin{center}
\begin{tabular}{c | c}
Row-lazy & Lazy \\ \hline
& \\
\psfig{figure=syrk_un/graphs/syrk_un_rowlazy_wrt_L_80.eps,width=3.0in,height=3.0in} &
\psfig{figure=syrk_un/graphs/syrk_un_lazy_wrt_L_80.eps,width=3.0in,height=3.0in}
\\ \hline
& \\
\psfig{figure=syrk_un/graphs/syrk_un_rowlazy_wrt_L_160.eps,width=3.0in,height=3.0in} &
\psfig{figure=syrk_un/graphs/syrk_un_lazy_wrt_L_160.eps,width=3.0in,height=3.0in}
\end{tabular}
\end{center}
\caption{Performance of the row lazy (left) and lazy (right) (w.r.t. $ L $) 
lower triangular matrix-matrix multiplication 
algorithms for a block size of $ 80 $ (top) and
$ 160 $ (bottom).
For these experiments, the ATLAS matrix-matrix multiplication
kernel was used.}
\label{fig:syrk_un:lazy-row-lazy:ATLAS}
\end{figure}

\begin{figure}[htbp]
\begin{center}
\begin{tabular}{c | c}
Row-lazy & Lazy \\ \hline
& \\
\psfig{figure=syrk_un/graphs/syrk_un_rowlazy_wrt_L_32.eps,width=3.0in,height=3.0in} &
\psfig{figure=syrk_un/graphs/syrk_un_lazy_wrt_L_32.eps,width=3.0in,height=3.0in}
\\ \hline
& \\
\psfig{figure=syrk_un/graphs/syrk_un_rowlazy_wrt_L_128.eps,width=3.0in,height=3.0in} &
\psfig{figure=syrk_un/graphs/syrk_un_lazy_wrt_L_128.eps,width=3.0in,height=3.0in}
\end{tabular}
\end{center}
\caption{Performance of the row lazy (left) and lazy (right) (w.r.t. $ L $) 
lower triangular matrix-matrix multiplication 
algorithms for a block size of $ 32 $ (top) and
$ 128 $ (bottom).
For these experiments, the ITXGEMM matrix-matrix multiplication
kernel was used.}
\label{fig:syrk_un:lazy-row-lazy:ITXGEMM}
\end{figure}

\begin{figure}[htbp]
\begin{center}
\begin{tabular}{c | c}
\psfig{figure=syrk_un/graphs/syrk_un_right_wrt_B_32.eps,width=3.0in,height=3.0in} &
\psfig{figure=syrk_un/graphs/syrk_un_right_wrt_B_128.eps,width=3.0in,height=3.0in}
\end{tabular}
\end{center}
\caption{Performance of the right-moving (w.r.t. $ B $) triangular matrix-matrix multiplication 
algorithms for a block size of $ 32 $ (left) and
$ 128 $ (right), using ITXGEMM.}
\label{fig:syrk_un:right-moving}
\end{figure}
\index{op}{matrix-matrix multiplication!$ A $ lower triangular!$ B \leftarrow A B $|)}%
