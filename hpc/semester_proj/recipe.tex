\documentclass{article}

\usepackage{psfig}
\usepackage{moreverb}
\usepackage{latexsym}
\usepackage{rotating}
\usepackage{makeidx}  % allows for indexgeneration

\input{flatex}
%
% ExampleBox draws a box around an example in the FLAME
% document
%

\newcommand{\ExampleBox}[2]{
\begin{flushright}
\fbox{
\begin{minipage}{5.5in}
{\bf Example #1}
#2
\end{minipage}
}
\end{flushright}
}

\newcommand{\NoteBox}[1]{
\begin{flushright}
\fbox{
\begin{minipage}{6.0in}
{\bf Note:}
#1
\end{minipage}
}
\end{flushright}
}

\newcommand{\ChapterAuthor}[1]{
{
\vspace{-0.2in}
\large 
by \\[0.1in]
\bf
#1
}
\vspace{0.5in}
}

\newcommand{\diagram}[1]{
}

\newcommand{\SummaryBox}[3]{
\index{rout}{{#2}}
\begin{tabular}{| p{2.0in} @{} | @{\hspace{1pt}} p{4.3in} | } \hline
\begin{minipage}{1.8in}
#1 
\end{minipage}
& 
\begin{minipage}{4.3in}
\begin{description}
\item
{\tt #2 #3}
\end{description}
\end{minipage}
\\ \hline
\end{tabular}
}

\newcommand{\un}{\_}%\hspace{-5pt}}

\newcommand{\parameter}[2]{
\begin{minipage}[t]{1.5in}
#1
\end{minipage}
\begin{minipage}[t]{4.25in}
#2
\end{minipage}
}

\newcommand{\purpose}[1]{
\vspace{-0.1in}
{\bf Purpose:} #1
\vspace{0.05in}
}

\newenvironment{FlaSpec}{
\begin{center}
\begin{tabular}{| p{6.25in} |} \hline
\vspace{-0.2in}
}
{
\\ \hline
\end{tabular}
\end{center}
}

\newcommand{\conj}[1]{{\tt conj}({#1})}

% \newcommand{\status}[1]{\addtocontents{toc}{{\\ \bf Status: #1}}}

\newcommand{\status}[1]{}

\newcounter{stepnumber}
% \addtocounter{stepnumber}{1}%

\newcommand{\step}{
\refstepcounter{stepnumber}%
\noindent
{\bf Step \thestepnumber:}
}

\newenvironment{codeexample}
{
\begin{quote}
}
{
\end{quote}
}

\newcommand{\becomes}{:=}

\newcommand{\equals}{=}

\newcommand{\implies}{\Rightarrow}

\newcommand{\longggarrow}{
\setlength{\unitlength}{0.5in}
\begin{picture}(5,1)
\put(5,0){\vector(-1,0){5}}
\end{picture}
}

\newcommand{\PPre}{P_{\rm pre}}
\newcommand{\PPost}{P_{\rm post}}
\newcommand{\PInv}{P_{\rm inv}}

\newcommand{\QBefore}{Q_{\rm before}}
\newcommand{\QAfter}{Q_{\rm after}}

\newcommand{\Qbefore}{Q_{\rm before}}
\newcommand{\Qafter}{Q_{\rm after}}


\newtheorem{theorem}{Theorem}[section]
\newtheorem{conjecture}[theorem]{Conjecture}
\newtheorem{corollary}[theorem]{Corollary}
\newtheorem{proposition}[theorem]{Proposition}
\newtheorem{lemma}[theorem]{Lemma}
\newtheorem{definition}[theorem]{Definition}
\newtheorem{remark}[theorem]{Remark}
\newtheorem{exercise}[theorem]{Exercise}
\newtheorem{example}[theorem]{Example}
\newtheorem{note}{Note}
\newtheorem{algorithm}{Algorithm}

% \newcommand{\PPre}{P_{\rm pre}}
% \newcommand{\PPost}{P_{\rm post}}
% \newcommand{\PInv}{P_{\rm inv}}

\title{
A Worksheet for Deriving
Dense Linear Algebra Algorithms
}
            
\author{
John Wiggins \\
The University of Texas at Austin
}

\begin{document}
            
\maketitle

\begin{abstract} 
Exercise 4.1.
\end{abstract}
            

\resetsteps

\begin{figure}[tbp]
\begin{center}
\worksheet
\end{center}
\caption{Worksheet for deriving linear algebra algorithms}
\end{figure}

\renewcommand{\operation}{B \becomes L^{-1} B}



\renewcommand{\precondition}{B = \hat{B} \wedge \cdots }


\renewcommand{\postcondition}{{ B = L^{-1} \hat{B} }}


\renewcommand{\invariant}{
\FlaTwoByOne{ B_T }
            { B_B } 
= 
\FlaTwoByOne{ L_{TL}^{-1} \hat{B}_T }
            { \hat{B}_B - L_{BL} L_{TL}^{-1} \hat{B}_T }
}


\renewcommand{\guard}{ \neg \SameSize( L_{TL}, L ) }


\renewcommand{\partitionings}{
$
L \rightarrow
\FlaTwoByTwo{ L_{TL} } { 0 }
              { L_{BL} } { L_{BR}   }
$,
$
B \rightarrow 
\FlaTwoByOne{ B_T }
            { B_B }
$ and
$
\hat{B} \rightarrow 
\FlaTwoByOne{ \hat{B}_T }
            { \hat{B}_B }
$
}


\renewcommand{\partitionsizes}{$ L_{TL} $ is $ 0 \times 0 $, and ... }




\renewcommand{\repartitionings}{ 
\normalsize
$
\FlaTwoByTwo{ L_{TL} } { 0 }
              { L_{BL} } { L_{BR}   } \rightarrow
\FlaThreeByThreeBR{ L_{00} }{ 0 }{ 0 }
              { L_{10} }{ L_{11} }{ 0 }
              { L_{20} }{ L_{21} }{ L_{22} }
$,
$
\FlaTwoByOne{ B_T }
            { B_B }
\rightarrow
\FlaThreeByOneB{ B_0 }
               { B_1 }
               { B_2 }
$, and \\
$
\FlaTwoByOne{ \hat{B}_T }
            { \hat{B}_B }
\rightarrow
\FlaThreeByOneB{ \hat{B}_0 }
               { \hat{B}_1 }
               { \hat{B}_2 }
$
}




\renewcommand{\repartitionsizes}{
$ B_1 $ and $ \hat{B}_1 $ are columns and $ L_{11} $ is a matrix}




\renewcommand{\moveboundaries}{ 
\normalsize
$
\FlaTwoByTwo{ L_{TL} } { 0 }
              { L_{BL} } { L_{BR}   } \leftarrow
\FlaThreeByThreeTL{ L_{00} }{ 0 }{ 0 }
              { L_{10} }{ L_{11} }{ 0 }
              { L_{20} }{ L_{21} }{ L_{22} }
$,
$
\FlaTwoByOne{ B_T }
            { B_B }
\leftarrow
\FlaThreeByOneT{ B_0 }
               { B_1 }
               { B_2 }
$, and \\
$
\FlaTwoByOne{ \hat{B}_T }
            { \hat{B}_B }
\leftarrow
\FlaThreeByOneT{ \hat{B}_0 }
               { \hat{B}_1 }
               { \hat{B}_2 }
$
}




\renewcommand{\beforeupdate}{
\FlaThreeByOneB{ B_0 }
              { B_1 }
              { B_2 }
=
\FlaThreeByOneB{ L_{00}^{-1} \hat{B}_0 }
              { \hat{B}_1 - L_{10} L_{00}^{-1} \hat{B}_0 }
              { \hat{B}_2 - L_{20} L_{00}^{-1} \hat{B}_0}
}




\renewcommand{\afterupdate}{
\FlaThreeByOneB{ B_0 }
              { B_1 }
              { B_2 }
=
\FlaThreeByOneB{ L_{00}^{-1} \hat{B}_0 }
              { L_{11}^{-1} ( \hat{B}_1 - L_{10} L_{00}^{-1} \hat{B}_0 ) }
              { \hat{B}_2 - ( L_{20} L_{00}^{-1} \hat{B}_0 + L_{21} L_{11}^{-1} ( \hat{B}_1 - L_{10} L_{00}^{-1} \hat{B}_0 ) ) }
}




\renewcommand{\update}{
$
\begin{array}{l}
B_1 \becomes L_{11}^{-1} B_1
\\
B_2 \becomes B_2 - L_{21} B_1
\end{array}
$
}


\begin{figure}[tbp]
\begin{center}
\worksheet
\end{center}
\caption{Worksheet for deriving algorithm for $ B \becomes L^{-1} B $.}
\end{figure}

\renewcommand{\partitionings}{
$
L \rightarrow
\FlaTwoByTwo{ L_{TL} } { 0 }
              { L_{BL} } { L_{BR}   }
$ and
$
B \rightarrow 
\FlaTwoByOne{ B_T }
            { B_B }
$
}


\renewcommand{\partitionsizes}{$ L_{TL} $ is $ 0 \times 0 $, and ... }




\renewcommand{\repartitionings}{ 
\normalsize
$
\FlaTwoByTwo{ L_{TL} } { 0 }
              { L_{BL} } { L_{BR}   } \rightarrow
\FlaThreeByThreeBR{ L_{00} }{ 0 }{ 0 }
              { l_{10}^T }{ \lambda_{11} }{ 0 }
              { L_{20} }{ l_{21} }{ L_{22} }
$ and
$
\FlaTwoByOne{ B_T }
            { B_B }
\rightarrow
\FlaThreeByOneB{ B_0 }
               { b_1^T }
               { B_2 }
$
}




\renewcommand{\repartitionsizes}{
$ b_1^T $ is a row and $ \lambda_{11} $ is a scalar}




\renewcommand{\moveboundaries}{ 
\normalsize
$
\FlaTwoByTwo{ L_{TL} } { 0 }
              { L_{BL} } { L_{BR}   } \leftarrow
\FlaThreeByThreeTL{ L_{00} }{ 0 }{ 0 }
              { l_{10}^T }{ \lambda_{11} }{ 0 }
              { L_{20} }{ l_{21} }{ L_{22} }
$ and
$
\FlaTwoByOne{ B_T }
            { B_B }
\leftarrow
\FlaThreeByOneT{ B_0 }
               { b_1^T }
               { B_2 }
$
}

\begin{figure}[tbp]
\begin{center}
\FlaAlgorithm
\end{center}
\caption{Algorithm for $ B \becomes L^{-1} B $.}
\end{figure}


\resetsteps

\renewcommand{\operation}{ \parbox{3.5in}{\ }  }


\renewcommand{\precondition}{ \parbox{3in}{\ } }


\renewcommand{\postcondition}{ \parbox{3in}{\ } } 


\renewcommand{\invariant}{ 
\begin{array}{c}
\\
\\
\\
\end{array} \parbox{3in}{\ }
}


\renewcommand{\guard}{ \parbox{1in}{ \ } }


\renewcommand{\partitionings}{
$
\begin{array}{c}
\\
\\
\\
\end{array} \parbox{3in}{\ }
$
}


\renewcommand{\partitionsizes}{\ }




\renewcommand{\repartitionings}{ 
$
\begin{array}{c}
\\
\\
\\
\\
\end{array} \parbox{3in}{\ }
$
}




\renewcommand{\repartitionsizes}{\ }



\renewcommand{\moveboundaries}{ 
$
\begin{array}{c}
\\
\\
\\
\\
\end{array} \parbox{3in}{\ }
$
}




\renewcommand{\beforeupdate}{
\begin{array}{c}
\\
\\
\\
\\
\end{array} \parbox{3in}{\ }
}




\renewcommand{\afterupdate}{
\begin{array}{c}
\\
\\
\\
\\
\end{array} \parbox{3in}{\ }
}




\renewcommand{\update}{
$
\begin{array}{c}
\\
\\
\\
\\
\end{array} \parbox{3in}{\ }
$
}

\begin{figure}[tbp]
\begin{center}
\worksheet
\end{center}
\caption{Blank Worksheet for deriving linear algebra algorithms}
\end{figure}

\end{document}
